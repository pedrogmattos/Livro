\part{{\scshape Análise}}

\chapter{Espaços Métricos}

\section{Os Espaços Métricos}

\subsection{Métricas}

\begin{defi}
Seja $M$ um conjunto. Uma \emph{métrica} em $M$ é uma função $d: M \times M \to \R$ que satisfaz
	\begin{enumerate}
	\item (Definição Positiva) Para todos $p_0,p_1 \in M$,
		\begin{equation*}
		d(p_0,p_1)=0 \sse p_0=p_1;
		\end{equation*}
	\item (Simetria) Para todos $p_0,p_1 \in M$
		\begin{equation*}
		d(p_0,p_1)=d(p_1,p_0);
		\end{equation*}
	\item (Desigualdade Triangular) Para todos $p_0,p_1,p_2 \in M$,
		\begin{equation*}
		d(p_1,p_3) \leq d(p_1,p_2) + d(p_2,p_3).
		\end{equation*}
	\end{enumerate}
A \emph{distância} de $p_0$ a $p_1$ é o número real $d(p_0,p_1)$.
\end{defi}

\begin{prop}
Sejam $M$ um conjunto e $d$ uma métrica em $M$. Então
	\begin{enumerate}
	\item (Positividade) Para todos $p_0,p_1 \in M$, $d(p_1,p_2) \geq 0$.
	\item (Desigualdade Triangular Generalizada) Para todos $p_0,\ldots,p_n \in M$,
		\begin{equation*}
		d(p_0,p_n) \leq \bigplus_{i=1}^{n-1} d(p_i,p_{i+1})
		\end{equation*}
%	\item Desigualdade Trianguar Alternativa \\
%	$\forall p_1,p_2,p_3 \in M \qquad |d(p_1,p_3)-d(p_2,p_3)| \leq d(p_1,p_2)$.
	\end{enumerate}
\end{prop}
\begin{proof}
	\begin{enumerate}
	\item Sejam $p_1,p_2 \in M$. Da desigualdade triangular e da simetria de $d$, segue que
	\begin{equation*}
	d(p_1,p_1) \leq d (p_1,p_2)+d(p_2,p_1)= 2 d(p_1,p_2),
	\end{equation*}
Mas $d(p_1,_1) = 0$, o que implica $d(p_1,p_1) \geq 0$.
	\item Para $n=1$, seja $p_1 \in M$; então $d(p_1,p_1)=0$ e $\bigplus_{i=1}^{0} d(p_i,p_{i+1})=0$, pois a soma é vazia. Para $n=2$, sejam $p_1,p_2 \in M$; então $d(p_1,p_2)$ e $\bigplus_{i=1}^{1} d(p_i,p_{i+1})=d(p_1,p_2)$, e vale a propriedade. Para $n=3$, sejam $p_1,p_2,p_3 \in M$; então a propriedade é a desigualdade triangulal. Agora, sejam $n \geq 4$, $p_1,\ldots,p_n \in M$ e assumamos que a propriedade vale para todo $k \in \N$, tal que $3 \leq k \leq n-1$. Então
	\begin{equation*}
	d(p_1,p_n) \leq \bigplus_{i=1}^{n-3} d(p_i,p_{i+1})+d(p_{n-2},p_n),
	\end{equation*}
pois essa soma tem $n-1$ termos e vale a hipótese de indução. Pela desigualdade triangular, vale que $d(p_{n-2},p_n) \leq d(p_{n-2},p_{n-1})+d(p_{n-1},p_n)$, e, portanto,
	\begin{align*}
	d(p_1,p_n) &\leq \bigplus_{i=1}^{n-3} d(p_i,p_{i+1})+d(p_{n-2},p_n) \\
			&\leq \bigplus_{i=1}^{n-3} d(p_i,p_{i+1}) + d(p_{n-2},p_{n-1})+d(p_{n-1},p_n) \\
			&= \bigplus_{i=1}^{n-1} d(p_i,p_{i+1}).
	\end{align*}
	\end{enumerate}
\end{proof}

\begin{prop}
	Sejam $M$ um conjunto e $d_0, \dots, d_n$ métricas em $M$. Então a função
	\begin{align*}
		\func{d}{M \times M}{\R}{(p_0,p_1)}{\bigplus_{i=0}^n d_i(p_0,p_1)}
	\end{align*}
é uma métrica em $M$.
\end{prop}
\begin{proof}
	\begin{enumerate}
	\item (Identidade dos Indiscerníveis) Sejam $p_1,p_2 \in M$. Suponhamos que
	\begin{equation*}
	d(p_1,p_2) = \bigplus_{i=1}^n d_i(p_1,p_2) = 0.
	\end{equation*}
Como, para todo $i \in \inte_n$, $d_i(p_1,p_2) \geq 0$, então, para todo $i \in \inte_n$, $d_i(p_1,p_2) = 0$. Logo $p_1=p_2$. Reciprocamente, suponhamos $p_1=p_2$. Então, para todo $i \in \inte_n$, $d_i(p_1,p_2)=0$, o que implica
	\begin{equation*}
	d(p_1,p_2) = \bigplus_{i=1}^n 0 = 0.
	\end{equation*}
	
	\item (Simetria) Sejam $p_1,p_2 \in M$. Então, pela simetria de $d_i$ para todo $i \in \inte_n$,
	\begin{equation*}
	d(p_1,p_2) = \bigplus_{i=1}^n d_i(p_1,p_2) = \bigplus_{i=1}^n d_i(p_2,p_1) = d(p_2,p_1).
	\end{equation*}
	
	\item (Desigualdade Triangular) Sejam $p_1,p_2,p_3 \in M$. Então, para todo $i \in \inte_n$, vale $d_i(p_1,p_3) \leq d_i(p_1,p_2)+d_i(p_2,p_3)$ pela desigualdade triangular de $d_i$, e segue que
	\begin{align*}
	d(p_1,p_3) &= \bigplus_{i=1}^n d_i(p_1,p_3) \\
				&\leq \bigplus_{i=1}^n (d_i(p_1,p_2)+d_i(p_2,p_3)) \\
				&= \bigplus_{i=1}^n d_i(p_1,p_2) + \bigplus_{i=1}^n d_i(p_2,p_3) \\
				&= d(p_1,p_2)+d(p_2,p_3).
	\end{align*}
	\end{enumerate}
\end{proof}

\begin{prop}
	Sejam $M$ um conjunto não vazio e $d_1,d_2$ métricas em $M$. Então a função
	\begin{align*}
	d: M \times M &\to \R \\
	 (p_1,p_2) &\mapsto \max\{d_1(p_1,p_2),d_2(p_1,p_2)\}
	\end{align*}
é uma métrica em $M$.
\end{prop}
\begin{proof}
	\begin{enumerate}
	\item (Definição Positiva) Sejam $p_1,p_2 \in M$. Suponhamos que
	\begin{equation*}
	d(p_1,p_2)=\max\{d_1(p_1,p_2),d_2(p_1,p_2)\}=0.
	\end{equation*}
Então $d_1(p_1,p_2)=0$ ou $d_2(p_1,p_2)=0$. Em ambos os casos, temos $p_1=p_2$. Reciprocamente, suponhamos que $p_1=p_2$. Então $d_1(p_1,p_2)=0$ e $d_2(p_1,p_2)=0$, o que implica $d(p_1,p_2)=\max\{d_1(p_1,p_2),d_2(p_1,p_2)\}=0$.
	
	\item (Simetria) Sejam $p_1,p_2 \in M$. Então
	\begin{equation*}
	d(p_1,p_2) = \max\{d_1(p_1,p_2),d_2(p_1,p_2)\} = \max\{d_1(p_2,p_1),d_2(p_2,p_1)\} = d(p_2,p_1).
	\end{equation*}
	
	\item (Desigualdade Triangular) Sejam $p_1,p_2,p_3 \in M$. Então $d(p_1,p_3)=d_1(p_1,p_3)$ ou $d(p_1,p_3)=d_2(p_1,p_3)$. No primeiro caso, segue que
	\begin{equation*}
	d(p_1,p_3) = d_1(p_1,p_3) \leq d_1(p_1,p_2) + d_1(p_2,p_3) \leq d(p_1,p_2) + d(p_2,p_3).
	\end{equation*}
	No segudo caso, segue que
	\begin{equation*}
	d(p_1,p_3) = d_2(p_1,p_3) \leq d_2(p_1,p_2) + d_2(p_2,p_3) \leq d(p_1,p_2) + d(p_2,p_3).
	\end{equation*}
	\end{enumerate}
\end{proof}

\begin{defi}
Um \emph{espaço métrico} é um par $\bm M=(M,d)$ em que $M$ é um conjunto e $d$ é uma métrica em $M$. Os elementos de $M$ são \emph{pontos}. Um \emph{subespaço métrico} de $\bm M$ é o par $\bm S=(S,d|_{S \times S})$.
\end{defi}

\subsection{Diâmetro, Bolas Abertas e Bolas Fechadas}

\begin{defi}
Sejam $\bm M$ um espaço métrico e $C \subseteq M$ um conjunto não vazio. O \emph{diâmetro} de $C$ é o número
	\begin{equation*}
	\diam(C) \coloneqq \sup(d(C \times C)) = \sup\set{d(p_0,p_1)}{p_0,p_1 \in C}
	\end{equation*}
se $d(C \times C)$ é limitado superiormente, e $\infty$, caso contrário. Um \emph{conjunto limitado em $\bm M$} é um conjunto $C \subseteq M$ tal que $\diam(C) \in \R$. Uma \emph{sequência limitada} em $\bm M$ é uma sequência $(p_n)_{n \in N}$ em $M$ tal que $\set{p_n}{n \in \N}$ é limitado.
\end{defi}

\begin{defi}
Sejam $\bm M$ um espaço métrico, $c \in M$ e $r \in \R$. A \emph{bola aberta} de centro $c$ e raio $r$ em $M$ é o conjunto
	\begin{equation*}
	\bola_r(c) \coloneqq \set{p \in M}{d(c,p) < r}.
	\end{equation*}
A \emph{bola fechada} de centro $c$ e raio $r$ em $M$ é o conjunto
	\begin{equation*}
	\bar\bola_r(c) \coloneqq \set{p \in M}{d(c,p) \leq r}.
	\end{equation*}
\end{defi}

\begin{prop}
Sejam $\bm M$ um espaço métrico e $C \subseteq M$ um conjunto. Então $C$ é um conjunto limitado se, e somente se, existem $c \in M$ e $r \in \R$ tais que $C \subseteq \bola_r(c)$.
\end{prop}

\section{Topologia dos Espaços Métricos}

\subsection{Interior e Pontos Interiores}

\begin{defi}
Sejam $\bm M$ um espaço métrico e $C \subseteq M$ um conjunto. Um \emph{ponto interior} de $C$ é um ponto $p \in C$ para o qual existe um número real $r > 0$ tal que $\bola_r(p) \subseteq C$. O \emph{interior} de $C$ é o conjunto $\inn{C}$ de todos pontos interiores de $C$. Um \emph{conjunto aberto} de $\bm M$ é um conjunto $A \subseteq M$ tal que $A = A^\circ$. O conjunto dos conjuntos abertos de $\bm M$ é denotado $\topo_{\bm M}$.
\end{defi}

\begin{prop}
Seja $\bm M = (M,d)$ um espaço métrico. Então
	\begin{enumerate}
	\item Para todo $c \in M$ e para todo número real $r > 0$, a bola aberta $B_r(c)$ é um conjunto aberto;
	\item O conjunto $\topo_{\bm M}$ é uma topologia de $M$.
	\end{enumerate}
\end{prop}
\begin{proof}
	\begin{enumerate}
	\item Sejam $c \in M$ e $r \in \R_+^*$. Queremos mostrar que $\bola_r(c)$ é aberto. Para isso, seja $p \in \bola_r(c)$. Então segue que $d \coloneqq d(c,p) < r$, pela definição de bola aberta, e, portanto, $r-d \in \R_+^*$. Para mostrar que essa bola centrada em $p$ está contida na bola maior centrada em $c$, seja $p' \in \bola_{r-d}(p)$. Então $d(p,p')<r-d$ e, pela desigualdade triangular, segue que
	\begin{equation*}
	d(c,p') \leq d(c,p) + d(p,p') < D + (r-d) = r,
	\end{equation*}
o que mostra que $p' \in \bola_r(c)$ e que, portanto, $\bola_s(p) \subseteq \bola_r(c)$. Assim, mostramos que $\bola_r(p)$ é aberta.
	
	\item
		\begin{enumerate}
		\item Podemos notar que $\emptyset$ é aberto por vacuidade, pois, se não fosse, existiria $p \in \emptyset$ para o qual não há $r \in \R_+^*$ satisfazendo $\bola_r(p) \subseteq A$, o que é absurdo.
	Para mostrar que $M$ é aberto, sejam $p \in M$ e $r \in \R_+^*$. Então $\bola_r(p) \subseteq M$, pois qualquer bola aberta é subconjunto de $M$. Portanto $M$ é aberto.
	
		\item Seja $(A_i)_{i \in I}$ uma família de abertos em $\bm M$ e seja $p \in (A_i)_{i \in I}$. Então existe $k \in I$ tal que $p \in A_k$. Como $A_k$ é aberto, então existe $r \in \R_+^*$ tal que $\bola_r(p) \subseteq A_k$. Como $A_k \subseteq (A_i)_{i \in I}$, segue que $\bola_r(p) \subseteq (A_i)_{i \in I}$ e que, portanto, $(A_i)_{i \in I}$ é aberto.
	
		\item Seja $(A_i)_{i \in \inte_n}$ uma sequência de abertos em $\bm M$ e seja $p \in (A_i)_{i \in \inte_n}$. Então, para todo $k \in \inte_n$, $p \in A_k$. Como, para todo $k \in \inte_n$, $A_k$ é aberto, segue que existe $r_k \in \R_+^*$ tal que $\bola_{r_k}(p) \subseteq A_k$, Seja $r \coloneqq \min \{r_k : k \in \inte_n\}$. Então, para todo $k \in \inte_n$, vale $\bola_r(p) \subseteq \bola_{r_k}(p)$, e segue que $\bola_r(p) \subseteq A_k$ e, portanto, $\bola_r(p) \subseteq (A_i)_{i \in \inte_n}$, o que mostra que $(A_i)_{i \in \inte_n}$ é aberto.
		\end{enumerate}		
	\end{enumerate}
\end{proof}

\subsection{Limites e Convergência de Sequências}

\begin{defi}
Sejam $\bm M$ um espaço métrico, $(p_n)_{n \in \N}$ uma sequência de pontos em $M$ e $p \in M$. A sequência $(p_n)_{n \in \N}$  \emph{converge} para o ponto $p$ se, e somente se, para todo número real $\varepsilon > 0$, existe um número natural $N$ tal que
	\begin{equation*}
	\forall n \in \N \qquad n \geq N \Rightarrow p_n \in \bola_\varepsilon(p).
	\end{equation*}
Denota-se $(p_n)_{n \in \N} \conv p$. O ponto $p$ é um \emph{limite} da sequência.  Caso contrário, a sequência não converge para $p$. Uma \emph{sequência convergente} é uma sequência que tem limite. Uma sequência \emph{divergente} é uma sequência que não tem limite.
\end{defi}

\begin{prop}
Todo espaço métrico $\bm M$ é um espaço topológico separado.
\end{prop}
\begin{proof}
Sejam $p,p' \in M$ pontos distintos. Mostraremos que existe um número real $r$ tal que $0 < r \leq \frac{1}{2} d(p,p')$, e que isso implica que $\bola_r(p) \cap \bola_r(p') = \emptyset$. Como $p \neq p'$, então $d(p,p') > 0$, portanto existe $r \in \R$ tal que $0 < r \leq \frac{1}{2} d(p,p')$. Suponhamos que existe $p'' \in \bola_r(p) \cap \bola_r(p')$. Então $d(p,p'')<r$ e $d(p',p'')<r$. Mas, pela desigualdade triangular, segue que
	\begin{equation*}
	d(p,p') \leq d(p,p'') + d(p'',p') < r + r \leq d(p,p'),
	\end{equation*}
o que é absurdo. Portanto $\bola_r(p) \cap \bola_r(p') = \emptyset$.
\end{proof}

\begin{coro}
Toda sequência convergente em um espaço métrico $\bm M$ tem limite único.
\end{coro}
\begin{proof}
Suponhamos que $p,p'$ são limites de $(p_n)_{n \in \N}$. Se $p \neq p'$, então $d(p,p')>0$. Seja $\varepsilon \in \R$ tal que $0 < \varepsilon \leq \frac{1}{2} d(p,p')$. Então existe $N_1 \in \N$ tal que, para todo $n \in \N$, se $n \geq N_1$, então $p_n \in \bola_\varepsilon(p)$, e existe $N_2 \in \N$ tal que, para todo $n \in \N$, se $n \geq N_2$, então $p_n \in \bola_\varepsilon(p')$. Assim, definindo $N \coloneqq \max \{N_1,N_2\}$, segue que, se $n \geq N$, então $n \geq N_1$ e $n \geq N_2$, e, portanto, que $p_n \in \bola_\varepsilon(p)$ e $p_n \in \bola_\varepsilon(p')$; ou seja, $p_n \in \bola_\varepsilon(p) \cap \bola_\varepsilon(p')$, mas isso é absurdo, pois $\bola_\varepsilon(p) \cap \bola_\varepsilon(p')=\emptyset$. Portanto $p=p'$.
\end{proof}

Essa proposição nos permite tratar o limite de uma sequência como um número único e, por isso, podemos usar a notação $\displaystyle\lim_{n \in \N} p_n = p$ para quando $(p_n)_{n \in \N} \conv p$.

\begin{prop}
Uma sequência de em um espaço métrico $\bm M$ é convergente se, e somente se, todas suas subsequências são convergentes.
\end{prop}
\begin{proof}
	Suponhamos que $(p_n) \conv p$ e seja $(p_{n_k})_{k \in \N}$ uma subsequência de $(p_n)_{n \in \N}$. Seja $\varepsilon \in \R$ tal que $\varepsilon > 0$. Como $(p_n) \conv p$, existe $N \in \N$ tal que, para todo $n \in \N$, se $n \geq N$, então $p_n \in \bola_\varepsilon(p)$; como $(n_k)_{k \in \N}$ é estritamente crescente, existe $K \in \N$ tal que, para todo $k \in \N$, se $k \geq K$, então $n_k \geq N$. Mas então
	\begin{equation*}
	k \geq K \Rightarrow n_k \geq N \Rightarrow p_{n_k} \in \bola_\varepsilon(p)
	\end{equation*}
e, portanto, $(p_{n_k}) \to p$.	Reciprocamente, se toda subsequência de $(p_n)_{n \in \N}$ converge para $p$, $(p_n)_{n \in \N}$, em particular, é uma dessas subsequências e, portanto, $(p_n) \conv p$.
\end{proof}

\begin{prop}
Toda sequência convergente em um espaço métrico $\bm M$ é limitada.
\end{prop}
\begin{proof}
	Seja $(p_n)_{n \in \N}$ uma sequência de pontos em $M$ tal que $(p_n) \conv p$. Então, para $\varepsilon = 1$, existe $N \in \N$ tal que, para todo $n \in \N$, se $n \geq N$, então $p_n \in \bola_1(p)$. Assim, seja $l \in \R$ tal que
	\begin{equation*}
	l > \max(\{1\} \cup \{d(p,p_n) : n \in \inte_N\}),
	\end{equation*}
seque que, para todo $n \in \N$, $p_n \in \bola_l(p)$ pois, se $0 \leq n \leq N$, $d(p,p_n) < l$ pela definição de $l$ e, se $n \geq N$, então $p_n \in \bola_1(p) \subseteq B_l(p)$, pois $1 < l$. Logo $(p_n)_{n \in \N}$ é limitada.
\end{proof}

\begin{prop}
Sejam $\bm M$ um espaço métrico, $C \subseteq M$ um conjunto e $p \in M$. Então existe uma sequência de pontos em $C$ que converge para $p$ se, e somente se, para todo número real $\varepsilon > 0$, $C \cap \bola_\varepsilon(p) \neq \emptyset$.
\end{prop}
\begin{proof}
	Suponhamos que exista uma sequência $(p_n)_{n \in \N}$ de pontos em $C$ tal que $(p_n) \conv p$. Então, para todo número real $\varepsilon > 0$, existe $N \in \N$ tal que, para todo $n \in \N$, se $n \geq N$, então $p_n \in \bola_\varepsilon(p)$. Mas isso implica que $p_n \in C \cap \bola_\varepsilon(p)$. Reciprocamente, suponhamos que, para todo número real $\varepsilon > 0$, $C \cap \bola_\varepsilon(p) \neq \emptyset$. Então, em particular, para todo $n \in \N$, escolhamos $p_n \in C \cap \bola_{\frac{1}{n}}(p)$. Assim, temos a sequência $(p_n)_{n \in \N}$. Para mostrar que $(p_n) \conv p$, seja $\varepsilon \in \R$ tal que $\varepsilon > 0$. Então existe $N \in \N$ tal que $\frac{1}{N} \leq \varepsilon$. Mas isso implica que, para todo número natural $n \geq N$, $\frac{1}{n} \leq \frac{1}{N}$, e segue que
	\begin{equation*}
	d(p,p_n) < \frac{1}{n} \leq \frac{1}{N} \leq \varepsilon
	\end{equation*}
e, portanto, $(p_n) \conv p$.
\end{proof}

\begin{prop}
Sejam $\bm M$ um espaço métrico, $p,q \in M$ e $(p_n)_{n \in \N}$ e $(q_n)_{n \in \N}$ sequências em $M$ que convergem para $p$ e $q$ respectivamente. Então a sequência $(d(p_n,q_n))_{n \in \N}$ em $\R$ converge para $d(p,q)$.
\end{prop}
\begin{proof}
	Para todo $n \in \N$, segue da desigualdade triangular que
	\begin{equation*}
	d(p_n,q_n) \leq d(p_n,p) + d(p,q) + d(q,q_n).
	\end{equation*}
Seja $\varepsilon > 0$ um número real. Então existem $N_1,_2 \in \N$ tais que
	\begin{equation*}
	\forall n \in \N \qquad n \geq N_1 \Rightarrow d(p,p_n) < \frac{\varepsilon}{2}
	\end{equation*}
e
	\begin{equation*}
	\forall n \in \N \qquad n \geq N_2 \Rightarrow d(q,q_n) < \frac{\varepsilon}{2}.
	\end{equation*}
Fazendo $N_3 \coloneqq \max\{N_1,N_2\}$, segue que
	\begin{equation*}
	\forall n \in \N \qquad n \geq N_3 \Rightarrow d(p_n,q_n) \leq d(p_n,p) + d(p,q) + d(q,q_n) < d(p,q) + \varepsilon;
	\end{equation*}
ou seja, $d(p_n,q_n) - d(p,q) < \varepsilon$. Analogamente, achamos $N_6 \in \N$ tal que
	\begin{equation*}
	\forall n \in \N \qquad n \geq N_3 \Rightarrow d(p_n,q_n) \leq d(p_n,p) + d(p,q) + d(q,q_n) < d(p,q) + \varepsilon
	\end{equation*}
e fazendo $n \coloneqq \max\{N_3,N_6\}$, segue que
	\begin{equation*}
	\forall n \in \N \qquad n \geq N \Rightarrow |d(p,q) - d(p_n,q_n)| < \varepsilon,
	\end{equation*}
o que mostra que $(d(p_n,q_n)) \conv d(p,q)$ em $\R$.
	
\end{proof}

\subsection{Fecho e Pontos Aderentes}

\begin{defi}
Sejam $\bm M$ um espaço métrico e $C \subseteq M$ um conjunto. Um \emph{ponto aderente} a $C$ é um ponto $p \in M$ para o qual existe uma sequência $(p_n)_{n \in \N}$ de pontos de $C$ que converge para $p$. O \emph{fecho} de $C$ é o conjunto $\fec{C}$ de todos os pontos aderentes a $C$. Um \emph{conjunto fechado} de $\bm M$ é um conjunto $F \subseteq M$ tal que $F = \fec{F}$.
\end{defi}

\begin{prop}
Sejam $\bm M$ um espaço métrico e $F \subseteq M$. Então $F$ é um conjunto fechado se, e somente se, $F^\complement$ é um conjunto aberto.
\end{prop}
\begin{proof}
Suponhamos que $F$ é um conjunto fechado. Se $F^\complement = \emptyset$, Mas $\emptyset$ é aberto pois, caso contrário, existe $p \in \emptyset$ para o qual não há número real $\varepsilon > 0$ tal que $\bola_\varepsilon(p) \subseteq \emptyset$, mas isso é absurdo. Se $F^\complement \neq \emptyset$, seja $p \in F^\complement$. Se não existe número real $\varepsilon > 0$ tal que $\bola_\varepsilon(p) \subseteq F^\complement$, então, para todo número real $\varepsilon > 0$, $F \cap \bola_\varepsilon(p) \neq \emptyset$. Mas isso implica que existe uma sequência $(p_n)_{n \in \N}$ de pontos em $F$ tal que $(p_n) \conv p$. Como $F$ é fechado, isso implica $p \in F$, o que é uma contradição. Então existe número real $\varepsilon > 0$ tal que $\bola_\varepsilon(p) \subseteq F^\complement$, e isso mostra que $F^\complement$ é aberto.
	
Reciprocamente, suponhamos que $F^\complement$ é aberto. Se $F = \emptyset$, então $F$ é fechado. Se $F \neq \emptyset$, seja $(p_n)_{n \in \N}$ uma sequência em $F$ que converge para $p \in M$. Suponhamos que $p \notin F$. Então $p \in F^\complement$ e, como $F^\complement$ é aberto, existe um número real $\varepsilon > 0$ tal que $\bola_\varepsilon(p) \subseteq F^\complement$. Como $(p_n) \conv p$, existe $N \in \N$ tal que, para todo $n \in \N$, se $n \geq N$, então $p_n \in \bola_\varepsilon(p)$. Mas isso implica que $p_N \in \bola_\varepsilon(p) \subseteq F^\complement$, o que é absurdo, pois $p_n \in F$. Portanto $p \in F$ e isso mostra que $F$ é fechado.
\end{proof}

\begin{prop}
Seja $\bm M$ um espaço métrico. Então, para todo $c \in M$ e para todo número real $r > 0$, a bola fechada $\bar\bola_r(c)$ é um conjunto fechado.
\end{prop}
\begin{proof}
Basta notar que $\bar\bola_r(c)^\complement$ é aberto.
\end{proof}

\subsection{Continuidade}

\begin{defi}
Sejam $\bm M_1$ e $\bm M_2$ espaços métricos, $f: M_1 \to M_2$ uma função e $p \in M$. A função $f$ é \emph{contínua em $p$} se, para todo número real $\varepsilon > 0$, existe um número real $\delta > 0$ tal que
%	\begin{equation*}
%	\forall x \in M_1 \qquad d_1(p,x) < \delta \Rightarrow d_2(f(p),f(x)) < \varepsilon.
%	\end{equation*}
	\begin{equation*}
	\forall x \in M_1 \qquad x \in \bola_\delta(p) \Rightarrow f(x) \in \bola_\varepsilon(f(p)).
	\end{equation*}
Caso contrário, a função $f$ não é contínua em $p$, ela é \emph{descontínua} em $p$.
\end{defi}

Denotamos as bolas abertas em $\bm M_1$ e em $\bm M_2$ por $\bola$, mas deve-se perceber que elas são relativas a métricas possivelmente diferentes. 

\begin{prop}
Sejam $\bm M_1$ e $\bm M_2$ espaços métricos, $f: M_1 \to M_2$ uma função e $p \in M_1$. Então $f$ é contínua em $p$ se, e somente se, para toda sequência $(p_n)_{n \in \N}$ de pontos em $M_1$ que converge para $p$, a sequência $(f(p_n))_{n \in \N}$ de pontos em $M_2$ converge para $f(p)$; ou seja
	\begin{equation*}
	\lim f(p_n) = f(\lim p_n).
	\end{equation*}
\end{prop}
\begin{proof}
	Suponhamos que $f$ é contínua em $p$. Seja $(p_n)_{n \in \N}$ uma sequência de pontos em $M_1$ que converge para $p$. Seja um número real $\varepsilon > 0$. Como $f$ é contínua, existe um número real $\delta > 0$ tal que $p_n \in \bola_\delta(p)$ implica $f(p_n) \in \bola_\varepsilon(f(p))$. Mas, como $(p_n) \conv p$, existe $N \in \N$ tal que
	\begin{equation*}
	\forall n \in \N \qquad n \geq N \Rightarrow p_n \in \bola_\delta(p) \Rightarrow f(p_n) \in \bola_\varepsilon(f(p))
	\end{equation*}
o que mostra que $(f(p_n)) \conv f(p)$.
	
	Reciprocamente, suponhamos que, para toda sequência $(p_n)_{n \in \N}$ em $M_1$ que converge para $p$, a sequência $(f(p_n))_{n \in \N}$ converge para $f(p)$. Suponhamos, por absurdo, que $f$ não é contínua em $p$. Então existe um número real $\varepsilon > 0$ tal que, para todo número real $\delta > 0$, existe $x \in M_1$ tal que $x \in \bola_\delta(p)$, mas $f(x) \notin \bola_\varepsilon(f(p))$. Vamos mostrar que isso implica que existe uma sequência $(p_n)_{n \in \N}$ em $M_1$ que converge para $p$, mas que a sequência $(f(p_n))_{n \in \N}$ não converge para $f(p)$; ou seja, que existe um número real $\varepsilon > 0$ tal que, para todo número natural $N$, existe $n \in \N$ tal que $n \geq N$, mas $f(p_n) \notin \bola_\varepsilon(f(p))$. Seja $n \in \N$ e tomemos $\delta = \frac{1}{n}$. Então existe $x \in M_1$ tal que $x \in \bola_\frac{1}{n}(p)$, mas $f(x) \notin \bola_\varepsilon(f(p))$. Nomeando esse $x \in M_1$ de $p_n$, obtemos uma sequência $(p_n)_{n \in \N}$ que converge para $p$ pois, para todo número real $\varepsilon' > 0$, existe um número natural $N \in \N$ tal que $\frac{1}{N} \leq \varepsilon'$ e isso implica que
\begin{equation*}
	\forall n \in \N \qquad n \geq N \Rightarrow \frac{1}{n} \leq \frac{1}{N} \leq \varepsilon' \Rightarrow p_n \in \bola_\frac{1}{n}(p) \subseteq \bola_\frac{1}{N}(p) \subseteq \bola_{\varepsilon'}(p).
	\end{equation*}
	No entanto, $(f(p_n))_{n \in \N}$ é uma sequência que não converge para $f(p)$ pois, considerando o $\varepsilon$ original tomado da descontinuidade de $f$, para todo número natural $N$, $f(p_N) \notin \bola_\varepsilon(f(p))$ e isso contradiz a hipótese de que, para toda sequência $(p_n)_{n \in \N}$ em $M_1$ que converge para $p$, a sequência $(f(p_n))_{n \in \N}$ converge para $f(p)$. Portanto $f$ é contínua.	
\end{proof}

\begin{defi}
Sejam $\bm M_1$ e $\bm M_2$ espaços métricos, $D \subseteq M_1$ e $f: D \to M_2$ uma função. A função $f$ é \emph{contínua} em $D$ se ela é contínua em todo ponto de $D$. Caso contrário, a função $f$ é \emph{descontínua} em $D$. Para $D=M_1$, dizemos simplesmente que $f$ é contínua ou descontínua.
\end{defi}

\subsection{Outras Definições}

\begin{defi}
Sejam $\bm M$ um espaço métrico e $C \subseteq M$ um conjunto. Um \emph{ponto limite} (ou \emph{ponto de acumulação}) de $C$ é um ponto $p \in M$ para o qual existe uma sequência $(p_n)_{n \in \N}$ de pontos de $C \setminus \{p\}$ que converge para $p$. O \emph{derivado} de $C$ é o conjunto de todos os pontos limites de $C$. 
\end{defi}

Da definição, segue que $C' \subseteq \overline C$. A inclusão contrária caracteriza a seção a seguir.

\begin{defi}
Sejam $\bm M$ um espaço métrico e $C \subseteq M$ um conjunto. Um \emph{ponto isolado} de $C$ é um ponto $p \in M$ que é um ponto aderente a $C$ mas que não é um ponto limite de $C$.
\end{defi}

Um ponto isolado de $C$ é um ponto $p \in \overline C \setminus C'$.

\section{Isometrias}

\begin{defi}
Sejam $\bm{M_1}$ e $\bm{M_2}$ espaços métricos. Uma \emph{isometria} de $\bm{M_1}$ para $\bm{M_2}$ é uma função $f: M_1 \to M_2$ que satisfaz, para todos $p,p' \in M_1$,
	\begin{equation*}
	d_1(p,p') = d_2(f(p),f(p')).
	\end{equation*}
\end{defi}

\begin{prop}
Toda isometria é injetiva.
\end{prop}

\section{Estrutura Uniforme}

\subsection{Sequências Aproximantes}

\begin{defi}
Seja $\bm M$ um espaço métrico. Uma sequência \emph{aproximante} em $\bm M$ é uma sequência $(p_n)_{n \in \N}$ de pontos em $M$ tal que, para todo número real $\varepsilon > 0$, existe um número natural $N$ satisfazendo
	\begin{equation*}
	\forall n,m \in \N \qquad n,m \geq N \Rightarrow d(p_n,p_m) < \varepsilon.
	\end{equation*}
\end{defi}

Essa sequências são conhecidas como \emph{sequências de Cauchy}. O nome aproximante se dá pelo fato de que os termos da sequência ficam cada vez mais próximos entre si, e será adotado por ser mais intuitivo, embora não seja a nomenclatura padrão.

\begin{prop}
Toda sequência convergente em um espaço métrico $\bm M$ é aproximante.
\end{prop}
\begin{proof}
Seja $(p_n)_{n \in \N}$ uma sequência em $M$ que converge para $p$. Seja $\varepsilon \in \R$ tal que $\varepsilon > 0$. Então $\frac{1}{2}\varepsilon > 0$ é um número real e segue que existe $N \in \N$ tal que, para todo número natural $n \geq N$, $p_n \in \bola_{\frac{1}{2}\varepsilon}(p)$. Assim, segue que	
	\begin{equation*}
	\forall n,m \in \N \qquad n,m \geq N \Rightarrow d(p_n,p_m) \leq d(p_n,p) + d(p,p_m) < \frac{\varepsilon}{2} + \frac{\varepsilon}{2} = \varepsilon,
	\end{equation*}
o que mostra que $(p_n)_{n \in \N}$ é uma sequência aproximante.
\end{proof}

\begin{prop}
Toda sequência aproximante em um espaço métrico $\bm M$ que tem uma subsequência convergente é convergente.
\end{prop}
\begin{proof}
	Seja $(p_{n_k})_{k \in \N}$ uma subsequência de $(p_n)_{n \in \N}$ que converge  para $p$. Seja $\varepsilon > 0$ um número real. Como $(p_n)_{n \in \N}$ é uma sequência de Cauchy e $\frac{1}{2}\varepsilon > 0$ é um número real, existe um número natural $N$ tal que
	\begin{equation*}
	\forall n,m \in \N \qquad n,m \geq N \Rightarrow d(p_n,p_m) < \frac{\varepsilon}{2}.
	\end{equation*}
Como $(p_{n_k})_{k \in \N}$ é uma subsequência convergente, existe $K_1 \in \N$ tal que
	\begin{equation*}
	\forall k \in \N \qquad k \geq K_1 \Rightarrow d(p,p_{n_k}) < \frac{\varepsilon}{2}.
	\end{equation*}
Como $(n_k)_{k \in \N}$ é uma sequência estritamente crescente, existe $K_2 \in \N$ tal que, para todo número natural $k \geq K_2$, $n_k \geq N$. Assim, tomando $K \coloneqq \max\{K_1,K_2\}$, segue que, para todo número natural $n \in \N$, existe $k \in \N$ tal que $n_k \geq N$ e,  pela desigualdade triangular, que
	\begin{equation*}
	\forall n \in \N \qquad n \geq N \Rightarrow d(p_n,p) \leq d(p_n,p_{n_k}) + d(p_{n_k},p) < \frac{\varepsilon}{2}+\frac{\varepsilon}{2}=\varepsilon.
	\end{equation*}
\end{proof}

\begin{prop}
Toda sequência aproximante em um espaço métrico $\bm M$ é limitada.
\end{prop}
\begin{proof}
Seja $(p_n)_{n \in \N}$ uma sequência aproximante em $\bm M$. Então, para $\varepsilon=1$, existe $N \in \N$ tal que
	\begin{equation*}
	\forall n,m \in \N \qquad n,m \geq N \Rightarrow d(p_n,p_m)<1.
	\end{equation*}
	Definamos $P \coloneqq \{p_n : n \in \N\}$. Então segue que
	\begin{align*}
	\diam(P) &= \sup \set{d(p_n,p_m)}{n,m \in \N} \\
		&= \max\{1 \cup \set{d(p_n,p_m)}{0 \leq n,m \leq N}\} \in \R,
	\end{align*}
o que mostra que $(p_n)_{n \in \N}$ é limitada.
\end{proof}

\subsection{Continuidade Uniforme}

\begin{defi}
Sejam $\bm M_1$ e $\bm M_2$ espaços métricos. Uma função \emph{uniformemente contínua} é uma função $f: M_1 \to M_2$ tal que, para todo número real $\varepsilon > 0$, existe um número real $\delta > 0$ tal que
	\begin{equation*}
	\forall p_1,p_2 \in M_1 \qquad d_1(p_1,p_2) < \delta \Rightarrow d_2(f(p_1),f(p_2)) < \varepsilon.
	\end{equation*}
\end{defi}

\begin{prop}
Sejam $\bm M_1$ e $\bm M_2$ espaços métricos, $f: M_1 \to M_2$ uma função uniformemente contínua e $(p_n)_{n \in \N}$ uma sequência aproximante em $M_1$. Então a sequência $(f(p_n))_{n \in \N}$ em $M_2$ é aproximante.
\end{prop}
\begin{proof}
Seja $\varepsilon > 0$ um número real. Da continuidade uniforme de $f$, existe um número real $\delta > 0$ tal que, para todo $p,p' \in M_1$, $d_1(p,p') < \delta$ implica $d_2(f(p),f(p')) < \varepsilon$. Como $(p_n)_{n \in \N}$ é sequência aproximante, existe $N \in \N$ tal que
	\begin{equation*}
	\forall n,m \in \N \qquad n,m \geq N \Rightarrow d_1(p_n,p_m) < \delta.
	\end{equation*}
Mas, da continuidade uniforme de $f$, isso implica que
	\begin{equation*}
	\forall n,m \in \N \qquad n,m \geq N \Rightarrow d_1(p_n,p_m) < \delta  \Rightarrow d_2(f(p_n),f(p_m)) < \varepsilon,
	\end{equation*}
e isso mostra que $(f(p_n))_{n \in \N}$ é uma sequênca aproximante.
\end{proof}

\subsection{Espaços Métricos Completos}

\begin{defi}
Um espaço métrico \emph{completo} é um espaço métrico em que todas sequências aproximantes convergem.
\end{defi}

\begin{prop}
Seja $\bm M$ um espaço métrico. Todo subespaço completo de $\bm M$ é um conjunto fechado em $\bm M$.
\end{prop}
\begin{proof}
Sejam $\bm C \subseteq \bm M$ subsepaço métrico completo e $(p_n)_{n \in \N}$ uma sequência convergente em $C$. Então $(p_n)_{n \in \N}$ é aproximante e, como $C$ é completo, converge para um ponto em $C$, o que significa que $C$ é fechado.
\end{proof}

\begin{prop}
Sejam $\bm M$ um espaço métrico, $\bm C \subseteq \bm M$ um subespaço completo e $F \subseteq C$ um conjunto fechado em $\bm M$. Então $\bm F$ é completo.
\end{prop}
\begin{proof}
Seja $(p_n)_{n \in \N}$ uma sequência aproximante em $F$. Então $(p_n)_{n \in \N}$ é uma sequência aproximante em $C$ e, como $C$ é completo, $(p_n)_{n \in \N}$ converge. Porém, como $F$ é fechado, então $(p_n)_{n \in \N}$ converge para um ponto em $F$, o que mostra que $F$ é completo.
\end{proof}

\begin{teo}
Seja $\bm M$ um espaço métrico. Então $\bm M$ é completo se, e somente se, para toda sequência descrescente $(F_n)_{n \in \N}$ de conjuntos não vazios e fechados em $\bm M$ tais que $(\diam(F_n))_{n \in \N} \conv 0$ em $\R$, vale que
	\begin{equation*}
	\bigcap_{n \in \N} A_n \neq \emptyset.
	\end{equation*}
\end{teo}

\begin{teo}
Sejam $\bm M_1$  um espaço métrico, $\bm M_2$ espaço métrico completo, $D \subseteq M_1$ um conjunto denso em $M_1$ e $f: D \to M_2$ uma função uniformemente contínua. Então $f$ tem uma única extensão para uma função uniformemente contínua $f^*: M_1 \to M_2$. Ainda, se $f$ é uma isometria, então $f^*$ é uma isometria.
\end{teo}
\begin{proof}
	Seja $p \in M_1$. Como $D$ é denso em $M_1$, existe uma sequência $(p_n)_{n \in \N}$ em $D$ que converge para $p$. Como $(p_n)_{n \in \N}$ é convergente, é uma sequência de Cauchy e, como $f$ é uniformemente contínua em $D$, segue que $(f(p_n))_{n \in \N}$ é uma sequência de Cauchy em $M_2$. Mas $M_2$ é completo, o que implica que $(f(p_n))_{n \in \N}$ converge para um ponto $p' \in M_2$. Definimos, portanto, a função $f^*$ em $p$ como $f^*(p)=p'$. Precisamos mostrar que $f^*$ independe da escolha da sequência em $D$ que converge para $p$. Se $(q_n)_{n \in \N}$ é uma sequência em $D$ que converge para $p$, definamos a sequência $(r_n)_{n \in \N}$ em $D$ por
	\begin{equation*}
	r_n \coloneqq
			\begin{cases}
			p_n &\text{se $n=2k$}\\
			q_n &\text{se $n=2k+1$}.
			\end{cases}
	\end{equation*}
A sequência $(r_n)_{n \in \N}$ converge para $p$ e, portanto, é uma sequência de Cauchy. A continuidade uniforme de $f$ implica que a sequência $(f(r_n))_{n \in \N}$ é de Cauchy e, portanto, como $(f(p_n))_{n \in \N}=(f(r_{2k}))_{k \in \N}$ é uma subsequência que converge para $p'$, a sequência $(f(r_n))_{n \in \N}$ converge para $p'$, o que implica que a subsequência $(f(q_n))_{n \in \N}=(f(r_{2k+1}))_{k \in \N}$ converge para $p'$. Assim, mostramos que $f^*$ está bem definida. Claramente, se $p \in D$, então $f(p)=f^*(p)$, pois, como $D$ é denso em $M_1$, se $(p_n)_{n \in \N}$ é uma sequência em $D$ que converge para $p$, então, como $f$ é contínua, segue que $f(p_n) \conv f(p)$, o que mostra que $f^*(p)=f(p)$.

	Agora, devemos mostrar que $f^*$ é uniformemente contínua. Seja $\varepsilon > 0$ um número real, então $\frac{1}{2}\varepsilon > 0$ é um número real e, como $f$ é uniformemente contínua, existe número real $\delta > 0$ tal que
	\begin{equation*}
	\forall p,p' \in M_1 \qquad d_1(p,p') < \delta \Rightarrow d_2(f(p),f(p')) < \frac{\varepsilon}{2}.
	\end{equation*}
Assim, sejam $p,q \in M_1$ tais que $d_1(p,q) < \delta$. Quremos mostrar que $d_2(f(p),f(p')) < \varepsilon$. Sejam $(p_n)_{n \in \N}$ e $(q_n)_{n \in \N}$ sequências que convergem para $p$ e $q$, respectivamente. Então $d_1(p_n,q_n) \conv d_1(p,q)$ em $\R$.

...

	A unicidade de $f^*$ ocorre pois, se existem $f^*$ e $f'^*$ uniformemente contínuas que extendem $f$, como $D$ é denso em $M_1$ e $f^*|_D = f'^*|_D$, segue que $f^* = f'^*$ .
	
	Por fim, mostramos que a isometria se preserva...
\end{proof}

\begin{defi}
Seja $\bm M_1$  um espaço métrico. Um \emph{completamento} de $\bm M$ é um espaço métrico $\bm M_2$ completo tal que $M_1$ é denso em $M_2$.
\end{defi}

\begin{prop}
Seja $\bm M$ um espaço métrico e $\bm M_1$ e $\bm M_2$ completamentos de $\bm M$. Então existe uma isometria entre $\bm M_1$ e $\bm M_2$ que é a função identidade quando restrita a $M$.
\end{prop}
\begin{proof}
	Seja $f$ a função identidade em $M$. Pela proposição anterior, existe uma única extensãouniformemente contínua de $f^*$ em $M_1$ ...
	
	...
\end{proof}


\begin{prop}
Sejam $K \subseteq M$ compacto e $f: M \to \bar M$ contínua. Então $f$ é uniformemente contínua.
\end{prop}
\begin{proof}
Suponhamos, por absurdo, que $f$ não é uniformemente contínua. Então existem $\varepsilon > 0$ e $(x_n)_{n \in \N},(y_n)_{n \in \N}$ sequências em $K$ tais que
	\begin{equation*}
	\nor{x_n - y_n} < \frac{1}{n} \e \nor{f(x_n)-f(y_n)} \geq \varepsilon.
	\end{equation*}
Como $K$ é compacto, existem subsequências $(x_{n_k})_{k \in \N}$  e $(y_{n_k})_{k \in \N}$ convergindo a $x \in K$ com $\nor{f(x_{n_k})-f(y_{n_k})} \geq \varepsilon$. Por continuidade de $f$, existe $\delta > 0$ tal que, se $x_{n_k},y_{n_k} \in B(x,\delta)$, então $\nor{f(x_{n_k})-f(x)} < \frac{\varepsilon}{2}$ e $\nor{f(y_{n_k})-f(x)} < \frac{\varepsilon}{2}$. Pela desigualdade triangular, temos um absudo.
\end{proof}



\chapter{Espaços Normados}

\section{Normas}

\begin{defi}
Seja $\bm E$ um espaço vetorial sobre um corpo $\bm C \subseteq \C$. Uma \emph{norma} em $\bm E$ é uma função $\nor{\cdot}: E \to \R$
que satisfaz
	\begin{enumerate}
	\item (Homogeneidade absoluta) Para todos $c \in C$ e $v \in E$,
		\begin{equation*}
		\nor{a v} = \abs{a}\nor{v};
		\end{equation*}
	\item (Subaditividade) Para todos $v_0,v_1 \in E$,
		\begin{equation*}
		\nor{v_0 + v_1} \leq \nor{v_0} + \nor{v_1};
		\end{equation*}
	\item (Definição positiva) Se $\nor{v}=0$, então $v=0$.
	\end{enumerate}
\end{defi}

Claramente $\abs{\cdot}: \C \to \R$ é uma norma em $\C$.

\begin{prop}
Sejam $\bm E$ um espaço vetorial sobre um corpo $\bm C \subseteq \C$ e $\nor{\cdot}$ uma norma em $\bm E$. Então
	\begin{enumerate}
	\item $\nor{0}=0$;
	\item Para todo $v \in E$, $\nor{-v}=-\nor{v}$;
	\item Para todo $v \in E$, $\nor{v} \geq 0$.
	\item (Subaditividade generalizada) Sejam $v_0,\dots,v_{n-1} \in E$. Então
		\begin{equation*}
		\nor{\bigplus_{0\leq i < n} v_i} \leq \bigplus_{0\leq i < n}\nor{v_i}
		\end{equation*}
	\end{enumerate}
\end{prop}

\begin{defi}
Um \emph{espaço normado} é um par $\E = (\bm E,\nor{\cdot})$ em que $\bm E$ é um espaço vetorial sobre um corpo $\bm C \subseteq \C$ e $\nor{\cdot}$ é uma norma em $\bm E$. A dimensão de $\E$ é a dimensão do espaço vetorial $\bm E$.
\end{defi}

\begin{prop}
Seja $\E$ um espaço normado. A função
	\begin{align*}
	\func{d}{E \times E}{\R}{(v_0,v_1)}{\nor{v_0 - v_1}}.
	\end{align*}
é uma métrica em $E$.
\end{prop}
\begin{proof}
(Definição positiva) Sejam $v,\bar v \in E$. Se $v = \bar v$, então segue da positividade que
	\begin{equation*}
	d(v,\bar v) = d(v,v) = \nor{v - v} = \nor{v-v}=\nor{0}=0.
	\end{equation*}
Reciprocamente, se $d(v,\bar v)=0$, então $\nor{v-\bar v}=0$. Segue da definição positiva $v-\bar v=0$, logo $v=\bar v$.
(Simetria)  Sejam $v,\bar v \in E$. Então segue da homogeneidade absoluta que
	\begin{equation*}
	d(v,\bar v)=\nor{v-\bar v}=\nor{-1(\bar v -v)}=\abs{-1}\nor{\bar v - v}=d(\bar v,v).
	\end{equation*}
(Desigualdade triangular) Sejam $v_0,v_1,v_2 \in E$. Então segue da subaditividade que
	\begin{align*}
	d(v_0,v_2) &= \nor{v_0-v_2} \\
		&=\nor{v_0-v_1+v_1-v_2} \\
		&\leq \nor{v_0-v_1}+\nor{v_1-v_2} \\
		&=d(v_0,v_1)+d(v_1,v_2).
	\end{align*}	
\end{proof}

\begin{defi}
Seja $\E$ um espaço normado. A \emph{métrica} em $\E$ é a função
	\begin{align*}
	\func{d}{E \times E}{\R}{(v,\bar v)}{\nor{v - \bar v}}.
	\end{align*}	
A \emph{topologia de $\E$} é a topologia de $(E,d)$.
\end{defi}

\begin{defi}
Seja $\E$ um espaço vetorial normado. O \emph{disco unitário} de $\E$ é o conjunto
	\begin{equation*}
	\disc \coloneqq \set{v \in E}{\nor{v}=1}.
	\end{equation*}
\end{defi}

O disco unitário é a bola fechada, de raio 1 e centro na origem, com respeito à métrica induzida pela norma.

\begin{prop}
Seja $\E$ um espaço normado. O disco unitário de $\disc$ é um conjunto convexo e centro-ssimétrico na origem.
\end{prop}
\begin{proof}
Sejam $t \in ]0,1[$ e $v,\bar v \in \disc$. Então
	\begin{equation*}
	\nor{(1-t)v+t\bar v} \leq (1-t)\nor{v}+t\nor{\bar v} =(1-t)+t=1,
	\end{equation*}
logo $(1-t)v+t\bar v \in \disc$, o que mostra que $\disc$ é convexo. Agora, seja $v \in \disc$. Então $1 \geq \nor{v} = \nor{-v}$, logo $-v \in \disc$, o que mostra a centrossimetria.
\end{proof}

\section{Espaços Normados de Dimensão Finita}

Espaços vetoriais de dimensão finita $\bm E$ sobre um corpo $\bm C$ podem ser identificados com $\bm{C^d}$, que que $d$ é a dimensão de $E$. Nesses casos, a menos que seja mencionado o contrário, sempre consideraremos a base canônica
	\begin{equation*}
	e_i = (0,\dots,0,\underbrace{1}_i,0,\dots,0)
	\end{equation*}
em $\bm{C^d}$ e todo vetor $v \in \bm{C^d}$ será representado como $v=(v_0,\dots,v_{d-1})$.

\begin{defi}
Sejam $\bm E$ um espaço vetorial finito sobre um corpo $\bm C \subseteq \C$ e $p \in [1,\infty[$. A \emph{$p$-norma} em $\bm E$ é a função
	\begin{align*}
	\func{\nor{\cdot}_p}{E}{\R}{v}{\left(\bigplus_{i=0}^{d-1}\abs{v_i}^p\right)^{\frac{1}{p}}}.
	\end{align*}
A \emph{$\infty$-norma} em $\bm E$ é a função
	\begin{align*}
	\func{\nor{\cdot}_\infty}{E}{\R}{v}{\max_{0 \leq i < d} \abs{v_i}}.
	\end{align*}
\end{defi}

Pode-se verificar que $\displaystyle\lim_{p \conv \infty} \nor{v}_p = \nor{v}_\infty$.

\begin{prop}
Para todo $p \in [1,\infty]$, a $p$-norma em $\E$ é uma norma.
\end{prop}

\begin{defi}
\emph{Normas equivalentes} são normas $\nor{\cdot}, |\cdot|: E \to \R$ para as quais existem $c,C \in \R$ tais que $0 < c$ e, para todo $v \in E$,
	\begin{equation*}
	c \nor{v} \leq |v| \leq C \nor{v}.
	\end{equation*}
\end{defi}

\begin{prop}
Equivalência de norma é uma relação de equivalência.
\end{prop}

\begin{prop}
Sejam $\bm E$ um espaço vetorial finito sobre um corpo $\bm C \subseteq \C$. Então toda norma em $\bm E$ são equivalentes.
\end{prop}
\begin{proof}
Vamos mostrar que toda norma em $\bm E$ é equivalente a $\nor{\cdot}_1$ e, como equivalência de normas é uma relação de equivalência, seguirá que todas normas são equivalentes em $\bm E$. Seja $\nor{\cdot}$ uma norma em $\bm E$. Para todo $v \in E$, definindo $C \coloneqq \max_{0\leq i <d} \nor{e_i}$, em que $\{e_i\}_{0 \leq i < d}$ é a base canônica de $\bm E$, segue que
	\begin{equation*}
	\nor{v} = \nor{\bigplus_{i=0}^{d-1} v_ie_i} \leq \bigplus_{i=0}^{d-1}\abs{v_i}\nor{e_i} \leq C\nor{v}_1.
	\end{equation*}
A outra parte da equivalência segue do fato de que toda todo conjunto fechado e limitado em $\R$ é sequencialmente compacto.
\end{proof}

\section{Funções Multilineares}

\begin{prop}
Sejam $\bm{E_0},\dots,\bm{E_{n-1}},\bm{E}$ espaços normados e $T: E_0 \times \cdots \times E_{n-1} \to E$ uma função $n$-linear. São equivalentes
	\begin{enumerate}
	\item $T$ é contínua;
	\item $T$ é contínua em $0$;
	\item Existe real $C>0$ tal que, para todos $v_0 \in \bm{E_0}$, $\ldots$, $v_{n-1} \in \bm{E_{n-1}}$,
		\begin{equation*}
		\nor{f(v_0,\dots,v_{n-1})} \leq C\nor{v_0}\cdots\nor{v_{n-1}};
		\end{equation*}	
	\end{enumerate}
\end{prop}

\begin{prop}
Sejam $\bm{E_1},\dots,\bm{E_n}$ espaços normados de dimensão finita, $\bm{E}$ um espaço normado e $T: E_1 \times \cdots \times E_n \to E$ uma função $n$-linear. Então existe $C \in \R_+$ tal que, para todos $v_1 \in \bm{E_1}$, $\ldots$, $v_n \in \bm{E_n}$,
	\begin{equation*}
	\nor{T(v_1,\ldots,v_n)} \leq C\nor{v_1}\cdots\nor{v_n}.
	\end{equation*}
\end{prop}
\begin{proof}
Para todo $1 \leq i \leq n$, sejam $d_i \coloneqq \dim E_i$ e $\{(b_i)_k\}_{0\leq k<{d_n}}$ uma base de $\bm{E_i}$. Todas normas em $\bm{E_i}$ são equivalentes, portanto usaremos a norma $\nor{\cdot}_\infty$. Assim, para todos $v_1 \in \bm{E_1}$, $\ldots$, $v_n \in \bm{E_n}$,
	\begin{align*}
	\nor{T(v_1,\ldots,v_n)} &= \nor{\bigplus_{\substack{0 \leq k_1 < d_1\\\cdots\\0\leq k_n< d_n}} (v_{k_1})_1\cdots(v_{k_n})_nT\big((b_{k_1})_1,\cdots,(b_{k_n})_n\big)} \\
		&\leq \bigplus_{\substack{0 \leq k_1 < d_1\\\cdots\\0\leq k_n< d_n}} \abs{(v_{k_1})_1}\cdots\abs{(v_{k_n})_n}\nor{T\big((b_{k_1})_1,\cdots,(b_{k_n})_n\big)} \\
		&\leq \nor{v_1}\cdots\nor{v_n}\bigplus_{\substack{0 \leq k_1 < d_1\\\cdots\\0\leq k_n< d_n}} \nor{T\big((b_{k_1})_1,\cdots,(b_{k_n})_n\big)}.
	\end{align*}
Definindo
	\begin{equation*}
	C \coloneqq \bigplus_{\substack{0 \leq k_1 < d_1\\\cdots\\0\leq k_n< d_n}}  \nor{T\big((b_{k_1})_1,\cdots,(b_{k_n})_n\big)},
	\end{equation*}
segue que
	\begin{equation*}
	\nor{T(v_1,\ldots,v_n)} \leq C \nor{v_1}\cdots\nor{v_n}.
	\end{equation*}
\end{proof}



























\section{Espaço Real}

\subsection{Notação de Vetores e Funções}

\section{Compactos}

\begin{teo}[Heine-Borel]
	Seja $C \subseteq \R^n$ um conjunto. Então são equivalentes:
	\begin{enumerate}
	\item $C$ é compacto.
	\item $C$ é fechado e limitado.
	\item Para toda sequência $(x_n)_{n \in \N} \subseteq C$, existe subsequência $(x_{n_k})_{k \in \N}$ que converge a $x \in C$.
	\end{enumerate}
\end{teo}

\begin{prop}
Um conjunto $C \subseteq \R^n$ é compacto se, e somente se, para toda função $f: C \to \R^m$ contínua, $f(C)$ é limitado.
\end{prop}
\begin{proof}

\end{proof}

\paragraph*{Exercício} Vale o mesmo se trocarmos limitado por fechado na proposição anterior?

\begin{prop}
	Sejam $C \subseteq \R^n$ compacto, $X \subseteq \R^m$ e $f: X \times C \to \R^l$ contínua. Seja $x_0 \in X$. Então, para todo $\varepsilon > 0$, existe $\delta > 0$ tal que, se $x \in X$ e $|x-x_0| < \delta$, então, para todo $\alpha \in C$, $|f(x,\alpha)-f(x_0,\alpha)|<\varepsilon$.
\end{prop}
\begin{proof}
	Suponha que existe $\varepsilon>0$ e $x_k \to x_0$ e $\alpha_k \in C$ tal que $|f(x_k,\alpha_k)-f(x_0, \alpha_k)| \geq \varepsilon$, $\alpha_k$ tem subcobertura $\alpha_{k_n} \to \alpha_0$
\end{proof}











\chapter{Diferenciabilidade no Espaço Real}

O espaço real $\E$ estudado neste capítulo será o espaço vetorial normado $(\R^d,+,\cdot)$ sobre $\R$. A base canônica de $\R^d$ será representada pelos vetores $\{\bm e_0, \ldots, \bm e_{d-1}\}$. Um vetor $x \in \R^d$ será também representado por $x=(x_0,\ldots,x_{d-1})$ e uma função $f: \R^{d_0} \to \R^{d_1}$ será também representada por $f=(f_0,\ldots,f_{d_1-1})$, de modo que $f_i \coloneqq \pi_i \circ f$, sendo $\pi_i$ a $i$-ésima projeção de $\R^{d_1}$ em $\R$. Como todas normas em $\R^d$ são equivalentes, não será feita referência à norma utilizada, apenas será usado o fato de que $\R^d$ é um espaço vetorial normado (e completo). Se necessário, a norma utilizada será explicitada e, quando não for, a norma usada será $\nor{\cdot}_2$. O estudo da diferenciabilidade em espaços de dimensão maior que 1 envolve o uso de funções contínuas e transformações lineares, e também de funções de um espaço real em um espaço de tranformações lineares. Por esse motivo, a notação pode ser confusa. Para simplificar a notação, uma transformação linear $T$ aplicada a um vetor $v$ será sempre denotada por $T \cdot v$. Desenvolveremos, a seguir, a teoria de diferenciabilidade de funções entre espaços reais, e as funções consideradas serão sempre da forma
	\begin{equation*}
	f: \R^d \to \R^c,
	\end{equation*}
mas toda teoria poderia ser desenvolvida para funções definidas em abertos de $\R^d$. O tratamento que adotaremos, no entanto, não prejudica a generalidade, pois todas propriedades desenvolvidas podem ser compreendidas localmente.

\section{Diferenciabilidade}

A ideia por trás dessa definição de diferenciabilidade é a de que a função $f$ pode ser aproximada em uma vizinhança de um ponto $p$ por seu valor no ponto mais o valor de uma transfomação linear aplicada num vetor $v$ de variação que mede quanto afastou-se do ponto $p$. Ser aproximada, nesse sentido, quer dizer que o erro da aproximação será da ordem da norma do vetor variação $v$, de modo que a razão entre os dois vá a zero quando a variação vai a zero. A definição de função contínua, de fato, pode ser pensada como um caso análogo: a função $f$ numa vizinhança do ponto $p$ pode ser aproximada por seu valor em $p$, e aproximada aqui quer dizer que a norma da diferença vai a zero quando o vator variação vai a zero. Mais à frente, as $k$-ésimas diferenciais da função $f$ serão definidas analogamente, considerando nesses casos funções multilineares.

\begin{defi}
Sejam $p \in \R^d$ e $f: \R^d \to \R^c$ uma função. Uma \emph{diferencial} de $f$ em $p$ é uma transformação linear $T : \R^d \to \R^c$ que satisfaz
	\begin{equation*}
	\lim_{v \conv 0} \frac{f(p+v)-f(p)-T \cdot v}{\nor{v}} = 0.
	\end{equation*}
Uma função \emph{diferenciável} em $p$ é uma função que tem diferencial em $p$.
\end{defi}

É válido notar que são equivalentes a essa condição
	\begin{equation*}
	\lim_{x \conv p} \frac{f(x)-f(p)-T\cdot(x-p)}{\nor{x-p}} = 0.
	\end{equation*}
e
	\begin{equation*}
	\lim_{v \conv 0} \frac{\nor{f(p+v)-f(p)-T \cdot v}}{\nor{v}} = 0.
	\end{equation*}

\begin{prop}[Diferenciabilidade implica continuidade]
Sejam $p \in \R^d$ e $f: \R^d \to \R^c$ uma função. Se $f$ é diferenciável em $p$, então $f$ é contínua em $p$.
\end{prop}
\begin{proof}
Se $f$ é diferenciável em $p$, então, como $\lim_{v \conv 0} T \cdot v = 0$,
	\begin{align*}
	\lim_{v \conv 0} (f(p+v)-f(p)) &= \lim_{v \conv 0}(f(p+v)-f(p)-T \cdot v) \\
		&= \lim_{v \conv 0}\nor{v} \frac{f(p+v)-f(p)-T \cdot v}{\nor{v}} \\
		&= 0,
	\end{align*}
logo $f$ é contínua em $p$.
\end{proof}

\begin{prop}[Unicidade da diferencial]
Sejam $p \in \R^d$ e $f:\R^d \to \R^c$ uma função diferenciável em $p$. Então existe uma única diferencial de $f$ em $p$.
\end{prop}
\begin{proof}
Sejam $T,S: \R^d \to \R^c$ diferenciais de $f$ em $p$. Nesse caso, temos que
	\begin{align*}
	\lim_{v \conv 0} &\frac{T \cdot v - S \cdot v}{\nor{v}} = \\
	&= \lim_{v \conv 0} \frac{T \cdot v - (f(p+v)-f(p)) + (f(p+v)-f(p)) - S \cdot v}{\nor{v}} \\
	& =  -\lim_{v \conv 0} \frac{f(p+v)-f(p) - T \cdot v}{\nor{v}} + \lim_{v \conv 0} \frac{f(p+v)-f(p) - S \cdot v}{\nor{v}} \\
	&=0.
	\end{align*}
Como $T$ e $S$ são transformações lineares, sabemos que $T \cdot 0 = S \cdot 0=0$. Para todo $v \in \R^d \setminus \{0\}$, temos que, quando $t \conv 0$, $tv \conv 0$. Ainda, como $T$ e $S$ são transformações lineares, $T \cdot (tv) = t(T \cdot v)$ e $S \cdot (tv) = t (S \cdot v)$, e segue que
	\begin{align*}
	0 &= \lim_{tv \conv 0} \frac{\nor{T \cdot (tv) - S \cdot (tv)}}{\nor{tv}} \\
		&= \lim_{t \conv 0} \frac{\abs{t}\nor{T \cdot v - S \cdot v}}{\abs{t} \nor{v}} \\
		&= \frac{\nor{T \cdot v - S \cdot v}}{\nor{v}},
	\end{align*}
o que implica $T \cdot v = S \cdot v$, pois $\nor{v} \neq 0$. Portanto $T=S$.
\end{proof}

\begin{nota}
Sejam $p \in \R^d$ e $f: \R^d \to \R^c$ uma função diferenciável em $p$. A diferencial de $f$ em $p$ é denotada $\D f(p): \R^d \to \R^c$ e satisfaz
	\begin{equation*}
	\lim_{v \conv 0} \frac{f(p+v) - f(p) - \D f(p) \cdot v}{\nor{v}} = 0.
	\end{equation*}
\end{nota}

Podemos ver que, se $f$ é diferenciável, então $\D f: \R^d \to L(\R^d,\R^c)$ é uma função que leva $p \in \R^d$ na diferencial $\D f(p)$ de $f$ em $p$.

\begin{prop}[Regra da cadeia]
Sejam $f: \R^n \to \R^m$ diferenciável em $p \in \R^n$ e $g: \R^m \to \R^l$ diferenciável em $f(p)$. Então $g \circ f: \R^n \to \R^l$ é diferenciável em $p$ e
	\begin{equation*}
	\D (g \circ f)(p) = \D g(f(p)) \circ \D f(p).
	\end{equation*}
\end{prop}
\begin{proof} Definamos
	\begin{equation*}
	r_1(v) \coloneqq f(p+v) - f(p) - \D f(p) \cdot v
	\end{equation*}
e
	\begin{equation*}
	r_2(v) \coloneqq g(f(p)+v) - g(f(p)) - \D g(f(p)) \cdot v,
	\end{equation*}
de modo que da diferenciabilidade de $f$ em $p$ e de $g$ em $f(p)$ segue
	\begin{equation*}
	\lim_{v \conv 0} \frac{r_1(v)}{\nor{v}} = \lim_{v \conv 0} \frac{r_2(v)}{\nor{v}} = 0.
	\end{equation*}	
Calculando $(g \circ f)(p+v)$, obtemos
	\begin{align*}
	(g \circ f)(p+v) &= g(f(p+v)) = g(f(p)+\D f(p) \cdot v + r_1(v)) \\
		&= g(f(p)) + \D g(f(p)) \cdot (\D f(p) \cdot v + r_1(v)) \\
		&\qquad + r_2(\D f(p) \cdot v + r_1(v)) \\
		&= (g \circ f)(p) + \big(\D g(f(p)) \circ \D f(p)\big) \cdot v + \D g(f(p)) \cdot r_1(v) \\
		&\qquad + r_2(\D f(p) \cdot v + r_1(v)).
	\end{align*}
Portanto
	\begin{align*}
	(g \circ f)(p+v) - (g \circ f)(p) - \big(\D g(f(p)) \circ \D f(p)\big) \cdot v \\
	\qquad = \D g(f(p)) \cdot r_1(v) + r_2(\D f(p) \cdot v + r_1(v)).
	\end{align*}
Como $\D g(f(p)) \circ \D f(p)$ é uma transformação linear de $\R^n$ para $\R^l$, basta mostrar que a expressão acima, dividida por $\nor{v}$, vai a zero. Mas
	\begin{equation*}
	\lim_{v \conv 0} \frac{\D g(f(p)) \cdot r_1(v)}{\nor{v}} = \lim_{v \conv 0} \D g(f(p)) \cdot \frac{r_1(v)}{\nor{v}} = 0
	\end{equation*}
e, como $\lim_{v \conv 0} \D f(p) \cdot v + r_1(v) = 0$ e $\D f(p) \cdot \frac{v}{\nor{v}}$ é limitado,
	\begin{align*}
	\lim_{v \conv 0} &\frac{r_2(\D f(p) \cdot v + r_1(v))}{\nor{v}} \\
		&= \lim_{v \conv 0} \frac{r_2(\D f(p) \cdot v + r_1(v))}{\nor{\D f(p) \cdot v + r_1(v)}} \frac{\nor{\D f(p) \cdot v + r_1(v)}}{\nor{v}}\\
		&= \lim_{v \conv 0} \frac{r_2(\D f(p) \cdot v + r_1(v))}{\nor{\D f(p) \cdot v + r_1(v)}} \nor{\D f(p) \cdot \frac{v}{\nor{v}}+ \frac{r_1(v)}{\nor{v}}} = 0.
	\end{align*}
Logo
	\begin{equation*}
	\lim_{v \conv 0} \frac{(g \circ f)(p+v) - (g \circ f)(p) - \big(\D g(f(p)) \circ \D f(p)\big) \cdot v}{\nor{v}} = 0,
	\end{equation*}
e concluímos que $\D g(f(p)) \circ \D f(p)$ é a diferencial de $g \circ f$ em $p$.
\end{proof}


\begin{prop}[Regra da cadeia iterada]
Sejam $f: \R^n \to \R^m$ $2$-diferenciável em $p \in \R^n$ e $g: \R^m \to \R^l$ diferenciável em $f(p)$. Então $g \circ f: \R^n \to \R^l$ é diferenciável em $p$ e
	\begin{equation*}
	\D^2 (g \circ f)(p,p) = \D^2 g(f(p),f(p)) \circ \D f(p) + \D g(f(p)) \circ \D^2 f(p)
	\end{equation*}
\end{prop}




\begin{prop}
	Sejam $f,g: \R^n \to \R^m$ diferenciáveis em $p \in \R^n$. Então
	\begin{enumerate}
	\item $D (f+g)(p) = D f(p) + D g(p);$
	\item $D (f \cdot g) = D f(p) \cdot g(p) + f(p) \cdot D g(p);$
	\item Se $g(a) \neq 0$,
	\begin{equation*}
	D \left(\frac{f}{g}\right) (p) = \frac{g(p) \cdot D f(a) - D g (a) \cdot f(p)}{g(p)^2}
	\end{equation*}
	\end{enumerate}
\end{prop}



\subsection{Diferenciais de Ordem Superior}

Generalizamos, agora, a ideia de uma diferencial para uma $k$-diferencial. Para isso, denotaremos um vetor $(v,\ldots,v)$ com $i$ entradas por $(v)^{\otimes i}$

\begin{defi}
Seja $p \in \R^d$. Uma função \emph{$k$-diferenciável} em $p$ é uma função $f: \R^d \to \R^c$ tal que, para todo $i \leq k$, existe uma função $i$-linear simétrica
	\begin{equation*}
	T_i: \R^d \times \cdots \times \R^d \to \R^c
	\end{equation*}
satisfazendo
	\begin{equation*}
	\lim_{v \conv 0} \frac{\displaystyle f(p+v)-f(p)- \bigplus_{i=1}^k \frac{1}{i!}T_i \cdot (v)^{\otimes i}}{\nor{v}^k} = 0.
	\end{equation*}
Uma função $T_i$ como acima é uma \emph{diferencial de ordem $i$} (ou \emph{$i$-ésima diferencial}) da $f$ em $p$.
\end{defi}

\begin{prop}
Sejam $p \in \R^d$ e $f: \R^d \to \R^c$ uma função $k$-diferenciável em $p$. Então $f$ é $(k-1)$-diferenciável em $p$.
\end{prop}
\begin{proof}
Primeiro notemos que
	\begin{equation*}
	\lim_{v \conv 0} \frac{T_k \cdot (v)^{\otimes k}}{\nor{v}^{k-1}} \leq \lim_{v \conv 0} \frac{\nor{T_k} \nor{v}^k}{\nor{v}^{k-1}} = \lim_{v \conv 0} \nor{T_k}\nor{v} = 0.
	\end{equation*}
Portanto segue que	
	\begin{align*}
	\lim_{v \conv 0} &\frac{\displaystyle f(p+v)-f(p)- \bigplus_{i=1}^{k-1} \frac{1}{i!}T_i \cdot (v)^{\otimes i}}{\nor{v}^{k-1}} \\
		&= \lim_{v \conv 0} \frac{\displaystyle f(p+v)-f(p)- \bigplus_{i=1}^{k-1} \frac{1}{i!}T_i \cdot (v)^{\otimes i}-\frac{1}{k!}T_k \cdot (v)^{\otimes k}}{\nor{v}^{k-1}}\\
		&= \lim_{v \conv 0} \nor{v}\frac{\displaystyle f(p+v)-f(p)- \bigplus_{i=1}^k \frac{1}{i!}T_i \cdot (v)^{\otimes i}}{\nor{v}^k} \\
		&= 0.
	\end{align*}
\end{proof}

\begin{prop}
Sejam $p \in \R^d$ e $f: \R^d \to \R^c$ uma função $k$-diferenciável em $p$. Então as $i$-ésimas diferenciais de $f$ em $p$ são únicas.
\end{prop}
\begin{proof}
Mostraremos por indução em $k$. Para $k=1$, temos a definição de função diferenciável, portanto a diferencial de $f$ em $p$ é única. Para o passo indutivo, suponhamos que toda função $(k-1)$-diferenciável tem únicas $i$--ésimas diferenciais para $0\leq i \leq k-1$. Consideremos uma função $f: \R^d \to \R^c$ $k$-diferenciável em $p$. Então ela é $(k-1)$-diferenciável em $p$ pela proposição anterior e segue que, para todo $i \leq k$, existe uma única função $i$-linear simétrica
	\begin{equation*}
	T_i: \R^d \times \cdots \times \R^d \to \R^c
	\end{equation*}
satisfazendo
	\begin{equation*}
	\lim_{v \conv 0} \frac{\displaystyle f(p+v)-f(p)- \bigplus_{i=1}^k \frac{1}{i!}T_i \cdot (v)^{\otimes i}}{\nor{v}^k} = 0.
	\end{equation*}
Agora, sejam $T,S$ diferenciais de ordem $k$ de $f$ em $p$ e definamos
	\begin{equation*}
	A(v) \coloneqq f(p+v)-f(p)- \bigplus_{i=1}^{k-1} \frac{1}{i!}T_i \cdot (v)^{\otimes i}.
	\end{equation*}
Segue que
	\begin{align*}
	&\lim_{v \conv 0} \frac{T \cdot (v)^{\otimes k} - S \cdot (v)^{\otimes k}}{\nor{v}^k} \\
	&= \lim_{v \conv 0} \frac{\frac{1}{k!} T \cdot (v)^{\otimes k} - A(v) + A(v) - \frac{1}{k!} S \cdot (v)^{\otimes k}}{\nor{v}^k} \\
	&= -\lim_{v \conv 0} \frac{\frac{1}{k!} T \cdot (v)^{\otimes k} - A(v)}{\nor{v}^{\otimes k}} + \lim_{v \conv 0} \frac{A(v) - \frac{1}{k!} S \cdot (v)^{\otimes k}}{\nor{v}^k} \\
	&= 0.
	\end{align*}
Como $T$ e $S$ são transformações $k$-lineares, sabemos que $T \cdot (0)^{\otimes k} = S \cdot (0)^{\otimes k} =0$. Para $v \in (\R^d)^k \setminus\{(0)^{\otimes k}\}$, temos que, quando $t \conv 0$, $(tv)^{\otimes k} \conv (0)^{\otimes k}$. Ainda, como $T$ e $S$ são $k$-lineares, $T \cdot (tv)^{\otimes k} = t^kT \cdot (v)^{\otimes k}$ e $S \cdot (tv)^{\otimes k} = t^kS \cdot (v)^{\otimes k}$, e segue que
	\begin{align*}
	0= &\lim_{tv \conv 0} \frac{T \cdot (tv)^{\otimes k} -  S \cdot (tv)^{\otimes k}}{\nor{tv}^k} \\
	&= \lim_{t \conv 0} \frac{(t)^k\big(T \cdot (v)^{\otimes k} -  S \cdot (v)^{\otimes k}\big)}{\abs{t}^k\nor{v}^k} \\
	&= \pm \frac{\big(T \cdot (v)^{\otimes k} -  S \cdot (v)^{\otimes k}\big)}{\nor{v}^k}
	\end{align*}
o que implica $T \cdot (v)^{\otimes k} =  S \cdot (v)^{\otimes k}$, pois $\nor{v} \neq 0$. Por fim, essa relação e a simetria de $T$ e $S$ implicam que elas são iguais em todos os pontos, portanto $T=S$.
\end{proof}

\begin{nota}
Sejam $p \in \R^d$ e $f: \R^d \to \R^c$ uma função $k$-diferenciável em $p$. A diferencial de ordem $k$ de $f$ em $p$ é denotada $\D^k f(p): \R^d \times \cdots \times \R^d \to \R^c$ e satisfaz
	\begin{equation*}
	\lim_{v \conv 0} \frac{\displaystyle f(p+v) - f(p) - \bigplus_{i=1}^k \frac{1}{i!}\D^i f(p) \cdot (v)^{\otimes i}}{\nor{v}^k} = 0.
	\end{equation*}
O polinômio
	\begin{equation*}
	P(v) = f(p) + \bigplus_{i=1}^k \frac{1}{i!}\D^i f(p) \cdot (v)^{\otimes i}
	\end{equation*}
é o \emph{polinômio diferencial de ordem $k$} de $f$ em $p$.
\end{nota}

\section{Derivadas Direcionais e a Geometria da Diferenciabilidade}

A partir dessa seção, cosideraremos funções $f: A \to \R^c$, em que $A \subseteq \R^d$ é um aberto. Toda a discussão feita na seção anterior considerou a diferenciabilidade em pontos do domínio. Agora, consideraremos a diferenciabilidade em conjuntos. A definição de diferenciabilidade da seção anterior pode ser facilmente adaptada parra funções $f: A \to \R^c$ pois essa função pode ser definida em $\R^d$ todo escolhendo qualquer valor para $f$ em $A^\complement$. Como as definições e resultados trataram de pontos, isso não atrapalha em nada. Os abertos serão necessários agora pois consideraremos curvas numa vizinhança de um ponto e relacionaremos as derivadas por essas curvas com derivadas parciais da função $f$.

\begin{defi}
Sejam $A \subseteq \R^n$ um aberto, $p \in A$, $v \in \R^n$ tal que $p+v \in A$ e $f: A \to \R^m$. A \emph{derivada direcional} de $f$ em $p$ na direção de $v$ é
	\begin{equation*}
	\partial_v f(p) \coloneqq \lim_{t \conv 0} \frac{f(p+tv)-f(p)}{t}.
	\end{equation*}
\end{defi}

Como $A$ é aberto, existe $\varepsilon$ tal que $p+tv \in A$ para todo $t \in ]{-\varepsilon},\varepsilon[$. Tomemos então a curva
	\begin{align*}
	\func{\gamma}{]{-\varepsilon},\varepsilon[}{A}{t}{p+tv},
	\end{align*}
de modo que temos $\gamma(0)=p$ e $\gamma'(0) = v$. Então, pela regra da cadeia,
	\begin{equation*}
	 \partial_v f(p) = \ D (f \circ \gamma)(0) = \D f(\gamma(0)) \cdot \gamma'(0) = \D f(p) \cdot v.
	\end{equation*}
Disso concluímos que a derivada direcional de $f$ em $p$ na direção de $v$ é a imagem de $v$ sob a transformação linear $\D f(p)$. Portanto definindo as derivadas direcionais $\partial_i f(x) \coloneqq \D f(x) \cdot e_i$ temos que
	\begin{equation*}
	\D f(x) \cdot v = \bigplus_{i=0}^{d-1} v_i \partial_i f(x).
	\end{equation*}















\chapter{Equações Diferenciais Ordinárias}

\section{Equações Diferenciais Ordinárias e Soluções}

\begin{defi}
	Sejam $n,d \in \N^*$ Uma \emph{equação diferencial ordinária de ordem $n$ e dimensão $d$} é uma expressão $E$ da forma
	\begin{equation*}
	x^{(n)} = F(t, x, x^{(1)}, \cdots,x^{(n-1)}),
	\end{equation*}
em que $t \in \R$ é a variável \emph{tempo}, $x : \R \to \R^d$ é uma função, $A \subseteq \R \times (\R^d)^n$ é um aberto e $F: A \to \R^d$ é uma função contínua.
\end{defi}

\begin{defi}
	Seja $E: x^{(n)} = F(t, x, x^{(1)}, \cdots,x^{(n-1)})$ uma equação diferencial ordinária de dimensão $d$. Uma \emph{solução de $E$} é uma curva $\varphi: I \subseteq \R \to \R^d$ $n$-diferenciável tal que
	\begin{equation*}
	\varphi^{(n)} = F(t, \varphi(t), \varphi^{(1)}(t), \cdots,\varphi^{(n-1)}(t)).
	\end{equation*}
\end{defi}

\begin{prop}[Redução de ordem]
	Seja $E: x^{(n)}(t) = F(t, x, x^{(1)}, \cdots,x^{(n-1)})$ uma equação diferencial ordinária de dimensão $d$ definida num aberto $A \subseteq \R \times (\R^d)^n$. Então existe equação diferencial ordinária $\bm E$ de dimensão $dn$
	\begin{equation*}
	 \bm x' = \bm F(t,\bm x)
	\end{equation*}	
tal que existe $\phi$ solução de $E$ se, e somente se, existe $\bm \phi$ solução de $\bm E$.
\end{prop}
\begin{proof}
	Definimos $\bm x = (x_0, \ldots,x_{n-1}) \coloneqq (x,x^{(1)}, \cdots,x^{(n-1)})$, de modo que
	\begin{align*}
	\bm x' &= ((x_0)',\ldots,(x_{n-1})',(x_{n-1})') \\
			&= (x^{(1)}, \ldots,x^{(n-1)},x^{(n)}) \\
			&= (x_1,\ldots,x_{n-1},F(t, x, x^{(1)}, \cdots,x^{(n-1)})) \\
			&= (x_1,\ldots,x_{n-1},F(t,\bm x))
	\end{align*}	

	Assim, definimos a função
	\begin{align*}
	\func{\bm F}{A}{\R^{dn}}{(t,\bm x)}{(x_1,\ldots,x_{n-1},F(t, \bm x))}
	\end{align*}
e então $\bm x' = \bm F(t,\bm x)$. Por fim, devemos mostrar que as soluções são equivalentes. Seja $\phi: I \subseteq \R \to \R^d$ solução de $E$. Então, para todo $k \in \{1,\ldots,n\}$, existe $\phi^{(k)}: \R \to \R^d$. Assim, definindo $\bm \phi \coloneqq (\phi,\phi^{(1)},\ldots,\phi^{(n-1)})$, segue que $\bm \phi$ é solução de $\bm E$.
\end{proof}

Por causa dessa proposição, podemos representar qualquer equação diferencial ordinária como uma de ordem 1.

\section{Existência e Unicidade de Soluções}



\section{Soluções Maximais}

\begin{defi}
	Seja $E: x'=F(t,x)$ uma equação diferencial ordinária de dimensão $d$. Uma \emph{solução maximal de $E$} é uma solução $\phi: I \to \R^d$ de $E$ para a qual vale que, para toda solução $\psi: J \to \R^d$ de $E$ tal que $I \subseteq J$ e $\psi|_I = \phi$, então $I = J$ e $\phi = \psi$. O intervalo $I$ é o \emph{intervalo maximal}.
\end{defi}

\begin{defi}
	Dado $(t_0,x_0) \in U$, defina
	\begin{equation*}
	S_{(t_0,x_0)} \coloneqq \set{\phi: I_\phi \to \R^d}{\phi \text{\ \ é solução do problema de Cauchy}}.
	\end{equation*}
Dados $\phi ,\psi \in S_{(t_0,x_0)}$, definimos $\phi \leq \psi $ se, e somente se, $I_{\phi } \subseteq I_{\psi}$ e $\psi|_{I_\phi} = \phi $.
\end{defi}

\begin{prop}
	A relação acima de fato é relação de ordem parcial.
\end{prop}

\begin{teo}
	Seja $F: U \to \R^d$ uma função contínua definida num aberto $U \subseteq \R \times \R^d$. Então, para cada $(t_0,x_0) \in U$ existe solução maximal para o problema de Cauchy
	\begin{equation*}
	\begin{cases}
		x' = F(t,x) \\
		x(t_0)=x_0
	\end{cases}
	\end{equation*}
\end{teo}









































\chapter{Teoria da Medida}

\section{Espaço Mensurável}

\subsection{Sigma-Álgebras e Sub-Sigma-Álgebras}

\begin{defi}
	Seja $X$ um conjunto. Uma \emph{sigma-álgebra} sobre $X$ é um conjunto $\Sigma \subseteq \p(X)$ de subconjuntos $X$ que satisfaz
	\begin{enumerate}
	\item (Vazio) $\emptyset \in \Sigma$.
	\item (Fechamento por complementação) Para todo $M \in \Sigma$, $M\complement \in \Sigma$.
	\item (Fehamento por união enumerável) Para toda sequência $(M_i)_{i \in \N}$ de conjuntos de $\Sigma$,
	\begin{equation*}
	\bigcup_{i \in \N} M_i \in \Sigma
	\end{equation*}
	\end{enumerate}
\end{defi}

	Vale notar que uma sigma-álgebra $\Sigma$ é uma álgebra booleana (\ref{prop:algeb.subconj}) e, portanto, todas propriedades de álgebras booleanas valem para uma sigma-álgebra. De fato, o \textit{sigma} no nome vem da terceira propriedade das sigma-álgebras, pois veremos que essa propriedade tem a ver com um tipo de soma de medidas a ser definido adiante.

\begin{prop}
	Seja $X$ um conjunto não vazio e $\Sigma$ uma sigma-álgebra sobre $X$. Então
	\begin{enumerate}
	\item (Universo) $X \in \Sigma$;
	\item (Fechamento por interseção enumerável) Para toda sequência $(M_i)_{i \in \N}$ de conjuntos de $\Sigma$,
	\begin{equation*}
	\bigcap_{i \in \N} M_i \in \Sigma.
	\end{equation*}
	\end{enumerate}
\end{prop}
\begin{proof}
	\begin{enumerate}
	\item Da primeira propriedade de $\Sigma$, tem-se que $\emptyset \in \Sigma$. Da segunda propriedade de $\Sigma$, tem-se que $X = \emptyset^\complement \in \Sigma$.
	\item Da segunda propriedade, tem-se que, para todo $i \in \N$, $M_i^\complement \in \Sigma$. Da terceira propriedade de $\Sigma$, tem-se que $\bigcup_{i \in \N} M_i^\complement \in \Sigma$. Das Leis de De Morgan (\ref{prop:de.morgan}), tem-se que
	\begin{equation*}
	\left(\bigcap_{i \in \N} M_i \right)^\complement = \bigcup_{i \in \N} (M_i)^\complement \in \Sigma,
	\end{equation*}
e conclui-se que $\displaystyle\bigcap_{i \in \N} M_i \in \Sigma$.
	\end{enumerate}
\end{proof}

\begin{ex}
	$\Sigma = \{\emptyset,X\}$ e $\Sigma = \p(X)$ são sigma-álgebras sobre $X$.
\end{ex}

\begin{defi}
Seja $X$ um conjunto não vazio e $\Sigma$ uma sigma-álgebra sobre $X$. Uma \emph{sub-sigma-álgebra} de $\Sigma$ é um conjunto $\Sigma' \subseteq \Sigma$ que é uma sigma-álgebra sobre $X$.
\end{defi}

\begin{defi}
	Um \emph{espaço mensurável} é um par $(X,\Sigma)$ em que $X$ é um conjunto não vazio e $\Sigma$ é uma sigma-álgebra sobre $X$. Um \emph{conjunto mensurável} é um elemento da sigma-álgebra $\Sigma$.
\end{defi}

\subsection{Sigma-Álgebras Geradas}

\begin{prop}
Seja $X$ um conjunto não vazio e $(\Sigma_i)_{i \in I}$ uma família de sigma-álgebras sobre $X$. Então
	\begin{equation*}
	\Sigma \coloneqq \bigcap_{i \in I} \Sigma_i
	\end{equation*}
é uma sigma-álgebra sobre $X$.
\end{prop}
\begin{proof}
	Como $\Sigma_i$ são sigma-álgebras, então $\emptyset \in \Sigma_i$ para todo $i \in I$. Assim, segue que $\emptyset \in \Sigma$. Ainda, se $A \in \Sigma$, então $A \in \Sigma_i$ para todo $i \in I$. Logo $A^\complement \in \Sigma_i$ para todo $i \in I$, o que implica $A^\complement \in \Sigma$. Por fim, se $(A_j)_{j \in \N}$ é uma sequência de conjuntos em $\Sigma$, então $A_j \in \Sigma$ para todo $j \in \N$. Mas isso implica que $A_j \in \Sigma_i$ para todo $j \in \N$, $i \in I$, o que, por sua vez, implica que, para todo $i \in I$,
	\begin{equation*}
	\bigcup_{j \in \N} A_j \in \Sigma_i.
	\end{equation*}
Então conclui-se que $\displaystyle\bigcup_{j \in \N} A_j \in \Sigma$ e, portanto, $\Sigma$ é uma sigma-álgebra sobre $X$.
\end{proof}

\begin{defi}
Seja $X$ um conjunto e $\mathcal C \in \p(X)$ um conjunto de subconjuntos de $X$. A \emph{sigma-álgebra gerada por} $\mathcal C$ é a interseção da família de todas as sigma-álgebras sobre $X$ de que $\mathcal C$ é subconjunto, denotada $\ger{\mathcal C}$.
\end{defi}
	
	A sigma-álgebra gerada por um conjunto é a menor sigma-álgebra que contém esse conjunto no sentido que não existe subconjunto dessa sigma-álgebra que contenha o conjunto e também seja uma sigma-álgebra.

\begin{ex}
	A sigma-álgebra sobre $X$ gerada por $\emptyset$ é $\{\emptyset, X\}$.
\end{ex}

%Tentei construir os elementos da sigma-álgebra gerada, mas dá um pouco mais de trabalho do que eu queria, tem a ver com ordinais e tal, vou fazer depois se der.

%\begin{prop}
%Seja $X$ um conjunto e $\mathcal C \subseteq \p(X)$ um conjunto de subconjuntos de $X$. Então
%	\begin{equation*}
%	\ger{\mathcal C} = \set{\bigcup_{i \in I} C_i}{\card{I}\leq \card{\N} \e \forall i \in I \left( C_i \in \mathcal C \ou C_i\comp \in \mathcal C\right)}.
%	\end{equation*}
%\end{prop}
%\begin{proof}
%Seja $\Sigma \coloneqq \set{\bigcup_{i \in I} C_i}{\card{I}\leq \card{\N} \e \forall i \in I \left( C_i \in \mathcal C \ou C_i\comp \in \mathcal C\right)}$. Primeiro notemos que $\mathcal C \subseteq \Sigma$. Para todo $C \in \mathcal C$, temos $C = \bigcup_{i \in \{0\}} C$ e $\card{\{0\}} \leq \card{\N}$. Logo $C \in \Sigma$ e, então, $\mathcal C \subseteq \Sigma$. Agora, mostremos que $\Sigma$ é uma sigma-álgebra. Primeiro notemos que $\emptyset \in \Sigma$, pois $\emptyset = \bigcup_{i \in \emptyset} C$ e $\card{\emptyset} \leq \card{\N}$.

%Agora, seja $M \in \Sigma$. Então $M=\bigcup_{i \in I} C_i$ tal que $\card{I}\leq \card{\N}$  e, para todo $i \in I$, $C_i \in \mathcal C$ ou $C_i\comp \in \mathcal C$. Como
%	\begin{equation*}
%	M\comp = \left( \bigcup_{i \in I} C_i \right)\comp = \bigcap_{i \in I} (C_i)\comp
%	\end{equation*}

%Por fim, seja $(M_n)_{n \in N}$ uma sequência de conjuntos de $\Sigma$. Então, para todo $n \in \N$, existe conjunto $I_n$ enumerável e conjuntos $(C_{n,i})_{i \in I_n}$ tais que
%	\begin{equation*}
%	M_n = \bigcup_{i \in I_n} C_{n,i}.
%	\end{equation*}
%Como, para todo $n \in \N$, o conjunto $I_n$ é enumerável, então temos que os conjuntos $\set{C_{n,i}}{i \in I_n}$ são enumeráveis e, portanto,
%	\begin{equation*}
%	E \coloneqq \bigcup_{n \in \N} \set{C_{n,i}}{i \in I_n}
%	\end{equation*}
%é enumerável, pois é união enumerável de conjuntos enumeráveis. Indexando $E$ como $(C_j)_{j \in J}$, segue que
%	\begin{equation*}
%	\bigcup_{n \in \N} M_n = \bigcup_{n \in \N} \bigcup_{i \in I_n} C_{n,i} = \bigcup_{j \in J} C_j
%	\end{equation*}
%é uma união enumerável e, portanto, $\bigcup_{n \in \N} M_n \in \Sigma$. Concluímos, assim, que $\Sigma$ é uma sigma-álgebra em que $\mathcal C$ está contido. Para notar que $\Sigma=\ger{\mathcal C}$, resta mostrar que $\Sigma$ é a menor sgma-álgebra em que $\mathcal C$ está contido. Para isso, seja $\Sigma'$ uma sigma-álgebra sobre $X$ tal que $\mathcal C \subseteq \Sigma$. Seja $M \in \Sigma$. Então $M=\bigcup_{i \in I} C_i$ tal que $\card{I}\leq \card{\N}$ e, para todo $i \in I$, $C_i \in \mathcal C$ ou $C_i\comp \in \mathcal C$. Como $\mathcal C \subseteq \Sigma'$, segue que, para todo $i \in I$, $C_i \in \Sigma'$ ou $C_i\comp \in \Sigma'$. Portanto, como $\Sigma'$ é uma sigma-álgebra, $M=\bigcup_{i \in I} C_i \in \Sigma'$. Concluímos que $\Sigma \subseteq \Sigma'$ e, então, que $\Sigma = \ger{\mathcal C}$.
%\end{proof}


\subsection{Limites de Conjuntos}

\begin{defi}
	Sejam $X$ um conjunto e $(A_n)_{n \in \N}$ uma sequência de subconjuntos de $X$. O \emph{limite inferior} de $(A_n)_{n \in \N}$ é o conjunto
	\begin{equation*}
	\liminf A_n \coloneqq \bigcup_{m=0}^\infty \left( \bigcap_{n=m}^\infty A_n \right).
	\end{equation*}
	Esse conjunto é o conjuto dos pontos que não pertencem todos menos finitos conjuntos $A_n$.
O \emph{limite superior} de $(A_n)_{n \in \N}$ é o conjunto
	\begin{equation*}
	\limsup A_n \coloneqq \bigcap_{m=0}^\infty \left( \bigcup_{n=m}^\infty A_n \right).
	\end{equation*}
	Esse conjunto é o conjunto dos pontos que pertencem a infinitos conjuntos $A_n$.
\end{defi}

\begin{prop}
	Sejam $X$ um conjunto e $(A_n)_{n \in \N}$ uma sequência de subconjuntos de $X$. Então
	\begin{equation*}
	\emptyset \subseteq \liminf A_n \subseteq \limsup A_n \subseteq X.
	\end{equation*}
\end{prop}

\begin{prop}
	Sejam $X$ um conjunto e $(A_n)_{n \in \N}$ uma sequência de subconjuntos de $X$. Então
	\begin{enumerate}
	\item Se $(A_n)_{n \in \N}$ é monótona cresente,
	\begin{equation*}
	\liminf A_n = \bigcup_{n=0}^\infty A_n = \limsup A_n.
	\end{equation*}
	
	\item Se $(A_n)_{n \in \N}$ é monótona decresente,
	\begin{equation*}
	\liminf A_n = \bigcap_{n=0}^\infty A_n = \limsup A_n.
	\end{equation*}
	\end{enumerate}
\end{prop}

\begin{defi}
	Sejam $X$ um conjunto e $(A_n)_{n \in \N}$ uma sequência de subconjuntos de $X$. Um \emph{limite} de $(A_n)_{n \in \N}$ é um conjunto $\lim A_n$ tal que
	\begin{equation*}
	\lim A_n = \liminf A_n = \limsup A_n.
	\end{equation*}
\end{defi}


\section{Funções mensuráveis}

\begin{defi}
Sejam $\bm X = (X,\Sigma_X)$ e $\bm Y = (Y,\Sigma_Y)$ espaços mensuráveis. Uma \emph{função mensurável} de $\bm X$ para $\bm Y$ é uma função $f: X \to Y$ que satisfaz
	\begin{equation*}
	\forall M \in \Sigma_Y \qquad f^{-1}(M) \in \Sigma_X.
	\end{equation*}
Denota-se $f: \bm X \to \bm Y$.
\end{defi}

\begin{prop}
Seja $\bm X$ um espaço mensurável. A função $\id_X: X \to X$ é uma função mensurável.
\end{prop}

\begin{prop}
Sejam $\bm{X_0}$, $\bm{X_1}$ e $\bm{X_2}$ espaços mensuráveis e $f_0: \bm{X_0} \to \bm{X_1}$ e $f_1: \bm{X_1} \to \bm{X_2}$ funções mensuráveis. Então $f_1 \circ f_0: \bm{X_0} \to \bm{X_2}$ é uma função mensurável.
\end{prop}

\subsection{Sigma-Álgebras Puxadas e Empurradas}

\begin{defi}
Sejam $X$ um conjunto, $\bm Y = (Y,\Sigma_Y)$ um espaço mensurável e $f: X \to Y$ uma função. A \emph{sigma-álgebra puxada} por $f$ é
	\begin{equation*}
	f^*(\Sigma_Y) \coloneqq \set{f^{-1}(M)}{M \in \Sigma_Y}.
	\end{equation*}
\end{defi}

\begin{prop}
Sejam $X$ um conjunto, $\bm Y = (Y,\Sigma_Y)$ um espaço mensurável e $f: X \to Y$ uma função. Então $\Sigma_X \coloneqq f^*(\Sigma_Y)$, a sigma-álgebra puxada por $f$, é uma sigma-álgebra sobre $X$.
\end{prop}
\begin{proof}
Primeiro, notemos que $\emptyset \in \Sigma_X$, pois $\emptyset \in \Sigma_Y$ e $f^{-1}(\emptyset) = \emptyset$ (\ref{prop:props.imag.inv}). Segundo, seja $B \in \Sigma_X$. Então existe $A \in \Sigma_Y$ tal que $B = f^{-1}(A)$. Como $\Sigma_Y$ é uma sigma-álgebra, então $A^\complement \in \Sigma_Y$, o que implica $f^{-1}(A^\complement) \in \Sigma_X$. Mas $(f^{-1}(A))^\complement = f^{-1}(A^\complement)$ (\ref{prop:props.imag.inv}). Então $B^\complement \in \Sigma_X$. Terceiro, seja $(B_i)_{i \in \N}$ uma sequência de conjuntos de $\Sigma_X$. Então, para todo $i \in I$, existe $A_i \in \Sigma_Y$ tal que $B_i = f^{-1}(A_i)$. Como $\Sigma_Y$ é uma sigma-álgebra, então $\bigcup_{i \in \N} A_i \in \Sigma_Y$. Isso implica que $f^{-1}(\bigcup_{i \in \N} A_i) \in \Sigma_X$. Mas $f^{-1}(\bigcup_{i \in \N} A_i) = \bigcup_{i \in \N} f^{-1}(A_i)$ (\ref{prop:props.imag.inv}). Então $\bigcup_{i \in \N} B_i \in \Sigma_X$ e, assim, conclui-se que $\Sigma_X$ é uma sigma-álgebra sobre $X$.
\end{proof}

\begin{defi}
Sejam $\bm X = (X,\Sigma_X)$ um espaço mensurável, $Y$ um conjunto e $f: X \to Y$ uma função. A \emph{sigma-álgebra empurrada} por $f$ é
	\begin{equation*}
	f_*(\Sigma_X) \coloneqq \set{M \subseteq Y}{f^{-1}(M) \in \Sigma_X}.
	\end{equation*}
\end{defi}

\begin{prop}
Sejam $\bm X = (X,\Sigma_X)$ um espaço mensurável, $Y$ um conjunto e $f: X \to Y$ uma função. Então $\Sigma_Y \coloneqq f_*(\Sigma_X)$, a sigma-álgebra empurrada por $f$, é uma sigma-álgebra sobre $Y$.
\end{prop}
\begin{proof}
Primeiro, notemos que $\emptyset \in \Sigma_Y$, pois $\emptyset \in \Sigma_X$ e $f^{-1}(\emptyset) = \emptyset$ (\ref{prop:props.imag.inv}). Segundo, seja $A \in \Sigma_Y$. Então $f^{-1}(A) \in \Sigma_ X$, o que implica $(f^{-1}(A))^\complement \in \Sigma_ X$, pois $\Sigma_ X$ é sigma-álgebra. Mas $(f^{-1}(A))^\complement = f^{-1}(A^\complement)$ (\ref{prop:props.imag.inv}), o que implica $f^{-1}(A^\complement) \in \Sigma_ X$ e, portanto, $A^\complement \in \Sigma_ Y$. Terceiro, seja $(A_i)_{i \in \N}$ uma sequência de conjuntos de $\Sigma_ Y$. Então, para todo $i \in \N$, $f^{-1}(A_i) \in \Sigma_ X$, o que implica que $\bigcup_{i \in \N} f^{-1}(A_i) \in \Sigma_ X$, pois $\Sigma_ X$ é uma sigma-álgebra. Mas, como $\bigcup_{i \in \N} f^{-1}(A_i) = f^{-1}(\bigcup_{i \in \N} A_i)$ (\ref{prop:props.imag.inv}), segue que $\bigcup_{i \in \N} A_i \in \Sigma_ Y$ e, assim, conclui-se que $\Sigma_ Y$ é uma sigma-álgebra sobre $Y$.
\end{proof}

\begin{prop}
Sejam $\bm X = (X,\Sigma_ X)$ e $\bm Y = (Y,\Sigma_ Y)$ espaços mensuráveis. Uma função $f: X \to Y$ é função mensurável de $\bm X$ para $\bm Y$ se, e somente se, a sigma-álgebra $f^*(\Sigma_Y)$ puxada por $f$ é uma sub-sigma-álgebra de $\Sigma_ X$.
	\begin{equation*}
	f \in \mensu(\bm X,\bm Y) \sse f^*(\Sigma_Y) \subseteq \Sigma_X.
	\end{equation*}
\end{prop}
\begin{proof}
Suponha que $f$ é uma função mensurável. Seja $B \in f^*(\Sigma_Y)$. Então existe $A \in \Sigma_ Y$ tal que $B = f^{-1}(A)$. Como $f$ é mensurável, vale $f^{-1}(A) \in \Sigma_ X$, o que implica $B \in \Sigma_ X$ e, portanto, $f^*(\Sigma_Y) \subseteq \Sigma_ X$. Como $f^*(\Sigma_Y)$ é uma sigma-álgebra sobre $X$ pela proprosição acima, segue que $f^*(\Sigma_Y)$ é uma sub-sigma-álgebra de $\Sigma_ X$. Reciprocamente, suponha que $f^*(\Sigma_Y)$ é uma sub-sigma-álgebra de $\Sigma_ X$. Seja $A \in \Sigma_ Y$. Então $f^{-1}(A) \in f^*(\Sigma_Y)$. Mas como $f^*(\Sigma_Y)$ é uma sub-sigma-álgebra de $\Sigma_ X$, segue que $f^{-1}(A) \in \Sigma_ X$, o que mostra que $f$ é mensurável.
\end{proof}


\clearpage
\section{Produto de espaços mensuráveis}

\begin{defi}
Seja $(\bm{X_i})_{i \in I} = (X_i,\Sigma_i)_{i \in I}$ uma família de espaços mensuráveis. O \emph{produto} da família $(\bm X_i)_{i \in I}$ é o par
	\begin{equation*}
	\prod_{i \in I} \bm{X_i} \coloneqq (X,\Sigma),
	\end{equation*}
em que $X \coloneqq \prod_{i \in I} X_i$ é o produto de conjuntos e
	\begin{equation*}
	\Sigma \coloneqq \ger{\bigcup_{i \in I} \pi_i^*(\Sigma_i)}.
	\end{equation*}
\end{defi}

\begin{prop}
Seja $(\bm X_i)_{i \in I} = (X_i,\Sigma_i)_{i \in I}$ uma família de espaços mensuráveis. Então o produto $\prod_{i \in I} \bm X_i$ é um espaço mensurável.
\end{prop}
\begin{proof}
Sejam $X \coloneqq \prod_{i \in I} X_i$ e $\Sigma=\ger{\bigcup_{i \in I} \pi_i^*(\Sigma_i)}$. Devemos somente argumentar que $\Sigma$ é uma sigma-álgebra sobre $X= \prod_{i \in I} X_i$. Para isso, notemos que, para cada $i \in I$, a sigma-álgebras $\pi_i^*(\Sigma_i)$ é a sigma-álegebras puxada por $\pi_i: X \to X_i$, portanto uma sigma-álgebra sobre $X$. Assim, sendo, $\bigcup_{i \in I} \pi_i^*(\Sigma_i) \subseteq \p(X)$ e, portanto, a sigma-álgebra $\Sigma$ gerada por esse conjunto é uma sigma-álgebra sobre $X$.
\end{proof}

\begin{prop}
Seja $(\bm X_i)_{i \in I} = (X_i,\Sigma_i)_{i \in I}$ uma família de espaços mensuráveis. Para todo $i \in I$, a projeção canônica $\pi_i: \prod_{i \in I} X_i \to X_i$ é uma função mensurável.
\end{prop}
\begin{proof}
Sejam $i \in I$ e $M \in \Sigma_i$. Então $\pi_i\inv(M) \in \pi_i^*(\Sigma_i)$ e, portanto, $\pi_i\inv(M) \in \Sigma$.
\end{proof}

\begin{prop}[Propriedade Universal]
Seja $(\bm X_i)_{i \in I} = (X_i,\Sigma_i)_{i \in I}$ uma família de espaços mensuráveis, $\bm Y = (Y,\Sigma_Y)$ um espaço mensurável e, para todo $i \in I$, $f_i: \bm Y \to \bm{X_i}$ uma função mensurável. Então existe uma única função mensurável $f: \bm Y \to \prod_{i \in I} \bm{X_i}$ tal que, para todo $i \in I$, $\pi_i \circ f = f_i$ (o diagrama comuta).
\begin{figure}
\centering
\begin{tikzpicture}[node distance=2.5cm, auto]
	\node (P) {$\displaystyle\prod_{i \in I} \bm{X_i}$};
	\node (Ci) [below of=P] {$\bm{X_i}$};
	\node (X) [left of=Ci] {$\bm{Y}$};
	\draw[->] (X) to node [swap] {$f_i$} (Ci);
	\draw[->, dashed] (X) to node {$f$} (P);
	\draw[->] (P) to node {$\pi_i$} (Ci);
\end{tikzpicture}
\end{figure}
\end{prop}
\begin{proof}
Defina a função
	\begin{align*}
	\func{f}{Y}{\prod_{i \in I} X_i}{y}{(f_i(y))_{i \in I}}.
	\end{align*}
Da propriedade universal para o produto de conjutos, $f$ é a única função tal que, para todo $i \in I$, $\pi_i \circ f = f_i$. Basta mostrar que $f$ é uma função mensurável. Para simplificar a notação, definamos $(X,\Sigma) \coloneqq \prod_{i \in I} \bm{X_i}$. Todo elemento de $\Sigma$ é formado a partir de complementos e uniões de elementos de $\bigcup_{i \in I} \pi_i^*(\Sigma)$. Sendo assim, como $f\inv$ preserva complemento e união, e $f\inv(\emptyset)=\emptyset$, se mostrarmos que, para todo $M \in \bigcup_{i \in I} \pi_i^*(\Sigma_i)$, $f\inv(M) \in \Sigma_Y$, seguirá que, para todo $M \in \Sigma$, $f\inv(M) \in \Sigma_Y$. Seja $M \in \bigcup_{i \in I} \pi_i^*(\Sigma_i)$. Então existe $i \in I$ tal que $M \in \pi_i^*(\Sigma_i)$ e, portanto, existe $M_i \in \Sigma_i$ tal que $M=\pi_i\inv(M_i)$. Então segue que
	\begin{equation*}
	f\inv(M) = f\inv(\pi_i\inv(M_i)) = (\pi_i \circ f)\inv(M_i) = f_i\inv(M_i)
	\end{equation*}
e portanto, como $f_i$ é mensurável, $f_i\inv(M_i) \in \Sigma_Y$, portanto $f\inv(M) \in \Sigma_Y$. Isso prova, pelos comentários anteriores, que para todo $M \in \Sigma$, $f\inv(M) \in \Sigma$ e, portanto, $f$ é mensurável.
\end{proof}







\clearpage
\section{Espaços Mensuráveis Topológicos}

\subsection{Boreliano e função mensurável em $\R$}

\begin{defi}
	A \emph{álgebra de Borel} $\bor$ é a sigma-álgebra sobre $\R$ gerada pelo conjunto de intervalos abertos de $\R$ da forma $(a,b)$, com $a,b \in \R$. Um \emph{conjunto de Borel} é um elemento de $\bor$.
\end{defi}

\begin{defi}
	A \emph{álgebra de Borel extendida} $\overline{\bor}$ é o conjunto de todos os conjuntos de Borel e de todos conjuntos de Borel unidos a $\{-\infty\}$, $\{+\infty\}$ ou $\{-\infty,+\infty\}$.
\end{defi}

\begin{defi}
	Seja $(X,\Sigma)$ um espaço mensurável. Uma função $f: X \to \R$ \emph{$X$-mensurável} (ou \emph{mensurável}) é uma função que satisfaz
	\begin{equation*}
	\forall \alpha \in \R \qquad \{x \in X : f(x) \leq \alpha\} \in \Sigma.
	\end{equation*}
\end{defi}

\begin{prop}
	Seja $(X,\Sigma)$ um espaço mensurável e $f: X \to \R$. Então as quatro afirmações abaixo são equivalentes
	\begin{enumerate}
	\item $\forall \alpha \in \R \qquad A_\alpha \coloneqq \set{x \in X}{f(x) \leq \alpha} \in \Sigma$;
	\item $\forall \alpha \in \R \qquad B_\alpha \coloneqq \set{x \in X}{f(x) < \alpha} \in \Sigma$;
	\item $\forall \alpha \in \R \qquad C_\alpha \coloneqq \set{x \in X}{f(x) \geq \alpha} \in \Sigma$;
	\item $\forall \alpha \in \R \qquad D_\alpha \coloneqq \set{x \in X}{f(x) > \alpha} \in \Sigma$.
	\end{enumerate}
\end{prop}
\begin{proof}
	Claramente, como $A_\alpha^\complement = D_\alpha$, (1) e (4) são equivalentes; do mesmo modo, como $B_\alpha^\complement = C_\alpha$, (2) e (3) são equivalentes. Vamos demonstrar que (1) e (2) são equivalentes, o que demonstrará a proposição.
	
	...	
\end{proof}

\begin{prop}
	Seja $(X,\Sigma)$ um espaço mensurável e $f: X \to \R$. Então $f$ é uma função mensurável de $(X,\Sigma)$ para $(\R,\bor)$ se, e somente se,
	\begin{equation*}
	\forall \alpha \in \R \qquad \set{x \in X}{f(x) \leq \alpha} \in \Sigma.
	\end{equation*}
\end{prop}
\begin{proof}
	Se $f$ é mensurável, seja $E \in$
	
	...
\end{proof}
\newpage









\section{Medida e Espaço de Medida}

\begin{defi}
Seja $\bm X=(X,\Sigma)$ um espaço mensurável. Uma \emph{medida} sobre $\bm X$ é uma função $\mu : \Sigma \to \RR$ que satisfaz
	\begin{enumerate}
	\item $\mu(\emptyset)=0$;
	\item Para todo $M \in \Sigma$,
	\begin{equation*}
	\mu(M) \geq 0;
	\end{equation*}
	\item Para toda sequência $(M_i)_{i \in \N} \in \Sigma^\N$ de conjuntos mensuráveis disjuntos aos pares,
	\begin{equation*}
	\mu\left(\bigcup_{i \in \N} M_i\right)=\bigplus_{i \in \N} \mu(M_i).
	\end{equation*}
	\end{enumerate}
\end{defi}

\begin{defi}
	Um \emph{espaço de medida} é uma tripla $(X,\Sigma,\mu)$ em que $(X,\Sigma)$ é um espaço mensurável e $\mu$ é uma medida sobre $(X,\Sigma)$.
\end{defi}

\begin{prop}
Sejam $(X, \Sigma,\mu)$ um espaço de medida e $M_1,M_2 \in \Sigma$ conjuntos mensuráveis tais que $M_1 \subseteq M_2$. Então
	\begin{enumerate}
	\item $\mu(M_1) \leq \mu(M_2)$;
	\item $\mu(M_1) < + \infty \entao \mu(M_2 \setminus M_1) = \mu(M_2) - \mu(M_1)$.
	\end{enumerate}
\end{prop}
\begin{proof}
	Como $M_2 = M_1 \cup (M_2 \setminus M_1)$ e $M_1 \cap (M_2 \setminus M_1) = \emptyset$, segue que
	\begin{equation*}
	\mu(M_2)=\mu(M_1)+\mu(M_2 \setminus M_1).
	\end{equation*}
	\begin{enumerate}
	\item Daí, como $\mu(M_2 \setminus M_1) \geq 0$, segue que $\mu(M_2) \geq \mu(M_1)$.
	\item Se $\mu(M_1) < + \infty$, então, subtraindo-a dos dois lados da equação, temos $\mu(M_2)-\mu(M_1)=\mu(M_2 \setminus M_1)$.
	\end{enumerate}
\end{proof}

\begin{prop}
Sejam $(X, \Sigma,\mu)$ um espaço de medida e $(M_n)_{n \in \N} \in \Sigma^\N$ uma sequência de conjuntos mensuráveis.
	\begin{enumerate}
	\item Se $(M_n)_{n \in \N}$ é crescente, então
		\begin{equation*}
		\mu \left( \bigcup_{n \in \N} M_n \right) = \lim_{n \to +\infty} \mu(M_n);
		\end{equation*}
	\item Se $(M_n)_{n \in \N}$ é descresente e $\mu(M_1) < + \infty$, então
		\begin{equation*}
		\mu \left( \bigcap_{n \in \N} M_n \right) = \lim_{n \to +\infty} \mu(M_n).
		\end{equation*}
	\end{enumerate}
\end{prop}
\begin{proof}
	...
\end{proof}





\chapter{Integração por Medida}

\section{Integral de Função Simples}

\begin{defi}
Sejam $X$ um conjunto e $C \subseteq X$. A \emph{função indicadora} de $C$ em $X$ é a função
	\begin{align*}
	\func{\idc_C}{X}{\R}{x}{
	\begin{cases}
		1,& x \in C \\
		0,& x \notin C.
	\end{cases}
	}
	\end{align*}
\end{defi}

\begin{prop}
Sejam $X$ um conjunto e $A,B \subseteq X$, e $I=\{0,\cdots,n\}.$. Então
	\begin{enumerate}
	\item $\idc_\emptyset=0$;
	\item $\idc_X=1$;
	\item $(\idc_A)^2=\idc_A$;
	\item $\idc_A\idc_B=\idc_B\idc_A$;
	\item $\idc_{A \cap B} = \idc_A\idc_B = \min\{\idc_A,\idc_B\}$;
	\item $\idc_{A \cup B} = \idc_A+\idc_B - \idc_A\idc_B = \max\{\idc_A,\idc_B\}$;
	\item $\idc_{A^\complement} = 1- \idc_A$;
	\item $\idc_{A \setminus B} = \idc_A-\idc_A\idc_B$;
	\item $\idc_{A \difsim B} = \idc_A+\idc_B-2\idc_A\idc_B$;
	\item $\displaystyle\idc_{\bigcap_{i \in I} A_i} = \bigtimes_{i \in I} \idc_{A_i}$;
	\item $\displaystyle\idc_{\bigcup_{i \in I} A_i} = \bigplus_{\substack{S \subseteq I\\S \neq \emptyset}} \left((-1)^{\card{S}-1} \bigtimes_{i \in S}\idc_{A_i}\right)$;
	\item $\displaystyle\idc_{\scalebox{1.2}{$\difsim$}_{i \in I} A_i} = \bigplus_{\substack{S \subseteq I\\S \neq \emptyset}} \left((-2)^{\card{S}-1} \bigtimes_{i \in S}\idc_{A_i} \right)$;
	\end{enumerate}
\end{prop}
\begin{proof}
	\begin{align*}
	\idc_{A \difsim B} &= \idc_{A\setminus B \cup B\setminus A} \\
			&= \idc_{A \setminus B} + \idc_{B \setminus A} - 	\idc_{A \setminus B}\idc_{B \setminus A}\\
			&=\idc_A-\idc_A\idc_B + \idc_B-\idc_B\idc_A - (\idc_A-\idc_A\idc_B)(\idc_B-\idc_B\idc_A)\\
			&=\idc_A+\idc_B-2\idc_A\idc_B-(\idc_A\idc_B-\idc_A\idc_B-\idc_A\idc_B+\idc_A\idc_B) \\
			&=\idc_A+\idc_B-2\idc_A\idc_B.
	\end{align*}
	
	\begin{align*}
	\idc_{A \difsim B \difsim C} &= \idc_{A \difsim B}+\idc_C-2\idc_{A \difsim B}\idc_C \\
			&= \idc_A+\idc_B-2\idc_A\idc_B+\idc_C-2(\idc_A+\idc_B-2\idc_A\idc_B)\idc_C \\
			&= \idc_A+\idc_B+\idc_C-2(\idc_A\idc_B+\idc_A\idc_C+\idc_B\idc_C)+4\idc_A\idc_B\idc_C.
	\end{align*}
\end{proof}

\begin{defi}
Seja $(X,\Sigma,\mu)$ um espaço de medida. Uma função \emph{simples} em $X$ é uma função $f:X \to \R$ tal que $f(X)$ é finito.
\end{defi}

\begin{defi}
Sejam $(X,\Sigma,\mu)$ um espaço de medida, $f: X \to \R$ uma função simples, $c_1,\cdots,c_n \in \R$ as constantes distintas tais que $f(X)=\{c_1,\cdots,c_n\}$ e $I \coloneqq \{1,\cdots,n\}$. A \emph{partição associada} a $f$ é o conjunto $\set{P_i}{i \in I}$ em que, para todo $i \in I$,
	\begin{equation*}
	P_i \coloneqq f\inv(c_i) = \set{x \in X}{f(x) = c_i}.
	\end{equation*}
\end{defi}

\begin{prop}
Sejam $(X,\Sigma,\mu)$ um espaço de medida, $f: X \to \R$ uma função simples, $c_1,\cdots,c_n \in \R$ as constantes distintas tais que $f(X)=\{c_1,\cdots,c_n\}$, $I \coloneqq \{1,\cdots,n\}$ e $\set{P_i}{i \in I}$ a partição associada a $f$. Então $\set{P_i}{i \in I}$ é uma partição de $X$ e 
	\begin{equation*}
	f = \bigplus_{i=1}^n c_i \idc_{P_i}.
	\end{equation*}
\end{prop}

\begin{prop}
Sejam $(X,\Sigma,\mu)$ um espaço de medida e $f: X \to \R$ uma função simples. Então existem únicas constantes distintas $c_1,\cdots,c_n \in \R$ e uma única partição de $X$ em conjuntos mensuráveis $P_1,\cdots,P_n \in \Sigma$ satisfazendo
	\begin{equation*}
	f = \bigplus_{i=1}^n c_i \idc_{P_i}.
	\end{equation*}
\end{prop}
\begin{proof}
Como $f(X)$ é finito, existem únicos $c_1,\cdots,c_n \in \R$ distintos tais que $f(X)=\{c_1,\cdots,c_n\}$. Seja $I \coloneqq \{1,\cdots,n\}$. Definem-se os conjuntos
	\begin{equation*}
	P_i \coloneqq \set{x \in X}{f(x) = c_i}.
	\end{equation*}
Esses conjuntos formam claramente uma partição de $X$. Além disso, são mensuráveis pelos resultados do capítulo anterior. Por fim, seja $x \in X$. Existe $j \in I$ tal que $f(x)=c_j$ e $x \in P_j$. Nesse caso, $\idc_{P_j}(x)=1$ e, para todo $i \in I\setminus\{j\}$, $\idc_{P_i}(x)=0$, pois os conjuntos $P_i$ são disjuntos. Portanto
	\begin{equation*}
	f(x) = c_j = c_j \idc_{P_j}(x) + \bigplus_{i \in I \setminus \{j\}} c_i \idc_{P_i}(x) = \bigplus_{i=1}^n c_i \idc_{P_i}(x),
	\end{equation*}o que mostra que
	\begin{equation*}
	f = \bigplus_{i=1}^n c_i \idc_{P_i}.
	\end{equation*}
\end{proof}

\begin{defi}
Sejam $(X,\Sigma,\mu)$ um espaço de medida, $f : X \to \R$ uma função simples tal que $f=\bigplus_{i=1}^n c_i \idc_{P_i}$ e $M \in \Sigma$ um conjunto mensurável. A \emph{integral} de $f$ sobre $M$ com respeito a $\mu$ é o número
	\begin{equation*}
	\int_M f \coloneqq \bigplus_{i=1}^n c_i \mu(P_i \cap M).
	\end{equation*}
Quando for necessário explicitar a medida $\mu$ usada, escreveremos $\displaystyle\int_{\mu,M} f$. No caso em que $M=X$, temos
	\begin{equation*}
	\int_X f = \bigplus_{i=1}^n c_i \mu(f\inv(c_i)) =  \bigplus_{i=1}^n c_i \mu(P_i).
	\end{equation*}
\end{defi}

\newpage

\begin{equation*}
\int_M f
\end{equation*}
\begin{equation*}
\prescript{}{\mu}\int_M f
\end{equation*}
\begin{equation*}
\int_M f \dd \mu
\end{equation*}
\begin{equation*}
\mu\text{-}\int\limits_{x \in X} f(x)
\end{equation*}




Para denotar a integral, a notação
	\begin{equation*}
	\int\limits_{x \in X} f(x)
	\end{equation*}
também pode ser usada, e tem a vantagem de se assemelhar mais com a notação de somatório
	\begin{equation*}
	\bigplus_{i \in I} f_i.
	\end{equation*}
A notação da definição, no entanto, tem a vantagem de evitar escrever a variável $x$, que é de fato desnecessária na maioria dos contextos. A notação adotada para a integral é comumente $\int_M f \mathrm{d}\mu$, mas não adotaremos essa convenção.

\begin{prop}
Sejam $(X,\Sigma,\mu)$ um espaço de medida, $f: X \to \R$ e $g: X \to \R$ funções simples, $M \in \Sigma$ um conjunto mensurável e $c \in \R$. Então
	\begin{enumerate}
	\item $\displaystyle\int_M cf = c\int_M f$.
	\item $\displaystyle\int_M (f+g) = \int_M f + \int_M g$.
	\end{enumerate}
\end{prop}
\begin{proof}
Sejam $f=\bigplus_{i=1}^n c_i \idc_{P_i}$, $g=\bigplus_{j=1}^m d_j \idc_{Q_j}$, $I \coloneqq \{1,\cdots,n\}$ e $J \coloneqq \{1,\cdots,m\}$.
\begin{enumerate}
	\item Se $c=0$, vale a igualdade, pois $0f=0\idc_X$, logo
	\begin{equation*}
	\int_M 0f = 0\mu(M \cap X) = 0 = 0\int_M f.
	\end{equation*}
Se $c \neq 0$, então $cf(X)=\{cc_1,\cdots,cc_n\}$ e as constantes $cc_1,\cdots,cc_n$ são todas distintas. Definindo, para todo $i \in I$, $R_i \coloneqq \set{x \in X}{cf(x)=cc_i}$, os conjuntos $R_1,\cdots,R_n$ formam uma partição de $X$ em conjuntos mensuráveis. Além disso, temos $R_i=P_i$ para todo $i \in I$ porque, como $c \neq 0$, segue que $f(x)=c_i$ se, e somente se, $cf(x)=cc_i$. Portanto
	\begin{equation*}
	\int_M cf = \bigplus_{i=1}^n cc_i \mu(R_i \cap M) = c\bigplus_{i=1}^n c_i \mu(P_i \cap M) = c\int_M f.
	\end{equation*}
	
	\item Como $f(X)$ e $g(X)$ são conjuntos finitos,
	\begin{equation*}
	(f+g)(X) \coloneqq \set{c_i+d_j}{(i,j) \in I \times J}
	\end{equation*}
é um conjunto finito. No entanto, não necessariamente $(f+g)(X)$ tem $mn$ elementos, pois podem existir $(i_1,j_1),(i_2,j_2) \in I \times J$ distintos tais que $c_{i_1}+d_{j_1}=c_{i_2}+d_{j_2}$. Sejam $e_1,\cdots,e_l \in \R$ as constantes distintas tais que $(f+g)(X)=\{e_1,\cdots,e_l\}$, $K \coloneqq \{1,\cdots,l\}$ e $R_k \coloneqq \set{x \in X}{(f+g)(x)=e_k}$. Nesse caso, $\set{R_k}{k \in K}$ é uma partição de $X$ em conjuntos mensuráveis e
	\begin{equation*}
	f+g=\bigplus_{k=1}^l e_k \idc_{R_k}.
	\end{equation*}
Por outro lado, temos
	\begin{align*}
	(f+g)(x) &= f(x)+g(x) \\
				&= \bigplus_{i=1}^n c_i \idc_{P_i}(x) + \bigplus_{j=1}^m d_j \idc_{Q_j}(x) \\
				&= 
	\end{align*}
	

	\begin{equation*}
	f+g = \bigplus_{i=1}^n \bigplus_{j=1}^m (c_i+d_j) \idc_{P_i \cap Q_j}.
	\end{equation*}
 Isso significa que os conjuntos
\end{enumerate}
\end{proof}