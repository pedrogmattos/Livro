\part{{\scshape Análise}}

\chapter{Teoria da Medida}

\section{Espaço Mensurável}

\subsection{Sigma-Álgebras e Sub-Sigma-Álgebras}

\begin{defi}
	Seja $X$ um conjunto não vazio. Uma \emph{sigma-álgebra} $\Sigma$ sobre $X$ é um subconjunto de $\p(X)$ que satisfaz
	\begin{enumerate}
	\item O vazio é mensurável.
	\begin{equation*}
	\emptyset \in \Sigma
	\end{equation*}

	\item Complementar de mensurável é mensurável.
	\begin{equation*}
	A \in \Sigma \entao A^\complement \in \Sigma
	\end{equation*}
	\item União enumerável de mensuráveis é mensurável.
	\begin{equation*}
	(A_i)_{i \in \N} \subseteq \Sigma \entao \bigcup_{i \in \N} A_i \in \Sigma
	\end{equation*}
	\end{enumerate}
\end{defi}

	Vale notar que uma sigma-álgebra $\Sigma$ é uma álgebra booleana (\ref{prop:algeb.subconj}) e, portanto, todas propriedades de álgebras booleanas valem para uma sigma-álgebra. De fato, o \textit{sigma} no nome vem da terceira propriedade das sigma-álgebras, pois veremos que essa propriedade tem a ver com um tipo de soma de medidas a ser definido adiante.

\begin{prop}
	Seja $X$ um conjunto não vazio e $\Sigma$ uma sigma-álgebra sobre $X$. Então
	\begin{enumerate}
	\item $X \in \Sigma$;
	\item $(A_i)_{i \in \N} \subseteq \Sigma \entao \displaystyle \bigcap_{i \in \N} A_i \in \Sigma$.
	\end{enumerate}
\end{prop}
\begin{proof}
	\begin{enumerate}
	\item Da primeira propriedade de $\Sigma$, tem-se que $\emptyset \in \Sigma$. Da segunda propriedade de $\Sigma$, tem-se que $X = \emptyset^\complement \in \Sigma$.
	\item Da segunda propriedade, tem-se que, para todo $i \in \N$, $A_i^\complement \in \Sigma$. Da terceira propriedade de $\Sigma$, tem-se que $\bigcup_{i \in \N} A_i^\complement \in \Sigma$. Das Leis de De Morgan (\ref{prop:de.morgan}), tem-se que
	\begin{equation*}
	\left(\bigcap_{i \in \N} A_i \right)^\complement = \bigcup_{i \in \N} (A_i)^\complement \in \Sigma,
	\end{equation*}
e conclui-se que $\displaystyle\bigcap_{i \in \N} A_i \in \Sigma$.
	\end{enumerate}
\end{proof}

\begin{ex}
	$\Sigma = \{\emptyset,X\}$ e $\Sigma = \p(X)$ são sigma-álgebras sobre $X$.
\end{ex}

\begin{defi}
	Seja $X$ um conjunto não vazio e $\Sigma$ uma sigma-álgebra sobre $X$. Uma \emph{sub-sigma-álgebra} de $\Sigma$ é um conjunto $\Sigma' \subseteq \Sigma$ que é uma sigma-álgebra sobre $X$.
\end{defi}

\begin{defi}
	Um \emph{espaço mensurável} é um par $(X,\Sigma)$ em que $X$ é um conjunto não vazio e $\Sigma$ é uma sigma-álgebra sobre $X$. Um \emph{conjunto mensurável} é um elemento da sigma-álgebra $\Sigma$.
\end{defi}

\subsection{Sigma-Álegbras Puxadas e Empurradas}

\begin{defi}
Sejam $X$ um conjunto, $\bm Y = (Y,\Sigma_Y)$ um espaço mensurável e $f: X \to Y$ uma função. A \emph{sigma-álgebra puxada} por $f$ é
	\begin{equation*}
	f^*(\Sigma_Y) \coloneqq \set{f^{-1}(M)}{M \in \Sigma_Y}.
	\end{equation*}
\end{defi}

\begin{prop}
Sejam $X$ um conjunto, $\bm Y = (Y,\Sigma_Y)$ um espaço mensurável e $f: X \to Y$ uma função. Então $\Sigma_X \coloneqq f^*(\Sigma_Y)$, a sigma-álgebra puxada por $f$, é uma sigma-álgebra sobre $X$.
\end{prop}
\begin{proof}
Primeiro, notemos que $\emptyset \in \Sigma_X$, pois $\emptyset \in \Sigma_Y$ e $f^{-1}(\emptyset) = \emptyset$ (\ref{prop:props.imag.inv}). Segundo, seja $B \in \Sigma_X$. Então existe $A \in \Sigma_Y$ tal que $B = f^{-1}(A)$. Como $\Sigma_Y$ é uma sigma-álgebra, então $A^\complement \in \Sigma_Y$, o que implica $f^{-1}(A^\complement) \in \Sigma_X$. Mas $(f^{-1}(A))^\complement = f^{-1}(A^\complement)$ (\ref{prop:props.imag.inv}). Então $B^\complement \in \Sigma_X$. Terceiro, seja $(B_i)_{i \in \N}$ uma sequência de conjuntos de $\Sigma_X$. Então, para todo $i \in I$, existe $A_i \in \Sigma_Y$ tal que $B_i = f^{-1}(A_i)$. Como $\Sigma_Y$ é uma sigma-álgebra, então $\bigcup_{i \in \N} A_i \in \Sigma_Y$. Isso implica que $f^{-1}(\bigcup_{i \in \N} A_i) \in \Sigma_X$. Mas $f^{-1}(\bigcup_{i \in \N} A_i) = \bigcup_{i \in \N} f^{-1}(A_i)$ (\ref{prop:props.imag.inv}). Então $\bigcup_{i \in \N} B_i \in \Sigma_X$ e, assim, conclui-se que $\Sigma_X$ é uma sigma-álgebra sobre $X$.
\end{proof}

\begin{defi}
Sejam $\bm X = (X,\Sigma_X)$ um espaço mensurável, $Y$ um conjunto e $f: X \to Y$ uma função. A \emph{sigma-álgebra empurrada} por $f$ é
	\begin{equation*}
	f_*(\Sigma_X) \coloneqq \set{M \subseteq Y}{f^{-1}(M) \in \Sigma_X}.
	\end{equation*}
\end{defi}

\begin{prop}
Sejam $\bm X = (X,\Sigma_X)$ um espaço mensurável, $Y$ um conjunto e $f: X \to Y$ uma função. Então $\Sigma_Y \coloneqq f_*(\Sigma_X)$, a sigma-álgebra empurrada por $f$, é uma sigma-álgebra sobre $Y$.
\end{prop}
\begin{proof}
Primeiro, notemos que $\emptyset \in \Sigma_Y$, pois $\emptyset \in \Sigma_X$ e $f^{-1}(\emptyset) = \emptyset$ (\ref{prop:props.imag.inv}). Segundo, seja $A \in \Sigma_Y$. Então $f^{-1}(A) \in \Sigma_ X$, o que implica $(f^{-1}(A))^\complement \in \Sigma_ X$, pois $\Sigma_ X$ é sigma-álgebra. Mas $(f^{-1}(A))^\complement = f^{-1}(A^\complement)$ (\ref{prop:props.imag.inv}), o que implica $f^{-1}(A^\complement) \in \Sigma_ X$ e, portanto, $A^\complement \in \Sigma_ Y$. Terceiro, seja $(A_i)_{i \in \N}$ uma sequência de conjuntos de $\Sigma_ Y$. Então, para todo $i \in \N$, $f^{-1}(A_i) \in \Sigma_ X$, o que implica que $\bigcup_{i \in \N} f^{-1}(A_i) \in \Sigma_ X$, pois $\Sigma_ X$ é uma sigma-álgebra. Mas, como $\bigcup_{i \in \N} f^{-1}(A_i) = f^{-1}(\bigcup_{i \in \N} A_i)$ (\ref{prop:props.imag.inv}), segue que $\bigcup_{i \in \N} A_i \in \Sigma_ Y$ e, assim, conclui-se que $\Sigma_ Y$ é uma sigma-álgebra sobre $Y$.
\end{proof}

\begin{prop}
	Sejam $\bm X = (X,\Sigma_ X)$ e $\bm Y = (Y,\Sigma_ Y)$ espaços mensuráveis. Uma função $f: X \to Y$ é função mensurável de $\bm X$ para $\bm Y$ se, e somente se, $\Sigma_ X' \coloneqq \set{f^{-1}(A)}{A \in \Sigma_ Y}$ é uma sub-sigma-álgebra de $\Sigma_ X$.
\end{prop}
\begin{proof}
	Suponha que $f$ é uma função mensurável. Seja $B \in \Sigma_ X'$. Então existe $A \in \Sigma_ Y$ tal que $B = f^{-1}(A)$. Como $f$ é mensurável, vale $f^{-1}(A) \in \Sigma_ X$, o que implica $B \in \Sigma_ X$ e, portanto, $\Sigma_ X' \subseteq \Sigma_ X$. Como $\Sigma_ X'$ é uma sigma-álgebra sobre $X$ pela proprosição acima, segue que $\Sigma_ X'$ é uma sub-sigma-álgebra de $\Sigma_ X$. Reciprocamente, suponha que $\Sigma_ X'$ é uma sub-sigma-álgebra de $\Sigma_ X$. Seja $A \in \Sigma_ Y$. Então $f^{-1}(A) \in \Sigma_ X'$. Mas como $\Sigma_ X'$ é uma sub-sigma-álgebra de $\Sigma_ X$, segue que $f^{-1}(A) \in \Sigma_ X$, o que mostra que $f$ é mensurável. 
\end{proof}

\subsection{Sigma-Álgebras Geradas}

\begin{prop}
Seja $X$ um conjunto não vazio e $(\Sigma_i)_{i \in I}$ uma família de sigma-álgebras sobre $X$. Então
	\begin{equation*}
	\Sigma \coloneqq \bigcap_{i \in I} \Sigma_i
	\end{equation*}
é uma sigma-álgebra sobre $X$.
\end{prop}
\begin{proof}
	Como $\Sigma_i$ são sigma-álgebras, então $\emptyset \in \Sigma_i$ para todo $i \in I$. Assim, segue que $\emptyset \in \Sigma$. Ainda, se $A \in \Sigma$, então $A \in \Sigma_i$ para todo $i \in I$. Logo $A^\complement \in \Sigma_i$ para todo $i \in I$, o que implica $A^\complement \in \Sigma$. Por fim, se $(A_j)_{j \in \N}$ é uma sequência de conjuntos em $\Sigma$, então $A_j \in \Sigma$ para todo $j \in \N$. Mas isso implica que $A_j \in \Sigma_i$ para todo $j \in \N$, $i \in I$, o que, por sua vez, implica que, para todo $i \in I$,
	\begin{equation*}
	\bigcup_{j \in \N} A_j \in \Sigma_i.
	\end{equation*}
Então conclui-se que $\displaystyle\bigcup_{j \in \N} A_j \in \Sigma$ e, portanto, $\Sigma$ é uma sigma-álgebra sobre $X$.
\end{proof}

\begin{defi}
Seja $X$ um conjunto e $\mathcal C \in \p(X)$ um conjunto de subconjuntos de $X$. A \emph{sigma-álgebra gerada por} $\mathcal C$ é a interseção da família de todas as sigma-álgebras sobre $X$ de que $\mathcal C$ é subconjunto, denotada $\ger{\mathcal C}$.
\end{defi}
	
	A sigma-álgebra gerada por um conjunto é a menor sigma-álgebra que contém esse conjunto no sentido que não existe subconjunto dessa sigma-álgebra que contenha o conjunto e também seja uma sigma-álgebra.

\begin{ex}
	A sigma-álgebra sobre $X$ gerada por $\emptyset$ é $\{\emptyset, X\}$.
\end{ex}

%Tentei construir os elementos da sigma-álgebra gerada, mas dá um pouco mais de trabalho do que eu queria, tem a ver com ordinais e tal, vou fazer depois se der.

%\begin{prop}
%Seja $X$ um conjunto e $\mathcal C \subseteq \p(X)$ um conjunto de subconjuntos de $X$. Então
%	\begin{equation*}
%	\ger{\mathcal C} = \set{\bigcup_{i \in I} C_i}{\card{I}\leq \card{\N} \e \forall i \in I \left( C_i \in \mathcal C \ou C_i\comp \in \mathcal C\right)}.
%	\end{equation*}
%\end{prop}
%\begin{proof}
%Seja $\Sigma \coloneqq \set{\bigcup_{i \in I} C_i}{\card{I}\leq \card{\N} \e \forall i \in I \left( C_i \in \mathcal C \ou C_i\comp \in \mathcal C\right)}$. Primeiro notemos que $\mathcal C \subseteq \Sigma$. Para todo $C \in \mathcal C$, temos $C = \bigcup_{i \in \{0\}} C$ e $\card{\{0\}} \leq \card{\N}$. Logo $C \in \Sigma$ e, então, $\mathcal C \subseteq \Sigma$. Agora, mostremos que $\Sigma$ é uma sigma-álgebra. Primeiro notemos que $\emptyset \in \Sigma$, pois $\emptyset = \bigcup_{i \in \emptyset} C$ e $\card{\emptyset} \leq \card{\N}$.

%Agora, seja $M \in \Sigma$. Então $M=\bigcup_{i \in I} C_i$ tal que $\card{I}\leq \card{\N}$  e, para todo $i \in I$, $C_i \in \mathcal C$ ou $C_i\comp \in \mathcal C$. Como
%	\begin{equation*}
%	M\comp = \left( \bigcup_{i \in I} C_i \right)\comp = \bigcap_{i \in I} (C_i)\comp
%	\end{equation*}

%Por fim, seja $(M_n)_{n \in N}$ uma sequência de conjuntos de $\Sigma$. Então, para todo $n \in \N$, existe conjunto $I_n$ enumerável e conjuntos $(C_{n,i})_{i \in I_n}$ tais que
%	\begin{equation*}
%	M_n = \bigcup_{i \in I_n} C_{n,i}.
%	\end{equation*}
%Como, para todo $n \in \N$, o conjunto $I_n$ é enumerável, então temos que os conjuntos $\set{C_{n,i}}{i \in I_n}$ são enumeráveis e, portanto,
%	\begin{equation*}
%	E \coloneqq \bigcup_{n \in \N} \set{C_{n,i}}{i \in I_n}
%	\end{equation*}
%é enumerável, pois é união enumerável de conjuntos enumeráveis. Indexando $E$ como $(C_j)_{j \in J}$, segue que
%	\begin{equation*}
%	\bigcup_{n \in \N} M_n = \bigcup_{n \in \N} \bigcup_{i \in I_n} C_{n,i} = \bigcup_{j \in J} C_j
%	\end{equation*}
%é uma união enumerável e, portanto, $\bigcup_{n \in \N} M_n \in \Sigma$. Concluímos, assim, que $\Sigma$ é uma sigma-álgebra em que $\mathcal C$ está contido. Para notar que $\Sigma=\ger{\mathcal C}$, resta mostrar que $\Sigma$ é a menor sgma-álgebra em que $\mathcal C$ está contido. Para isso, seja $\Sigma'$ uma sigma-álgebra sobre $X$ tal que $\mathcal C \subseteq \Sigma$. Seja $M \in \Sigma$. Então $M=\bigcup_{i \in I} C_i$ tal que $\card{I}\leq \card{\N}$ e, para todo $i \in I$, $C_i \in \mathcal C$ ou $C_i\comp \in \mathcal C$. Como $\mathcal C \subseteq \Sigma'$, segue que, para todo $i \in I$, $C_i \in \Sigma'$ ou $C_i\comp \in \Sigma'$. Portanto, como $\Sigma'$ é uma sigma-álgebra, $M=\bigcup_{i \in I} C_i \in \Sigma'$. Concluímos que $\Sigma \subseteq \Sigma'$ e, então, que $\Sigma = \ger{\mathcal C}$.
%\end{proof}

\section{Funções mensuráveis}

\begin{defi}
Sejam $\bm X = (X,\Sigma_X)$ e $\bm Y = (Y,\Sigma_Y)$ espaços mensuráveis. Uma \emph{função mensurável} de $\bm X$ para $\bm Y$ é uma função $f: X \to Y$ que satisfaz
	\begin{equation*}
	\forall A \in \Sigma_Y \qquad f^{-1}(A) \in \Sigma_X.
	\end{equation*}
Denota-se $f: \bm X \to \bm Y$.
\end{defi}

\begin{prop}
Seja $\bm X = (X,\Sigma_X)$ um espaço mensurável. A função $\id: X \to X$ é um 
\end{prop}


\clearpage
\section{Produto de espaços mensuráveis}

\begin{defi}
Seja $(\bm{X_i})_{i \in I} = (X_i,\Sigma_i)_{i \in I}$ uma família de espaços mensuráveis. O \emph{produto} da família $(\bm X_i)_{i \in I}$ é o par
	\begin{equation*}
	\prod_{i \in I} \bm{X_i} \coloneqq (X,\Sigma),
	\end{equation*}
em que $X \coloneqq \prod_{i \in I} X_i$ é o produto de conjuntos e
	\begin{equation*}
	\Sigma \coloneqq \ger{\bigcup_{i \in I} \pi_i^*(\Sigma_i)}.
	\end{equation*}
\end{defi}

\begin{prop}
Seja $(\bm X_i)_{i \in I} = (X_i,\Sigma_i)_{i \in I}$ uma família de espaços mensuráveis. Então o produto $\prod_{i \in I} \bm X_i$ é um espaço mensurável.
\end{prop}
\begin{proof}
Sejam $X \coloneqq \prod_{i \in I} X_i$ e $\Sigma=\ger{\bigcup_{i \in I} \pi_i^*(\Sigma_i)}$. Devemos somente argumentar que $\Sigma$ é uma sigma-álgebra sobre $X= \prod_{i \in I} X_i$. Para isso, notemos que, para cada $i \in I$, a sigma-álgebras $\pi_i^*(\Sigma_i)$ é a sigma-álegebras puxada por $\pi_i: X \to X_i$, portanto uma sigma-álgebra sobre $X$. Assim, sendo, $\bigcup_{i \in I} \pi_i^*(\Sigma_i) \subseteq \p(X)$ e, portanto, a sigma-álgebra $\Sigma$ gerada por esse conjunto é uma sigma-álgebra sobre $X$.
\end{proof}

\begin{prop}
Seja $(\bm X_i)_{i \in I} = (X_i,\Sigma_i)_{i \in I}$ uma família de espaços mensuráveis. Para todo $i \in I$, a projeção canônica $\pi_i: \prod_{i \in I} X_i \to X_i$ é uma função mensurável.
\end{prop}
\begin{proof}
Sejam $i \in I$ e $M \in \Sigma_i$. Então $\pi_i\inv(M) \in \pi_i^*(\Sigma_i)$ e, portanto, $\pi_i\inv(M) \in \Sigma$.
\end{proof}

\begin{prop}[Propriedade Universal]
Seja $(\bm X_i)_{i \in I} = (X_i,\Sigma_i)_{i \in I}$ uma família de espaços mensuráveis, $\bm Y = (Y,\Sigma_Y)$ um espaço mensurável e, para todo $i \in I$, $f_i: \bm Y \to \bm{X_i}$ uma função mensurável. Então existe uma única função mensurável $f: \bm Y \to \prod_{i \in I} \bm{X_i}$ tal que, para todo $i \in I$, $\pi_i \circ f = f_i$ (o diagrama comuta).
\begin{figure}
\centering
\begin{tikzpicture}[node distance=2.5cm, auto]
	\node (P) {$\displaystyle\prod_{i \in I} \bm{X_i}$};
	\node (Ci) [below of=P] {$\bm{X_i}$};
	\node (X) [left of=Ci] {$\bm{Y}$};
	\draw[->] (X) to node [swap] {$f_i$} (Ci);
	\draw[->, dashed] (X) to node {$f$} (P);
	\draw[->] (P) to node {$\pi_i$} (Ci);
\end{tikzpicture}
\end{figure}
\end{prop}
\begin{proof}
Defina a função \func{f}{Y}{\prod_{i \in I} X_i}{y}{(f_i(y))_{i \in I}.}
Da propriedade universal para o produto de conjutos, $f$ é a única função tal que, para todo $i \in I$, $\pi_i \circ f = f_i$. Basta mostrar que $f$ é uma função mensurável. Para simplificar a notação, definamos $(X,\Sigma) \coloneqq \prod_{i \in I} \bm{X_i}$. Todo elemento de $\Sigma$ é formado a partir de complementos e uniões de elementos de $\bigcup_{i \in I} \pi_i^*(\Sigma)$. Sendo assim, como $f\inv$ preserva complemento e união, e $f\inv(\emptyset)=\emptyset$, se mostrarmos que, para todo $M \in \bigcup_{i \in I} \pi_i^*(\Sigma_i)$, $f\inv(M) \in \Sigma_Y$, seguirá que, para todo $M \in \Sigma$, $f\inv(M) \in \Sigma_Y$. Seja $M \in \bigcup_{i \in I} \pi_i^*(\Sigma_i)$. Então existe $i \in I$ tal que $M \in \pi_i^*(\Sigma_i)$ e, portanto, existe $M_i \in \Sigma_i$ tal que $M=\pi_i\inv(M_i)$. Então segue que
	\begin{equation*}
	f\inv(M) = f\inv(\pi_i\inv(M_i)) = (\pi_i \circ f)\inv(M_i) = f_i\inv(M_i)
	\end{equation*}
e portanto, como $f_i$ é mensurável, $f_i\inv(M_i) \in \Sigma_Y$, portanto $f\inv(M) \in \Sigma_Y$. Isso prova, pelos comentários anteriores, que para todo $M \in \Sigma$, $f\inv(M) \in \Sigma$ e, portanto, $f$ é mensurável.
\end{proof}







\clearpage
\section{Espaços Mensuráveis Topológicos}

\subsection{Boreliano e função mensurável em $\R$}

\begin{defi}
	A \emph{álgebra de Borel} $\bor$ é a sigma-álgebra sobre $\R$ gerada pelo conjunto de intervalos abertos de $\R$ da forma $(a,b)$, com $a,b \in \R$. Um \emph{conjunto de Borel} é um elemento de $\bor$.
\end{defi}

\begin{defi}
	A \emph{álgebra de Borel extendida} $\overline{\bor}$ é o conjunto de todos os conjuntos de Borel e de todos conjuntos de Borel unidos a $\{-\infty\}$, $\{+\infty\}$ ou $\{-\infty,+\infty\}$.
\end{defi}

\begin{defi}
	Seja $(X,\Sigma)$ um espaço mensurável. Uma função $f: X \to \R$ \emph{$X$-mensurável} (ou \emph{mensurável}) é uma função que satisfaz
	\begin{equation*}
	\forall \alpha \in \R \qquad \{x \in X : f(x) \leq \alpha\} \in \Sigma.
	\end{equation*}
\end{defi}

\begin{prop}
	Seja $(X,\Sigma)$ um espaço mensurável e $f: X \to \R$. Então as quatro afirmações abaixo são equivalentes
	\begin{enumerate}
	\item $\forall \alpha \in \R \qquad A_\alpha \coloneqq \set{x \in X}{f(x) \leq \alpha} \in \Sigma$;
	\item $\forall \alpha \in \R \qquad B_\alpha \coloneqq \set{x \in X}{f(x) < \alpha} \in \Sigma$;
	\item $\forall \alpha \in \R \qquad C_\alpha \coloneqq \set{x \in X}{f(x) \geq \alpha} \in \Sigma$;
	\item $\forall \alpha \in \R \qquad D_\alpha \coloneqq \set{x \in X}{f(x) > \alpha} \in \Sigma$.
	\end{enumerate}
\end{prop}
\begin{proof}
	Claramente, como $A_\alpha^\complement = D_\alpha$, (1) e (4) são equivalentes; do mesmo modo, como $B_\alpha^\complement = C_\alpha$, (2) e (3) são equivalentes. Vamos demonstrar que (1) e (2) são equivalentes, o que demonstrará a proposição.
	
	...	
\end{proof}

\begin{prop}
	Seja $(X,\Sigma)$ um espaço mensurável e $f: X \to \R$. Então $f$ é uma função mensurável de $(X,\Sigma)$ para $(\R,\bor)$ se, e somente se,
	\begin{equation*}
	\forall \alpha \in \R \qquad \set{x \in X}{f(x) \leq \alpha} \in \Sigma.
	\end{equation*}
\end{prop}
\begin{proof}
	Se $f$ é mensurável, seja $E \in$
	
	...
\end{proof}
\newpage









\section{Medida e Espaço de Medida}

\begin{defi}
Seja $\bm X=(X,\Sigma)$ um espaço mensurável. Uma \emph{medida} sobre $\bm X$ é uma função $\mu : \Sigma \to \RR$ que satisfaz
	\begin{enumerate}
	\item $\mu(\emptyset)=0$;
	\item Para todo $M \in \Sigma$,
	\begin{equation*}
	\mu(M) \geq 0;
	\end{equation*}
	\item Para toda sequência $(M_i)_{i \in \N} \in \Sigma^\N$ de conjuntos mensuráveis disjuntos aos pares,
	\begin{equation*}
	\mu\left(\bigcup_{i \in \N} M_i\right)=\bigplus_{i \in \N} \mu(M_i).
	\end{equation*}
	\end{enumerate}
\end{defi}

\begin{defi}
	Um \emph{espaço de medida} é uma tripla $(X,\Sigma,\mu)$ em que $(X,\Sigma)$ é um espaço mensurável e $\mu$ é uma medida sobre $(X,\Sigma)$.
\end{defi}

\begin{prop}
Sejam $(X, \Sigma,\mu)$ um espaço de medida e $M_1,M_2 \in \Sigma$ conjuntos mensuráveis tais que $M_1 \subseteq M_2$. Então
	\begin{enumerate}
	\item $\mu(M_1) \leq \mu(M_2)$;
	\item $\mu(M_1) < + \infty \entao \mu(M_2 \setminus M_1) = \mu(M_2) - \mu(M_1)$.
	\end{enumerate}
\end{prop}
\begin{proof}
	Como $M_2 = M_1 \cup (M_2 \setminus M_1)$ e $M_1 \cap (M_2 \setminus M_1) = \emptyset$, segue que
	\begin{equation*}
	\mu(M_2)=\mu(M_1)+\mu(M_2 \setminus M_1).
	\end{equation*}
	\begin{enumerate}
	\item Daí, como $\mu(M_2 \setminus M_1) \geq 0$, segue que $\mu(M_2) \geq \mu(M_1)$.
	\item Se $\mu(M_1) < + \infty$, então, subtraindo-a dos dois lados da equação, temos $\mu(M_2)-\mu(M_1)=\mu(M_2 \setminus M_1)$.
	\end{enumerate}
\end{proof}

\begin{prop}
Sejam $(X, \Sigma,\mu)$ um espaço de medida e $(M_n)_{n \in \N} \in \Sigma^\N$ uma sequência de conjuntos mensuráveis.
	\begin{enumerate}
	\item Se $(M_n)_{n \in \N}$ é crescente, então
		\begin{equation*}
		\mu \left( \bigcup_{n \in \N} M_n \right) = \lim_{n \to +\infty} \mu(M_n);
		\end{equation*}
	\item Se $(M_n)_{n \in \N}$ é descresente e $\mu(M_1) < + \infty$, então
		\begin{equation*}
		\mu \left( \bigcap_{n \in \N} M_n \right) = \lim_{n \to +\infty} \mu(M_n).
		\end{equation*}
	\end{enumerate}
\end{prop}
\begin{proof}
	...
\end{proof}





\chapter{Teoria da Integração por Medida}

\section{Integral de Função Simples}

\begin{defi}
Sejam $X$ um conjunto e $C \subseteq X$. A \emph{função indicadora} de $C$ em $X$ é a função \func{\idc_C}{X}{\R}{x}{
	\begin{cases}
		1,& x \in C \\
		0,& x \notin C.
	\end{cases}
	}
\end{defi}

\begin{prop}
Sejam $X$ um conjunto e $A,B \subseteq X$, e $I=\{0,\cdots,n\}.$. Então
	\begin{enumerate}
	\item $\idc_\emptyset=0$;
	\item $\idc_X=1$;
	\item $(\idc_A)^2=\idc_A$;
	\item $\idc_A+\idc_B=\idc_B+\idc_A$;
	\item $\idc_A\idc_B=\idc_B\idc_A$;
	\item $\idc_{A \cap B} = \min\{\idc_A,\idc_B\}= \idc_A\idc_B$;
	\item $\idc_{A \cup B} = \max\{\idc_A,\idc_B\}= \idc_A+\idc_B - \idc_A\idc_B$;
	\item $\idc_{A^\complement} = 1- \idc_A$;
	\item $\idc_{A \setminus B} = \idc_A-\idc_A\idc_B$;
	\item $\idc_{A \difsim B} = \idc_A+\idc_B-2\idc_A\idc_B$;
	\item $\displaystyle\idc_{\bigcap_{i \in I} A_i} = \bigtimes_{i \in I} \idc_{A_i}$;
	\item $\displaystyle\idc_{\bigcup_{i \in I} A_i} = \bigplus_{\substack{S \subseteq I\\S \neq \emptyset}} \left((-1)^{\card{S}-1} \bigtimes_{i \in S}\idc_{A_i}\right)$;
	\item $\displaystyle\idc_{\scalebox{1.2}{$\difsim$}_{i \in I} A_i} = \bigplus_{\substack{S \subseteq I\\S \neq \emptyset}} \left((-2)^{\card{S}-1} \bigtimes_{i \in S}\idc_{A_i} \right)$;
	\end{enumerate}
\end{prop}
\begin{proof}
	\begin{align*}
	\idc_{A \difsim B} &= \idc_{A\setminus B \cup B\setminus A} \\
			&= \idc_{A \setminus B} + \idc_{B \setminus A} - 	\idc_{A \setminus B}\idc_{B \setminus A}\\
			&=\idc_A-\idc_A\idc_B + \idc_B-\idc_B\idc_A - (\idc_A-\idc_A\idc_B)(\idc_B-\idc_B\idc_A)\\
			&=\idc_A+\idc_B-2\idc_A\idc_B-(\idc_A\idc_B-\idc_A\idc_B-\idc_A\idc_B+\idc_A\idc_B) \\
			&=\idc_A+\idc_B-2\idc_A\idc_B.
	\end{align*}
	
	\begin{align*}
	\idc_{A \difsim B \difsim C} &= \idc_{A \difsim B}+\idc_C-2\idc_{A \difsim B}\idc_C \\
			&= \idc_A+\idc_B-2\idc_A\idc_B+\idc_C-2(\idc_A+\idc_B-2\idc_A\idc_B)\idc_C \\
			&= \idc_A+\idc_B+\idc_C-2(\idc_A\idc_B+\idc_A\idc_C+\idc_B\idc_C)+4\idc_A\idc_B\idc_C.
	\end{align*}
\end{proof}

\begin{defi}
Seja $(X,\Sigma,\mu)$ um espaço de medida. Uma função \emph{simples} em $X$ é uma função $f:X \to \R$ tal que $f(X)$ é finito.
\end{defi}

\begin{defi}
Sejam $(X,\Sigma,\mu)$ um espaço de medida, $f: X \to \R$ uma função simples, $c_1,\cdots,c_n \in \R$ as constantes distintas tais que $f(X)=\{c_1,\cdots,c_n\}$ e $I \coloneqq \{1,\cdots,n\}$. A \emph{partição associada} a $f$ é o conjunto $\set{P_i}{i \in I}$ em que, para todo $i \in I$,
	\begin{equation*}
	P_i \coloneqq f\inv(c_i) = \set{x \in X}{f(x) = c_i}.
	\end{equation*}
\end{defi}

\begin{prop}
Sejam $(X,\Sigma,\mu)$ um espaço de medida, $f: X \to \R$ uma função simples, $c_1,\cdots,c_n \in \R$ as constantes distintas tais que $f(X)=\{c_1,\cdots,c_n\}$, $I \coloneqq \{1,\cdots,n\}$ e $\set{P_i}{i \in I}$ a partição associada a $f$. Então $\set{P_i}{i \in I}$ é uma partição de $X$ e 
	\begin{equation*}
	f = \bigplus_{i=1}^n c_i \idc_{P_i}.
	\end{equation*}
\end{prop}

\begin{prop}
Sejam $(X,\Sigma,\mu)$ um espaço de medida e $f: X \to \R$ uma função simples. Então existem únicas constantes distintas $c_1,\cdots,c_n \in \R$ e uma única partição de $X$ em conjuntos mensuráveis $P_1,\cdots,P_n \in \Sigma$ satisfazendo
	\begin{equation*}
	f = \bigplus_{i=1}^n c_i \idc_{P_i}.
	\end{equation*}
\end{prop}
\begin{proof}
Como $f(X)$ é finito, existem únicos $c_1,\cdots,c_n \in \R$ distintos tais que $f(X)=\{c_1,\cdots,c_n\}$. Seja $I \coloneqq \{1,\cdots,n\}$. Definem-se os conjuntos
	\begin{equation*}
	P_i \coloneqq \set{x \in X}{f(x) = c_i}.
	\end{equation*}
Esses conjuntos formam claramente uma partição de $X$. Além disso, são mensuráveis pelos resultados do capítulo anterior. Por fim, seja $x \in X$. Existe $j \in I$ tal que $f(x)=c_j$ e $x \in P_j$. Nesse caso, $\idc_{P_j}(x)=1$ e, para todo $i \in I\setminus\{j\}$, $\idc_{P_i}(x)=0$, pois os conjuntos $P_i$ são disjuntos. Portanto
	\begin{equation*}
	f(x) = c_j = c_j \idc_{P_j}(x) + \bigplus_{i \in I \setminus \{j\}} c_i \idc_{P_i}(x) = \bigplus_{i=1}^n c_i \idc_{P_i}(x),
	\end{equation*}o que mostra que
	\begin{equation*}
	f = \bigplus_{i=1}^n c_i \idc_{P_i}.
	\end{equation*}
\end{proof}

\begin{defi}
Sejam $(X,\Sigma,\mu)$ um espaço de medida, $f : X \to \R$ uma função simples tal que $f=\bigplus_{i=1}^n c_i \idc_{P_i}$ e $M \in \Sigma$ um conjunto mensurável. A \emph{integral} de $f$ sobre $M$ com respeito a $\mu$ é o número
	\begin{equation*}
	\int_M f \coloneqq \bigplus_{i=1}^n c_i \mu(P_i \cap M).
	\end{equation*}
Quando for necessário explicitar a medida $\mu$ usada, escreveremos $\displaystyle\int_{\mu,M} f$. No caso em que $M=X$, temos
	\begin{equation*}
	\int_X f = \bigplus_{i=1}^n c_i \mu(f\inv(c_i)) =  \bigplus_{i=1}^n c_i \mu(P_i).
	\end{equation*}
\end{defi}

\newpage

\begin{equation*}
\int_M f
\end{equation*}
\begin{equation*}
\prescript{}{\mu}\int_M f
\end{equation*}
\begin{equation*}
\int_M f \dd \mu
\end{equation*}
\begin{equation*}
\mu\text{-}\int\limits_{x \in X} f(x)
\end{equation*}




Para denotar a integral, a notação
	\begin{equation*}
	\int\limits_{x \in X} f(x)
	\end{equation*}
também pode ser usada, e tem a vantagem de se assemelhar mais com a notação de somatório
	\begin{equation*}
	\bigplus_{i \in I} f_i.
	\end{equation*}
A notação da definição, no entanto, tem a vantagem de evitar escrever a variável $x$, que é de fato desnecessária na maioria dos contextos. A notação adotada para a integral é comumente $\int_M f \mathrm{d}\mu$, mas não adotaremos essa convenção.

\begin{prop}
Sejam $(X,\Sigma,\mu)$ um espaço de medida, $f: X \to \R$ e $g: X \to \R$ funções simples, $M \in \Sigma$ um conjunto mensurável e $c \in \R$. Então
	\begin{enumerate}
	\item $\displaystyle\int_M cf = c\int_M f$.
	\item $\displaystyle\int_M (f+g) = \int_M f + \int_M g$.
	\end{enumerate}
\end{prop}
\begin{proof}
Sejam $f=\bigplus_{i=1}^n c_i \idc_{P_i}$, $g=\bigplus_{j=1}^m d_j \idc_{Q_j}$, $I \coloneqq \{1,\cdots,n\}$ e $J \coloneqq \{1,\cdots,m\}$.
\begin{enumerate}
	\item Se $c=0$, vale a igualdade, pois $0f=0\idc_X$, logo
	\begin{equation*}
	\int_M 0f = 0\mu(M \cap X) = 0 = 0\int_M f.
	\end{equation*}
Se $c \neq 0$, então $cf(X)=\{cc_1,\cdots,cc_n\}$ e as constantes $cc_1,\cdots,cc_n$ são todas distintas. Definindo, para todo $i \in I$, $R_i \coloneqq \set{x \in X}{cf(x)=cc_i}$, os conjuntos $R_1,\cdots,R_n$ formam uma partição de $X$ em conjuntos mensuráveis. Além disso, temos $R_i=P_i$ para todo $i \in I$ porque, como $c \neq 0$, segue que $f(x)=c_i$ se, e somente se, $cf(x)=cc_i$. Portanto
	\begin{equation*}
	\int_M cf = \bigplus_{i=1}^n cc_i \mu(R_i \cap M) = c\bigplus_{i=1}^n c_i \mu(P_i \cap M) = c\int_M f.
	\end{equation*}
	
	\item Como $f(X)$ e $g(X)$ são conjuntos finitos,
	\begin{equation*}
	(f+g)(X) \coloneqq \set{c_i+d_j}{(i,j) \in I \times J}
	\end{equation*}
é um conjunto finito. No entanto, não necessariamente $(f+g)(X)$ tem $mn$ elementos, pois podem existir $(i_1,j_1),(i_2,j_2) \in I \times J$ distintos tais que $c_{i_1}+d_{j_1}=c_{i_2}+d_{j_2}$. Sejam $e_1,\cdots,e_l \in \R$ as constantes distintas tais que $(f+g)(X)=\{e_1,\cdots,e_l\}$, $K \coloneqq \{1,\cdots,l\}$ e $R_k \coloneqq \set{x \in X}{(f+g)(x)=e_k}$. Nesse caso, $\set{R_k}{k \in K}$ é uma partição de $X$ em conjuntos mensuráveis e
	\begin{equation*}
	f+g=\bigplus_{k=1}^l e_k \idc_{R_k}.
	\end{equation*}
Por outro lado, temos
	\begin{align*}
	(f+g)(x) &= f(x)+g(x) \\
				&= \bigplus_{i=1}^n c_i \idc_{P_i}(x) + \bigplus_{j=1}^m d_j \idc_{Q_j}(x) \\
				&= 
	\end{align*}
	

	\begin{equation*}
	f+g = \bigplus_{i=1}^n \bigplus_{j=1}^m (c_i+d_j) \idc_{P_i \cap Q_j}.
	\end{equation*}
 Isso significa que os conjuntos
\end{enumerate}
\end{proof}





































































































\chapter{Espaço Real}

\section{Notação de Vetores e Funções}

	O espaço real $\R^n$ estudado neste capítulo do livro será o espaço vetorial $(\R^n,+,\cdot)$ sobre $\R$. A base canônica de $\R^n$ será representada pelos vetores $\{\bm e_1, \ldots, \bm e_n\}$. Um vetor $x \in \R^n$ será também representado por $x=(x_1,\ldots,x_n)$ e uma função $f: \R^n \to \R^m$ será também representada por $f=(f_1,\ldots,f_m)$, em que $f_i \coloneqq \pi_i \circ f$, sendo $\pi_i$ a $i$-ésima projeção de $\R^m$ em $\R$.

\section{Normas no Espaço Real}

	O espaço real $n$-dimensional será visto como um espaço métrico e herdará, portanto, todas a topologia associada a espaços métricos em geral. A métrica usada será dada por uma norma. Mostraremos que muitas normas podem ser associadas a $\R^n$ de modo que a mesma topologia é gerada. A seguir, analisemos algumas dessas normas.
	
\begin{defi}
	Seja $p \in [1,+\infty[$. A \emph{norma $p$} em $\R^n$ é a função $\nor{\cdot}_p: \R^n \to \R$ definida por
	\begin{equation*}
	\nor{\bm x}_p \coloneqq \left( \sum_{i=1}^n (x_i)^p \right)^{\frac{1}{p}}.
	\end{equation*}
	
	A \emph{norma $\infty$} em $\R^n$ é a função $\nor{\cdot}_\infty \R^n \to \R$ definida por
	\begin{equation*}
	\nor{\bm x}_\infty \coloneqq \max \{x_i : i \in \inte_n \}.
	\end{equation*}
\end{defi}

\begin{prop}
	Desigualdade de nao sei quem (Minkowski?).
\end{prop}


\begin{prop}
	A norma $p$ e a norma $\infty$ em $\R^n$ são normas.
\end{prop}
\begin{proof}
	Consideremos primeiro as normas $p$. Seja $c \in \R$ e $x \in \R^n$. Então
	\begin{equation*}
	\nor{c\bm x}_p = \left( \sum_{i=1}^n (cx_i)^p \right)^{\frac{1}{p}} = \left(c^p \sum_{i=1}^n (x_i)^p \right)^{\frac{1}{p}} = |c| \left( \sum_{i=1}^n (x_i)^p \right)^{\frac{1}{p}} = |c|\nor{\bm x}_p .
	\end{equation*}
	
\end{proof}


\begin{defi}
	\emph{Normas equivalentes} são normas $\nor{\cdot}, |\cdot|: \R^n \to \R$ para as quais existem $c,C \in \R$ tais que $0 < c < C$ e, para todo $x \in \R^n$,
	\begin{equation*}
	c \nor{x} \leq |x| \leq C \nor{x}.
	\end{equation*}
\end{defi}

\begin{prop}
Equivalência de norma é uma relação de equivalência.
\end{prop}

\begin{prop}
	Todas normas em $\R^n$ são equivalentes.
\end{prop}
\begin{proof}
	Seja $\nor{\cdot}$ uma norma em $\R^n$. Vamos mostrar que essa norma é equivalente a $\nor{\cdot}_1$ e, como equivalência de norm é uma relação de equivalência, seguirá que todas normas são equivalentes em $\R^n$. Definamos $c^{-1} \coloneqq \max\{\nor{e_1},\ldots,\nor{e_n}\} < \infty$. Então
	\begin{equation*}
	\nor{x} = \nor{\sum_{i=1}^n x_i e_i} \leq \sum_{i=1}^n |x_i|\nor{e_i} \leq c^{-1} \nor{x}_1
	\end{equation*}
	Assim, segue que $c \nor{x} \leq \nor{x}_1$.
\end{proof}


\subsection{Compactos}

\begin{defi}
	Seja $C \subseteq \R^n$. Uma \emph{cobertura} de $C$ é uma família $(A_i)_{i \in I}$ de abertos de $\R^n$ tal que
	\begin{equation*}
	C \subseteq \bigcup_{i \in I} A_i
	\end{equation*}
	Uma \emph{subcobertura} de $C$ relativa a $(A_i)_{i \in I}$ é uma cobertura $(A_j)_{j \in J}$ de $C$ tal que $J \subseteq I$.
\end{defi}

\begin{defi}
	Um conjunto \emph{compacto} em $\R^n$ é um conjunto $C \subseteq \R^n$ tal que, para toda cobertura $(A_I)_{i \in I}$ de $C$, existe uma subcobertura finita de $C$ relativa a $(A_i)_{i \in I}$.
\end{defi}

\begin{teo}[Heine-Borel]
	Seja $C \subseteq \R^n$ um conjunto. Então são equivalentes:
	\begin{enumerate}
	\item $C$ é compacto.
	\item $C$ é fechado e limitado.
	\item Para toda sequência $(x_n)_{n \in \N} \subseteq C$, existe subsequência $(x_{n_k})_{k \in \N}$ que converge a $x \in C$.
	\end{enumerate}
\end{teo}
\begin{proof}

\end{proof}

\begin{teo}
	Sejam $C \subseteq \R^n$ compacto e $f: C \to \R^m$ contínua. Então $f(C)$ é compacto.
\end{teo}
\begin{proof}

\end{proof}

\begin{prop}
	Sejam $C \subseteq \R^n$ compacto e $f: C \to \R^m$ contínua. Então $f(C)$ é limitado e $f$ adimite máximo e mínimo.
\end{prop}

\begin{prop}
	Um conjunto $C \subseteq \R^n$ é compacto se, e somente se, para toda função $f: C \to \R^m$ contínua, $f(C)$ é limitado.
\end{prop}
\begin{proof}

\end{proof}

exercício

vale o mesmo se trocarmos limitado por fechado na prop anterior?


\begin{prop}
	Sejam $C \subseteq \R^n$ compacto e $f: C \to \R^m$ contínua. Então $f$ é uniformemente contínua.
\end{prop}
\begin{proof}
	Suponhamos, por absurdo, que $f$ não é uniformemente contínua. Então existem $\varepsilon > 0$ e $(x_n),(y_n) \subseteq C$ tais que $|x_n - y_n| < \frac{1}{n}$ e $|f(x_n)-f(y_n)| \geq \varepsilon$. Como $C$ é compacto, existem subsequências $(x_{n_k})$  e $(y_{n_k})$ convergindo a $x \in C$ com $|f(x_{n_k})-f(y_{n_k})| \geq \varepsilon$. Por continuidade, existe $\delta > 0$ tal que, se $x_{n_k},y_{n_k} \in B(x,\delta)$, então $|f(x_{n_k})-f(x)| < \frac{\varepsilon}{2}$ e $|f(y_{n_k})-f(x)| < \frac{\varepsilon}{2}$. Pela desigualdade triangular, temos um absudo.
\end{proof}

\begin{prop}
	Sejam $C \subseteq \R^n$ compacto, $X \subseteq \R^m$ e $f: X \times C \to \R^l$ contínua. Seja $x_0 \in X$. Então, para todo $\varepsilon > 0$, existe $\delta > 0$ tal que, se $x \in X$ e $|x-x_0| < \delta$, então, para todo $\alpha \in C$, $|f(x,\alpha)-f(x_0,\alpha)|<\varepsilon$.
\end{prop}
\begin{proof}
	Suponha que existe $\varepsilon>0$ e $x_k \to x_0$ e $\alpha_k \in C$ tal que $|f(x_k,\alpha_k)-f(x_0, \alpha_k)| \geq \varepsilon$, $\alpha_k$ tem subcobertura $\alpha_{k_n} \to \alpha_0$
\end{proof}











\chapter{Diferenciabilidade no Espaço Real}

\section{Diferenciabilidade}

\begin{defi}
	Sejam $A \subseteq \R^n$ um aberto e $p \in A$. Uma função \emph{diferenciável em $p$} é uma função $f: A \to \R^m$ para a qual existe uma transformação linear $T : \R^n \to \R^m$ tal que
	\begin{equation*}
	\lim_{v \to 0} \frac{\Vert f(p+v)-f(p)-T(v) \Vert}{\Vert v \Vert} = 0.
	\end{equation*}
\end{defi}

Isso é equivalente a
	\begin{equation*}
	\lim_{x \to p} \frac{\Vert f(x)-f(p)-T(x-p) \Vert}{\Vert x-p \Vert} = 0.
	\end{equation*}

\begin{prop}[Unicidade da derivada]
	Sejam $A \subseteq \R^n$ um aberto, $p \in A$ e $f: A \to \R^m$ uma função diferenciável em $p$. Então existe uma única transformação linear $T: \R^n \to \R^m$ tal que
	\begin{equation*}
	\lim_{v \to 0} \frac{\Vert f(p+v)-f(p)-T(v) \Vert}{\Vert v \Vert} = 0.
	\end{equation*}
\end{prop}
\begin{proof}
	Suponhamos que exista tranformação linear $S: \R^n \to \R^m$ que satisfaz
	\begin{equation*}
	\lim_{v \to 0} \frac{\Vert f(p+v)-f(p)-S(v) \Vert}{\Vert v \Vert} = 0.
	\end{equation*}
Nesse caso, temos que
	\begin{align*}
	& \lim_{v \to 0} \frac{\Vert T(v)-S(v) \Vert}{\Vert v \Vert} = \\
	&= \lim_{v \to 0} \frac{\Vert T(v) - (f(p+v)-f(p)) + (f(p+v)-f(p)) - S(v) \Vert}{\Vert v \Vert} \\
	& \leq  \lim_{v \to 0} \frac{\Vert f(p+v)-f(p)-T(v) \Vert}{\Vert v \Vert} + \lim_{v \to \bm 0} \frac{\Vert f(p+v)-f(p)-S(v) \Vert}{\Vert v \Vert} \\
	&=0.
	\end{align*}
	
	Como $T$ e $S$ são tranformações lineares, sabemos que $T(0)=S(0)=0$. Para todo $x \in \R^m \setminus \{0\}$, temos que, quando $t \to 0$, $tx \to 0$. Ainda, como $T$ e $S$ são tranformações lineares, $T(tx)=tT(x)$ e $T(tx)=tT'(x)$, e segue que
	\begin{align*}
	0 &= \lim_{tx \to 0} \frac{\Vert T(tx)-S(tx) \Vert}{\Vert tx \Vert} \\
		&= \lim_{t \to 0} \frac{|t| \Vert T(x)-S(x) \Vert}{|t| \Vert x \Vert} \\
		&= \frac{\Vert T(x)-S(x) \Vert}{\Vert x \Vert},
	\end{align*}
o que implica que $T(x) = S(x)$. Logo $T=S$.
\end{proof}

\begin{nota}
	Sejam $p \in \R^n$ e $f: \R^n \to \R^m$ uma função diferenciável em $p$. A \emph{derivada de $f$ em $p$} é a transformação linear $Df(p): \R^n \to \R^m$ que satisfaz
	\begin{equation*}
	\lim_{v \to \bm 0} \frac{\Vert f(p+v) - f(p) - Df(p)(v) \Vert}{\Vert v \Vert} = 0.
	\end{equation*}
\end{nota}

Podemos ver que, se $f$ é diferenciável, então $Df: \R^n \to L(\R^n,\R^m)$ é uma função que leva $p \in \R^n$ na derivada $Df(p)$ de $f$ em $p$.

\begin{prop}[Derivabilidade implica continuidade]
	Sejam $A \subseteq \R^m$ um aberto, $p \in A$ e $f: A \to \R^n$ uma função diferenciável em $p$. Então $f$ é contínua em $p$.
\end{prop}
\begin{proof}
	Como $f$ é diferenciável em $p$, para todo $\varepsilon > 0$ existe $\delta > 0$ tal que, se $0<\nor{v}< \delta$, então
	\begin{equation*}
	\frac{\nor{f(p+v) - f(p) - Df(p)(v)}}{\nor{v}} < \varepsilon.
	\end{equation*}
Essa última desigualdade é equivalente a $\nor{f(p+v) - f(p) - Df(p)(v)}<\varepsilon\nor{v}$. Agora, notemos que
	\begin{equation*}
	\nor{f(p+v)-f(p)} - \nor{Df(p)(v)} \leq \nor{f(p+v) - f(p) - Df(p)(v)},
	\end{equation*}
o que implica que
	\begin{equation*}
	\nor{f(p+v)-f(p)} < \varepsilon\nor{v} + \nor{Df(p)(v)}.
	\end{equation*}
	Por fim, basta notar que, como $Df(p)$ é linear, para todo $v$ vale $\nor{Df(p)(v)} \leq \nor{Df(p)}\nor{v}$, em que $\nor{Df(p)}$ é a norma do supremo de funções. Assim, segue que
	\begin{equation*}
	\nor{f(p+v)-f(p)} < \varepsilon\nor{v} + \nor{Df(p)}\nor{v} = \Big(\varepsilon+\nor{Df(p)}\Big) \nor{v}.
	\end{equation*}
%Assim, dado $\varepsilon' > 0$, basta tomar $\delta'=$
\end{proof}

\begin{defi}
	Seja $A \subseteq \R^n$ um aberto. Uma função \emph{diferenciável} é  uma função $f: A \to \R^m$ que é diferenciável em todo ponto $p \in A$.
\end{defi}

\begin{prop}[Regra da cadeia]
	Sejam $f: \R^n \to \R^m$ diferenciável em $p \in \R^n$ e $g: \R^m \to \R^l$ diferenciável em $f(p)$. Então $g \circ f: \R^n \to \R^l$ é diferenciável em $p$ e
	\begin{equation*}
	D (g \circ f)(p) = D g(f(p)) \circ D f(p)
	\end{equation*}
\end{prop}
\begin{proof} Definamos
	\begin{align*}
	r_1(v) &\coloneqq f(v) - f(p) - Df(p)(v-p); \\
	r_2(v) &\coloneqq g(v) - g(f(p)) - Dg(f(p))(v-f(p)); \\
	r_3(v) &\coloneqq (g \circ f)(v) - (g \circ f)(p) - (Dg(f(p)) \circ Df(p))(v-p).
	\end{align*}
Como $f$ e $g$ são diferenciáveis nos respectivos pontos do enunciado, segue que
	\begin{equation*}
	\lim_{v \to p} \frac{\nor{r_1(v)}}{\nor{v}} = 0 \e \lim_{v \to p} \frac{\nor{r_2(v)}}{\nor{v}} = 0.
	\end{equation*}
Queremos mostrar que
	\begin{equation*}
	\lim_{v \to p} \frac{\nor{r_3(v)}}{\nor{v}} = 0.
	\end{equation*}	
	
Denotando $d_1 \coloneqq f^{(1)}(p)$, $d_2 \coloneqq g^{(1)}(f(p))$ por simplicidade, temos que
	\begin{align*}
	r_3(v) &= (g \circ f)(v) - (g \circ f)(p) - (d_2 \circ d_1)(v-p) \\
			&= g(f(v)) - g(f(p)) - d_2(d_1(v-p)) \\
			&= g(f(v)) - g(f(p)) - d_2(f(v) - f(p) - r_1(v)) \\
			&= g(f(v)) - g(f(p)) - d_2(f(v) - (f(p)) + d_2(r_1(v))) \\
			&= r_2(f(v)) + d_2(r_1(v)).
	\end{align*}

Como 
	
	
	
	
	
	
	
	










	
	
	
	
	
	
	\newpage

	\begin{align*}
	& \frac{\nor{(g \circ f)(p+v) - (g \circ f)(p) - \big(Dg(f(p)) \circ Df(p)\big)(v)}}{\nor{v}}=\\
	&= \frac{\nor{g (f(p+v)) - g(f(p)) - (Dg(f(p))(Df(p))(v))}}{\nor{v}}
	\end{align*}
	
	\begin{equation*}
	 \frac{\nor{f(p+v) - f(p) - Df(p)(v)}}{\nor{v}}
	\end{equation*}
	\begin{equation*}
	 \frac{\nor{g (f(p)+w) - g(f(p)) - (Dg(f(p))(w)}}{\nor{w}}
	\end{equation*}
\end{proof}

\begin{prop}
	Sejam $f,g: \R^n \to \R^m$ diferenciáveis em $p \in \R^n$. Então
	\begin{enumerate}
	\item $D (f+g)(p) = D f(p) + D g(p);$
	\item $D (f \cdot g) = D f(p) \cdot g(p) + f(p) \cdot D g(p);$
	\item Se $g(a) \neq 0$,
	\begin{equation*}
	D \left(\frac{f}{g}\right) (p) = \frac{g(p) \cdot D f(a) - D g (a) \cdot f(p)}{g(p)^2}
	\end{equation*}
	\end{enumerate}
\end{prop}








































\newpage
\section{Derivadas direcionais}

\begin{defi}
	Sejam $A \subseteq \R^n$ um aberto, $p \in A$, $v \in \R^n$ tal que $p+v \in A$ e $f: A \to \R^m$. A derivada direcional de $f$ em $p$ com relação a $v$ é
	\begin{equation*}
	\partial_v f \coloneqq \lim_{t \to 0} \frac{f(p+v)-f(p)}{t}.
	\end{equation*}
\end{defi}






















\chapter{Equações Diferenciais Ordinárias}

\section{Equações Diferenciais Ordinárias e Soluções}

\begin{defi}
	Sejam $n,d \in \N^*$ Uma \emph{equação diferencial ordinária de ordem $n$ e dimensão $d$} é uma expressão $E$ da forma
	\begin{equation*}
	x^{(n)} = F(t, x, x^{(1)}, \cdots,x^{(n-1)}),
	\end{equation*}
em que $t \in \R$ é a variável \emph{tempo}, $x : \R \to \R^d$ é uma função, $A \subseteq \R \times \R^{nd}$ é um aberto e $F: A \to \R^d$ é uma função contínua.
\end{defi}

\begin{defi}
	Seja $E: x^{(n)} = F(t, x, x^{(1)}, \cdots,x^{(n-1)})$ uma equação diferencial ordinária de dimensão $d$. Uma \emph{solução de $E$} é uma curva $\varphi: I \subseteq \R \to \R^d$ $n$ vezes diferenciável tal que
	\begin{equation*}
	\varphi^{(n)} = F(t, \varphi(t), \varphi^{(1)}(t), \cdots,\varphi^{(n-1)}(t)).
	\end{equation*}
\end{defi}

\begin{prop}[Redução de ordem]
	Seja $E: x^{(n)}(t) = F(t, x, x^{(1)}, \cdots,x^{(n-1)})$ uma equação diferencial ordinária de dimensão $d$ definida num aberto $A \subseteq \R \times \R^{nd}$. Então existe equação diferencial ordinária $E'$ de dimensão $nd$
	\begin{equation*}
	 \bm x' = \bm F(t,\bm x)
	\end{equation*}	
tal que existe $\phi$ solução de $E$ se, e somente se, existe $\bm \phi$ solução de $E'$.
\end{prop}
\begin{proof}
	Definimos $\bm x = (x_0, \ldots,x_{n-1}) \coloneqq (x,x^{(1)}, \cdots,x^{(n-1)})$, de modo que
	\begin{align*}
	\bm x' &= ((x_0)',\ldots,(x_{n-1})',(x_{n-1})') \\
			&= (x^{(1)}, \ldots,x^{(n-1)},x^{(n)}) \\
			&= (x_1,\ldots,x_{n-1},F(t, x, x^{(1)}, \cdots,x^{(n-1)})) \\
			&= (x_1,\ldots,x_{n-1},F(t,\bm x))
	\end{align*}	

	Assim, definimos a função
	\begin{align*}
	\bm F: A &\to \R^{nd} \\
	(t,\bm x) &\mapsto (x_1,\ldots,x_{n-1},F(t, \bm x))
	\end{align*}
e então $\bm x' = \bm F(t,\bm x)$. Por fim, devemos mostrar que as soluções são equivalentes. Seja $\phi: I \subseteq \R \to \R^d$ solução de $E$. Então, para todos $k \in \{1,\ldots,n\}$, existe $\phi^{(k)}: \R \to \R^d$. Assim, definindo $\bm \phi \coloneqq (\phi,\phi^{(1)},\ldots,\phi^{(n-1)})$, segue que $\bm \phi$ é solução de $E'$.
\end{proof}

Por causa dessa proposição, podemos representar qualquer EDO como tendo ordem 1.

\section{Existência e Unicidade de Soluções}



\section{Soluções Maximais}

\begin{defi}
	Seja $E: x'=F(t,x)$ uma equação diferencial ordinária de dimensão $d$. Uma \emph{solução maximal de $E$} é uma solução $\phi: I \to \R^d$ de $E$ para a qual vale que, para toda solução $\psi: J \to \R^d$ de $E$ tal que $I \subseteq J$ e $\psi|_I = \phi$, então $I = J$ e $\phi = \psi$. O intervalo $I$ é o \emph{intervalo maximal}.
\end{defi}

\begin{defi}
	Dado $(t_0,x_0) \in U$, defina
	\begin{equation*}
	S_{(t_0,x_0)} \coloneqq \set{\phi: I_\phi \to \R^d}{\phi \text{\ \ é solução do problema de Cauchy}}.
	\end{equation*}
Dados $\phi ,\psi \in S_{(t_0,x_0)}$, definimos $\phi \leq \psi $ se, e somente se, $I_{\phi } \subseteq I_{\psi}$ e $\psi|_{I_\phi} = \phi $.
\end{defi}

\begin{prop}
	A relação acima de fato é relação de ordem parcial.
\end{prop}

\begin{teo}
	Seja $F: U \to \R^d$ uma função contínua definida num aberto $U \subseteq \R \times \R^d$. Então, para cada $(t_0,x_0) \in U$ existe solução maximal para o problema de Cauchy
	\begin{equation*}
	\begin{cases}
		x' = F(t,x) \\
		x(t_0)=x_0
	\end{cases}
	\end{equation*}
\end{teo}









































