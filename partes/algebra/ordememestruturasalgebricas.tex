\chapter{Ordem em estruturas algébricas}

%\section{Grupos Ordenados}

\section{Corpos ordenados}

Nesta seção, pretendemos formalizar a ideia de elementos positivos e negativos em um corpo, e para isso usaremos o conceito de ordem. Uma ordem, também chamada de ordem total, ordem linear ou cadeia, é uma relação reflexiva, antissimétrica, transitiva e total em um conjunto $X$. As três primeiras propriedades definem uma ordem parcial, e a quarta, também chamada de conexidade, significa existe relação entre qualquer par de elementos. De fato, as três últimas implicam a reflexividade, mas definimos assim para podermos dizer que uma ordem total é uma ordem parcial que satisfaz a totalidade.

Um corpo é um conjunto em que existe adição, subtração, multiplicação e divisão. Para unir os dois conceitos de modo que as estruturas de corpo e de ordem se relacionem, devemos exigir que elas satisfaçam algumas propriedades. Para isso, devemos relembrar dos conceitos de função monótona, de translação e de expansão. Uma função monótona crescente entre conjuntos ordenados é uma função $f$ tal que, se dois elementos do domínio $x,x'$ têm uma certa relação de ordem $x \leq x'$  (um é menor ou igual ao outro), então suas imagens têm a mesma relação de ordem, $f(x) \leq f(x')$. Isso quer dizer que a função preserva a ordem dos conjuntos. Uma função monótona decrescente inverte a relação de ordem, de modo que se dois elementos têm uma relação de ordem $x \leq x'$, suas imagens têm a relação de ordem dual, ou invertida, $f(x') \leq f(x)$. Lembremos ainda que, em uma ordem total, definimos para cada $e \in X$ os seguintes intervalos
	\begin{align*}
	\intfa{e}{\infty} &= \set{e \in X}{e \leq x} \\
	\intaa{e}{\infty} &= \set{e \in X}{e < x} \\
	\intaf{-\infty}{e} &= \set{e \in X}{x \leq e} \\
	\intaa{-\infty}{e} &= \set{e \in X}{x < e}.
	\end{align*}
Claramente, supomos que $\infty$ não é um símbolo usado para representar um elemento de $X$, de modo a não gerar confusão.

Para entrelaçar as estruturas de corpo e de ordem, usamos as translações e expansões em corpos e exigimos que elas sejam monótonas. Seja $\bm C$ um corpo e $c \in C$. A translação em $C$ por $c$ é a função
	\begin{align*}
	\func{T_c}{C}{C}{c'}{c+c'}
	\end{align*}
e a expansão de $C$ por $c$ é a função
	\begin{align*}
	\func{E_c}{C}{C}{c'}{cc'}.
	\end{align*}
Aqui assumimos que os corpos são comutativos, por isso não é necessário considerar translações e expansões à esquerda e à direita.

Pedir que as funções sejam monótonas é bem natural para preservar a estrutura de ordem do corpo. No entanto, se escolhemos ter translação monótona, não podemos ter qualquer expansão monótona, já que se $E_{-1}$ fosse monótona, seguiria que, para todos $c,c' \in C$,
	\begin{equation*}
	c \leq c' \Rightarrow -c \leq -c'
	\end{equation*}
e da monotonicidade da adição seguiria que
	\begin{equation*}
	c' = (c'+c) + (-c) \leq (c'+c) + (-c') = c.
	\end{equation*}
Como $c \leq c'$ e $c' \leq c$, seguiria então da antissimetria da ordem que $c=c'$. Como a ordem é total, isso implicaria ainda que todos elementos são iguais, já que quaisquer dois elementos podem ser comparados. Isso nos sugere, então, a imitar a relação de ordem nos números racionais e reais, e exigir que a expansão por elementos positivos sejam crescente e por elementos negativos seja decrescente. Claro, isso nos obriga a definir o que é ser positivo ou negativo. Positivo significa simplesmente ser maior ou igual a $0$, e negativo significa ser menor ou igual a $0$. A definição de $0$ como positivo ou negativo é, na maioria dos casos, irrelevante, mas em geral é mais conveniente considerá-lo ambos, em vez de nenhum dos dois.

\begin{definition}
Um \emph{corpo ordenado} é um par $(\bm C, \leq)$ em que $\bm C$ é um corpo e $\leq$ é uma ordem em $C$ que satisfaz:
	\begin{enumerate}
	\item (Monotonicidade da translação) Para todo $c \in C$, a translação $T_c\colon C \to C$ é uma função monótona crescente: para todos $c',c'' \in C$
		\begin{equation*}
		c' \leq c'' \Rightarrow c+c' \leq c+c'';
		\end{equation*}
	\item (Monotonicidade da expansão) Para todo $c \in \intfa{0}{\infty}$ ($c \in C$ tal que $c \geq 0$), a expansão $E_c\colon C \to C$ é uma função monótona crescente: para todos $c',c'' \in C$
		\begin{equation*}
		c' \leq c'' \Rightarrow cc' \leq cc''.
		\end{equation*}
 	\end{enumerate}
Os elementos \emph{positivos}, \emph{negativos}, \emph{estritamente positivos} e \emph{estritamente negativos} de $C$ são os elementos $c \in C$ tais que $c \geq 0$, $c \leq 0$, $c>0$ e $c<0$, respectivamente. O conjuntos desses elementos são denotados, respectivamente, $C_{\geq 0} := \intfa{0}{\infty}$, $C_{\leq 0} := \intaf{-\infty}{0}$, $C_{> 0} := \intaa{0}{\infty}$ e $C_{< 0} := \intaa{-\infty}{0}$.
\end{definition}

Poderia-se perguntar por que devemos ter uma ordem total, por que não só uma ordem parcial. Se queremos definir o conceito de positividade e negatividade, devemos garantir que todos os elementos sejam relacionados com o $0$. Suponhamos então que temos uma ordem parcial tal que todos elementos são relacionados com $0$ e mostremos que essa ordem é total. Para todos $c,c' \in C$, consideremos então $c'-c \in C$. Como todos elementos podem ser comparados com $0$, segue que $c'-c \geq 0$ ou $c'-c \leq 0$. Da monotonicidade crescente da translação por $c$, segue que $c' \geq c$ ou $c' \leq c$, portanto a ordem é total. O conceito de monotonicidade da expansão ser crescente ou decrescente de acordo com a positividade do elemento pelo qual se multiplica exige, em si, que todos elementos sejam relacionados com $0$. Nas seções seguintes veremos como o conceito de cone positivo se relaciona com o de ordem em corpos.
%Suponhamos que $\leq$ é uma ordem parcial em $C$ satisfazendo as monotonicidades da adição e multiplicação. Sejam $c,c' \in C$ tais que $c \leq c'$ e $c \neq c'$. Então $0 \leq c'-c$, o que implica, ao dividir por $c'-c$, que $0 \leq 1$. Assim sendo, para todo $n \in \N_C$, temos que $0 \leq n$ e $-n \leq 0$. Ainda, para todos $(p,q) \in \N_C \times \N_C \setminus \{0\}$, $0 \leq \frac{p}{q}$ e $-\frac{p}{q} \leq 0$. Isso significa que $\leq$ é uma ordem total em $\Q_C$.

\begin{proposition}
Seja $(\bm C,\leq)$ um corpo ordenado.
	\begin{enumerate}
	\item Para todo $c \in C$, $c \in C_{\geq 0}$ se, e somente se, $-c \in C_{\leq 0}$;

	\item Para todo $c \in C_{\leq 0}$, a expansão $E_c\colon C \to C$ é uma função monótona decrescente: para todos $c',c'' \in C$
		\begin{equation*}
		c' \leq c'' \Rightarrow cc'' \leq cc'.
		\end{equation*}
		
	\item Para todos $c,c' \in C$,
		\begin{enumerate}
		\item se $c,c' \in C_{\geq 0}$, então $c+c' \in C_{\geq 0}$;
		
		\item se $c,c \in C_{\leq 0}$, então $c+c' \in C_{\leq 0}$;
		
%		\item se $c \in C_{\geq 0}$ e $c' \in C_{\leq 0}$, então $c+c' \in C_{\geq 0}$ se, e somente se, $c \geq -c'$;
		\end{enumerate}
	
	\item Para todos $c,c' \in C$, $cc' \in C_{\geq 0}$ se, e somente se, $c,c' \in C_{\geq 0}$ ou $c,c' \in C_{\leq 0}$;
	
	\item A inversa da adição $-\colon C \to C$ é decrescente: para todos $c,c' \in C$ tais que $c \leq c'$, $-c' \leq -c$;
	
	\item A inversa da multiplicação ${}\inv\colon C \setminus \{0\} \to C \setminus \{0\}$ é decrescente: para todos $c,c' \in C \setminus \{0\}$ tais que $c \leq c'$, ${c'}\inv \leq c\inv$;
	
	\item Para todo $c \in C$, $c^2 \geq 0$;
	
	\item $-1 \leq 0 \leq 1$;
	
	\item Para todos $c,c',d,d' \in C$ tais que $c \leq c'$ e $d \leq d'$, $c+d \leq c'+d'$;
	\end{enumerate}
\end{proposition}
\begin{proof}
	\begin{enumerate}	
%	\item Seja $c \in C_{\geq 0} \cap C_{\leq 0}$. Então $c \geq 0$ e $c \leq 0$, logo da antissimetria da ordem segue que $c=0$.

	\item Seja $c \in C_{\geq 0}$. Isso significa que $c \geq 0$. Pela monotonicidade crescente da translação por $-c$, segue que
		\begin{equation*}
		-c = -c+0 \leq -c+c = 0,
		\end{equation*}
portanto $-c \in C_{\leq 0}$. A recíproca segue direto de $c=-(-c)$.

	\item Sejam $c \in C_{\leq 0}$ e $c',c'' \in C$ tais que $c' \leq c''$. Então temos que $-c \in C_{\leq 0}$, portanto
	\begin{equation*}
	-cc' \leq -cc''
	\end{equation*}
e somando $cc'+cc''$, segue que
	\begin{equation*}
	cc'' = (cc'+cc'')-cc' \leq (cc'+cc'')-cc'' = cc'.
	\end{equation*}
	
	\item 
	
	\item Se $c,c' \in C_{\geq 0}$, então $c \geq 0$ e $c' \geq 0$, logo da monotonicidade crescente de $E_c$ segue que $cc' \geq c0=0$. Se $c,c' \in C_{\leq 0}$, então $c \leq 0$ e $c' \leq 0$, logo da monotonicidade decrescente de $E_c$ segue que $cc' \geq c0=0$.

Reciprocamente, se $c \in C_{\geq 0}$ e $c' \in C_{\leq 0}$, então $c \geq 0$ e $c' \leq 0$, logo da monotonicidade crescente de $E_c$ segue que $cc' \leq c0=0$. O caso em que $c \in C_{\leq 0}$ e $c' \in C_{\geq 0}$ é o mesmo, trocando $c$ e $c'$.
	\end{enumerate}
\end{proof}

\begin{proposition}[Relação de ordem e ciclicidade]
Seja $(\bm C,\leq)$ um corpo ordenado. Se $\car(\bm C) \neq 0$, então $\bm C$ é o corpo trivial ($C=\{0\}$).
\end{proposition}
\begin{proof}
Seja $\car(C)=n \in \N \setminus \{0\}$. Consideremos $0,1,n_C \in C$. Pela monotonicidade da adição, segue que,
	\begin{equation*}
	0 \leq 1 \leq 1+1 \leq \ldots \leq n_C=0,
	\end{equation*}
o que implica, pela antissimetria da ordem, que $1=0$.
\end{proof}

\subsection{Imersão dos inteiros e racionais}

Lembremos que a função
	\begin{align*}
	\func{h}{\Z}{C}{n}{
		\begin{cases}
			\sum_{i \in [n]} 1_C,& n>0 \\
			0_C,& n=0 \\
			\sum_{i \in [-n]} (-1_C),& n<0
		\end{cases}
	}
	\end{align*}
é um homomorfismo de anel. Como $(\bm C,\leq)$ é um corpo ordenado não trivial, ele não tem torção, logo $h$ é injetivo e sua imagem é $\Z_C$. A função
	\begin{align*}
	\func{\bar h}{\Q}{C}{\frac{n}{d}}{h(n)h(d)\inv}
	\end{align*}
é um homomorfismo de corpo injetivo cuja imagem é $\Q_C$. O homomorfismo $\bar h$ é o único homomorfismo injetivo que estende $h$, de acordo com a propriedade universal dos corpos de frações \ref{alg:prop.univer.corpo.frac}. Assim, todo corpo sem torção, em particular um corpo ordenado não trivial, tem um subcorpo isomorfo aos racionais. 	Note que $\bar h$ é crescente, pois $\Q_C$ tem a mesma relação de ordem que $\Q$.

\begin{proposition}
Seja $(\bm C,\leq)$ um corpo ordenado não trivial. Para todo $\frac{n}{d},\frac{n'}{d'} \in \Q_C$,
	\begin{equation*}
	\frac{p}{q} \leq \frac{p'}{q'} \Leftrightarrow pq' \leq qp'
	\end{equation*}
\end{proposition}
\begin{proof}
Tomemos $q,q' \geq 0$ sem perda de generalidade. Suponhamos que $\frac{p}{q} \leq \frac{p'}{q'}$. Da monotonicidade crescente de $E_{qq'}$, segue que
	\begin{equation*}
	pq' = qq'\frac{p}{q} \leq qq'\frac{p'}{q'} = qp'.
	\end{equation*}
A recíproca é evidente.
\end{proof}


\subsection{Cones positivos}

Apresentamos brevemente uma definição alternativa de ordem em corpos.

\begin{definition}
Seja $\bm C$ um corpo. Um \emph{cone pré-positivo} em $C$ é um conjunto $P \subseteq C$ tal que
	\begin{enumerate}
	\item (Fechamento da adição) Para todos $p,p' \in P$, $p+p' \in P$;
	\item (Fechamento da multiplicação) Para todos $p,p' \in P$, $pp' \in P$;
	\item (Positividade do quadrado) Para todo $c \in C$, $c^2 \in P$;
	\item (Não-positividade do oposto da unidade) $-1 \notin P$.
	\end{enumerate}

Um \emph{cone positivo} é um cone pré-positivo $P \subseteq C$ tal que $C = P \cup -P$.
\end{definition}

\begin{proposition}
Sejam $\bm C$ um corpo e $P \subseteq C$ um cone pré-positivo.
	\begin{enumerate}
	\item $0,1 \in P$;
	\item Para todo $p \in P \setminus \{0\}$, $p \inv \in P$;
	\item $P \cap -P = \{0\}$;
	\item $P \cup -P$ é um subcorpo de $C$.
	\end{enumerate}
\end{proposition}
\begin{proof}
	\begin{enumerate}
	\item Pela positividade do quadrado, segue que $0=0^2 \in P$ e $1=1^2 \in P$.
	
	\item Exercício.%Seja $p \in P \setminus \{0\}$. Queremos mostrar que $p \inv \in P$. 
	
	\item Seja $p \in P \cap -P$. Então $p,-p \in P$. Se $p \neq 0$, então $p\inv \in P$ e segue que $-1 = -pp\inv \in P$, o que é uma contradição. Logo $p=0$.
	
	\item Exercício.
	
	\end{enumerate}
\end{proof}

\begin{proposition}
Seja $(\bm C,\leq)$ um corpo ordenado. O conjunto $C_{\geq  0}$ de elementos positivos de $C$ é um cone positivo.
\end{proposition}
\begin{proof}
Segue direto das proposições anteriores.
\end{proof}

\begin{definition}
Sejam $\bm C$ um corpo e $P \subseteq C$ um cone pré-positivo.  A \emph{ordem} induzida por $P$ é a relação $\leq_P$ definida por: para todos $c,c' \in C$, $c \leq c'$ se, e somente se, $c' - c \geq 0$.
\end{definition}

\begin{proposition}
Sejam $\bm C$ um corpo e $P \subseteq C$ um cone pré-positivo.
	\begin{enumerate}
	\item A ordem $\leq_P$ induzida por $P$ é uma ordem parcial em $C$;
	
	\item Para todo $c \in C$, a translação $T_c\colon C \to C$ é monótona crescente;
	
	\item Para todo $c \in C$, 
		\begin{enumerate}
		\item se $c \in P$, a expansão $E_c\colon C \to C$ é monótona crescente;
		
		\item se $c \in -P$, a expansão $E_c\colon C \to C$ é monótona decrescente.
		\end{enumerate}
	\end{enumerate}
\end{proposition}
\begin{proof}
	\begin{enumerate}
	\item (Reflexividade) Seja $c \in C$. Isso significa que $c-c =0 \in P$, logo $c \leq_P c$.

(Antissimetria) Sejam $c,c' \in C$ tais que $c \leq_P c'$ e $c' \leq_P c$. Isso significa que $c'-c \in P$ e $c-c'=-(c'-c) \in P$, o que implica que $c'-c=0$, logo $c=c'$.

(Transitividade) Sejam $c,c',c'' \in C$ tais que $c \leq_P c'$ e $c' \leq_P c''$. Isso significa que $c'-c \in P$ e $c''-c' \in P$, logo $(c''-c') + (c'-c) = c''-c \in P$, portanto $c \leq_P c''$.

	\item Sejam $c \in C$ e $c',c'' \in C$ tais que $c' \leq_P c''$. Isso significa que $c''-c \in P$, portanto
	\begin{equation*}
	(c+c'') - (c+c') = c''-c' \in P,
	\end{equation*}
o que significa que $c+c' \leq_P c+c''$. Isso mostra que $T_c$ é monótona crescente.

	\item Sejam $c \in C$ e $c',c'' \in C$ tais que $c' \leq_P c''$. Isso significa que $c''-c' \in P$.
		\begin{enumerate}
		\item Se $c \in P$, segue do fechamento por multiplicação que
	\begin{equation*}
	cc''-cc' = c(c''-c') \in P,
	\end{equation*}
o que significa que $cc' \leq_P cc''$. Isso mostra que $E_c$ é monótona crescente.
		
		\item Se $c \in -P$, então $-c \in P$, e segue do fechamento por multiplicação que
	\begin{equation*}
	cc'-cc'' = -c(c''-c') \in P,
	\end{equation*}
o que significa que $cc'' \leq_P cc'$. Isso mostra que $E_c$ é monótona decrescente.
		\end{enumerate}
	\end{enumerate}
\end{proof}

No entanto, para elementos $c \in C$ que não estão em $P \cup -P$, não podemos garantir a monotonicidade da expansão por ele. Note que $C \neq P \cup -P$ é equivalente a termos elementos que não são relacionados com $0$.

\begin{proposition}
Sejam $\bm C$ um corpo e $P \subseteq C$ um cone positivo. O par $(\bm C,\leq_P)$ é um corpo ordenado.
\end{proposition}
\begin{proof}
Pela proposição anterior, $\leq_P$ é uma ordem parcial. Basta mostrar a totalidade. Sejam $c,c' \in C$. Como $C=P \cup -P$, então $c'-c \in P \cup -P$. Isso implica que $c'-c \in P$ ou $c'-c \in -P$, logo $c \leq_P c'$ ou $c' \leq_P c$. Assim concluímos que $\leq_P$ é uma ordem.

Pela proposição anterior, a translação é monótona. Como $C=P \cup -P$, então $c \geq_P 0$ se, e somente se, $c \in P$ e $c \leq_P 0$ se, e somente se, $c \in -P$, e segue da proposição anterior que, se $c \geq_P 0$, a expansão $E_c$ é monótona crescente e, se $c \leq_P 0$, a expansão $E_c$ é monótona decrescente.
\end{proof}

\begin{proposition}
Seja $\bm C$ um corpo. Existe uma bijeção entre as ordens em $\bm C$ tais que $(\bm C,\leq)$ é um corpo ordenado e os cones positivos de $\bm C$.
\end{proposition}
\begin{proof}
Usando os resultados anteriores, essa demonstração fica simples. A bijeção é dada por $\leq \mapsto C_{\geq 0}$ e sua inversa é $P \mapsto \leq_P$.
\end{proof}

\subsection{Corpos ordenados completos}

%\subsubsection{Números Reais \ensuremath{\R}}

\begin{definition}
Um corpo ordenado \emph{completo} é um corpo ordenado $(\bm C,\leq)$ tal que todo conjunto não-vazio limitado superiormente admite supremo: para todo $A \subseteq C$ não-vazio imitado superiormente, existe $s \in C$ tal que
	\begin{equation*}
	s = \sup A.
	\end{equation*}
\end{definition}

\begin{proposition}
Seja $(\bm C,\leq)$ um corpo ordenado completo (não-trivial). Existe isomorfismo de corpos crescente  $h\colon \R \to C$.
\end{proposition}
\begin{proof}
%Vamos construir o isomorfismo $h$ por etapas. Primeiro, para todo inteiro $i \in \Z$ definimos $h(i) := i_C \in C$. Como $\car(C) \neq 0$(pois $\bm C$ é não-trivial), temos bijeção entre $\Z$ e $\Z_C$ que é crescente. Agora, para todos $i,j \in \Z$, $j \neq 0$, definimos $h(\frac{i}{j}) := \frac{h(i)}{h(j)} = \frac{i_C}{j_C} \in C$. Assim, temos novamente uma bijeção entre $\Q$ e $\Q_C$ que é crescente. Finalmente, para $x \in \R$, definimos o conjunto $L := \set{q \in \Q}{q \leq x}$. Como $L \subseteq \R$ é limitado superiormente por $x \in \R$, existe cota superior $i \in \Z$ de $L$, logo $h(L) \subseteq C$ é limitado superiormente por $h(i) = i_C \in $, e pela completude de $\bm C$ segue que existe $\sup h(L)$. Portanto definimos
%	\begin{equation*}
%	h(x) := \sup h(L) = \set{h(q)}{q \in \Q,\ q \leq x}.
%	\end{equation*}
Definimos $h$ como
	\begin{align*}
	\func{h}{\R}{C}{x}{\sup \set{\frac{i_C}{j_C}}{(i,j) \in \Z \times \Z \setminus \{0\}, \frac{i}{j} \leq x}}.
	\end{align*}
Definamos $L(x) := \set{q \in \Q}{q \leq x}$. Mostremos que $h$ é uma bijeção. (Injetividade) Sejam $x,x' \in \R$ tais que $x \neq x'$. Da totalidade da ordem, $x < x'$ ou $x' < x$, logo $\sup L(x) < \sup L(x')$ ou $\sup L(x') < \sup L(x)$. No primeiro caso, segue que $h(x) < h(x')$, e no segundo $h(x') < h(x)$, portanto $h(x) \neq h(x')$. (Sobrejetividade) Seja $c \in C$. Então $c = \sup \set{q \in \Q_C}{q \leq c}$, portanto definindo $x := \sup \set{\frac{i_C}{j_C}}{(i,j) \in \Z \times \Z \setminus \{0\}, \frac{i_C}{j_C} \leq c}$, temos que $h(x)=y$.

Agora, mostremos que $h$ é isomorfismo de corpos. (Homomorfismo de grupo aditivo) Sejam $x,x' \in \R$. Então $x+x' = \sup L(x+x')$, logo
	\begin{align*}
	h(x+x') = h(\sup L(x+x')) = \sup h(L(x+x')) = h(x) + h(x').
	\end{align*}
(Homomorfismo de grupo multiplicativo) Sejam $x,x' \in \R$. Então $xx' = \sup L(xx')$, logo
	\begin{align*}
	h(xx') = h(\sup L(xx')) = \sup h(L(xx')) = h(x)h(x').
	\end{align*}
Os detalhes podem ser completados pelo leitor.

O isomorfismo $h$ é crescente pois, se $x \leq x'$, então $L(x) \subseteq L(x')$, logo $h(L(x)) \subseteq h(L(x'))$, o que implica $h(x) \leq h(x')$.
\end{proof}

Essa proposição mostra que corpos ordenados completos são, basicamente, cópias de $\R$. Isso nos permite usar $\R$ sempre que quisermos usar algum corpo que seja ordenado e completo, e tira um pouco a arbitrariedade algébrica de usar o corpo dos números reais.

\subsection{Corpos ordenados infinitesimais}

\begin{definition}
Um \emph{corpo não-infinitesimal} (ou \emph{arquimediano}) é um corpo ordenado $(\bm C,\leq)$ tal que $\N_C$ é ilimitado superiormente: para todo $c \in C$, existe $n \in \N_C$ tal que $c < n$.
\end{definition}

\begin{proposition}
Seja $(\bm C,\leq)$ um corpo ordenado. São equivalentes:
	\begin{enumerate}
	\item $(\bm C,\leq)$ é arquimediano;
	\item Para todos $c \in C_{>0}$ e $c' \in C$, existe $n \in \N_C$ tal que $nc>c'$;
	\item Para todo $c \in C_{>0}$, existe $n \in \N_C$ tal que $0<\frac{1}{n}<c$;
	\item Para todo $c \in C_{>0}$, existe $n \in \N_C$ tal que $0<\frac{1}{2^n}<c$.
	\item Para todos $c,c' \in C$ tais que $c<c'$, existe $q \in \Q_C$ tal que $c < q < c'$.
	\end{enumerate}
\end{proposition}

% Topologicamente falando, qual a consequência de um corpo ser infinitesimal/arquimediano?
% Descobri usando a propriedade final acima: C é não infinitesimal se, e somente se, \Q_C é denso em C (com respeito à topologia de ordem ou da distancia, que são a mesma).


\subsection{Valor absoluto e distância}


\begin{definition}
Seja $(\bm C,\leq)$ um corpo ordenado. O \emph{valor absoluto} de $C$ é a função
	\begin{align*}
	\func{\abs{\var}}{C}{C_{\geq 0}}{c}{
		\begin{cases}
			c,& c \in C_{\geq 0} \\
			-c,& c \in C_{\leq 0}.
		\end{cases}
	}
	\end{align*}
\end{definition}

\begin{proposition}
Seja $(\bm C,\leq)$ um corpo ordenado. O valor absoluto satisfaz
	\begin{enumerate}
	\item (Separação) Para todo $c \in C$, $\abs{c}=0$ se, e somente se, $c=0$;
	\item (Multiplicatividade) Para todos $c,c' \in C$, $\abs{cc'}=\abs{c}\abs{c'}$;
	\item (Subaditividade) Para todos $c,c' \in C$, $\abs{c+c'} \leq \abs{c} + \abs{c'}$;
	
	\item Para todo $c \in C$, $\abs{-c} = \abs{c}$;
	\item Para todo $c \in C \setminus \{0\}$, $\abs{c\inv} = \abs{c}\inv$.
	\end{enumerate}
\end{proposition}
\begin{proof}
	\begin{enumerate}
	\item Como $0 \in C_{\geq 0}$, $\abs{0}=0$. Reciprocamente, se $\abs{c}=0$, então $c=0$ ou $-c=0$. Ambos os casos implicam $c=0$.
	
	\item Sejam $c,c' \in C$. Consideramos quatro casos. Se $c,c' \in C_{\geq 0}$, então $cc' \in C_{\geq 0}$, logo
	\begin{equation*}
	\abs{cc'} = cc' = \abs{c}\abs{c'}.
	\end{equation*}
Se $c,c' \in C_{\leq 0}$, então $cc' \in C_{\geq 0}$, logo
	\begin{equation*}
	\abs{cc'} = cc' = (-c)(-c') =\abs{c}\abs{c'}.
	\end{equation*}
Se $c \in C_{\geq 0}$ e $c \in C_{\leq 0}$, então $cc' \in C_{\leq 0}$, logo
	\begin{equation*}
	\abs{cc'} = -cc' = \abs{c}\abs{c'}.
	\end{equation*}
O caso em que $c \in C_{\leq 0}$ e $c \in C_{\geq 0}$ é o mesmo, trocando $c$ e $c'$.

	\item Sejam $c,c' \in C$. Consideramos quatro casos. Se $c,c' \in C_{\geq 0}$, então $c+c' \in C_{\geq 0}$, logo
	\begin{equation*}
	\abs{c+c'} = c+c' = \abs{c}+\abs{c'}.
	\end{equation*}
Se $c,c' \in C_{\leq 0}$, então $c+c' \in C_{\leq 0}$, logo
	\begin{equation*}
	\abs{c+c'} = -(c+c') = (-c)+(-c') = \abs{c}+\abs{c'}.
	\end{equation*}
Se $c \in C_{\geq 0}$ e $c' \in C_{\leq 0}$, então $-c \leq c$ e $c' \leq -c'$. Consideramos dois casos. Se $c+c' \in C_{\geq 0}$, então
	\begin{equation*}
	\abs{c+c'} = c+c' \leq c+(-c') = \abs{c}+\abs{c'}
	\end{equation*}
Se $c+c' \in C_{\leq 0}$, então
	\begin{equation*}
	\abs{c+c'} = -(c+c') = (-c)+(-c') \leq c+(-c') = \abs{c}+\abs{c'}
	\end{equation*}
	
	\item Seja $c \in C$. Se $c \in C_{\geq 0}$, então $\abs{c}=c$ e $-c \in C_{\leq 0}$, logo $\abs{-c}= c = \abs{c}$; se $c \in C_{\leq 0}$, então $\abs{c}=-c$ e $-c \in C_{\geq 0}$, logo $\abs{-c} = -c = \abs{c}$.
	
	\item Seja $c \in C \setminus \{0\}$. Então
		\begin{equation*}
		\abs{c\inv} = \abs{c\inv}\abs{c}\abs{c}\inv = \abs{c\inv c}\abs{c}\inv = \abs{1}\abs{c}\inv = \abs{c}\inv.
		\end{equation*}
	\end{enumerate}
\end{proof}

\begin{definition}
Seja $(\bm C,\leq)$ um corpo ordenado. A \emph{distância} de $C$ é a função
	\begin{align*}
	\func{\dist{\var}{\var}}{C \times C}{C_{\geq 0}}{(c,c')}{\abs{c'-c}.}
	\end{align*}
\end{definition}

\begin{proposition}
Seja $(\bm C,\leq)$ um corpo ordenado. A distância $\dist{\var}{\var}\colon C \times C \to C_{\geq 0}$ satisfaz:
	\begin{enumerate}
	\item (Separação) Para todos $c,c' \in C$,
		\begin{equation*}
		\dist{c}{c'} = 0 \sse c=c';
		\end{equation*}

	\item (Simetria) Para todos $c,c' \in C$,
		\begin{equation*}
		\dist{c}{c'} = \dist{c'}{c};
		\end{equation*}

	\item (Desigualdade Triangular) Para todos $c,c',c'' \in C$,
		\begin{equation*}
		\dist{c}{c''} \leq \dist{c}{c'} + \dist{c'}{c''};
		\end{equation*}
	
	\item (Invariância por Translação) Para todos $c,c',c'' \in C$,
		\begin{equation*}
		\dist{c+c'}{c+c''} = \dist{c'}{c''};
		\end{equation*}
	
	\item (Homogeneidade Absoluta) Para todos $c,c',c'' \in C$,
		\begin{equation*}
		\dist{cc'}{cc''} = \abs{c}\dist{c'}{c''}.
		\end{equation*}
	
	\item Para todos $c,c' \in C$,
		\begin{equation*}
		\dist{-c}{-c'} = \dist{c}{c'};
		\end{equation*}
	
	\item Para todo $c \in C$,
		\begin{equation*}
		\abs{c} = \dist{0}{c}.
		\end{equation*}
	
%	\item Para todos $c,c' \in C_{\geq 0}$,
%		\begin{equation*}
%		\dist{c\inv}{{c'}\inv} = (\dist{c}{c'})\inv ;
%		\end{equation*}

%	\begin{align*}
%	\abs{\frac{d'}{n'} - \frac{d}{n}} &= \abs{\frac{d'n-n'd}{n'n}} \\
%		&= \frac{\abs{d'n-n'd}}{\abs{n'}\abs{n}}
%	\end{align*}
%	\begin{align*}
%	{c'}\inv - c\inv &= \frac{d'}{n'} - \frac{d}{n} \\
%		&= \left(\frac{d'}{n'} - \frac{d}{n}\right)\left(\frac{n'}{d'} - \frac{n}{d}\right)\inv\left(\frac{n'}{d'} - \frac{n}{d}\right) \\
%		&= \frac{d'n-n'd}{n'n}\frac{d'd}{n'd-d'n}\left(\frac{n'}{d'} - \frac{n}{d}\right) \\
%		&= \frac{d'n-n'd}{n'd-d'n}\frac{d'}{n'}\frac{d}{n}\left(\frac{n'}{d'} - \frac{n}{d}\right) \\
%		&= \frac{d'n-n'd}{n'd-d'n} {c'}\inv c\inv (c'-c)
%	\end{align*}
	\end{enumerate}
\end{proposition}







\subsection{Conicidade}

\begin{definition}[Cone]
Seja $\bm V$ um espaço linear sobre um corpo ordenado $(\bm C,\leq)$. Um \emph{cone} (ou  \emph{conjunto cônico}) em $V$ é um conjunto $X \subseteq V$ tal que, para todo $c \in C_{\geq 0}$,
	\begin{equation*}
	cX \subseteq X.
	\end{equation*}
\end{definition}

Isso é equivalente a, para todos $c \in C_{\geq 0}$ e $x \in X$, $cx \in X$.

\begin{definition}[Combinação cônica]
Sejam $\bm V$ um espaço linear sobre um corpo ordenado $(\bm C,\leq)$ e $v_0,\cdots,v_{n-1} \in V$. Uma \emph{combinação cônica} de $v_0,\cdots,v_{n-1}$ é um vetor $v \in V$ para o qual existem $c_0,\cdots,c_{n-1} \in C_{\geq 0}$ tais que
	\begin{equation*}
	v = \sum_{i \in [n]} c_iv_i.
	\end{equation*}
\end{definition}

Um \emph{cone agudo} é um cone $X$ tal que $X \cap -X = \{0\}$.

\begin{proposition}
Sejam $\bm V$ um espaço linear sobre um corpo ordenado $(\bm C,\leq)$ e $X \subseteq V$ um cone.
	\begin{enumerate}
	\item Se $X \neq \emptyset$, então $0 \in X$;
	\item $X \cap -X$ é o maior subespaço linear contido em $X$.
	\end{enumerate}
\end{proposition}

\begin{definition}
Sejam $X$ um conjunto e $(\bm C,\leq)$ um corpo ordenado. Uma \emph{função positiva} de $X$ para $C$ é uma função de $X$ para $C_{\geq 0}$; ou seja, uma função $f\colon X \to C$ tal que, para todo $x \in X$, $f(x) \geq 0$. O espaço de funções positivas de $X$ para $C$ é $C^X_{\geq 0}$.
\end{definition}

%\begin{definition}
%Sejam $X$ um conjunto, $\bm V$ um espaço linear sobre um corpo ordenado $(\bm C,\leq)$ e $N \subseteq V$ um cone. Uma \emph{função positiva} de $X$ para $L$ com respeito a $N$ é uma função de $X$ para $N$; ou seja, uma função $f\colon X \to L$ tal que, para todo $x \in X$, $f(x) \in N$.
%\end{definition}

\begin{proposition}
Sejam $X$ um conjunto e $(\bm C,\leq)$ um corpo ordenado. O espaço $C^X_{\geq 0}$ de funções positivas é um cone em $C^X$.
\end{proposition}
\begin{proof}
Sejam $c \in C_{\geq 0}$ e $f \in C^X_{\geq 0}$. Então, para todo $x \in X$, $f(x) \geq 0$, portanto $cf(x) \geq 0$, logo $cf \geq 0$.
\end{proof}

% O mesmo vale para funções de um conjunto em um cone de um espaço linear.




\subsection{Convexidade}

\begin{definition}[Conjunto convexo]
Seja $\bm V$ um espaço linear sobre um corpo ordenado $(\bm C,\leq)$. Um \emph{conjunto convexo} em $V$ é um conjunto $X \subseteq V$ tal que, para todo $c \in \intff{0}{1}$,
	\begin{equation*}
	(1-c)X + cX \subseteq X.
	\end{equation*}
\end{definition}

Isso é equivalente a dizer que, para todos $x,x' \in X$ e $c \in \intff{0}{1}$,
	\begin{equation*}
	(1-c)x + cx' \in X.
	\end{equation*}

Uma combinação convexa é uma combinação cônica cuja soma dos coeficientes é $1$.

\begin{definition}[Combinação convexa]
Sejam $\bm V$ um espaço linear sobre um corpo ordenado $(\bm C,\leq)$ e $v_0,\cdots,v_{n-1} \in V$. Uma \emph{combinação convexa} de $v_0,\cdots,v_{n-1}$ é um vetor $v \in V$ para o qual existem $c_0,\cdots,c_{n-1} \in C_{\geq 0}$ tais que $\sum_{i \in [n]} c_i = 1$ e
	\begin{equation*}
	v = \sum_{i \in [n]} c_iv_i.
	\end{equation*}
\end{definition}

Um conjunto convexo é um conjunto fechado por combinações convexas.

\begin{proposition}
Sejam $\bm V$ um espaço linear sobre um corpo ordenado $(\bm C,\leq)$ e $X$ um conjunto convexo em $V$. Para todos $c_0,\cdots,c_{n-1} \in C_{\geq 0}$ tais que $\sum_{i \in [n]} c_i = 1$,
	\begin{equation*}
	\sum_{i \in [n]} c_iX \subseteq X.
	\end{equation*}
%Para todos $x_0,\cdots,x_{n-1} \in X$ e $c_0,\cdots,c_{n-1} \in C_{\geq 0}$ tais que $\sum_{i \in [n]} c_i = 1$,
%	\begin{equation*}
%	\sum_{i \in [n]} c_ix_i \in X.
%	\end{equation*}
\end{proposition}

\begin{definition}[Independência convexa (afim)]
Seja $\bm V$ um espaço linear sobre um corpo ordenado $(\bm C,\leq)$. Uma lista de vetores \emph{convexamente independentes} (ou \emph{afimente independentes}) é uma lista $v_0, \ldots, v_d \subseteq V$ tais que $v_1-v_0, \ldots, v_d-v_0$ são linearmente independentes.
\end{definition}

\subsubsection{Simplexos}

\begin{definition}
Sejam $(\bm C,\leq)$ um corpo ordenado e $d \in \N$. O \emph{$d$-simplexo unitário} (ou simplexo unitário $d$-dimensional) sobre $\bm C$ é o conjunto
	\begin{equation*}
	\simp^d_{\bm C} = \set{c \in C_{\geq 0}^{d+1}}{\sum_{k \in [d+1]} c_k = 1}.
	\end{equation*}

O \emph{$d$-simplexo unitário real} é o $d$-simplexo unitário sobre o corpo dos números reais, denotado
	\begin{equation*}
	\Sx^d := \simp^d_{\R} = \set{x \in \R_{\geq 0}^{d+1}}{\sum_{k \in [d+1]} x_k = 1}.
	\end{equation*}
\end{definition}

O simplexo unitário é portanto o conjunto de pontos de $c = (c_0,\ldots,c_d) \in C^{d+1}$ tais que, para todo $k \in [d+1]$, $c_k \geq 0$ e satisfazendo $\sum_{k \in [d+1]} c_k = 1$.

\begin{figure}
\centering

%%%%%%%%%%%%
% 0-Simplexo
%%%%%%%%%%%%
\begin{subfigure}[b]{0.5\textwidth}
	\centering

	\begin{tikzpicture}[
		scale=2,
		eixo/.style={dotted,cinza},
		vertice/.style={preto,circle,fill,minimum size=1pt,inner sep=0pt,outer sep=0pt},
		]
		% Coordenadas
		\coordinate (O) at (0,0);
		\coordinate (A) at (1,0);
		% Eixos
	%	\draw[eixo] (O) node {$0$};
		\node[vertice,eixo] at (O) {};
		\draw[eixo] (O) node[anchor=east] {$0$};
		\draw[eixo] (O) -- (A) node[anchor=west] {$1$};
		% Simplexo
		\node[anchor=south] at (A) {$\Sx^0$};
		%% Vértice
		\node[vertice] at (A) {};
	\end{tikzpicture}

	\caption{O $0$-simplexo é um ponto.}
	\label{sfig:simplexos.ponto}
\end{subfigure}%\hfill
%%%%%%%%%%%%
% 1-Simplexo
%%%%%%%%%%%%
\begin{subfigure}[b]{0.5\textwidth}
	\centering

	\begin{tikzpicture}[
		scale=2,
		eixo/.style={dotted,cinza},
		vertice/.style={preto,circle,fill,minimum size=1pt,inner sep=0pt,outer sep=0pt},
		aresta/.style={preto,line width=1pt}
		]
		% Coordenadas
		\coordinate (O) at (0,0);
		\coordinate (A) at (1,0);
		\coordinate (B) at (0,1);
		% Eixos
	%	\draw[eixo] (O) node {$0$};
		\node[vertice,eixo] at (O) {};
		\draw[eixo] (O) node[anchor=north east] {$0$};
		\draw[eixo] (O) -- (A) node[anchor=west] {$(1,0)$};
		\draw[eixo] (O) -- (B) node[anchor=south] {$(0,1)$};
		% Simplexo
	%	\draw (0.5,0.5) node[anchor=south west] {$\Sx^1$};
		\node[anchor=south west] at ($(A)!0.5!(B)$) {$\Sx^1$};
		%% Vértices
		\node[vertice] at (A) {};
		\node[vertice] at (B) {};
		%% Aresta
		\draw[aresta] (A) -- (B);
	\end{tikzpicture}

	\caption{O $1$-simplexo é um intervalo.}
	\label{sfig:simplexos.intervalo}
\end{subfigure}

%%%%%%%%%%%%
% 2-Simplexo
%%%%%%%%%%%%
\begin{subfigure}[b]{0.5\textwidth}
	\centering

	\tdplotsetmaincoords{60}{60}
	\begin{tikzpicture}[
		scale=2,
		tdplot_main_coords,
		eixo/.style={dotted,cinza},
		vertice/.style={preto,circle,fill,minimum size=1pt,inner sep=0pt,outer sep=0pt},
		aresta/.style={preto,line width=1pt},
		face/.style={preto,opacity=0.75},% opacity=0.9
		]
		% Coordenadas
		\coordinate (O) at (0,0,0);
		\coordinate (A) at (1,0,0);
		\coordinate (B) at (0,1,0);
		\coordinate (C) at (0,0,1);
		% Eixos
	%	\draw[eixo] (O) node {$0$};
		\node[vertice,eixo] at (O) {};
		\draw[eixo] (O) node[anchor=north east] {$0$};
		\draw[eixo] (O) -- (A) node[anchor=north] {$(1,0,0)$};
		\draw[eixo] (O) -- (B) node[anchor=west] {$(0,1,0)$};
		\draw[eixo] (O) -- (C) node[anchor=south] {$(0,0,1)$};
		% Simplexo
	%	\node[tdplot_main_coords,anchor=south west] at (0,1/2,1/2) {$\Sx^2$};
		\node[tdplot_main_coords,anchor=south west] at ($(B)!0.5!(C)$) {$\Sx^2$};
		%% Vértices
		\node[vertice] at (A) {};
		\node[vertice] at (B) {};
		\node[vertice] at (C) {};
		%% Arestas
		\draw[aresta] (A) -- (B);
		\draw[aresta] (B) -- (C);
		\draw[aresta] (C) -- (A);
		%% Face
		\fill[face] (A) -- (B) -- (C) -- cycle;
	\end{tikzpicture}
	
	\caption{O $2$-simplexo é um triângulo.}
	\label{sfig:simplexos.triangulo}
\end{subfigure}%\hfill
%%%%%%%%%%%%
% 3-Simplexo
%%%%%%%%%%%%
\begin{subfigure}[b]{0.5\textwidth}
	\centering

	\tdplotsetmaincoords{60}{60}
	\begin{tikzpicture}[
		scale=2,
		tdplot_main_coords,
		eixo/.style={dotted,cinza},
		vertice/.style={preto,circle,fill,minimum size=1pt,inner sep=0pt,outer sep=0pt},
		aresta/.style={preto,line width=1pt},
		face/.style={preto,opacity=0.5},% opacity=0.75
		]
		% Coordenadas
		\coordinate (O) at (0,0,0);
		\coordinate (A) at (1,0,0);
		\coordinate (B) at (0,1,0);
		\coordinate (C) at (0,0,1);
		\coordinate (D) at (-1/3,-1/3,-1/3);
		% Eixos
	%	\draw[eixo] (O) node {$0$};
		\node[vertice,eixo] at (O) {};
		\draw[eixo] (O) node[anchor=south east] {$0$};
		\draw[eixo] (O) -- (A) node[anchor=north] {$(1,0,0,0)$};
		\draw[eixo] (O) -- (B) node[anchor=west] {$(0,1,0,0)$};
		\draw[eixo] (O) -- (C) node[anchor=south] {$(0,0,1,0)$};
		\draw[eixo] (O) -- (D) node[anchor=east] {$(0,0,0,1)$};
		% Simplexo
		\node[tdplot_main_coords,anchor=south west] at ($(B)!0.5!(C)$) {$\Sx^3$};
		%% Vértices
		\node[vertice] at (A) {};
		\node[vertice] at (B) {};
		\node[vertice] at (C) {};
		\node[vertice] at (D) {};
		%% Arestas
		\draw[aresta] (A) -- (B);
		\draw[aresta] (A) -- (C);
		\draw[aresta] (A) -- (D);
		\draw[aresta] (B) -- (C);
		\draw[aresta,dashed] (B) -- (D);
		\draw[aresta] (C) -- (D);
		%% Faces
		\fill[face] (A) -- (B) -- (C) -- cycle;
		\fill[face] (A) -- (B) -- (D) -- cycle;
		\fill[face] (B) -- (C) -- (D) -- cycle;
		\fill[face] (A) -- (C) -- (D) -- cycle;
	\end{tikzpicture}
	
	\caption{O $3$-simplexo é um tetraedro.}
	\label{sfig:simplexos.tetraedro}
\end{subfigure}

\caption[Simplexos unitários reais de dimensões $0$, $1$, $2$ e $3$]{
	Simplexos unitários reais $\Sx^0$, $\Sx^1$, $\Sx^2$ e $\Sx^3$. Os eixos estão representados em cinza para referência; os simplexos estão representados em preto e, nos casos de dimensão $2$ e $3$, suas faces estão representadas com opacidade para que os eixos sejam visíveis.% Os vértices são a base canônica $\ii_k$.
}
\label{fig:simplexos}
\end{figure}

\begin{definition}
Sejam $\bm V$ um espaço linear sobre um corpo ordenado $(\bm C,\leq)$ e $v_0, \ldots, v_d \in E$ convexamente independentes. O \emph{simplexo gerado} por $v_0, \ldots, v_d$ é o conjunto
	\begin{equation*}
	\simpger[d]{v_0, \ldots, v_d} = \set{\sum_{k \in [d+1]} c_k v_k = 1}{c \in \simp^d_C}
	\end{equation*}
e o interior desse simplexo é
	\begin{equation*}
	\simpgera[d]{v_0, \ldots, v_d} = \set{\sum_{k \in [d+1]} c_k v_k = 1}{c \in \simp^d_C}.
	\end{equation*}
\end{definition}

No caso real, denota-se
\begin{equation*}
	\Sxger[d]{v_0, \ldots, v_d} = \set{\sum_{k \in [d+1]} x_k v_k = 1}{x \in \Sx^d}
	\end{equation*}
e
	\begin{equation*}
	\Sxgera[d]{v_0, \ldots, v_d} = \set{\sum_{k \in [d+1]} x_k v_k = 1}{x \in \Sx^d}.
	\end{equation*}

Em particular, o $1$-simplexo é o conjunto $\simp^1 = \set{(1-t,t)}{t \in \intff{0}{1}}$ e dados $v,v' \in E$,
	\begin{equation*}
	\simpger{v,v'} = \set{(1-t)v + tv'}{t \in \intff{0}{1}}
	\end{equation*}
e
	\begin{equation*}
	\simpgera{v,v'} = \set{(1-t)v + tv'}{t \in \intaa{0}{1}}.
	\end{equation*}
















%\section{Espaços Lineares Ordenados}