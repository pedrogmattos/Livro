\chapter{Álgebras}

\section{Álgebra e conceitos básicos}

%\begin{definition}
%Seja $\bm C$ um corpo. Uma \emph{álgebra} sobre $\bm C$ é um par $(\bm A,\times)$ em que $\bm A$ é um espaço vetorial sobre $\bm C$ e $\times\colon A \times A \to A$ é uma função bilinear. Uma álgebra é \emph{associativa, comutativa} ou \emph{antissimétrica} conforme a respectiva propriedade do produto $\times$, e é unitária se $\times$ tem identidade.
%\end{definition}

\begin{definition}
Seja $\bm C$ um corpo. Uma \emph{álgebra} sobre $\bm C$ é uma lista
	\begin{equation*}
	\bm{A} := ((A,+,-,0,\times),\cdot)                                                                                                       \end{equation*}
tal que $((A,+,-,0),\cdot)$ é um espaço linear sobre $\bm C$ e $\fun{\times}{A \times A}{A}$ é uma função bilinear. Uma álgebra é \emph{associativa, comutativa, alternativa} ou \emph{antissimétrica} conforme a respectiva propriedade de $\times$.

Uma \emph{álgebra unitária} sobre $\bm C$ é uma lista
	\begin{equation*}
	((A,+,-,0,\times,1),\cdot)
	\end{equation*}
tal $((A,+,-,0,\times),\cdot)$ é uma álgebra sobre $\bm C$ e $1 \in A$ é identidade de $(A,\times)$.

Uma \emph{álgebra de divisão associativa} sobre $\bm C$ é uma lista
	\begin{equation*}
	((A,+,-,0,\times,1,\div),\cdot)                                                                                                    \end{equation*}
tal $((A,+,-,0,\times,1),\cdot)$ é uma álgebra unitária associativa sobre $\bm C$ e $\fun{\div}{A \setminus \{0\}}{A \setminus \{0\}}$ é uma inversão de $(A,\times,1)$.
\end{definition}

Por simplicidade, denotamos uma álgebra por $(\bm A,\times)$ se queremos ressaltar a notação para a multiplicação de $A$.

\begin{proposition}
Seja $\bm V$ um espaço linear sobre um corpo $\bm C$. Então
	\begin{equation*}
	((\lin(\bm V),+,-,0,\circ,\Id),\cdot)
	\end{equation*}
é uma álgebra unitária associativa sobre $\bm C$.
\end{proposition}
\begin{proof}
Sabemos que $((\lin(\bm V),+,-,0),\cdot)$ é um espaço linear sobre $\bm C$. Para mostrar que é uma álgebra, devemos mostrar que a composição de funções $\circ$ é bilinear. Sejam $L,L',L'' \in \lin(\bm V)$ e $c \in C$. Para todo $v \in V$, segue que
	\begin{align*}
	((cL+L') \circ L'')(v) &= (cL+L')(L''(v)) \\
		&= cL(L''(v))+L'(L''(v)) \\
		&= cL \circ L''(v) + L' \circ L''(v) \\
		&= (cL \circ L'' + L' \circ L'')(v),
	\end{align*}
portanto $(cL+L') \circ L'' = cL \circ L'' + L' \circ L''$.

Para todo $v \in V$, segue da linearidade de $L$ que
	\begin{align*}
	(L \circ (cL'+L''))(v) &= L((cL'+L'')(v)) \\
		&= L(cL'(v)+L''(v)) \\
		&= cL(L'(v))+L(L''(v)) \\
		&= cL \circ L'(v)+L \circ L''(v) \\
		&= (cL \circ L' + L \circ L'')(v),
	\end{align*}
portanto $L \circ (cL'+L'') = cL \circ L' + L \circ L''$. A composição de funções é associativa, portanto a álgebra é associativa. Por fim, a função identidade $\Id$ é identidade da composição.
\end{proof}

\begin{proposition}
Sejam $(\bm A,\times)$ uma álgebra sobre um corpo $\bm C$ e $I$ um conjunto. Então $(A^I,\times)$, em que $\times\colon A^I \times A^I \to A^I$ é o produto entrada a entrada, é uma álgebra sobre $\bm C$. Se o produto de $A$ é associativo ou comutativo, então o produto de $A^I$ é, respectivamente, associativo ou comutativo, e é se $A$ é unitária, $(1)_{i \in I}$ é identidade do produto de $A^I$.
\end{proposition}
\begin{proof}
Sabemos que $A^I$ é um espaço linear sobre $\bm C$. Basta mostrar que $\times$ é um produto bilinear. Sejam $(a_i)_{i \in I},(a'_i)_{i \in I},(a''_i)_{i \in I} \in A^I$ e $c \in C$. Então
	\begin{align*}
	(c(a_i)_{i \in I} + (a'_i)_{i \in I}) \times (a''_i)_{i \in I} &= (ca_i + a'_i)_{i \in I} \times (a''_i)_{i \in I} \\
	&= ((ca_i + a'_i) \times a''_i)_{i \in I} \\
	&= (ca_i \times a''_i + a'_i \times a''_i)_{i \in I} \\
	&= c(a_i \times a''_i)_{i \in I} + (a'_i \times a''_i)_{i \in I} \\
	&= c(a_i)_{i \in I} \times (a''_i)_{i \in I} + (a'_i)_{i \in I} \times (a''_i)_{i \in I}.
	\end{align*}
A demonstração da linearidade na segunda entrada é análoga, e as demonstrações de associatividade e comutatividade e identidade são triviais.
\end{proof}

\subsection{Ação adjunta}

Como de costume, denotamos a multiplicação à esquerda por um elemento fixo $a \in A$ por
	\begin{align*}
	\func{a\times}{A}{A}{a'}{a \times a'}.
	\end{align*}
Pela bilinearidade da multiplicação, temos que $a\times \in \lin(\bm A)$ e, portanto está bem definida uma função que leva cada $a \in A$ em uma função linear $a\times$, como definimos a seguir.

\begin{definition}
Sejam $(\bm A,\times)$ uma álgebra sobre um corpo $\bm C$ e $a \in A$. A \emph{ação adjunta} de $\bm A$ é a função linear
	\begin{align*}
	\func{\adj}{A}{\lin(\bm A)}{a}{
		\adj_a :=
		\begin{aligned}[t]
		\func{a\times}{A}{A}{a'}{a \times a'}.
		\end{aligned}
	}
	\end{align*}
\end{definition}

\begin{exercise}
Seja $(\bm A,\times)$ uma álgebra sobre um corpo $\bm C$. A álgebra $\bm A$ é associativa se, e somente se, para todos $a,a' \in A$,
	\begin{equation*}
	\adj_{a \times a'} = \adj_a \circ \adj_{a'}.
	\end{equation*}
\end{exercise}
Explicitamente, a condição anterior é
	\begin{equation*}
	(a \times a') \times = (a \times) \circ (a' \times).
	\end{equation*}

A ação adjunta é uma representação quando a álgebra é associativa.

\subsection{Colchete comutador}

\begin{definition}
Seja $(\bm A,\times)$ uma álgebra sobre um corpo $\bm C$. O \emph{colchete comutador} de $(\bm A,\times)$ é a função
	\begin{align*}
	\func{\col{\var}{\var}}{A \times A}{A}{(a,a')}{a \times a' - a' \times a.}
	\end{align*}
\end{definition}

\begin{proposition}
Seja $(\bm A,\times)$ uma álgebra sobre um corpo $\bm C$. Então
	\begin{enumerate}
	\item $(\bm A,\col{\var}{\var})$ é uma álgebra alternativa sobre $\bm C$;
	\item O produto $\times$ é comutativo se, e somente se, $\col{\var}{\var} = 0$;
	\end{enumerate}
\end{proposition}
\begin{proof}
	\begin{enumerate}
	\item Primeiro, notemos que, para todos $a,a' \in A$,
		\begin{equation*}
		\col{a}{a'} = a \times a' - a' \times a = -(a' \times a - a \times a') = -\col{a'}{a}.
		\end{equation*}
	portanto $\col{\var}{\var}$ é antissimétrico. Sendo assim, para mostrar que $\col{\var}{\var}$ é bilinear, basta mostrar que ela é linear na primeira entrada. Para todos $a,a',a'' \in A$ e $c \in C$,
		\begin{align*}
		\col{ca+a'}{a''} &= (ca+a') \times a'' - a'' \times (ca+a') \\
			&= ca \times a'' + a' \times a'' - ca'' \times a - a'' \times a' \\
			&= ca \times a'' - ca'' \times a + a' \times a'' - a'' \times a' \\
			&= c\col{a}{a''} + \col{a'}{a''}.
		\end{align*}
	Por fim, para todo $a \in A$,
		\begin{equation*}
		\col{a}{a} = a \times a - a \times a = 0.
		\end{equation*}

	\item Suponhamos, primeiro, que $\times$ é comutativo. Então, para todos $a,a' \in A$,
		\begin{equation*}
		\col{a}{a'} = a \times a' - a' \times a = a \times a' - a \times a' = 0.
		\end{equation*}
Reciprocamente, suponhamos que $\col{\var}{\var}=0$. Então, para todos $a,a' \in A$,
		\begin{equation*}
		a \times a' = a \times a' + 0 = a \times a' + \col{a'}{a} = a \times a' + a' \times a - a \times a' = a' \times a.
		\end{equation*}
	\end{enumerate}
\end{proof}



\subsection{Derivação}

\begin{definition}
Seja $(\bm A,\times)$ uma álgebra sobre um corpo $\bm C$. Uma \emph{derivação} em $\bm A$ é uma função linear $\fun{D}{A}{A}$ tal que
	\begin{itemize}
	\item (Derivabilidade\footnote{A propriedade de derivabilidade é também chamada de `regra do produto' ou `regra de Leibniz'.}) Para todos $a,a' \in A$,
		\begin{equation*}
		D(a \times a') = D(a) \times a' + a \times D(a').
		\end{equation*}
	\end{itemize}
O conjunto dessas derivações é $\Der(\bm A)$.
\end{definition}

Note que a propriedade acima nem sempre é equivalente a
	\begin{equation*}
	D(a \times a') = a' \times D(a) + a \times D(a'),
	\end{equation*}
pois o produto $\times$ nem sempre é comutativo, mas sempre é equivalente a
	\begin{equation*}
	D(a \times a') = a \times D(a') + D(a) \times a',
	\end{equation*}
pois a soma $+$ é comutativa.

\begin{proposition}
Sejam $(\bm A,\times)$ uma álgebra sobre um corpo $\bm C$ e $\fun{D}{A}{A}$ uma derivação em $\bm A$.
	\begin{enumerate}
	\item (Regra do produto generalizada) Para todos $a_0,\ldots,a_{n-1} \in A$,
		\begin{equation*}
		D(a_0 \cdots a_{n-1}) = \sum_{i \in [n]} a_0 \cdots D(a_i) \cdots a_{n-1};
		\end{equation*}
	\item Se $\times$ é comutativa, então, para todos $a \in A$ e $n \in \N \setminus \{0\}$,
		\begin{equation*}
		D(a^n) = na^{n-1}D(a);
		\end{equation*}
	\item Se existe unidade $1 \in A$ do produto, então
		\begin{equation*}
		D(1)=0.
		\end{equation*}
	\item (Regra do produto de ordem superior) Para todos $a,a' \in A$ e $n \in \N$,
		\begin{equation*}
		D^n(aa') = \sum_{i \in [n+1]} \binom{n}{i} D^{n-i}(a)D^i(a').
		\end{equation*}
	\end{enumerate}
\end{proposition}

Uma derivação pode ser equivalentemente descrita usando o comutador de composição de funções.

\begin{exercise}
Sejam $\bm A$ uma álgebra sobre um corpo $\bm C$ e $\fun{D}{A}{A}$ uma função linear. São equivalentes
	\begin{enumerate}
	\item $D$ é uma derivação;
	\item Para todo $a \in A$, $(D(a))\times = \col{D}{a \times}$;
	\item Para todo $a \in A$, $\times(D(a)) = \col{D}{\times a}$.
	\end{enumerate}
\end{exercise}

\begin{proposition}
\label{prop:algebra.derivacao.subespaco}
Seja $\bm A$ uma álgebra sobre um corpo $\bm C$. O conjunto $\Der(\bm A)$ das derivações em $\bm A$ é um subespaço linear de $\lin(\bm A)$.
\end{proposition}
\begin{proof}
Para mostrar que $\Der(\bm A)$ é subespaço linear de $\lin(\bm A)$, mostraremos que $\Der(\bm A)$ é fechado pela adição e pela ação por escalar pontuais. Para todo $c \in C$ e todas derivações $D,D' \in \Der(\bm A)$, a função $cD+D'$ é linear, pois $D$ e $D'$ são lineares; resta mostrar que ela é uma derivação. Para todos $a,a' \in A$,
	\begin{align*}
	(cD+D')(aa') &= (cD)(aa') + D'(aa') \\
		&= cD(aa') + D'(aa') \\
		&= cD(a)a' + caD(a') + D'(a)a' + aD'(a') \\
		&= (cD(a)+D'(a))a' + a(cD(a')+D'(a')) \\
		&= (cD+D')(a)a' + a(cD+D')(a').
	\end{align*}
portanto $cD+D'$ é uma derivação.
\end{proof}



\section{Álgebra derivativa}

As álgebras que descreveremos nesta seção são conhecidas como `álgebras de Lie'. Elas são álgebras cujo produto é antissimétrico e satisfaz uma propriedade importante, conhecida por `identidade de Jacobi', mas que chamamos de derivação adjunta pois ela é equivalente a dizer que a adjunta de todo elemento é uma derivação da álgebra.

\begin{proposition}
Sejam $\bm A$ uma álgebra sobre um corpo $\bm C$ e $a \in A$. A função adjunta $\adj_a$ é uma derivação em $\bm A$ se, e somente se,
	\begin{itemize}
	\item (Derivação adjunta) Para todos $a',a'' \in A$,
		\begin{equation*}
		a \times (a' \times a'') = (a \times a') \times a'' + a' \times (a \times a'').
		\end{equation*}
	\end{itemize}
\end{proposition}

A demonstração é imediata. Como comentado, essa propriedade é conhecida às vezes como `identidade de Jacobi'. No entanto, a identidade mais conhecida como `identidade de Jacobi' é
	\begin{equation*}
	a \times (a' \times a'') + a' \times (a'' \times a) + a'' \times (a \times a') = 0,
	\end{equation*}
que é equivalente à anterior se o produto é antissimétrico. Na maioria das vezes em que se usa essa identidade o produto é de fato antissimétrico, o que torna as duas propriedades equivalentes.

\begin{definition}
Seja $\bm C$ um corpo. Uma \emph{álgebra derivativa} sobre $\bm C$ é uma álgebra
	\begin{equation*}
	\bm{\mathfrak{A}} := ((\mathfrak{A},+,-,0,\col{\var}{\var}),\cdot)
	\end{equation*}
sobre $\bm C$ cuja multiplicação $\fun{\col{\var}{\var}}{\mathfrak{A} \times \mathfrak{A}}{\mathfrak{A}}$ satisfaz
	\begin{itemize}
	\item (Alternatividade) Para todo $a \in \mathfrak{A}$, $\col{a}{a} = 0$;
%	\item (Derivação adjunta) Para todo $a \in \mathfrak{A}$, $\ad{a} := \adj_a$ é uma derivação em $\bm{\mathfrak{A}}$.
	\item (Derivação adjunta) Para todo $a,a',a'' \in \mathfrak{A}$,
		\begin{equation*}
		\col{a}{\col{a'}{a''}} = \col{\col{a}{a'}}{a''} + \col{a'}{\col{a}{a''}}.
		\end{equation*}
	\end{itemize}
A multiplicação $\col{\var}{\var}$ é o \emph{colchete de derivação} da álgebra. A ação adjunta de $\mathfrak{A}$ é denotada
	\begin{align*}
	\func{\ad{\var}}{\mathfrak{A}}{\lin(\bm{\mathfrak{A}})}{a}{
		\begin{aligned}[t]
		\func{\ad{a}}{\mathfrak{A}}{\mathfrak{A}}{a'}{\col{a}{a'}}.
		\end{aligned}
	}
	\end{align*}
\end{definition}

\begin{comment}

Como $\col{\var}{\var}$ é alternativa, é antissimétrica, portanto a bilinearidade é equivalente à linearidade na segunda entrada de $\col{\var}{\var}$. A alternância é equivalente ao produto de um elemento com ele mesmo ser $0$, o que é o mesmo que a derivação adjunta baseada em um elemento aplicada a esse elemento ser $0$.

As três propriedades de $\fun{\col{\var}{\var}}{A \times A}{A}$ são equivalentes a
	\begin{enumerate}
	\item (Linearidade na 2ª entrada) Para todos $a,a',a'' \in A$ e $c \in C$,
		\begin{equation*}
		\col{a}{ca'+a''} = c\col{a}{a'} + \col{a}{a''};
		\end{equation*}
	\item (Alternância) Para todo $a \in A$,
		\begin{equation*}
		\col{a}{a} = 0;
		\end{equation*}
	\item (Derivação adjunta) Para todo $a \in A$, $\adj_a$ é uma derivação: para todos $a',a'' \in A$,
		\begin{equation*}
		\col{a}{\col{a'}{a''}} = \col{\col{a}{a'}}{a''} + \col{a'}{\col{a}{a''}}.
		\end{equation*}
	\end{enumerate}

Nesse caso em que a função adjunta é sempre uma derivação, pode-se também denotar $\ad{a} := \adj_a$, de modo que as propriedades acima se reduzem a termos: para todo $a \in A$, $\ad{a}$ é uma derivação tal que $\ad{a}(a)=0$. A partir de agora, denotaremos a adjunta de $a \in A$ em álgebras de derivação adjunta como
	\begin{align*}
	\func{\ad{a}}{A}{A}{a'}{\col{a}{a'}}
	\end{align*}
e a \emph{representação adjunta} será a função
	\begin{align*}
	\func{\ad{\var}}{A}{\lin(\bm A)}{a}{
		\begin{aligned}[t]
		\func{\ad{a}}{A}{A}{a'}{\col{a}{a'}},
		\end{aligned}
	}
	\end{align*}
em que $\lin(\bm A) = \lin(A,A)$ é o espaço das funções lineares de $A$ para $A$.

\end{comment}

\begin{proposition}
Seja $(\bm A,\times)$ uma álgebra associativa sobre um corpo $\bm C$. A álgebra $(\bm A,\col{\var}{\var})$, em que $\col{\var}{\var}$ é o colchete comutador de $\times$, é uma álgebra derivativa sobre $\bm C$.
\end{proposition}
\begin{proof}
Já demonstramos que $(\bm A,\col{\var}{\var})$ é uma álgebra alternativa, resta mostrar a derivação adjunta. Para todos $a,a',a'' \in A$, segue da associatividade de $\times$ que
	\begin{align*}
	\col{a}{\col{a'}{a''}} &= a \times (a' \times a'' - a'' \times a') - (a' \times a'' - a'' \times a') \times a \\
		&= a \times (a' \times a'') - a \times (a'' \times a') - (a' \times a'') \times a + (a'' \times a') \times a \\
		&= (a \times a') \times a'' - (a \times a'') \times a' - a' \times (a'' \times a) + a'' \times (a' \times a) \\
		&= (a \times a') \times a'' - (a \times a'') \times a' - a' \times (a'' \times a) + a'' \times (a' \times a) \\
		&+ a' \times (a \times a'') - (a' \times a) \times a'' + (a'' \times a) \times a' - a'' \times (a \times a') \\
		&= (a \times a') \times a'' - (a' \times a) \times a'' - a'' \times (a \times a') + a'' \times (a' \times a) \\
		&+ a' \times (a \times a'') - a' \times (a'' \times a) - (a \times a'') \times a' + (a'' \times a) \times a' \\
		&= (a \times a' - a' \times a) \times a'' - a'' \times (a \times a' - a' \times a) \\
		&+ a' \times (a \times a'' - a'' \times a) - (a \times a'' - a'' \times a) \times a' \\
		&= \col{\col{a}{a'}}{a''} + \col{a'}{\col{a}{a''}}.
		\qedhere
	\end{align*}
\end{proof}

Isso mostra, em particular, que a álgebra de funções lineares em um espaço linear $\bm V$ com composição $((\lin(\bm V),+,-,0,\circ),\cdot)$ gera uma álgebra derivativa
	\begin{equation*}
	((\lin(\bm V),+,-,0,\col{\var}{\var}),\cdot),                                                                                                                                                                                                                  \end{equation*}
em que o colchete comutador da álgebra é dado, para todas $L,L' \in \lin(\bm V)$, por
	\begin{equation*}
	\col{L}{L'} = L \circ L' - L' \circ L.
	\end{equation*}

Para diferenciá-la da álgebra associativa de composição de funções lineares, a denotaremos como $\Linder(\bm V)$. A estrutura de espaço linear subjacente é a mesma para as duas álgebras, mas a multiplicação é diferente.

\begin{definition}
Seja $\bm V$ um espaço vetorial sobre um corpo $\bm C$. A \emph{álgebra derivativa linear} de $\bm V$ é a álgebra derivativa
	\begin{equation*}
	\bm{\Linder(\bm V)} := ((\Linder(\bm V),+,-,0,\col{\var}{\var}),\cdot)
	\end{equation*}
em que $\Linder(\bm V) := \lin(\bm V)$, $((\Linder(\bm V),+,-,0),\cdot)$ é o espaço linear de funções lineares e $\col{\var}{\var}$ é o colchete comutador da composição $\circ$ de funções.
\end{definition}

Consideremos, agora, o conjunto $\Der(\bm A)$ das derivações em uma álgebra $\bm A$. Por \ref{prop:algebra.derivacao.subespaco}, o espaço $\Der(\bm A)$ é um subespaço linear de $\lin(\bm A)$. No entanto, $\Der(\bm A)$ não é uma subálgebra de $\lin(\bm A)$ com o produto de composição de funções, pois, para todos $D,D' \in \Der(\bm A)$ e $a,a' \in A$,
	\begin{equation*}
	(D \circ D')(aa') = (D \circ D')(a)a' + a (D \circ D')(a') + D'(a)D(a') + D(a)D'(a')
	\end{equation*}
o que significa que $D \circ D'$ nem sempre é uma derivação, pois $D'(a) \times D(a') + D(a) \times D'(a')$ nem sempre é $0$. Pra resolver esse problema, devemos considerar em $\Der(\bm A)$ o colchete comutador da composição de funções%, de modo que $(\Der(\bm A),\col{\var}{\var})$ será uma subálgebra de $(\lin(\bm A),\col{\var}{\var})$

\begin{proposition}
\label{alge:prop.algebra.colchete.deriv}
Seja $\bm A$ uma álgebra sobre um corpo $\bm C$. A álgebra de derivações $(\Der(\bm A),\col{\var}{\var})$ é uma álgebra derivativa.% é uma subálgebra de $(\Linder(\bm A),\col{\var}{\var})$.
\end{proposition}
\begin{proof}
Por \ref{prop:algebra.derivacao.subespaco}, o espaço $((\Der(\bm A),+,-,0),\cdot)$ é um subespaço linear de $((\lin(\bm A),+,-,0),\cdot)$. Primeiro mostramos que $\Der(\bm A)$ é fechado pelo colchete comutador. Denotaremos o produto $a \times a'$ por $aa'$ para simplificar a notação. Para todas $D,D' \in \Der(\bm A)$ e $a,a' \in A$,
	\begin{align*}
		(D \circ D')(aa') &= D(D'(aa')) \\
			&= D(D'(a)a' + aD'(a')) \\
			&= D(D'(a)a') + D(aD'(a')) \\
			&= D(D'(a))a' + D'(a)D(a') + D(a)D'(a') + a D(D'(a')) \\
			&= (D \circ D')(a)a' + a (D \circ D')(a') + D'(a)D(a') + D(a)D'(a').
		\end{align*}
Trocando $D$ e $D'$, segue que
		\begin{equation*}
		(D' \circ D)(aa') = (D' \circ D)(a)a' + a (D' \circ D)(a') + D(a)D'(a') + D'(a)D(a'),
		\end{equation*}
Dessas duas expressões, segue que
		\begin{align*}
		\col{D}{D'}(aa') &= (D \circ D')(aa') - (D' \circ D)(aa') \\
			&= (D \circ D')(a)a' + a (D \circ D')(a') - (D' \circ D)(a)a' - a (D' \circ D)(a') \\
			&= \col{D}{D'}(a)a' + a\col{D}{D'}(a').
		\end{align*}
portanto $\col{D}{D'} \in \Der(\bm A)$.

%Já demonstramos que $(\bm A,\col{\var}{\var})$ é uma álgebra alternativa, resta mostrar a derivação adjunta.
\end{proof}

%	\begin{align*}
%	(DD')(aa') &= D(aa')D'(aa') \\
%		&= (D(a)a'+aD(a'))(D'(a)a'+aD'(a')) \\
%		&= D(a)a'D'(a)a' + D(a)a'aD'(a') + aD(a')D'(a)a' + aD(a')aD'(a'),
%	\end{align*}
%e
%	\begin{align*}
%	(DD')(a)a' + a(DD')(a') &= D(a)D'(a)a' + aD(a')D'(a').
%	\end{align*}
%Se notarmos que
%	\begin{align*}
%	(D'D)(aa') &= D'(aa')D(aa') \\
%		&= (D'(a)a'+aD'(a'))(D(a)a'+aD(a')) \\
%		&= D'(a)a'D(a)a' + D'(a)a'aD(a') + aD'(a')D(a)a' + aD'(a')aD(a').
%	\end{align*}
% EU ESTAVA TENTANDO CONSIDERAR O PRODUTO PONTUAL EM L(A,A), MAS ISSO É UM ERRO, O CORRETO É CONSIDERAR O PRODUTO COMPOSIÇÃO DE FUNÇÕES.
\begin{comment}
Agora, devemos mostrar que $\mathrm{Der}(A)$ é fechado por $\col{\var}{\var}$. Notaremos que $\mathrm{Der}(A)$ não é fechado pelo produto $\times$ induzido pontualmente na demonstração, e por isso precisamos do comutador. Além disso, ainda não usamos a associatividade do produto $\times$. Ela será usada agora. Sejam $D,D' \in \mathrm{Der}(A)$. Para todos $a,a' \in A$, primeiro notemos que, pela associatividade, segue que
	\begin{align*}
	(D \times D')(a)a' + a(D \times D')(a') &= (D(a)D'(a))a' + a(D(a')D'(a')) \\
		&= D(a)D'(a)a' + aD(a')D'(a').
	\end{align*}
Então,
	\begin{align*}
	(D \times D')(a,a') &= D(aa')D'(aa') \\
		&= (D(a)a' + aD(a'))(D'(a)a' + aD'(a')) \\
		&= (D(a)a')(D'(a)a') + (aD(a'))(D'(a)a') \\
			&\qquad + (D(a)a')(aD'(a')) + (aD(a'))(aD'(a')) \\
		&= (D(a)a'D'(a))a' + a(D(a')D'(a)a') \\
			&\qquad + (D(a)a')(aD'(a')) + (aD(a'))(aD'(a')) \\
		&= (D \times D')(a)a' -
	\end{align*}
e
	\begin{align*}
	(D' \times D)(a,a') &= D'(aa')D(aa') \\
		&= (D'(a)a' + aD'(a'))(D(a)a' + aD(a')) \\
		&= (D'(a)a')(D(a)a') + (aD'(a'))(D(a)a') \\
			&\qquad + (D'(a)a')(aD(a')) + (aD'(a'))(aD(a')) \\
		&= (D'(a)a'D(a))a' + a(D'(a')D(a)a') \\
			&\qquad + (D'(a)a')(aD(a')) + (aD'(a'))(aD(a'))
	\end{align*}
\end{comment}




\section{Álgebra involutiva}

Antes de definirmos uma álgebra involutiva, definimos o que é um anel (e um corpo) involutivos.

\begin{definition}
Uma \emph{anel involutivo} é um par $(\bm A,\invol{})$ tal que $\bm A$ é um anel e $\fun{\invol{}}{A}{A}$ é uma operação unária que satisfaz
	\begin{itemize}
	\item (Involutividade) Para todo $a \in A$, $\invol{(\invol{a})} = a$;
	\item (Aditividade) Para todos $a,a' \in A$, $\invol{(a+a')} = \invol{a} + \invol{(a')}$;
	\item (Anti-multiplicatividade) Para todos $a,a' \in A$, $\invol{(aa')} = \invol{(a')}\invol{a}$;
	\item (Identidade multiplicativa) $\invol{1} = 1$.
	\end{itemize}
\end{definition}

\begin{proposition}
Seja $(\bm A,*)$ um anel involutivo.
	\begin{enumerate}
	\item (Identidade aditiva) $\invol{0} = 0$;
	\item (Invertibilidade aditiva) Para todo $a \in A$, $\invol{(-a)} = -\invol{a}$;
	\item (Invertibilidade multiplicativa) Para todo $a \in A^{\div}$, $\invol{a} \in A^{\div}$ e $\invol{(a\inv)} = (\invol{a})\inv$.
	\end{enumerate}
\end{proposition}
\begin{proof}
	\begin{enumerate}
	\item Segue da aditividade que
		\begin{equation*}
		0 = \invol{0} - \invol{0} = \invol{0+0} - \invol{0} = \invol{0} + \invol{0} - \invol{0} = \invol{0}.
		\end{equation*}

	\item Para todo $a \in A$, segue da aditividade e do item anterior que
		\begin{align*}
		\invol{-a} &= \invol{-a} + 0 \\
			&= \invol{-a} + \invol{a} - \invol{a} \\
			&= \invol{(-a+a)} - \invol{a} \\
			&= \invol{0} - \invol{a} \\
			&= 0 - \invol{a} \\
			&= \invol{a}.
		\end{align*}

	\item Para todo $a \in A^{\div}$, segue da multiplicatividade e da identidade multiplicativa que
		\begin{align*}
		1 = \invol{1} = \invol{a a\inv} = \invol{a}\invol{(a\inv)}
		\end{align*}
	e
		\begin{align*}
		1 = \invol{1} = \invol{a\inv a} = \invol{(a\inv)}\invol{a}
		\end{align*}
	portanto $\invol{a} \in A^{\div}$ e $\invol{(a\inv)} = (\invol{a})\inv$.
%		\begin{align*}
%		\invol{(a\inv)} &= \invol{(a\inv)}1 \\
%			&= \invol{(a\inv)}\invol{a}(\invol{a})\inv \\
%			&= \invol{(a\inv a)}(\invol{a})\inv \\
%			&= \invol{1}(\invol{a})\inv \\
%			&= 1(\invol{a})\inv \\
%			&= (\invol{a})\inv.
%		\end{align*}
	\end{enumerate}
\end{proof}

\begin{notation}
Por causa da propriedade de invertibilidade multiplicativa, denota-se, para todo $a \in A^{\div}$,
	\begin{equation*}
	\invinvol{a} := \invol{(a\inv)}.
	\end{equation*}
\end{notation}


\begin{example}
	\begin{enumerate}
	\item Todo anel com a involução trivial dada pela identidade.
	\item Os números complexos com a conjugação complexa.
	\item O anel de matrizes reais com a transposta.
	\item O anel de matrizes reais com a transposta conjugada.
	\item A adjunta de uma transformação linear contínua em um espaço de produto interno completo.
	\end{enumerate}
\end{example}

\begin{definition}
Seja $(\bm C,\invol{})$ um corpo involutivo. Uma \emph{álgebra involutiva} sobre $\bm C$ é uma lista $((A,+,-,0,\times,1,\invol{}),\cdot)$ em que $((A,+,-,0,\times,1),\cdot)$ é um álgebra unitária sobre $\bm C$ e $\fun{\invol{}}{A}{A}$ é uma operação unária que satisfaz
	\begin{itemize}
	\item (Involutividade) Para todo $a \in A$, $\invol{(\invol{a})} = a$;
	\item (Aditividade) Para todos $a,a' \in A$, $\invol{(a+a')} = \invol{a} + \invol{(a')}$;
	\item (Anti-multiplicatividade) Para todos $a,a' \in A$, $\invol{(aa')} = \invol{(a')}\invol{a}$;
	\item (Identidade multiplicativa) $\invol{1} = 1$;
	\item (Compatibilidade) Para todos $c \in C$ e $a \in A$, $\invol{(ca)} = \invol{c}\invol{a}$.
	\end{itemize}
\end{definition}

\begin{definition}
Seja $\bm A$ uma álgebra involutiva sobre um corpo involutivo $\bm C$. A norma quadrática de $\bm A$ é a função
	\begin{align*}
	\func{\qua{\var}}{A}{A}{x}{\qua{x} := x\invol{x}}.
	\end{align*}
\end{definition}

\begin{proposition}
Seja $\bm A$ uma álgebra involutiva sobre um corpo involutivo $\bm C$.
	\begin{enumerate}
	\item Para todo $c \in C$ e $x \in A$, $\qua{cx} = \qua{c}\qua{x}$;
	\item Para todo $x \in A$, $ \qua{\invol{x}} = \invol{x}x$;
	\item Para todo $x \in A$, $ \invol{\qua{x}} = \qua{x}$;
	\item A função $B(x,y) := \qua{x+y}-\qua{x}-\qua{y}$ é satisfaz $B(x,y) = x\invol{y} + y\invol{x}$, é simétrica e comuta com adição e inversa da adição em cada entrada.% bi-aditiva.
	\end{enumerate}
\end{proposition}
\begin{proof}
	\begin{enumerate}
	\item Para todos $c \in C$ e $x \in A$, segue da bilinearidade da multiplicação e compatibilidade da involução que
		\begin{equation*}
		\qua{cx} = (cx)\invol{(cx)} = c\invol{c}x\invol{x} = \qua{c}\qua{x}.
		\end{equation*}
	\item Para todo $x \in A$,
		\begin{equation*}
		\qua{\invol{x}} = \invol{x}\invol{(\invol{x})} = \invol{x}x;
		\end{equation*}
	\item Para todo $x \in A$,
		\begin{equation*}
		\invol{\qua{x}} = \invol{(x\invol{x})} = \invol{(\invol{x})}\invol{x} = x\invol{x} = \qua{x};
		\end{equation*}
	\item Para todos $x,y \in A$,
		\begin{align*}
		B(x,y) &= \qua{x+y}-\qua{x}-\qua{y} \\
			&= (x+y)\invol{(x+y)} - x\invol{x} - y\invol{y} \\
			&= x(\invol{x}+\invol{y}) + y(\invol{x}+\invol{y}) - x\invol{x} - y\invol{y} \\
			&= x\invol{x} + x\invol{y} + y\invol{x} + y\invol{y} - x\invol{x} - y\invol{y} \\
			&= x\invol{y} + y\invol{x},
		\end{align*}
	portanto $B$ é simétrica. Além disso, para todos $x,x',y \in A$,
		\begin{align*}
		B(x,y) &= (x+x')\invol{y} + y\invol{(x+x')} \\
			&= x\invol{y} + x'\invol{y} + y\invol{x} + y\invol{(x')} \\
			&= x\invol{y} + y\invol{x} + x'\invol{y} + y\invol{(x')} \\
			&= B(x,y) + B(x',y)
		\end{align*}
		\begin{align*}
		B(-x,y) = (-x)\invol{y} + y\invol{(-x)} = -x\invol{y} - y\invol{x} = -B(x,y).
		\end{align*}
	portanto $B$ comuta com adição e inversa da adição em cada entrada.
%	e, para todos $c \in C$ e $x,x',y \in A$,
%		\begin{align*}
%		B(cx+x',y) &= (cx+x')\invol{y} + y\invol{(cx+x')} \\
%			&= cx\invol{y} + x'\invol{y} + \invol{c}y\invol{x} + y\invol{(x')} \\
%			&= cx\invol{y} + \invol{c}y\invol{x} + x'\invol{y} + y\invol{(x')} \\
%			&= cx\invol{y} + \invol{c}y\invol{x} + B(x',y) \\
%		\end{align*}
	\end{enumerate}
\end{proof}

\paragraph{Produto torcido}

Essa construção de uma álgebra produto é conhecida por `construção de Cayley-Dickson'.
Definamos, a partir de uma álgebra involutiva $\bm A$, uma álgebra involutiva na soma direta $A^2$.

\begin{definition}
Seja $\bm A$ uma álgebra involutiva sobre um corpo involutivo $\bm C$. O \emph{produto torcido} de $\bm A$ (por $\bm A$) é
	\begin{equation*}
	((A^2,+,-,0,\times,1,\invol{}),\cdot)
	\end{equation*}
em que
	\begin{itemize}
	\item $((A^2,+,-,0),\cdot)$ é o produto (soma direta) como espaço linear;
	\item $\times$ é a \emph{multiplicação torcida} definida por
%		\begin{align*}
%		\func{\times}{A^2 \times A^2}{A^2}{((a_{00},a_{01}), (a_{10},a_{11}))}{(a_{00}a_{10}-\invol{a_{11}}a_{01}, a_{11}a_{00} + a_{01}\invol{a_{10}})}
%		\end{align*}
		\begin{align*}
		\func{\times}{A^2 \times A^2}{A^2}{((x_0,x_1), (y_0,y_1))}{(x_0y_0-\invol{y_1}x_1, y_1x_0 + x_1\invol{y_0})}
		\end{align*}
	\item $\invol{}$ é a \emph{involução torcida} definida por
%		\begin{align*}
%		\func{\invol{}}{A^2}{A^2}{(a_0,a_1)}{(\invol{a_0},-a_1)}
%		\end{align*}
		\begin{align*}
		\func{\invol{}}{A^2}{A^2}{(x_0,x_1)}{(\invol{x_0},-x_1)}
		\end{align*}
	\item $1$ é a unidade definida por $1 := (1,0)$.
	\end{itemize}
\end{definition}

\begin{comment}

\begin{proposition}
Seja $\bm A$ uma álgebra involutiva sobre um corpo involutivo $\bm C$. O produto torcido de $\bm A$ é uma álgebra involutiva.
\end{proposition}
\begin{proof}
Basta mostrar as propriedades de multiplicação, involução e unidade, pois o produto de espaços lineares é espaço linear.
	\begin{itemize}
	\item (Bilinearidade) Para todos
	\end{itemize}
\end{proof}

Vale então que
	\begin{align*}
	\invol{(yx)} &= \invol{(x_0y_0-\invol{y_1}x_1, y_1x_0 + x_1\invol{y_0})} \\
		&= (\invol{(x_0y_0-\invol{y_1}x_1)}, -(y_1x_0 + x_1\invol{y_0})) \\
		&= (\invol{(x_0y_0)}-\invol{(\invol{y_1}x_1)}, -y_1x_0 - x_1\invol{y_0}) \\
		&= (\invol{y_0}\invol{x_0}-\invol{x_1}y_1, -x_1\invol{y_0} - y_1x_0) \\
		&= (\invol{y_0}\invol{x_0}-\invol{(-x_1)}(-y_1), (-x_1)\invol{y_0} + (-y_1)\invol{(\invol{x_0})}) \\
		&= (\invol{y_0},-y_1)(\invol{x_0},-x_1) \\
		&= \invol{y}\invol{x}.
	\end{align*}

Vale também que
	\begin{equation*}
	(x_0,0)(y_0,0) = (x_0y_0-\invol{0}0, 0x_0 + 0\invol{y_0}) = (x_0y_0,0)
	\end{equation*}
e
	\begin{equation*}
	(0,x_1)(0,y_1) = (00-\invol{y_1}x_1, y_10 + x_1\invol{0}) = -(\invol{y_1}x_1,0).
	\end{equation*}


	\begin{align*}
	\qua{x} &= x\invol{x} \\
		&= (x_0,x_1)(\invol{x_0},-x_1) \\
		&= (x_0\invol{x_0}-\invol{(-x_1)}x_1, (-x_1)x_0 + x_1\invol{(\invol{x_0})}) \\
		&= (x_0\invol{x_0}+\invol{x_1}x_1, (-x_1)x_0 + x_1x_0) \\
		&= (\qua{x_0} + \qua{\invol{x_1}},0)
	\end{align*}
e
	\begin{align*}
	\qua{\invol{x}} &= \invol{x}x \\
		&= (\invol{x_0},-x_1)(x_0,x_1) \\
		&= ((\invol{x_0})x_0-\invol{x_1}(-x_1), x_1(\invol{x_0}) + (-x_1)\invol{x_0}) \\
		&= (\invol{x_0}x_0+\invol{x_1}x_1, x_1\invol{x_0} - x_1\invol{x_0}) \\
		&= (\qua{\invol{x_0}} + \qua{\invol{x_1}},0)
	\end{align*}
%Se a álgebra é associativa (e começamos indutivamente do corpo), $\qua{xy} = \qua{x}\qua{y}$. Para provar isso, n
Notemos que
	\begin{align*}
	\qua{xy} &= (\qua{(x_0y_0-\invol{y_1}x_1)} + \qua{\invol{(y_1x_0 + x_1\invol{y_0})}},0) \\
		&= (\qua{x_0y_0-\invol{y_1}x_1} + \qua{\invol{x_0}\invol{y_1} + y_0\invol{x_1}},0) \\
		&= (\qua{x_0y_0}-\qua{\invol{y_1}x_1} + \qua{\invol{x_0}\invol{y_1}} + \qua{y_0\invol{x_1}},0) .
	\end{align*}
%	\begin{align*}
%	\qua{xy} &= (x_0y_0-\invol{y_1}x_1, y_1x_0 + x_1\invol{y_0})\invol{(x_0y_0-\invol{y_1}x_1, y_1x_0 + x_1\invol{y_0})} \\
%		&= (x_0y_0-\invol{y_1}x_1, y_1x_0 + x_1\invol{y_0})(\invol{(x_0y_0-\invol{y_1}x_1)}, -(y_1x_0 + x_1\invol{y_0})) \\
%		&= (x_0y_0-\invol{y_1}x_1, y_1x_0 + x_1\invol{y_0})(\invol{y_0}\invol{x_0}-\invol{x_1}y_1, -x_1\invol{y_0} - y_1x_0);
%	\end{align*}
%a primeira entrada do produto dá
%	\begin{align*}
%	(\qua{xy})_0 &= (x_0y_0-\invol{y_1}x_1)(\invol{y_0}\invol{x_0}-\invol{x_1}y_1)-\invol{(-x_1\invol{y_0} - y_1x_0)}(y_1x_0 + x_1\invol{y_0}) \\
%		&= (x_0y_0-\invol{y_1}x_1)(\invol{y_0}\invol{x_0}-\invol{x_1}y_1)-(-y_0\invol{x_1} - \invol{x_0}\invol{y_1})(y_1x_0 + x_1\invol{y_0}) \\
%		&= (x_0y_0-\invol{y_1}x_1)(\invol{y_0}\invol{x_0}-\invol{x_1}y_1)+(y_0\invol{x_1} + \invol{x_0}\invol{y_1})(y_1x_0 + x_1\invol{y_0}) \\
%		&= (x_0y_0)(\invol{y_0}\invol{x_0}) - (x_0y_0)(\invol{x_1}y_1) - (\invol{y_1}x_1)(\invol{y_0}\invol{x_0}) + (\invol{y_1}x_1)(\invol{x_1}y_1) \\
%		&+ (y_0\invol{x_1})(y_1x_0) + (y_0\invol{x_1})(x_1\invol{y_0}) +(\invol{x_0}\invol{y_1})(y_1x_0) + (\invol{x_0}\invol{y_1})(x_1\invol{y_0}) \\
%	\end{align*}
%e a segunda entrada do produto dá (não usa associatividade)
%	\begin{align*}
%	(\qua{xy})_1 &= (-x_1\invol{y_0} - y_1x_0)(x_0y_0-\invol{y_1}x_1) + (y_1x_0 + x_1\invol{y_0})\invol{(\invol{y_0}\invol{x_0}-\invol{x_1}y_1)} \\
%		&= -(x_1\invol{y_0} + y_1x_0)(x_0y_0-\invol{y_1}x_1) + (y_1x_0 + x_1\invol{y_0})(x_0y_0-\invol{y_1}x_1) \\
%		&= - (x_1\invol{y_0})(x_0y_0) + (x_1\invol{y_0})(\invol{y_1}x_1) - (y_1x_0)(x_0y_0) + (y_1x_0)(\invol{y_1}x_1) \\
%		&+ (y_1x_0)(x_0y_0) - (y_1x_0)(\invol{y_1}x_1) + (x_1\invol{y_0})(x_0y_0) - (x_1\invol{y_0})(\invol{y_1}x_1) \\
%		&= 0.
%	\end{align*}
Por outro lado,
	\begin{align*}
	\qua{x}\qua{y} &= (x\invol{x})(y\invol{y}) \\
		&= (\qua{x_0} + \qua{\invol{x_1}},0)(\qua{y_0} + \qua{\invol{y_1}},0) \\
		&= ((\qua{x_0} + \qua{\invol{x_1}})(\qua{y_0} + \qua{\invol{y_1}}),0) \\
		&= (\qua{x_0}\qua{y_0} + \qua{x_0}\qua{\invol{y_1}} + \qua{\invol{x_1}}\qua{y_0} + \qua{\invol{x_1}}\qua{\invol{y_1}},0) \\
	\end{align*}
%a primeira entrada do produto dá
%	\begin{align*}
%	(\qua{x}\qua{y})_0 &= (x_0x_0-\invol{x_1}x_1)(y_0y_0-\invol{y_1}y_1)-\invol{(y_1y_0 + y_1\invol{y_0})}(x_1x_0 + x_1\invol{x_0}) \\
%		&= (x_0x_0-\invol{x_1}x_1)(y_0y_0-\invol{y_1}y_1)-(\invol{y_0}\invol{y_1} + y_0\invol{y_1})(x_1x_0 + x_1\invol{x_0}) \\
%		&= (x_0x_0)(y_0y_0) - (x_0x_0)(\invol{y_1}y_1) - (\invol{x_1}x_1)(y_0y_0) + (\invol{x_1}x_1)(\invol{y_1}y_1) \\
%		&- (\invol{y_0}\invol{y_1})(x_1x_0) - (\invol{y_0}\invol{y_1})(x_1\invol{x_0}) - (y_0\invol{y_1})(x_1x_0) - (y_0\invol{y_1})(x_1\invol{x_0}) \\
%	\end{align*}
%e a segunda entrada do produto dá
%	\begin{align*}
%	(\qua{x}\qua{y})_1 &= (y_1y_0 + y_1\invol{y_0})(x_0x_0-\invol{x_1}x_1) + (x_1x_0 + x_1\invol{x_0})\invol{(y_0y_0-\invol{y_1}y_1)} \\
%		&= (y_1y_0 + y_1\invol{y_0})(x_0x_0-\invol{x_1}x_1) + (x_1x_0 + x_1\invol{x_0})(\invol{y_0}\invol{y_0}-\invol{y_1}y_1) \\
%		&= (y_1y_0)(x_0x_0) - (y_1y_0)(\invol{x_1}x_1) + (y_1\invol{y_0})(x_0x_0) - (y_1\invol{y_0})(\invol{x_1}x_1) \\
%		&+ (x_1x_0)(\invol{y_0}\invol{y_0}) - (x_1x_0)(\invol{y_1}y_1) + (x_1\invol{x_0})(\invol{y_0}\invol{y_0}) - (x_1\invol{x_0})(\invol{y_1}y_1) \\
%		&=
%	\end{align*}

\end{comment}




























\section{Álgebras reais}

Nesta seção, descreveremos algumas álgebras sobre o corpo dos números reais $\R$. A álgebra real $\R$ é simplesmente o corpo $\R$. Analisaremos com detalhes os casos das álgebras dos complexos $\R^2$ e dos quatérnios $\R^4$. Descreveremos brevemente os números $2$-hiperbólicos $\R^{1,1}$, às vezes chamados de `complexos hiperbólicos', e os números $4$-hiperbólicos $\R^{1,3}$.

\subsection{Complexos \texorpdfstring{$\R^2$}{}}

Consideremos o espaço linear $\R^2$ com a norma $2$
	\begin{align*}
	\func{\nor{\var}}{\R^2}{\R}{x}{((x_0)^2 + (x_1)^2)^{\frac{1}{2}}}.
	\end{align*}

A base canônica de $\R^2$ será denotada por
	\begin{align*}
	\ii_0 &:= (1,0) \\
	\ii_1 &:= (0,1) .
	\end{align*}

\begin{definition}
A \emph{multiplicação complexa} em $\R^2$ é a operação
	\begin{align*}
	\func{\times}{\R^2 \times \R^2}{\R^2}{(x,y)}{xy := (x_0y_0-x_1y_1,x_0y_1+x_1y_0)}.
	\end{align*}
A \emph{unidade} de $(\R^2,\times)$ é
	\begin{equation*}
	1_{\R^2} := \ii_0 = (1,0).
	\end{equation*}
A \emph{inversão} de $(\R^2,\times,1)$ é a operação
	\begin{align*}
	\func{\div}{\R^2 \setminus \{0\}}{\R^2 \setminus \{0\}}{x}{x\inv := \nor{x}^{-2}(x_0,-x_1)}.
	\end{align*}
\end{definition}

Em geral, por não haver ambiguidade, denotaremos $1_{\R^2}$ simplesmente por $1$ e $\ii_1$ simplesmente por $\ii$. Assim, todo $x \in \R^2$ pode ser denotado
	\begin{equation*}
	x = x_0 1 + x_1 \ii.
	\end{equation*}

Como a notação acima sugere, omitiremos o símbolo de multiplicação $\times$ sempre que possível pois isso não gerará ambiguidade.

Notemos que
	\begin{equation*}
	\ii^2 = (0,1)(0,1) = (0 \times 0 - 1 \times 1, 0 \times 1 + 1 \times 0) = (-1,0) = -1.
	\end{equation*}

\begin{proposition}
A lista
	\begin{equation*}
	(\R^2,+,-,0,\times,1,\cdot)
	\end{equation*}
é uma álgebra associativa, comutativa, unitária e invertível sobre $\R$.
\end{proposition}
\begin{proof}
Como $(\R^2,+,-,0,\cdot)$ é um espaço linear sobre $\R$, basta mostrar as propriedades de $\times$.
\begin{itemize}
	\item (Comutatividade) Para todos $x,y \in \R^2$,
		\begin{equation*}
		xy = (x_0y_0-x_1y_1,x_0y_1+x_1y_0) = (y_0x_0-y_1x_1,y_0x_1+y_1x_0) = yx.
		\end{equation*}
	\item (Bilinearidade) Como a multiplicação é comutativa, basta mostrar a linearidade na primeira entrada. Para todos $x,x',y \in \R^2$ e todo $c \in \R$,
		\begin{align*}
		(cx+x')y &= ((cx_0+x'_0)y_0-(cx_1+x'_1)y_1,(cx_0+x'_0)y_1+(cx_1+x'_1)y_0) \\
			&= (cx_0y_0+x'_0y_0-cx_1y_1-x'_1y_1,cx_0y_1+x'_0y_1+cx_1y_0+x'_1y_0) \\
			&= c(x_0y_0-x_1y_1,x_0y_1+x_1y_0) + (x'_0y_0-x'_1y_1,x'_0y_1x'_1y_0) \\
			&= c(xy) + x'y.
		\end{align*}
	\item %(Associatividade) Para todos $x,y,z \in \R^2$,
%		\begin{align*}
%		(xy)z &= (x_0y_0-x_1y_1,x_0y_1+x_1y_0)z \\
%			&= ((x_0y_0-x_1y_1)z_0 - (x_0y_1+x_1y_0)z_1 , (x_0y_0-x_1y_1)z_1 + (x_0y_1+x_1y_0)z_0 ) \\
%			&= (x_0(y_0z_0-y_1z_1)-x_1(y_0z_1+y_1z_0),x_0(y_0z_1+y_1z_0)+x_1(y_0z_0-y_1z_1)) \\
%			&= x(y_0z_0-y_1z_1,y_0z_1+y_1z_0) \\
%			&= x(yz).
%		\end{align*}
	(Associatividade) Faremos as contas por coordenadas. Para todos $x,y,z \in \R^2$, segue da definição do produto que as coordenadas de $xy$ são
		\begin{align*}
		(xy)_0 &= x_0y_0 - x_1y_1 \\
		(xy)_1 &= x_0y_1 + x_1y_0
		\end{align*}
	e que as coordenadas de $yz$ são
		\begin{align*}
		(yz)_0 &= y_0z_0 - y_1z_1 \\
		(yz)_1 &= y_0z_1 + y_1z_0.
		\end{align*}
	Ainda da definição do produto, as coordenadas de $(xy)z$ são
		\begin{align*}
		((xy)z)_0 &= (x_0y_0 - x_1y_1)z_0 - (x_0y_1 + x_1y_0)z_1 \\
		((xy)z)_1 &= (x_0y_0 - x_1y_1)z_1 + (x_0y_1 + x_1y_0)z_0
		\end{align*}
	e as coordenadas de $x(yz)$ são
		\begin{align*}
		(x(yz))_0 &= x_0(y_0z_0-y_1z_1) - x_1(y_0z_1+y_1z_0) \\
		(x(yz))_1 &= x_0(y_0z_1+y_1z_0) + x_1(y_0z_0-y_1z_1).
		\end{align*}
	Para a coordenada $0$, segue que
		\begin{align*}
		((xy)z)_0 &= (x_0y_0 - x_1y_1)z_0 - (x_0y_1 + x_1y_0)z_1 \\
			&= x_0(y_0z_0-y_1z_1) - x_1(y_0z_1+y_1z_0) \\
			&= (x(yz))_0
		\end{align*}
	e para a coordenada $1$ segue que
		\begin{align*}
		((xy)z)_1 &= (x_0y_0 - x_1y_1)z_1 + (x_0y_1 + x_1y_0)z_0 \\
			&= x_0(y_0z_1+y_1z_0) + x_1(y_0z_0-y_1z_1) \\
			&= (x(yz))_1.
		\end{align*}
	Portanto
		\begin{equation*}
		x(yz) = ((xy)z)_0 1 + ((xy)z)_1 \ii = (x(yz))_0 1 + (x(yz))_1 \ii = x(yz).
		\end{equation*}
	\item (Identidade) Para todo $x \in \R^2$,
		\begin{equation*}
		1x = (1x_0-0x_1,1x_1+0x_0) = (x_0,x_1) = x.
		\end{equation*}
	\item (Inversa) Para todo $x \in \R^2 \setminus \{0\}$,
		\begin{align*}
		x\inv x &= \nor{x}^{-2}(x_0,-x_1)(x_0,x_1) \\
			&= \nor{x}^{-2} (x_0x_0 - (-x_1)x_1,x_0(-x_1)+x_1x_0) \\
			&= \nor{x}^{-2} (\nor{x}^2,0) \\
			&= (1,0) \\
			&= 1.
			\qedhere
		\end{align*}
\end{itemize}
\end{proof}

Como o vetor $1=(1,0)$ é uma unidade, em geral o omitiremos da notação da base, denotando $x_0 1$ por $x_0$ e
	\begin{equation*}
	x = x_0 + x_1\ii,
	\end{equation*}
de modo que a multiplicação $\times$ nessa notação é a multiplicação usual dos números complexos
	\begin{equation*}
	xy = (x_0 + x_1 \ii)(y_0 + y_1 \ii) = (x_0y_0-x_1y_1) + (x_0y_1+x_1y_0) \ii.
	\end{equation*}

\begin{definition}
A \emph{conjugação} em $\R^2$ é a operação
	\begin{align*}
	\func{\conju{}}{\R^2}{\R^2}{x}{\conju{x} := (x_0,-x_1)}.
	\end{align*}
\end{definition}

\begin{exercise}
	\begin{enumerate}
	\item (Involutividade) Para todo $x \in \R^2$, $\conju{(\conju{x})} = x$;
	\item (Aditividade) Para todos $x,x' \in \R^2$, $\conju{(x+x')} = \conju{x} + \conju{(x')}$;
	\item (Identidade aditiva) $\conju{0} = 0$;
	\item (Invertibilidade aditiva) Para todo $x \in \R^2$, $\conju{(-x)} = -\conju{x}$;
	\item (Anti-multiplicatividade) Para todos $x,x' \in \R^2$, $\conju{(xx')} = \conju{(x')}\conju{x}$;
	\item (Identidade multiplicativa) $\conju{1} = 1$;
	\end{enumerate}
\end{exercise}

\begin{exercise}
Seja $x \in \R^2$.
	\begin{enumerate}
	\item $\nor{x}^2 = \abs{\esc{x}}^2 + \nor{\vec{x}}^2$;
	\item $\nor{x}^2 = x\conju{x}$;
%	\item $\conju{x} = \frac{1}{2}(1x(-1) + \ii_1 x (-\ii_1) + \ii_2 x (-\ii_2) + \ii_3 x (-\ii_3))$.
	\end{enumerate}
\end{exercise}


A multiplicação $\times$ pode também ser vista como uma ação de $\R^2$ aditivo sobre $\R^2$ aditivo. Cada $x \in \R^2$ pode ser identificado com uma transformação linear $x\colon \R^2 \to \R^2$. Os elementos $x \in \S^1 \subseteq \R^2$ são as rotações: se $x \in \S^1$, então existe $\theta \in \intfa{0}{\tau}$ tal que
	\begin{equation*}
	x = \cos(\theta) + \sin(\theta) \ii.
	\end{equation*}

Sendo assim, para todo $\cos(\theta) + \sin(\theta) \ii \in \S^1$, e todo $x \in \R^2$,
	\begin{align*}
	(\cos(\theta) + \sin(\theta)) \ii \times x &= (\cos(\theta)x_0-\sin(\theta)x_1) +(\cos(\theta)x_1+\sin(\theta)x_0) \ii = R_\theta(x).
	\end{align*}

Os elementos $x = x_0 \in \R \subseteq \R^2$ são as expansões e contrações. Os elementos de $\R^2$ podem ser decompostos como
	\begin{equation*}
	x = \nor{x}(\cos(\theta) + \sin(\theta) \ii).
	\end{equation*}

\begin{proposition}[Decomposição polar]
Seja $x \in \R^2 \setminus \{0\}$. Então
	\begin{equation*}
	x = \nor{x}(\cos(\theta) + \sin(\theta) \ii),
	\end{equation*}
em que $\theta \in \intfa{0}{\tau}$ é definido
	\begin{equation*}
	\theta :=	\begin{cases}
				\cos\inv\left(\frac{x_0}{\nor{x}}\right),			& x_1 > 0 \text{\ \ ou\ \ } x_0 = 1, \\
				\tau - \cos\inv\left(\frac{x_0}{\nor{x}}\right)		& x_1 < 0 \text{\ \ ou\ \ } x_0 = -1.
				\end{cases}
	\end{equation*}
\end{proposition}
\begin{proof}
Como $x \neq 0$, $\nor{x} \neq 0$, logo da igualdade $\nor{x}^2 = {x_0}^2 + {x_1}^2$ segue que
	\begin{equation*}
	1 = \left(\frac{x_0}{\nor{x}}\right)^2 + \left(\frac{x_1}{\nor{x}}\right)^2,
	\end{equation*}
portanto $\frac{\abs{x_0}}{\nor{x}} \leq 1$, o que é equivalente a $\frac{x_0}{\nor{x}} \in \intff{-1}{1}$. Disso segue que $\phi := \cos\inv\left(\frac{x_0}{\nor{x}}\right) \in \intff{0}{\tau \div 2}$, de modo que $\frac{\abs{x_1}}{\nor{x}} = (1-(\cos(\phi))^2)^{1 \div 2}$, o que implica que $\frac{\abs{x_1}}{\nor{x}} = \sin(\phi)$ ou $\frac{\abs{x_1}}{\nor{x}} = -\sin(\phi)$. Como $\phi \in \intff{0}{\tau \div 2}$, concluímos que $\sin(\phi) = \frac{\abs{x_1}}{\nor{x}}$.

Da definição de $\theta$ segue que $x_0 = \nor{x}\cos(\theta)$ e $x_1 = \nor{x}\sin(\theta)$. Para a demonstração disso, consideramos dois casos.
	\begin{itemize}
	\item ($x_1 > 0$ ou $x_0 = 1$) Nesse caso, $\abs{x_1}=x_1$ e $\theta = \phi$, portanto
		\begin{equation*}
		\cos(\theta) = \cos(\phi) = \frac{x_0}{\nor{x}}
		\end{equation*}
	e
		\begin{equation*}
		\sin(\theta) = \sin(\phi) = \frac{\abs{x_1}}{\nor{x}} = \frac{x_1}{\nor{x}}.
		\end{equation*}

	\item ($x_1 < 0$ ou $x_0 = -1$) Nesse caso, $\abs{x_1}=-x_1$ e $\theta = \tau - \phi$, portanto
		\begin{equation*}
		\cos(\theta) = \cos(\tau - \phi) = \cos(-\phi) = \cos (\phi)= \frac{x_0}{\nor{x}}
		\end{equation*}
	e
		\begin{equation*}
		\sin(\theta) = \sin(\tau - \phi) = \sin(-\phi) = -\sin(\phi) = -\frac{\abs{x_1}}{\nor{x}} = \frac{x_1}{\nor{x}}.
		\end{equation*}
	\end{itemize}

Assim, concluímos que
	\begin{equation*}
	x = x_0 + x_1 \ii = \nor{x}\cos(\theta) + \nor{x}\sin(\theta) \ii = \nor{x}(\cos(\theta) + \sin(\theta) \ii).
	\qedhere
	\end{equation*}
\end{proof}

\begin{comment}

\begin{figure}
\centering
\begin{tikzpicture}[scale=2]
	\draw (-1,0) node[anchor=north] {$-1$} -- (0,0) node[anchor=north] {$0$} -- (1,0) node[anchor=north] {$1$};
	\draw (0,0) -- (0,pi/2) node[anchor=west] {$\displaystyle\frac{\tau}{4}$} -- (0,pi) node[anchor=west] {$\displaystyle\frac{\tau}{2}$};
	\draw[dotted] (1,0) -- (1,pi) -- (-1,pi) -- (-1,0);
	\draw plot [domain=0:pi,smooth] ({cos(\x r)},\x);
\end{tikzpicture}
\caption{Gráfico da função $\cos\inv\colon \intff{-1}{1} \to \intff{0}{\frac{\tau}{2}}$.}
%\label{fig:cossenoinv}
\end{figure}

\begin{figure}
\centering
\begin{tikzpicture}[scale=2]
	\draw (-1,0) node[anchor=east] {$-1$} -- (0,0) node[anchor=east] {$0$} -- (1,0) node[anchor=west] {$1$};
	\draw (0,-pi/2) node[anchor=east] {$-\displaystyle\frac{\tau}{4}$} -- (0,pi/2) node[anchor=east] {$\displaystyle\frac{\tau}{4}$};
	\draw[dotted] (-1,-pi/2) rectangle (1,pi/2);
	\draw plot [domain=-pi/2:pi/2,smooth] ({sin(\x r)},\x);
\end{tikzpicture}
\caption{Gráfico da função $\sin\inv\colon \intff{-1}{1} \to \intff{-\frac{\tau}{4}}{\frac{\tau}{4}}$.}
\label{fig:senoinv}
\end{figure}


\begin{figure}
\centering
\begin{tikzpicture}[scale=1]
%	\draw (-1,0) node[anchor=north] {$-1$} -- (0,0) node[anchor=north] {$0$} -- (1,0) node[anchor=north] {$1$};
%	\draw (0,0) -- (0,pi/2) node[anchor=west] {$\displaystyle\frac{\tau}{4}$} -- (0,pi) node[anchor=west] {$\displaystyle\frac{\tau}{2}$};
%	\draw[dotted] (1,0) -- (1,pi) -- (-1,pi) -- (-1,0);
	\draw plot [domain=-1:1,smooth] (\x,{tan(\x r));
\end{tikzpicture}
\caption{Gráfico da função $\tan\inv\colon \R \to \intaa{\frac{\tau}{2}}{\frac{\tau}{2}}$.}
%\label{fig:cossenoinv}
\end{figure}

\end{comment}




\subsubsection{Rotações em \ensuremath{\R^2}}

\begin{proposition}
A função
	\begin{align*}
	\func{R}{\S^1}{\SO(2)}{u}{
		\begin{aligned}[t]
			\func{R_u}{\R^2}{\R^2}{x}{ux}
		\end{aligned}
	}
	\end{align*}
é um difeomorfismo e isomorfismo de grupos.
\end{proposition}
\begin{proof}
A função $R$ é diferenciável porque $R_u(x)$ é um polinômio em cada entrada. Primeiro, vamos mostrar que $R$ de fato está bem definida, ou seja, que, para todo $u \in \S^1$, $R_u \in \SO(2)$. Seja $u \in \S^1$. A função $R_u$ é linear, pois, para todos $x,x' \in \R^2$ e $c \in \R$, segue da bilinearidade da multiplicação complexa que
	\begin{equation*}
	R_u(cx+c') = u(cx+c') = cux + ux' = cR_u(x) + R_u(x').
	\end{equation*}
Isso mostra que $R_u \in \GL(2,\R)$. Além disso, $R_u$ preserva a norma, pois
	\begin{equation*}
	\nor{R_u(x)} = \nor{ux} = \nor{u}\nor{x} = \nor{x}.
	\end{equation*}
Isso mostra que $R_u \in \GO(2)$. Agora, como $R$ é diferenciável, em particular é contínua, logo, como $\S^1$ é conexo, $R(\S^1)$ é conexo. Isso implica que $R(\S^1) \subseteq \SO(2)$.

Seja $u \in \S^3$ tal que $R_u = \Id$. Então, para todo $x \in \R^2$, $ux = R_u(x) = x$. Em particular, $u = u1 = 1$, o que mostra que $R$ é injetiva.

Para a sobrejetividade, primeiro notamos que, como $R(\S^1) \subseteq \SO(2)$ é compacto, pois $\S^1$ é compacto, e $\SO(2)$ é separado\footnote{Hausdorff}, segue que $R(\S^1)$ é fechado. Além disso, $R$ é difeomorfismo local, pois o núcleo $\{1\}$ é um subgrupo discreto (portanto tem álgebra de Lie trivial), logo $R$ é difeomorfismo local na identidade, e portanto em todo $\S^1$, já que um morfismo de grupos diferenciáveis\footnote{Grupos de lie} tem posto constante. Assim, segue que $R(\S^1) \subseteq \SO(2)$ é aberto. Como $\SO(2)$ é conexo, segue que $R(\S^1) = \SO(2)$ e a inversa de $R$ é diferenciável.
\end{proof}

Para calcular exatamente qual é a rotação que $R_u$ representa em $\SO(2)$, lembremos que, pela decomposição polar de $\R^2$, todo $u \in \S^1$ pode ser escrito como
	\begin{equation*}
	\e^{\theta\ii} = \cos(\theta) + \sin(\theta)\ii,
	\end{equation*}
em que $\theta \in \intff{0}{\tau}$.

A rotação de $x \in \R^2$ por um ângulo $\theta \in \intfa{0}{\tau}$ (em torno da origem) é dada por
	\begin{equation*}
	R^\theta = (\cos(\theta)x_0 - \sin(\theta)x_1) + (\sin(\theta)x_0 + \cos(\theta)x_1) \ii.
	\end{equation*}

Assim segue que, para todo $x \in \R^2$,
	\begin{align*}
	R_{\e^{\theta \ii}}(x) &= (\cos(\theta) + \sin(\theta)\ii)(x_0 + x_1 \ii) \\
		&= (\cos(\theta)x_0 - \sin(\theta)x_1) + (\sin(\theta)x_0 + \cos(\theta)x_1) \ii \\
		&= R^\theta(x),
	\end{align*}
portanto $R(\e^{\theta \ii}) = R^{\theta}$.

Isso mostra que
	\begin{align*}
	\func{R}{\S^1}{\SO(2)}{\e^{\theta \ii}}{
		\begin{aligned}[t]
			\func{R^{\theta}}{\R^2}{\R^2}{x}{\e^{\theta \ii} x}.
		\end{aligned}
	}
	\end{align*}



\subsection{Quatérnios \texorpdfstring{$\R^4$}{}}

\subsubsection{Multiplicação quaterniônica}

Consideremos o espaço linear $\R^4$ com a norma $2$
	\begin{align*}
	\func{\nor{\var}}{\R^4}{\R}{x}{({x_0}^2+{x_1}^2+{x_2}^2+{x_3}^2)^{\frac{1}{2}}}.
	\end{align*}

A base canônica de $\R^4$ será denotada por
	\begin{align*}
	\ii_0 &:= (1,0,0,0) \\
	\ii_1 &:= (0,1,0,0) \\
	\ii_2 &:= (0,0,1,0) \\
	\ii_3 &:= (0,0,0,1).
	\end{align*}

\begin{definition}
A \emph{multiplicação quaterniônica} em $\R^4$ é a operação
%	\begin{align*}
%	\func{\times}{\R^4 \times \R^4}{\R^4}{(x,y)}{
%		\begin{aligned}[t]
%		xy := &(x_0y_0 - x_1y_1 - x_2y_2 - x_3y_3) 1 \\
%			+ &(x_0y_1 + x_1y_0 + x_2y_3 - x_3y_2) \ii_1 \\
%			+ &(x_0y_2 - x_1y_3 + x_2y_0 + x_3y_1) \ii_2 \\
%			+ &(x_0y_3 + x_1y_2 - x_2y_1 + x_3y_0) \ii_3.
%		\end{aligned}
%	}
%	\end{align*}
	\begin{align*}
	\func{\times}{\R^4 \times \R^4}{\R^4}{(x,y)}{
		\begin{aligned}[t]
		xy := (&x_0y_0 - x_1y_1 - x_2y_2 - x_3y_3, \\
			&x_0y_1 + x_1y_0 + x_2y_3 - x_3y_2, \\
			&x_0y_2 - x_1y_3 + x_2y_0 + x_3y_1, \\
			&x_0y_3 + x_1y_2 - x_2y_1 + x_3y_0).
		\end{aligned}
	}
	\end{align*}
A \emph{unidade} de $(\R^4,\times)$ é
	\begin{equation*}
	1_{\R^4} := \ii_0 = (1,0,0,0).
	\end{equation*}
A \emph{inversão} de $(\R^4,\times,1)$ é a operação
%	\begin{align*}
%	\func{\div}{\R^4 \setminus \{0\}}{\R^4 \setminus \{0\}}{x}{x\inv := \frac{x_0 1 - x_1 \ii_1 - x_2 \ii_2 - x_3 \ii_3}{\nor{x}^2}}.
%	\end{align*}
	\begin{align*}
	\func{\div}{\R^4 \setminus \{0\}}{\R^4 \setminus \{0\}}{x}{x\inv := \nor{x}^{-2} (x_0, -x_1, -x_2, -x_3)}.
	\end{align*}
\end{definition}

Em geral, por não haver ambiguidade, denotaremos $1_{\R^4}$ simplesmente por $1$. Assim, todo $x \in \R^4$ pode ser escrito
	\begin{equation*}
	x = x_0 1 + x_1\ii_1 + x_2\ii_2 + x_3\ii_3.
	\end{equation*}
Como a notação acima sugere, omitiremos o símbolo de multiplicação $\times$ sempre que possível pois isso não gerará ambiguidade.% Omitiremos também a unidade $1$ dos produtos, de modo que todo $x \in \R^4$ pode ser denotado
%	\begin{equation*}
%	x = x_0 + x_1\ii_1 + x_2\ii_2 + x_3\ii_3.
%	\end{equation*}

%Notemos que
%	\begin{align*}
%	{\ii_1}^2 &= (0,1,0,0)(0,1,0,0) = -1, \\
%	{\ii_2}^2 &= (0,0,1,0)(0,0,1,0) = -1, \\
%	{\ii_3}^2 &= (0,0,0,1)(0,0,0,1) = -1.
%	\end{align*}

%Pode-se mostrar que $\times$ é um produto bilinear associativo (mas não comutativo), que $1$ é identidade de $(\R^4,\times)$ e que $\div$ é inversa de $(\R^4,\times,1)$, e que além disso todas operações são preservadas pela norma. As contas são longas, mas diretas, portanto não as apresentaremos aqui. Alguns resultados, como a de que $x\inv$ é a operação inversa, serão provados mais à frente usando a notação de componente escalar, vetorial e conjugado de um quatérnio que simplificará as contas. Ressaltamos esse fato na seguinte proposição.% Isso significa que


\begin{table}
	\centering

	\begin{tabular}{c | c c c c}
	\toprule
	$\times$&	$1$		&	$\ii_1$	&	$\ii_2$	&	$\ii_3$	\\
	\hline
	$1$		&	$1$		&	$\ii_1$	&	$\ii_2$	&	$\ii_3$		\\
	$\ii_1$	&	$\ii_1$	&	$-1$	&	$\ii_3$	&	$-\ii_2$	\\
	$\ii_2$	&	$\ii_2$	&	$-\ii_3$&	$-1$	&	$\ii_1$		\\
	$\ii_3$	&	$\ii_3$	&	$\ii_2$	&	$-\ii_1$&	$-1$		\\
	\bottomrule
	\end{tabular}

	\caption{Tabela de multiplicação dos quatérnios.}
	\label{tab:multiplicacao.quaternionica}
\end{table}

\begin{figure}
	\centering

	\begin{tikzpicture}

	\def \raio{1.5cm}

	\coordinate (A) at (-90:\raio);
	\coordinate (B) at (30:\raio);
	\coordinate (C) at (150:\raio);

	\foreach \n in {0,1,2,3}
		{
			\draw[->] (-90+120*\n:\raio) arc (-90+120*\n:-29+120*\n:\raio);
			\draw (-30+120*\n:\raio) arc (-30+120*\n:30+120*\n:\raio);
		}

	\foreach \P/\n in {A/1, B/2, C/3}
		\node[draw,circle,fill=amarelo,inner sep=1.5pt] at (\P) {$\ii_\n$};

	\end{tikzpicture}

	\caption{Grafo para memorizar multiplicação quaterniônica.}

\end{figure}

\begin{proposition}
A lista
	\begin{equation*}
	((\R^4,+,-,0,\cdot,\times,\div,1),\nor{\var})
	\end{equation*}
é uma álgebra\footnote{A álgebra de $\R^4$ com essa multiplicação costuma ser denotada por $\mathbb{H}$, mas manteremos a notação $\R^4$ por simplicidade de notação, pois nenhuma ambiguidade resultará disso. A letra `H' é usada em homenagem ao matemático irlandês William Rowan Hamilton (1805 -- 1865).} (associativa) invertível normada sobre $\R$.
\begin{enumerate}
	\item (Bilinearidade) A multiplicação quaterniônica $\fun{\times}{\R^4 \times \R^4}{\R^4}$ é uma função bilinear;
	\item (Associatividade) Para todos $x,y,z \in \R^4$,
		\begin{equation*}
		(xy)z = x(yz);
		\end{equation*}
	\item (Identidade) Para todo $x \in \R^4$,
		\begin{equation*}
		x1 = 1x = x;
		\end{equation*}
	\item (Inversa) Para todo $x \in \R^4$,
		\begin{equation*}
		x\inv x = 1 = xx\inv;
		\end{equation*}
	\item (Submultiplicatividade) Para todos $x,y \in \R^4$,
		\begin{equation*}
		\nor{xx'} \leq \nor{x}\nor{x'}.
		\end{equation*}
	\item (Unitariedade) $\nor{1}=1$;
	\item (Invertibilidade) Para todo $x \in \R^4$,
		\begin{equation*}
		\nor{x\inv} = \nor{x}\inv.
		\end{equation*}
\end{enumerate}
\end{proposition}
\begingroup
\allowdisplaybreaks
\begin{proof}
Como $(\R^4,+,-,0,\cdot)$ é um espaço linear sobre $\R$, basta mostrar as propriedades de $\times$.
\begin{itemize}
	\item (Identidade) Para todo $x \in \R^4$,
		\begin{align*}
		1x &= (1,0,0,0)(x_0,x_1,x_2,x_3) \\
			&= (1 x_0 - 0 x_1 - 0 x_2 - 0 x_3)1 \\
			&+ (1 x_1 + 0 x_0 + 0 x_3 - 0 x_2)\ii_1 \\
			&+ (1 x_2 - 0 x_3 + 0 x_0 + 0 x_1)\ii_2 \\
			&+ (1 x_3 + 0 x_2 - 0 x_1 + 0 x_0)\ii_3 \\
			&= (x_0,x_1,x_2,x_3) \\
			&= x
		\end{align*}
	e
		\begin{align*}
		x1 &= (x_0,x_1,x_2,x_3)(1,0,0,0) \\
			&= (x_0 1 - x_1 0 - x_2 0 - x_3 0)1 \\
			&+ (x_0 0 + x_1 1 + x_2 0 - x_3 0)\ii_1 \\
			&+ (x_0 0 - x_1 0 + x_2 1 + x_3 0)\ii_2 \\
			&+ (x_0 0 + x_1 0 - x_2 0 + x_3 1)\ii_3 \\
			&= (x_0,x_1,x_2,x_3) \\
			&= x.
		\end{align*}

	\item (Bilinearidade) Para mostrar que o produto é bilinear, primeiro mostremos quais são seus valores na base $\{1,\ii_2,\ii_2,\ii_3\}$. Como $1$ é identidade do produto, basta calcular os valores do produto $\ii_1$, $\ii_2$ e $\ii_3$. Cálculos simples mostram que
		\begin{align*}
		\ii_1 \ii_1 &= -1 &\ii_1 \ii_2 = \ii_3 = -\ii_2 \ii_1 \\
		\ii_2 \ii_2 &= -1 &\ii_2 \ii_3 = \ii_1 = -\ii_3 \ii_2 \\
		\ii_3 \ii_3 &= -1 &\ii_3 \ii_1 = \ii_2 = -\ii_1 \ii_3
		\end{align*}
	e, sendo assim, basta mostrar que uma função bilinear $\fun{b}{\R^4 \times \R^4}{\R^4}$ com esses mesmos valores na base canônica coincide com o produto quaterniônico definido. Da bilinearidade segue que
		\begin{align*}
		b(x,y) &= b(x_0 1 + x_1 \ii_1 + x_2 \ii_2 + x_3 \ii_3, y_0 1 + y_1 \ii_1 + y_2 \ii_2 + y_3 \ii_3) \\
			&= x_0 b(1, y_0 1 + y_1 \ii_1 + y_2 \ii_2 + y_3 \ii_3) \\
			&+ x_1 b(\ii_1, y_0 1 + y_1 \ii_1 + y_2 \ii_2 + y_3 \ii_3) \\
			&+ x_2 b(\ii_2, y_0 1 + y_1 \ii_1 + y_2 \ii_2 + y_3 \ii_3) \\
			&+ x_3 b(\ii_3, y_0 1 + y_1 \ii_1 + y_2 \ii_2 + y_3 \ii_3) \\
			&= x_0 (y_0 b(1,1) + y_1 b(1,\ii_1) + y_2 b(1,\ii_2) + y_3 b(1,\ii_3)) \\
			&+ x_1 (y_0 b(\ii_1,1) + y_1 b(\ii_1,\ii_1) + y_2 b(\ii_1,\ii_2) + y_3 b(\ii_1,\ii_3)) \\
			&+ x_2 (y_0 b(\ii_2,1) + y_1 b(\ii_2,\ii_1) + y_2 b(\ii_2,\ii_2) + y_3 b(\ii_2,\ii_3)) \\
			&+ x_3 (y_0 b(\ii_3,1) + y_1 b(\ii_3,\ii_1) + y_2 b(\ii_3,\ii_2) + y_3 b(\ii_3,\ii_3)) \\
			&= x_0 (y_0 1 + y_1 \ii_1 + y_2 \ii_2 + y_3 \ii_3) \\
			&+ x_1 (y_0 \ii_1 + y_1 (-1) + y_2 \ii_3 + y_3 (-\ii_2)) \\
			&+ x_2 (y_0 \ii_2 + y_1 (-\ii_3) + y_2 (-1) + y_3 \ii_1) \\
			&+ x_3 (y_0 \ii_3 + y_1 \ii_2 + y_2 (-\ii_1) + y_3 (-1)) \\
			&= (x_0y_0 - x_1y_1 - x_2y_2 - x_3y_3) 1 \\
			&+ (x_0y_1 + x_1y_0 + x_2y_3 - x_3y_2) \ii_1 \\
			&+ (x_0y_2 - x_1y_3 + x_2y_0 + x_3y_1) \ii_2 \\
			&+ (x_0y_3 + x_1y_2 - x_2y_1 + x_3y_0) \ii_3 \\
			&= xy.
		\end{align*}

	\item (Associatividade) Faremos as contas por coordenadas. Para todos $x,y,z \in \R^4$, segue da definição do produto que as coordenadas de $xy$ são
		\begin{align*}
		(xy)_0 &= x_0y_0 - x_1y_1 - x_2y_2 - x_3y_3 \\
		(xy)_1 &= x_0y_1 + x_1y_0 + x_2y_3 - x_3y_2 \\
		(xy)_2 &= x_0y_2 - x_1y_3 + x_2y_0 + x_3y_1 \\
		(xy)_3 &= x_0y_3 + x_1y_2 - x_2y_1 + x_3y_0
		\end{align*}
	e que as coordenadas de $yz$ são
		\begin{align*}
		(yz)_0 &= y_0z_0 - y_1z_1 - y_2z_2 - y_3z_3 \\
		(yz)_1 &= y_0z_1 + y_1z_0 + y_2z_3 - y_3z_2 \\
		(yz)_2 &= y_0z_2 - y_1z_3 + y_2z_0 + y_3z_1 \\
		(yz)_3 &= y_0z_3 + y_1z_2 - y_2z_1 + y_3z_0.
		\end{align*}
	Ainda da definição do produto, as coordenadas de $(xy)z$ são
		\begin{align*}
		((xy)z)_0 &= (xy)_0z_0 - (xy)_1z_1 - (xy)_2z_2 - (xy)_3z_3 \\
		((xy)z)_1 &= (xy)_0z_1 + (xy)_1z_0 + (xy)_2z_3 - (xy)_3z_2 \\
		((xy)z)_2 &= (xy)_0z_2 - (xy)_1z_3 + (xy)_2z_0 + (xy)_3z_1 \\
		((xy)z)_3 &= (xy)_0z_3 + (xy)_1z_2 - (xy)_2z_1 + (xy)_3z_0 \\
		\end{align*}
	e as coordenadas de $x(yz)$ são
		\begin{align*}
		(x(yz))_0 &= x_0(yz)_0 - x_1(yz)_1 - x_2(yz)_2 - x_3(yz)_3 \\
		(x(yz))_1 &= x_0(yz)_1 + x_1(yz)_0 + x_2(yz)_3 - x_3(yz)_2 \\
		(x(yz))_2 &= x_0(yz)_2 - x_1(yz)_3 + x_2(yz)_0 + x_3(yz)_1 \\
		(x(yz))_3 &= x_0(yz)_3 + x_1(yz)_2 - x_2(yz)_1 + x_3(yz)_0.
		\end{align*}
	Para a coordenada $0$, segue que
		\begin{align*}
		((xy)z)_0 &= (xy)_0z_0 - (xy)_1z_1 - (xy)_2z_2 - (xy)_3z_3 \\
			&= (x_0y_0 - x_1y_1 - x_2y_2 - x_3y_3)z_0 \\
			&+ (x_0y_1 + x_1y_0 + x_2y_3 - x_3y_2)z_1 \\
			&+ (x_0y_2 - x_1y_3 + x_2y_0 + x_3y_1)z_2 \\
			&+ (x_0y_3 + x_1y_2 - x_2y_1 + x_3y_0)z_3 \\
			&= x_0(y_0z_0 - y_1z_1 - y_2z_2 - y_3z_3) \\
			&+ x_1(y_0z_1 + y_1z_0 + y_2z_3 - y_3z_2) \\
			&+ x_2(y_0z_2 - y_1z_3 + y_2z_0 + y_3z_1) \\
			&+ x_3(y_0z_3 + y_1z_2 - y_2z_1 + y_3z_0) \\
			&= x_0(yz)_0 + x_1(yz)_1 + x_2(yz)_2 + x_3(yz)_3 \\
			&= (x(yz))_0;
		\end{align*}
	para a coordenada $1$, segue que
		\begin{align*}
		((xy)z)_1 &= (xy)_0z_1 + (xy)_1z_0 + (xy)_2z_3 - (xy)_3z_2 \\
			&= (x_0y_0 - x_1y_1 - x_2y_2 - x_3y_3)z_1 \\
			&+ (x_0y_1 + x_1y_0 + x_2y_3 - x_3y_2)z_0 \\
			&+ (x_0y_2 - x_1y_3 + x_2y_0 + x_3y_1)z_3 \\
			&- (x_0y_3 + x_1y_2 - x_2y_1 + x_3y_0)z_2 \\
			&= x_0(y_0z_1 + y_1z_0 + y_2z_3 - y_3z_2) \\
			&+ x_1(y_0z_0 - y_1z_1 - y_2z_2 - y_3z_3) \\
			&+ x_2(y_0z_3 + y_1z_2 - y_2z_1 + y_3z_0) \\
			&- x_3(y_0z_2 - y_1z_3 + y_2z_0 + y_3z_1) \\
			&= x_0(yz)_1 + x_1(yz)_0 + x_2(yz)_3 - x_3(yz)_2 \\
			&= (x(yz))_1;
		\end{align*}
	para a coordenada $2$, segue que
		\begin{align*}
		((xy)z)_2 &= (xy)_0z_2 - (xy)_1z_3 + (xy)_2z_0 + (xy)_3z_1 \\
			&= (x_0y_0 - x_1y_1 - x_2y_2 - x_3y_3)z_2 \\
			&- (x_0y_1 + x_1y_0 + x_2y_3 - x_3y_2)z_3 \\
			&+ (x_0y_2 - x_1y_3 + x_2y_0 + x_3y_1)z_0 \\
			&+ (x_0y_3 + x_1y_2 - x_2y_1 + x_3y_0)z_1 \\
			&= x_0(y_0z_2 - y_1z_3 + y_2z_0 + y_3z_1) \\
			&- x_1(y_0z_3 + y_1z_2 - y_2z_1 + y_3z_0) \\
			&+ x_2(y_0z_0 - y_1z_1 - y_2z_2 - y_3z_3) \\
			&+ x_3(y_0z_1 + y_1z_0 + y_2z_3 - y_3z_2) \\
			&= x_0(yz)_2 - x_1(yz)_3 + x_2(yz)_0 + x_3(yz)_1 \\
			&= (x(yz))_2;
		\end{align*}
	e, para a coordenada $3$, segue que
		\begin{align*}
		((xy)z)_3 &= (xy)_0z_3 + (xy)_1z_2 - (xy)_2z_1 + (xy)_3z_0 \\
			&= (x_0y_0 - x_1y_1 - x_2y_2 - x_3y_3)z_3 \\
			&+ (x_0y_1 + x_1y_0 + x_2y_3 - x_3y_2)z_2 \\
			&- (x_0y_2 - x_1y_3 + x_2y_0 + x_3y_1)z_1 \\
			&+ (x_0y_3 + x_1y_2 - x_2y_1 + x_3y_0)z_0 \\
			&= x_0(y_0z_3 + y_1z_2 - y_2z_1 + y_3z_0) \\
			&+ x_1(y_0z_2 - y_1z_3 + y_2z_0 + y_3z_1) \\
			&- x_2(y_0z_1 + y_1z_0 + y_2z_3 - y_3z_2) \\
			&+ x_3(y_0z_0 - y_1z_1 - y_2z_2 - y_3z_3) \\
			&= x_0(yz)_3 + x_1(yz)_2 - x_2(yz)_1 + x_3(yz)_0 \\
			&= (x(yz))_3.
		\end{align*}
	Portanto
		\begin{align*}
		((xy)z) &= ((xy)z)_0 1 + ((xy)z)_1 \ii_1 + ((xy)z)_2 \ii_2 + ((xy)z)_3 \ii_3 \\
			&= (x(yz))_0 1 + (x(yz))_1 \ii_1 + (x(yz))_2 \ii_2 + (x(yz))_3 \ii_3 \\
			&= (x(yz)).
		\end{align*}

	\item (Inversa) Para todo $x \in \R^4 \setminus \{0\}$,
		\begin{align*}
		x\inv x &= \nor{x}^{-2} (x_0,-x_1,-x_2,-x_3)(x_0,x_1,x_2,x_3) \\
			&= \nor{x}^{-2} (x_0x_0 - (-x_1)x_1 - (-x_2)x_2 - (-x_3)x_3) 1 \\
			&+ \nor{x}^{-2} (x_0x_1 + (-x_1)x_0 + (-x_2)x_3 - (-x_3)x_2) \ii_1 \\
			&+ \nor{x}^{-2} (x_0x_2 - (-x_1)x_3 + (-x_2)x_0 + (-x_3)x_1) \ii_2 \\
			&+ \nor{x}^{-2} (x_0x_3 + (-x_1)x_2 - (-x_2)x_1 + (-x_3)x_0) \ii_3 \\
			&= \nor{x}^{-2} ({x_0}^2 + {x_1}^2 + {x_2}^2 + {x_3}^2) 1 \\
			&= \nor{x}^{-2} \nor{x}^2 1 \\
			&= 1
		\end{align*}
	e
		\begin{align*}
		x x\inv &= (x_0,x_1,x_2,x_3) \nor{x}^{-2}(x_0,-x_1,-x_2,-x_3) \\
			&= \nor{x}^{-2} (x_0x_0 - x_1(-x_1) - x_2(-x_2) - x_3(-x_3)) 1 \\
			&+ \nor{x}^{-2} (x_0(-x_1) + x_1x_0 + x_2(-x_3) - x_3(-x_2)) \ii_1 \\
			&+ \nor{x}^{-2} (x_0(-x_2) - x_1(-x_3) + x_2x_0 + x_3(-x_1)) \ii_2 \\
			&+ \nor{x}^{-2} (x_0(-x_3) + x_1(-x_2) - x_2(-x_1) + x_3x_0) \ii_3 \\
			&= \nor{x}^{-2} ({x_0}^2 + {x_1}^2 + {x_2}^2 + {x_3}^2) 1 \\
			&= \nor{x}^{-2} \nor{x}^2 1 \\
			&= 1.
		\end{align*}
\end{itemize}
\begin{itemize}
	\item (Submultiplicatividade) Demonstramos de fato a multiplicatividade da nor\-ma, que implica a submultiplicatividade. Para todos $x,y \in \R^4$,
		\begin{align*}
		\nor{xy}^2 &= (x_0y_0 - x_1y_1 - x_2y_2 - x_3y_3)^2 \\
			&+ (x_0y_1 + x_1y_0 + x_2y_3 - x_3y_2)^2 \\
			&+ (x_0y_2 - x_1y_3 + x_2y_0 + x_3y_1)^2 \\
			&+ (x_0y_3 + x_1y_2 - x_2y_1 + x_3y_0)^2 \\
%			&= (x_0y_0)^2 + (x_1y_1)^2 + (x_2y_2)^2 + (x_3y_3)^2 \\
%			&+ 2 (-x_0y_0x_1y_1 - x_0y_0x_2y_2 - x_0y_0x_3y_3 + x_1y_1x_2y_2 + x_1y_1x_3y_3 + x_2y_2x_3y_3) \\
			&= (x_0y_0)^2 + (x_1y_1)^2 + (x_2y_2)^2 + (x_3y_3)^2 \\
			&- 2 (x_0y_0x_1y_1 + x_0y_0x_2y_2 + x_0y_0x_3y_3 \\
			&\qquad- x_1y_1x_2y_2 - x_1y_1x_3y_3 - x_2y_2x_3y_3) \\
%			&+ (x_0y_1)^2 + (x_1y_0)^2 + (x_2y_3)^2 + (x_3y_2)^2 \\
%			&+ 2 (x_0y_1x_1y_0 + x_0y_1x_2y_3 - x_0y_1x_3y_2 + x_1y_0x_2y_3 - x_1y_0x_3y_2 - x_2y_3x_3y_2) \\
			&+ (x_0y_1)^2 + (x_1y_0)^2 + (x_2y_3)^2 + (x_3y_2)^2 \\
			&+ 2 (x_0y_1x_1y_0 + x_0y_1x_2y_3 - x_0y_1x_3y_2 \\
			&\qquad+ x_1y_0x_2y_3 - x_1y_0x_3y_2 - x_2y_3x_3y_2) \\
%			&+ (x_0y_2)^2 + (x_1y_3)^2 + (x_2y_0)^2 + (x_3y_1)^2 \\
%			&+ 2 (-x_0y_2x_1y_3 + x_0y_2x_2y_0 + x_0y_2x_3y_1 - x_1y_3x_2y_0 - x_1y_3x_3y_1 + x_2y_0x_3y_1) \\
			&+ (x_0y_2)^2 + (x_1y_3)^2 + (x_2y_0)^2 + (x_3y_1)^2 \\
			&- 2 (x_0y_2x_1y_3 - x_0y_2x_2y_0 - x_0y_2x_3y_1 \\
			&\qquad+ x_1y_3x_2y_0 + x_1y_3x_3y_1 - x_2y_0x_3y_1) \\
%			&+ (x_0y_3)^2 + (x_1y_2)^2 + (x_2y_1)^2 + (x_3y_0)^2 \\
%			&+ 2 (x_0y_3x_1y_2 - x_0y_3x_2y_1 + x_0y_3x_3y_0 - x_1y_2x_2y_1 + x_1y_2x_3y_0 - x_2y_1x_3y_0) \\
			&+ (x_0y_3)^2 + (x_1y_2)^2 + (x_2y_1)^2 + (x_3y_0)^2 \\
			&+ 2 (x_0y_3x_1y_2 - x_0y_3x_2y_1 + x_0y_3x_3y_0 \\
			&\qquad- x_1y_2x_2y_1 + x_1y_2x_3y_0 - x_2y_1x_3y_0) \\
			&= (x_0y_0)^2 + (x_1y_1)^2 + (x_2y_2)^2 + (x_3y_3)^2 \\
			&+ (x_0y_1)^2 + (x_1y_0)^2 + (x_2y_3)^2 + (x_3y_2)^2 \\
			&+ (x_0y_2)^2 + (x_1y_3)^2 + (x_2y_0)^2 + (x_3y_1)^2 \\
			&+ (x_0y_3)^2 + (x_1y_2)^2 + (x_2y_1)^2 + (x_3y_0)^2 \\
			&= {x_0}^2({y_0}^2+{y_1}^2+{y_2}^2+{y_3}^2) \\
			&+ {x_1}^2({y_0}^2+{y_1}^2+{y_2}^2+{y_3}^2) \\
			&+ {x_2}^2({y_0}^2+{y_1}^2+{y_2}^2+{y_3}^2) \\
			&+ {x_3}^2({y_0}^2+{y_1}^2+{y_2}^2+{y_3}^2) \\
			&= ({x_0}^2+{x_1}^2+{x_2}^2+{x_3}^2)({y_0}^2+{y_1}^2+{y_2}^2+{y_3}^2) \\
			&= \nor{x}^2\nor{y}^2.
		\end{align*}

	\item (Unitariedade)
		\begin{equation*}
		\nor{1} = (1^2+0^2+0^2+0^2)^{\frac{1}{2}} = 1.
		\end{equation*}

	\item (Invertibilidade) Para todo $x \in \R^4 \setminus \{0\}$,
		\begin{align*}
		\nor{x\inv} &= \left(\left(\frac{x_0}{\nor{x}^2}\right)^2 + \left(\frac{-x_1}{\nor{x}^2}\right)^2 + \left(\frac{-x_2}{\nor{x}^2}\right)^2 + \left(\frac{-x_3}{\nor{x}^2}\right)^2\right)^{\frac{1}{2}} \\
			&= \frac{1}{\nor{x}^2}((x_0)^2+(-x_1)^2+(-x_2)^2+(-x_3)^2)^{\frac{1}{2}} \\
			&= \frac{\nor{x}}{\nor{x}^2} \\
			&= \nor{x}\inv.
			\qedhere
		\end{align*}
\end{itemize}
\end{proof}
\endgroup

Demonstramos de fato a multiplicatividade da norma, mas a submultiplicatividade, junto às outras propriedades da norma, implica que, para todos $x,x' \in \R^4$,
	\begin{equation*}
	\nor{xx'} = \nor{x}\nor{x'}.
	\end{equation*}
Como o vetor $1 = (1,0,0,0)$ é uma unidade, em geral o omitiremos da notação da base, denotando $x_0 1$ por $x_0$ e
	\begin{equation*}
	x = x_0 + x_1 \ii_1 + x_2 \ii_2 + x_3 \ii_3.
	\end{equation*}

\begin{comment}
Segue direto da definição do produto que valem as seguintes relações de produto entre os elementos da base:
	\begin{equation*}
	\ii_1\ii_2\ii_3 = {\ii_1}^2 = {\ii_2}^2 = {\ii_3}^2 = -1.
	\end{equation*}
Dessas relações, deduzem-se
	\begin{equation*}
	\ii_1\ii_2 = -\ii_2\ii_1 = \ii_3, \qquad \ii_2\ii_3 = -\ii_3\ii_2 = \ii_1, \qquad \ii_3\ii_1 = -\ii_1\ii_3 = \ii_2.
	\end{equation*}

De fato, em vez da expressão explícita para $\times$ em termos das entradas de $x$ e $y$, poderíamos somente definir o produto entre os elementos de uma base, no caso $\{1,\ii_1, \ii_2, \ii_3\}$, e o produto se estenderia linearmente. Essa definição seria dada como nas relações acima. Com essas relações e representando os elementos $x \in \R^4$ como
	\begin{equation*}
	x = x_0 + x_1 \ii_1 + x_2 \ii_2 + x_3 \ii_3,
	\end{equation*}
pode-se calcular o produto quaterniônico $\times$ em $\R^4$ usando linearidade.


Por linearidade, deve valer
	\begin{align*}
	xy &= (x_0 1  + x_1 \ii_1 + x_2 \ii_2 + x_3 \ii_3)(y_0 1  + y_1 \ii_1 + y_2 \ii_2 + y_3 \ii_3) \\
		&= x_0 y_0 1  1  + x_0 y_1 1  \ii_1 + x_0 y_2 1  \ii_2 + x_0 y_3 1  \ii_3 \\
		&+ x_1 y_0 \ii_1  1  + x_1 y_1 \ii_1  \ii_1 + x_1y_2 \ii_1  \ii_2 + x_1 y_3 \ii_1  \ii_3 \\
		&+ x_2 y_0 \ii_2 1  + x_2 y_1 \ii_2 \ii_1 + x_2 y_2 \ii_2 \ii_2 + x_2 y_3 \ii_2 \ii_3 \\
		&+ x_3 y_0 \ii_3 1  + x_3 y_1 \ii_3 \ii_1 + x_3 y_2 \ii_3 \ii_2 + x_3 y_3 \ii_3 \ii_3
	\end{align*}

Portanto basta definir os valores dos produtos entre $1,i,j,k$. Escolhendo $1$ para ser a unidade, já que deve haver uma unidade (e essa notação não foi escolhida por acaso para $(1,0,0,0)$), basta determinarmos
	\begin{equation*}
	\ii_1\ii_1,\ii_1\ii_2,\ii_1\ii_3,\ii_2\ii_1,\ii_2\ij,\ii_2\ii_3,\ii_3\ii_1,\ii_3\ii_2,\ii_3\ii_3.
	\end{equation*}

\end{comment}

\begin{definition}
Seja $x \in \R^4$. A \emph{componente escalar} de $x$ é (o número) $\esc{x} := x_0 1$ e a \emph{componente vetorial} de $x$ é (o vetor)
	\begin{equation*}
	\vec{x} := x_1 \ii_1 + x_2 \ii_2 + x_3 \ii_3.
	\end{equation*}

O \emph{subespaço de escalares} de $\R^4$ é o subespaço
	\begin{equation*}
	\esc{\R} := \set{x \in \R^4}{\vec{x}=0}
	\end{equation*}
e o \emph{subespaço de vetores} de $\R^4$ é o subespaço
	\begin{equation*}
	\vec{\R}^3 := \set{x \in \R^4}{\esc{x}=0}.
	\end{equation*}
A \emph{esfera de vetores} de $\R^4$ é o conjunto
	\begin{equation*}
	\vec{\S}^2 := \set{x \in \S^3}{\esc{x}=0} = \S^3 \cap \vec{\R}^3.
	\end{equation*}
\end{definition}

Isso nos permite escrever todo quatérnio $x \in \R^4$ unicamente como
	\begin{equation*}
	x = \esc{x} + \vec{x}
	\end{equation*}
e decompor $\R^4$ como a soma direta
	\begin{equation*}
	\R^4 = \esc{\R} + \vec{\R}^3.
	\end{equation*}

%\begin{definition}
%Seja $x \in \R^4$. O \emph{conjugado} de $x$ é
%	\begin{equation*}
%	\conju{x} := \esc{x}-\vec{x}.
%	\end{equation*}
%Um \emph{versor} é um quatérnio $u \in \R^8$ tal que $\esc{u}=0$ e $\nor{\vec{u}}=1$.
%\end{definition}

\begin{definition}
A \emph{conjugação} em $\R^4$ é a operação
	\begin{align*}
	\func{\conju{}}{\R^4}{\R^4}{x}{\conju{x} := \esc{x}-\vec{x}}.
	\end{align*}
\end{definition}

\begin{exercise}
	\begin{enumerate}
	\item (Involutividade) Para todo $x \in \R^4$, $\conju{(\conju{x})} = x$;
	\item (Aditividade) Para todos $x,x' \in \R^4$, $\conju{(x+x')} = \conju{x} + \conju{(x')}$;
	\item (Identidade aditiva) $\conju{0} = 0$;
	\item (Invertibilidade aditiva) Para todo $x \in \R^4$, $\conju{(-x)} = -\conju{x}$;
	\item (Anti-multiplicatividade) Para todos $x,x' \in \R^4$, $\conju{(xx')} = \conju{(x')}\conju{x}$;
	\item (Identidade multiplicativa) $\conju{1} = 1$;
	\end{enumerate}
\end{exercise}

\begin{exercise}
Seja $x \in \R^4$.
	\begin{enumerate}
	\item $\nor{x}^2 = \abs{\esc{x}}^2 + \nor{\vec{x}}^2$;
	\item $\nor{x}^2 = x\conju{x}$;
	\item $\conju{x} = \frac{1}{2}(1x(-1) + \ii_1 x (-\ii_1) + \ii_2 x (-\ii_2) + \ii_3 x (-\ii_3))$.
	\end{enumerate}
\end{exercise}

\begin{proposition}[Decomposição polar]
Seja $x \in \R^4 \setminus \{0\}$. Existem únicos $\phi_0 \in \intff{0}{\tau \div 2}$ e $\hat{x} \in \vec{\S}^2$ tais que
	\begin{equation*}
	x = \nor{x}(\cos(\phi_0) + \sin(\phi_0) \hat{x}).
	\end{equation*}
Eles são dados por $\phi_0 := \cos\inv\left( \frac{\esc{x}}{\nor{x}} \right) \in \intff{0}{\tau \div 2}$ e $\hat{x} := \frac{\vec{x}}{\nor{\vec{x}}} \in \vec{\S}^2$.
\end{proposition}
%\begin{proposition}[Decomposição polar]
%	Seja $x \in \R^4 \setminus \{0\}$. Então % existem $\phi_0 \in \intff{0}{\tau \div 2}$ e $\hat{x} \in \vec{\S}^2$ tais que
%		\begin{equation*}
%		x=\nor{x}(\cos(\phi_0) + \sin(\phi_0) \hat{x}),
%		\end{equation*}
%	em que $\phi_0 := \cos\inv\left( \frac{\esc{x}}{\nor{x}} \right) \in \intff{0}{\tau \div 2}$ e $\hat{x} := \frac{\vec{x}}{\nor{\vec{x}}} \in \vec{\S}^2$.
%\end{proposition}
\begin{proof}
Mostraremos primeiro a decomposição. Como $x \neq 0$, $\nor{x} \neq 0$, logo da igualdade $\nor{x}^2 = \abs{\esc{x}}^2 + \nor{\vec{q}}^2$ segue que
	\begin{equation*}
	1 = \left( \frac{\abs{\esc{x}}}{\nor{x}} \right)^2 + \left( \frac{\nor{\vec{x}}}{\nor{x}} \right)^2,
	\end{equation*}
portanto $\frac{\abs{\esc{x}}}{\nor{x}} \leq 1$, o que é equivalente a $\frac{\esc{x}}{\nor{x}} \in \intff{-1}{1}$. Disso segue que $\phi_0 = \cos\inv\left( \frac{\esc{x}}{\nor{x}} \right) \in \intff{0}{\tau \div 2}$, de modo que $\frac{\nor{\vec{x}}}{\nor{x}} = (1-(\cos(\phi_0))^2)^{1 \div 2}$, o que implica $\frac{\nor{\vec{x}}}{\nor{x}}=\sin(\phi_0)$ ou $\frac{\nor{\vec{x}}}{\nor{x}}=-\sin(\phi_0)$. Como para $\phi_0 \in \intff{0}{\tau \div 2}$ vale $\sin(\phi_0) \geq 0$, concluímos que $\frac{\nor{\vec{x}}}{\nor{x}}=\sin(\phi_0)$.

Das definições de $\phi_0$ e $\hat{x}$, segue que $\esc{x} = \nor{x}\cos(\phi_0)$ e $\vec{x} = \nor{x}\sin(\phi_0)\hat{x}$, portanto
	\begin{align*}
	x &= \esc{x} + \vec{x} \\
		&= \nor{x}\cos(\phi_0) + \nor{x}\sin(\phi_0)\hat{x} \\
		&= \cos(\phi_0) + \sin(\phi_0) \frac{\vec{x}}{\nor{\vec{x}}} \\
		&= \nor{x} (\cos(\phi_0) + \sin(\phi_0) \hat{x}).
		\qedhere
	\end{align*}

Para a unicidade, sejam $\phi_0' \in \intff{0}{\tau \div 2}$ e $\hat{x}' \in \vec{\S}^2$ tais que
	\begin{equation*}
	x = \nor{x}(\cos(\phi_0) + \sin(\phi_0) \hat{x}) = \nor{x}(\cos(\phi_0') + \sin(\phi_0') \hat{x}').
	\end{equation*}
Então $\cos(\phi_0) = \cos(\phi_0')$ e $\sin(\phi_0) \hat{x} = \sin(\phi_0') \hat{x}'$. Da primeira igualdade segue que existe $n \in \Z$ tal que $\phi_0' = \phi_0 + n\tau$ ou $\phi_0' = -\phi_0 + n\tau$. Como $\phi_0,\phi_0' \in \intff{0}{\tau \div 2}$, segue que $\phi_0 = \phi_0'$. Isso implica que $\sin(\phi_0) = \sin(\phi_0')$, portanto da segunda igualdade segue que $\hat{x} = \hat{x}'$.
\end{proof}

\subsubsection{Produtos interno, escalar e vetorial}

\begin{definition}
O \emph{produto interno quaterniônico} em $\R^4$ é a função
	\begin{align*}
	\func{\inteq{\var}{\var}}{\R^4 \times \R^4}{\R}{(x,y)}{\inteq{x}{y} := x_0y_0 - (x_1y_1 + x_2y_2 + x_3y_3)}.
	\end{align*}

	O \emph{produto escalar} em $\vec{\R}^3$ é a função
	\begin{align*}
	\func{\pesc{}{}}{\vec{\R}^3 \times \vec{\R}^3}{\R}{(\vec{x},\vec{y})}{\pesc{\vec{x}}{\vec{y}} := x_1y_1 + x_2y_2 + x_3y_3}.
	\end{align*}

O \emph{produto vetorial} em $\vec{\R}^3$ é a função
	\begin{align*}
	\func{\pvec{}{}}{\vec{\R}^3 \times \vec{\R}^3}{\vec{\R}^3}{(\vec{x},\vec{y})}{
		\begin{aligned}[t]
			\pvec{\vec{x}}{\vec{y}} := &(x_2y_3-x_3y_2) \ii_1 \\
			+ &(-x_1y_3+x_3y_1) \ii_2 \\
			+ &(x_1y_2-x_2y_1) \ii_3.
		\end{aligned}
	}
	\end{align*}
\end{definition}

O produto interno quaterniônico difere do produto interno usual em $\R^4$, mas os outros dois produtos são o produto interno e o produto vetorial usuais em $\R^3$.

\begin{table}
	\centering

	\begin{tabular}{c | c c c c}
	\toprule
	$\pvec{}{}$&	$\ii_1$	&	$\ii_2$	&	$\ii_3$	\\
	\hline
	$\ii_1$	&	$0$		&	$\ii_3$	&	$-\ii_2$	\\
	$\ii_2$	&	$-\ii_3$&	$0$		&	$\ii_1$		\\
	$\ii_3$	&	$\ii_2$	&	$-\ii_1$&	$0$			\\
	\bottomrule
	\end{tabular}

	\caption{Tabela de produto vetorial em $\vec{\R}^3$.}
	\label{tab:multiplicacao.quaternionica.vetorial}
\end{table}

\begin{exercise}
	\begin{enumerate}
	\item Para todos $x,y \in \R^4$,
		\begin{equation*}
		\inteq{x}{y} = \esc{x}\esc{y}-\pesc{\vec{x}}{\vec{y}}.
		\end{equation*}

	\item O produto escalar $\pesc{}{}\colon \vec{\R}^3 \times \vec{\R}^3 \to \R$ é um produto interno no subespaço vetorial $\vec{\R}^3$;

	\item O produto vetorial $\pvec{}{}\colon \vec{\R}^3 \times \vec{\R}^3 \to \vec{\R}^3$ é uma função bilinear alternada tal que, para todos $\vec{x},\vec{y} \in \vec{\R}^3$,
		\begin{enumerate}
		\item $\pesc{(\pvec{\vec{x}}{\vec{y}})}{\vec{x}} = \pesc{(\pvec{\vec{x}}{\vec{y}})}{\vec{y}} = 0$;

		\item $\pesc{(\pvec{\vec{x}}{\vec{y}})}{(\pvec{\vec{x}}{\vec{y}})} =
				\det{
					\begin{bmatrix}
					\pesc{\vec{x}}{\vec{x}} & \pesc{\vec{x}}{\vec{y}} \\
					\pesc{\vec{y}}{\vec{x}} & \pesc{\vec{y}}{\vec{y}}
					\end{bmatrix}
				}$.
		\end{enumerate}

	\item Para todos $\vec{x},\vec{y},\vec{z} \in \vec{\R}^3$,
		\begin{equation*}
		\pvec{(\pvec{\vec{x}}{\vec{y}})}{\vec{z}} = (\pesc{\vec{x}}{\vec{z}})\vec{y} - (\pesc{\vec{y}}{\vec{z}})\vec{x};
		\end{equation*}

	\item A função
		\begin{align*}
		\func{\vol_{\Yup}}{\vec{\R}^3 \times \vec{\R}^3 \times \vec{\R}^3}{\vec{\R}^3}{(\vec{x},\vec{y},\vec{z})}{\pesc{(\pvec{\vec{x}}{\vec{y}})}{\vec{z}}}
		\end{align*}
é uma função trilinear alternada.

	\end{enumerate}
\end{exercise}


\begin{proposition}
	\begin{enumerate}
	\item Para todos $x,y \in \vec{\R}^3$ ($\esc{x} = \esc{y} = 0$),
		\begin{equation*}
		xy = - \pesc{\vec{x}}{\vec{y}} + \pvec{\vec{x}}{\vec{y}};
		\end{equation*}
	\item Para todos $x,y \in \R^4$,
		\begin{equation*}
		xy = \esc{x}\esc{y} - \pesc{\vec{x}}{\vec{y}} + \esc{x}\vec{y} + \esc{y}\vec{x} + \pvec{\vec{x}}{\vec{y}},
		\end{equation*}
sendo
		\begin{align*}
		\esc{(xy)} &= \esc{x}\esc{y} - \pesc{\vec{x}}{\vec{y}}, \\
		\vec{(xy)} &= \esc{x}\vec{y} + \esc{y}\vec{x} + \pvec{\vec{x}}{\vec{y}}.
		\end{align*}

	\item Para todos $x,y \in \R^4$, $xy=yx$ se, e somente se, $\pvec{\vec{x}}{\vec{y}}=0$;% (ou seja, $\vec{x} \parallel \vec{y}$);

	\item Para todo $x \in \R^4 \setminus \{0\}$,
		\begin{equation*}
		x\inv = \frac{\conju{x}}{\nor{x}^2};
		\end{equation*}

	\item $\vec{\S}^2 = \set{x \in \R^4}{x^2 = -1}$;

	\item Para todos $x \in \R^4 \setminus \{0\}$ e $y \in \R^4$,
		\begin{equation*}
		xyx\inv = \esc{y} + 2\frac{\pesc{\vec{x}}{\vec{y}}}{\nor{x}^2}\vec{x} + \frac{\inteq{x}{x}}{\nor{x}^2}\vec{y} + 2\frac{\esc{x}}{\nor{x}^2}(\pvec{\vec{x}}{\vec{y}}).
		\end{equation*}

	\item Para todos $x,y \in \R^4 \setminus \{0\}$ tais que $xy \in \esc{\R}$, % vale que $\vec{x}=0$ ou $\vec{y}=0$ ou
	 $\vec{x} \parallel \vec{y}$.
%	  e
%		\begin{equation*}
%		xy = \esc{x}\esc{y} - \nor{\vec{x}}\nor{\vec{y}}.
%		\end{equation*}
	\end{enumerate}
\end{proposition}
\begin{proof}
\begin{enumerate}
	\item Como $x_0=y_0=0$,
	\begin{align*}
	xy =& -(x_1y_1 + x_2y_2 + x_3y_3) \\
			&+ (x_2y_3-x_3y_2) \ii_1 \\
			&+ (-x_1y_3+x_3y_1) \ii_2 \\
			&+ (x_1y_2-x_2y_1) \ii_3 \\
		=& - \pesc{\vec{x}}{\vec{y}} + \pvec{\vec{x}}{\vec{y}}.
	\end{align*}

	\item Segue da bilinearidade do produto e de $\vec{x}$ e $\vec{y}$ serem puramente vetoriais que
	\begin{align*}
	xy &= (\esc{x}+\vec{x})(\esc{y}+\vec{y}) \\
		&= \esc{x}\esc{y}+\esc{x}\vec{y}+\esc{y}\vec{x}+\vec{x}\vec{y} \\
		&= \esc{x}\esc{y}+\esc{x}\vec{y}+\esc{y}\vec{x}+(-\pesc{\vec{x}}{\vec{y}} + \pvec{\vec{x}}{\vec{y}}) \\
		&= (\esc{x}\esc{y} - \pesc{\vec{x}}{\vec{y}}) + (\esc{x}\vec{y}+\esc{y}\vec{x}+\pvec{\vec{x}}{\vec{y}}).
	\end{align*}

	\item Para todos $x,y \in \R^4$,
		\begin{align*}
		xy - yx &= \esc{x}\esc{y} - \pesc{\vec{x}}{\vec{y}}+\esc{x}\vec{y}+\esc{y}\vec{x}+\pvec{\vec{x}}{\vec{y}} \\
			&- \esc{y}\esc{x} + \pesc{\vec{y}}{\vec{x}}-\esc{y}\vec{x}-\esc{x}\vec{y}-\pvec{\vec{y}}{\vec{x}} \\
			&= \pvec{\vec{x}}{\vec{y}} - \pvec{\vec{y}}{\vec{x}} \\
			&= 2\pvec{\vec{x}}{\vec{y}}.
		\end{align*}
Isso implica que $xy-yx=0$ se, e somente se, $\pvec{\vec{x}}{\vec{y}} - \pvec{\vec{y}}{\vec{x}}=0$, o que ocorre se, e somente se, $\pvec{\vec{x}}{\vec{y}}=0$.%, que por sua vez é equivalente a $\vec{x} \parallel \vec{y}$.

	\item Para todo $x \in \R^4 \setminus \{0\}$,
	\begin{align*}
	xx\inv &= (\esc{x}+\vec{x})\nor{x}^{-2}(\esc{x}-\vec{x}) \\
		&= \nor{x}^{-2}(\esc{x}^2-\pesc{\vec{x}}{(-\vec{x})}+\esc{x}(-\vec{x})+\esc{x}\vec{x}+\pvec{\vec{x}}{(-\vec{x})}) \\
		&= \nor{x}^{-2}(\esc{x}^2+\pesc{\vec{x}}{\vec{x}}-\esc{x}\vec{x}+\esc{x}\vec{x}-\pvec{\vec{x}}{\vec{x}}) \\
		&= \nor{x}^{-2}\nor{x}^2 \\
		&= 1
	\end{align*}
e, como $\pvec{\vec{x}}{(-\vec{x})} = 0$, $x\inv x = xx\inv = 1$.

	\item Para todo $x \in \R^4$,
		\begin{align*}
		x^2 &= \esc{x}\esc{x} - \pesc{\vec{x}}{\vec{x}}+\esc{x}\vec{x}+\esc{x}\vec{x}+\pvec{\vec{x}}{\vec{x}} \\
			&= \esc{x}^2 - \nor{\vec{x}}^2 + 2\esc{x}\vec{x}.
		\end{align*}
Segue que $\esc{x}=0$ e $\nor{\vec{x}}=1$ se, e somente se, $x^2=-1$.

	\item Para todos $x,y \in \R^4$,
		\begin{align*}
		\esc{(xyx\inv)} &= \esc{(xy)}\esc{(x\inv)} - \pesc{\vec{(xy)}}{\vec{(x\inv)}} \\
			&= (\esc{x}\esc{y} - \pesc{\vec{x}}{\vec{y}})\frac{\esc{x}}{\nor{x}^2} - \pesc{(\esc{x}\vec{y} + \esc{y}\vec{x} + \pvec{\vec{x}}{\vec{y}})}{\frac{(-\vec{x})}{\nor{x}^2}} \\
			&= \frac{\esc{x}^2\esc{y} - \esc{x}(\pesc{\vec{x}}{\vec{y}}) + \esc{x}\pesc{\vec{y}}{\vec{x}} + \esc{y}\pesc{\vec{x}}{\vec{x}} + \pesc{(\pvec{\vec{x}}{\vec{y}})}{\vec{x}}}{\nor{x}^2} \\
			&= \frac{\esc{x}^2\esc{y} + \nor{\vec{x}}^2\esc{y}}{\nor{x}^2} \\
			&= \esc{y}
		\end{align*}
e
		\begin{align*}
		\vec{(xyx\inv)} &= \esc{(xy)}\vec{(x\inv)} + \esc{(x\inv)}\vec{(xy)} + \pvec{\vec{(xy)}}{\vec{(x\inv)}} \\
			&= (\esc{x}\esc{y} - \pesc{\vec{x}}{\vec{y}})\frac{(-\vec{x})}{\nor{x}^2}
			+ \frac{\esc{x}}{\nor{x}^2}(\esc{x}\vec{y} + \esc{y}\vec{x} + \pvec{\vec{x}}{\vec{y}}) \\
				&\qquad + \pvec{(\esc{x}\vec{y} + \esc{y}\vec{x} + \pvec{\vec{x}}{\vec{y}})}{\frac{(-\vec{x})}{\nor{x}^2}} \\
			&= \frac{(\pesc{\vec{x}}{\vec{y}})\vec{x} + \esc{x}^2\vec{y} + \esc{x}(\pvec{\vec{x}}{\vec{y}})}{\nor{x}^2} \\
				&\qquad - \frac{\esc{x}(\pvec{\vec{y}}{\vec{x})} + \pvec{(\pvec{\vec{x}}{\vec{y}})}{\vec{x}}}{\nor{x}^2} \\
			&= \frac{(\pesc{\vec{x}}{\vec{y}})\vec{x}+ \esc{x}^2\vec{y} + 2\esc{x}(\pvec{\vec{x}}{\vec{y}}) - (\pesc{\vec{x}}{\vec{x}})\vec{y} + (\pesc{\vec{y}}{\vec{x}})\vec{x}}{\nor{x}^2} \\
			&= \frac{2(\pesc{\vec{x}}{\vec{y}})\vec{x} + (\esc{x}^2-\nor{\vec{x}}^2)\vec{y} + 2\esc{x}(\pvec{\vec{x}}{\vec{y}})}{\nor{x}^2} \\
			&= 2\frac{\pesc{\vec{x}}{\vec{y}}}{\nor{x}^2}\vec{x} + \frac{\inteq{x}{x}}{\nor{x}^2}\vec{y} + 2\frac{\esc{x}}{\nor{x}^2}(\pvec{\vec{x}}{\vec{y}}).
		\end{align*}

Portanto
		\begin{equation*}
		xyx\inv = \esc{y} + 2\frac{\pesc{\vec{x}}{\vec{y}}}{\nor{x}^2}\vec{x} + \frac{\inteq{x}{x}}{\nor{x}^2}\vec{y} + 2\frac{\esc{x}}{\nor{x}^2}(\pvec{\vec{x}}{\vec{y}}).
		\end{equation*}

	\item Sejam $x,y \in \R^4 \setminus \{0\}$ tais que $xy \in \esc{\R}$. Então
		\begin{equation*}
		0 = \esc{x}\vec{y}+\esc{y}\vec{x}+\pvec{\vec{x}}{\vec{y}}.
		\end{equation*}
	%Se $\vec{x} \parallel \vec{y}$, então $\vec{x} \times \vec{y}=0$ e existe $c \in \R$ tal que $\vec{y}=c\vec{x}$ ou $\vec{x}=c\vec{y}$. Isso significa que a condição anterior se reduz a
	%	\begin{equation*}
	%	0 = \esc{x}\vec{y}+\esc{y}\vec{x} = (\esc{x}c+\esc{y})\vec{x}
	%	\end{equation*}
	%ou
	%	\begin{equation*}
	%	0 = \esc{x}\vec{y}+\esc{y}\vec{x} = (\esc{x}+\esc{y}c)\vec{y},
	%	\end{equation*}
	%o que implica que $\vec{x}=\vec{y}=0$ ou $c=-\esc{y}/\esc{x}$ ou $c=-\esc{x}/\esc{y}$.
	%
	%Se $\vec{x} \perp \vec{y}$, então $(\vec{x},\vec{y},\vec{x} \times \vec{y})$ é uma base de $\vec{\HH}$
	%
	Isso implica que $(\vec{x},\vec{y},\pvec{\vec{x}}{\vec{y}})$ não é uma base de $\vec{\R}^3$, pois caso contrário
		\begin{equation*}
		\esc{x}\vec{y}+\esc{y}\vec{x}+\vec{x} \times \vec{y} \neq 0,
		\end{equation*}
	já que $(\esc{x},\esc{y},1) \neq (0,0,0)$. Mas a tripla não é base se, e somente se, $\vec{x}=0$ ou $\vec{y}=0$ ou $\pvec{\vec{x}}{\vec{y}}=0$, porque sempre vale $\pesc{\vec{x}}{(\pvec{\vec{x}}{\vec{y}})} = \pesc{\vec{y}}{(\pvec{\vec{x}}{\vec{y}})}=0$. Consideramos cada caso.
		\begin{itemize}
		\item ($\vec{x}=0$) Nesse caso, segue que $\esc{x} \neq 0$, pois $x \neq 0$, e que $\vec{x} \times \vec{y}=0$, logo
			\begin{equation*}
			0 = \esc{x}\vec{y}.
			\end{equation*}
		o que implica que $\vec{y}=0$, portanto $\vec{x} \parallel \vec{y}$ e $x,y \in \esc{\R}$.

		\item ($\vec{y}=0$) Nesse caso, segue que $\esc{y} \neq 0$, pois $y \neq 0$, e que $\vec{x} \times \vec{y}=0$, logo
			\begin{equation*}
			0 = \esc{y}\vec{x},
			\end{equation*}
		o que implica que $\vec{x}=0$, portanto $\vec{x} \parallel \vec{y}$ e $x,y \in \esc{\R}$.

		\item ($\vec{x} \neq 0$, $\vec{y} \neq 0$ e $\pvec{\vec{x}}{\vec{y}}=0$) Nesse caso, segue que $\vec{x} \parallel \vec{y}$.
%		 e que
%			\begin{equation*}
%			\esc{y}\vec{x} = -\esc{x}\vec{y}.
%			\end{equation*}
%		Se $\esc{x} = 0$, então $\esc{y}\vec{x} = 0$, logo $\esc{y} = 0$. Se $\esc{x} \neq 0$, então
%			\begin{equation*}
%			\frac{\esc{y}}{\esc{x}}\vec{x} = -\vec{y} \neq 0,
%			\end{equation*}
%		logo $\esc{y} \neq 0$.
		\end{itemize}

%	Nos primeiros dois casos, $\pesc{\vec{x}}{\vec{y}} = 0 = \nor{\vec{x}}\nor{\vec{y}}$, e no terceiro caso, $\pesc{\vec{x}}{\vec{y}}=\nor{\vec{x}}\nor{\vec{y}}$, portanto
%		\begin{equation*}
%		xy = \esc{(xy)} = \esc{x}\esc{y} - \nor{\vec{x}}\nor{\vec{y}}.
%		\qedhere
%		\end{equation*}
\end{enumerate}
\end{proof}

\begin{comment}

\subsubsection{Construção de Cayley-Dickson}

Todo $x \in \R^4$ pode ser escrito como
	\begin{equation*}
	x = (x_0 + x_1\ii_1) + (x_2 + x_3\ii_1)\ii_2,
	\end{equation*}
de modo que $(x_0 + x_1\ii_1)$ e $(x_2 + x_3\ii_1)$ podem ser vistos como elementos de $\R^2$. Definimos homomorfismos
	\begin{align*}
	\func{h_0}{\R^4}{\R^4}{x}{x_0 + x_1\ii_1}
	\end{align*}
	\begin{align*}
	\func{h_1}{\R^4}{\R^4}{x}{(x_2 + x_3\ii_1)}
	\end{align*}
e segue que
	\begin{equation*}
	x = (x_0 + x_1\ii_1) + (x_2 + x_3\ii_1)\ii_2 = h_0(x) + h_1(x)\ii_2.
	\end{equation*}
Multiplicando $x,y \in \R^4$, obtemos
%	\begin{align*}
%	xy &= ((x_0 + x_1\ii_1) + (x_2 + x_3\ii_1)\ii_2)((y_0 + y_1\ii_1) + (y_2 + y_3\ii_1)\ii_2) \\
%		&= ((x_0 + x_1\ii_1)(y_0 + y_1\ii_1) - \conju{(y_2 + y_3\ii_1)}(x_2 + x_3\ii_1)) \\
%		&+ ((y_2 + y_3\ii_1)(x_0 + x_1\ii_1) + (x_2 + x_3\ii_1)\conju{(y_0 + y_1\ii_1)})\ii_2
%	\end{align*}
	\begin{align*}
	xy &= (h_0(x) + h_1(x)\ii_2)(h_0(y) + h_1(y)\ii_2) \\
		&= (h_0(x)h_0(y) - \conju{h_1(y)}h_1(y)) \\
		&+ (h_1(y)h_0(x) + h_1(x)\conju{h_0(y)})\ii_2.
	\end{align*}
%	\begin{align}
%	xy := &(x_0y_0 - x_1y_1 - x_2y_2 - x_3y_3) 1 \\
%		+ &(x_0y_1 + x_1y_0 + x_2y_3 - x_3y_2) \ii_1 \\
%		+ &(x_0y_2 - x_1y_3 + x_2y_0 + x_3y_1) \ii_2 \\
%		+ &(x_0y_3 + x_1y_2 - x_2y_1 + x_3y_0) \ii_3.
%	\end{align}




\end{comment}




\subsubsection{Rotações em \ensuremath{\R^3}}

\begin{proposition}
A função
	\begin{align*}
	\func{R}{\S^3}{\SO(3)}{u}{
		\begin{aligned}[t]
		\func{R_u}{\vec{\R}^3}{\vec{\R}^3}{x}{uxu\inv}.
		\end{aligned}
	}
	\end{align*}
é um homomorfismo sobrejetivo de grupos diferenciável e $\nuc(R)=\S^0$.
\end{proposition}
\begin{proof}
A função $R$ é diferenciável porque $R_u(x)$ é um polinômio em cada entrada. Primeiro, vamos mostrar que $R$ de fato está bem definida, ou seja, que, para todo $u \in \S^3$, $R_u \in \SO(3)$. Seja $u \in \S^3$. Pela fórmula de conjugação por quatérnio, temos que, para todo $x \in \vec{\R}^3$,
	\begin{equation*}
	\esc{(R_u(x))} = \esc{(uxu\inv)} = \esc{x} = 0,
	\end{equation*}
portanto $R_u(x) \in \vec{\R}^3$ e então $\fun{R_u}{\vec{\R}^3}{\vec{\R}^3}$. A função $R_u$ é linear, pois, para todos $x,x' \in \vec{\R}^3$ e $c \in \R$, segue da bilinearidade do produto quaterniônico que
	\begin{equation*}
	R_u(cx+x') = u(cx+x')u\inv = (cux+ux')u\inv = cuxu\inv + ux'u\inv = cR_u(x) + R_u(x').
	\end{equation*}
Isso mostra que $R_u \in \GL(3,\R)$. Além disso, $R_u$ preserva norma, pois
	\begin{equation*}
	\nor{R_u(x)} = \nor{uxu\inv} = \nor{u}\nor{x}\nor{u}\inv = \nor{x}.
	\end{equation*}
Isso mostra que $R_u \in \GO(3)$. Agora, como $R$ é diferenciável, em particular é contínua, logo, como $\S^3$ é conexo, $R(\S^3)$ é conexo. Isso implica que $R(\S^3) \subseteq \SO(3)$.

Seja $u \in \S^3$ tal que $R_u=\Id$. Então, para todo $x \in \vec{\R}^3$, $uxu\inv = R_u(x) = x$, logo $ux=xu$. Isso ocorre se, e somente se, $\pvec{\vec{u}}{x} = 0$. Como isso vale para todo $x \in \vec{\R}^3$, segue que $\vec{u}=0$, portanto $u=\esc{u} \in \esc{\R}$; como $u \in \S^3$, concluímos que $u \in \S^0 = \{1,-1\}$, ou $\nuc(R) = \S^0$.

%A sobrejetividade se dá a seguir mostrando qual é a rotação que $R_u$ representa em $\SO(3)$.
Para a sobrejetividade, primeiro notamos que, como $R(\S^3) \subseteq \SO(3)$ é compacto, pois $\S^3$ é compacto, e $\SO(3)$ é separado\footnote{Hausdorff}, segue que $R(\S^3)$ é fechado. Além disso, $R$ é difeomorfismo local, pois o núcleo $\S^0$ é um subgrupo discreto (portanto tem álgebra de Lie trivial), logo $R$ é difeomorfismo local na identidade, e portanto em todo $\S^3$, já que um morfismo de grupos diferenciáveis\footnote{Grupos de lie} tem posto constante. Assim, segue que $R(\S^3) \subseteq \SO(3)$ é aberto. Como $\SO(3)$ é conexo, segue que $R(\S^3) = \SO(3)$.
\end{proof}

Em particular, obtemos desse recobrimento duplo que
	\begin{equation*}
	\quo{\S^3}{\S^0} \simeq \SO(3).
	\end{equation*}

Para calcular exatamente qual é a rotação que $R_u$ representa em $\SO(3)$, lembremos que, pela decomposição polar, todo $u \in \S^3$ pode ser escrito como
	\begin{equation*}
	u = \cos(\phi) + \sin(\phi)\hat{u} = \e^{\phi \hat{u}},
	\end{equation*}
com $\phi = \cos(\esc{u}) \in \intff{0}{\tau \div 2}$ e $\hat{u} = \frac{\vec{u}}{\nor{\hat{u}}} \in \vec{\S}^2$.

A rotação de $x \in \R^3$ por um ângulo $\theta \in \intff{0}{\tau}$ em torno de um vetor unitário $\hat{u} \in \S^2$ é dada por
	\begin{equation*}
	R^{\theta}_{\hat{u}}(x) = \proj_{\parallel \hat{u}}(x) + \cos(\theta) \proj_{\perp \hat{u}}(x) + \sin(\theta) \pvec{\hat{u}}{x}.
	\end{equation*}
%Note que genericamente $(\proj_{\parallel u}(x),\proj_{\perp u}(x),u \times x)$ é uma base de $\R^3$. %se $u$ e $x$ são ...
Isso ocorre pois, se rotacionamos um vetor $x$ em torno de um eixo $\hat{u}$, a componente de $x$ que é paralela a $\hat{u}$, dada pela projeção $\proj_{\parallel \hat{u}}(x)$ de $x$ em $\hat{u}$, fica constante, não rotaciona pois é paralela ao eixo de rotação, enquanto que a componente de $x$ que é perpendicular a $\hat{u}$, dada pela projeção $\pro_{\perp \hat{u}}(x)$, rotaciona no plano perpendicular a $\hat{u}$, dado pela base $\pro_{\perp \hat{u}}(x)$ e $\pvec{\hat{u}}{x}$, e essa rotação é por um ângulo $\theta$, portanto resultará em $\cos(\theta) \proj_{\perp \hat{u}}(x) + \sin(\theta) \pvec{\hat{u}}{x}$. Um detalhe é que o plano perpendicular é dado por essa base se, e somente se, $\hat{u}$ e $x$ não são paralelos, mas quando eles são paralelos vale $\proj_{\parallel \hat{u}}(x)=x$ e $\proj_{\perp \hat{u}}(x) = \pvec{\hat{u}}{x} = 0$, portanto a fórmula acima ainda vale.

Isso significa que
	\begin{equation*}
	R^{\theta}_{\hat{u}} = \proj_{\parallel \hat{u}} + \cos(\theta) \proj_{\perp \hat{u}} + \sin(\theta){ \pvec{\hat{u}}{}} .
	\end{equation*}

Como
	\begin{align*}
	2\sin(\phi)^2 &= 1-\cos(2\phi) \\
	\cos(\phi)^2 - \sin(\phi)^2 &= \cos(2\phi) \\
	2\cos(\phi)\sin(\phi) &= \sin(2\phi)
	\end{align*}
segue que da fórmula de conjugação por quatérnio que
	\begin{align*}
	\e^{\phi \hat{u}}x\e^{-\phi \hat{u}} &= 2(\pesc{\vec{u}}{x})\vec{u} + \inteq{u}{u}x + 2\esc{u}(\pvec{\vec{u}}{x}) \\
		&= 2\sin(\phi)(\pesc{\hat{u}}{x})(\sin(\phi)\hat{u}) \\
		&\qquad+ (\cos(\phi)^2-\sin(\phi)^2)x + 2\cos(\phi)\sin(\phi)(\pvec{\hat{u}}{x}) \\
		&= (1-\cos(2\phi))(\pesc{x}{\hat{u}})\hat{u} + \cos(2\phi)x + \sin(2\phi)(\pvec{\hat{u}}{x}) \\
		&= (\pesc{x}{\hat{u}})\hat{u} + \cos(2\phi)(x-(\pesc{x}{\hat{u}})\hat{u}) + \sin(2\phi)(\pvec{\hat{u}}{x}) \\
		&= \proj_{\parallel \hat{u}}(x) + \cos(2\phi)\proj_{\perp \hat{u}}(x) + \sin(2\phi)(x) \\
		&= R^{2\phi}_{\hat{u}}(x).
	\end{align*}

Isso mostra que
	\begin{align*}
	\func{R}{\S^3}{\SO(3)}{\e^{\phi \hat{u}}}{
		\begin{aligned}[t]
		\func{R^{2\phi}_{\hat{u}}}{\vec{\R}^3}{\vec{\R}^3}{x}{\e^{\phi \hat{u}}x\e^{-\phi \hat{u}}}.
		\end{aligned}
	}
	\end{align*}

Como $\nuc(R) = \S^0$, sabemos que $R\inv(R^{\theta}_{\hat{u}}) = \{\e^{\frac{\theta}{2}\hat{u}},-\e^{\frac{\theta}{2}\hat{u}}\}$. Podemos achar $-\e^{\frac{\theta}{2}\hat{u}}$ explicitamente em função de seu ângulo em $\intff{0}{\tau \div 2}$ e eixo de rotação em $\vec{\S}^2$. Como para todo $\phi \in \intff{0}{\tau \div 2}$ vale
	\begin{align*}
	-\cos(\phi) &= \cos((\tau \div 2) - \phi) \\
	\sin(\phi) &= \sin((\tau \div 2) - \phi),
	\end{align*}
segue que
	\begin{align*}
	-\e^{\frac{\theta}{2}\hat{u}} &= -(\cos(\theta \div 2) + \sin(\theta \div 2) \hat{u}) \\
		&= -\cos(\theta \div 2) - \sin(\theta \div 2) \hat{u} \\
		&= \cos((\tau-\theta) \div 2) + \sin((\tau - \theta) \div 2)(-\hat{u}) \\
		&= \e^{\frac{\tau - \theta}{2}(-\hat{u})}.
	\end{align*}
Notemos que como $(\tau \div 2) - \phi \in \intff{0}{\tau \div 2}$ e $-\hat{u} \in \vec{\S}^2$, $\e^{\frac{\tau - \theta}{2}(-\hat{u})}$ é a decomposição polar de $-\e^{\frac{\theta}{2}\hat{u}}$. Obtemos assim
	\begin{equation*}
	R\inv(R^{\theta}_{\hat{u}}) = \{\e^{\frac{\theta}{2}\hat{u}},-\e^{\frac{\tau - \theta}{2}(-\hat{u}})\}.
	\end{equation*}

%Por outro lado, como
%	\begin{align*}
%	-\cos(\theta) &= \cos(\theta + (\tau \div 2)) \\
%	-\sin(\theta) &= \sin(\theta + (\theta \div 2)),
%	\end{align*}
%segue que
%	\begin{align*}
%	-\e^{\frac{\theta}{2}\hat{u}} &= -(\cos(\theta \div 2) + \sin(\theta \div 2) \hat{u}) \\
%		&= -\cos(\theta \div 2) - \sin(\theta \div 2) \hat{u} \\
%		&= \cos((\theta \div 2) + (\tau \div 2)) + \sin((\theta \div 2) + (\theta \div 2))\hat{u} \\
%		&= \e^{\frac{\theta + \tau}{2}\hat{u}},
%	\end{align*}
%portanto temos que
%	\begin{equation*}
%	R\inv(R^{\theta}_{\hat{u}}) = \{\e^{\frac{\theta}{2}\hat{u}},-\e^{\frac{\theta + \tau}{2}\hat{u}}\}.
%	\end{equation*}
% MAS nesse caso $(\theta + \tau) \div 2 \in \intff{0}{\tau \div 2}$





As funções $\proj_{\parallel u}$, $\proj_{\perp u}$ e $\pvec{u\!}{}$ são funções lineares e, na base canônica de $\R^3$, são dadas pelas matrizes
	\begin{equation*}
	[\proj_{\parallel u}] =
		\begin{bmatrix}
			{u_1}^2 & u_1u_2 & u_1u_3 \\
			u_1u_2 & {u_2}^2 & u_2u_3 \\
			u_1u_3 & u_2u_3 & {u_3}^2
		\end{bmatrix},
	\end{equation*}
	\begin{equation*}
	[\proj_{\perp u}] =
		\begin{bmatrix}
			1-{u_1}^2 & -u_1u_2 & -u_1u_3 \\
			-u_1u_2 & 1-{u_2}^2 & -u_2u_3 \\
			-u_1u_3 & -u_2u_3 & 1-{u_3}^2
		\end{bmatrix}
	\end{equation*}
e
	\begin{equation*}
	[\pvec{u\!}{}] =
		\begin{bmatrix}
			0 & -u_3 & u_2 \\
			u_3 & 0 & -u_1 \\
			-u_2 & u_1 & 0
		\end{bmatrix}.
	\end{equation*}

Matricialmente, $[R_u]$ é dada por
%	\begin{align*}
%	[R_u] &=
%	\begin{bmatrix}
%			{u_1}^2 & u_1u_2 & u_1u_3 \\
%			u_1u_2 & {u_2}^2 & u_2u_3 \\
%			u_1u_3 & u_2u_3 & {u_3}^2
%	\end{bmatrix} \\
%	&\qquad+\cos(\theta)
%	\begin{bmatrix}
%		1-{u_1}^2 & -u_1u_2 & -u_1u_3 \\
%		-u_1u_2 & 1-{u_2}^2 & -u_2u_3 \\
%		-u_1u_3 & -u_2u_3 & 1-{u_3}^2
%	\end{bmatrix}
%	+ \sin(\theta)
%	\begin{bmatrix}
%			0 & -u_3 & u_2 \\
%			u_3 & 0 & -u_1 \\
%			-u_2 & u_1 & 0
%	\end{bmatrix}
%	\end{align*}
%ou
%	\begin{equation*}
%	(1-\cos(\theta))
%	\begin{bmatrix}
%			{u_1}^2 & u_1u_2 & u_1u_3 \\
%			u_1u_2 & {u_2}^2 & u_2u_3 \\
%			u_1u_3 & u_2u_3 & {u_3}^2
%	\end{bmatrix}
%	+
%	\cos(\theta)\Id + \sin(\theta)
%	\begin{bmatrix}
%			0 & -u_3 & u_2 \\
%			u_3 & 0 & -u_1 \\
%			-u_2 & u_1 & 0
%	\end{bmatrix}
%	\end{equation*}
%
	\begin{align*}
	[R_u] &= [\proj_{\parallel u}] + \cos(\theta)[\proj_{\perp u}] + \sin(\theta)[\pvec{u\!}{}] \\
	&=
	(1-\cos(\theta))
	\begin{bmatrix}
			{u_1}^2 & u_1u_2 & u_1u_3 \\
			u_1u_2 & {u_2}^2 & u_2u_3 \\
			u_1u_3 & u_2u_3 & {u_3}^2
	\end{bmatrix}
	+
	\cos(\theta)\Id + \sin(\theta)
	\begin{bmatrix}
			0 & -u_3 & u_2 \\
			u_3 & 0 & -u_1 \\
			-u_2 & u_1 & 0
	\end{bmatrix}
	\end{align*}







\begin{comment}

\subsubsection{Rotações em \ensuremath{\R^3} por quatérnios [Antiga]}

A rotação de $v \in \R^3$ por um ângulo $\theta$ em torno de um vetor unitário $u \in \R^3$ é dada por
	\begin{equation*}
	R^\theta_u(v) = \proj_{\parallel u}(v) + \cos(\theta) \proj_{\perp u}(v) + \sin(\theta) u \times v.
	\end{equation*}
%ou
%	\begin{equation*}
%	R^\theta_u(v) =(1-\cos(\theta)) \proj_{\parallel u}(v) + \cos(\theta) \Id + \sin(\theta) u \times v.
%	\end{equation*}

Note que genericamente $(\proj_{\parallel u}(v),\proj_{\perp u}(v),u \times v)$ é uma base de $\R^3$. %se $u$ e $v$ são ...
Isso significa que
	\begin{equation*}
	R^\theta_u = \proj_{\parallel u} + \cos(\theta) \proj_{\perp u} + \sin(\theta){ u \times} .
	\end{equation*}
As funções $\proj_{\parallel u}$, $\proj_{\perp u}$ e $u \times$ são funções lineares e, na base canônica de $\R^3$, são dadas pelas matrizes
	\begin{equation*}
	[\proj_{\parallel u}] =
	\begin{bmatrix}
		{u_1}^2 & u_1u_2 & u_1u_3 \\
		u_1u_2 & {u_2}^2 & u_2u_3 \\
		u_1u_3 & u_2u_3 & {u_3}^2
	\end{bmatrix}
	\end{equation*}
	\begin{equation*}
	[\proj_{\perp u}] =
	\begin{bmatrix}
		1-{u_1}^2 & -u_1u_2 & -u_1u_3 \\
		-u_1u_2 & 1-{u_2}^2 & -u_2u_3 \\
		-u_1u_3 & -u_2u_3 & 1-{u_3}^2
	\end{bmatrix}
	\end{equation*}
e
	\begin{equation*}
	[u \times] =
	\begin{bmatrix}
			0 & -u_3 & u_2 \\
			u_3 & 0 & -u_1 \\
			-u_2 & u_1 & 0
	\end{bmatrix}
	\end{equation*}




Os quatérnios unitários são os elementos de $\S^3 \subseteq \R^4$ e $\SO(3)$ é o grupo de rotações de $\R^3$. Definimos a função
	\begin{align*}
	\func{R}{\S^3}{\SO(3)}{q}{
		\begin{aligned}[t]
		\func{R_q}{\R^3}{\R^3}{v}{qvq\inv}.
		\end{aligned}
		}
	\end{align*}
Mostremos que essa função está bem definida. Precisamos mostrar que $qvq\inv \in \R^3$ e que $R_q$ é uma rotação. Primeiro, seja $q \in \S^3$. Então, como $\nor{q}=1$,
	\begin{equation*}
	q\inv = \esc{q}-\vec{q}.
	\end{equation*}
Como $v \in \R^3$ é um quatérnio vetorial puro, temos
	\begin{equation*}
	\esc{(vq\inv)} = \esc{v}\esc{q} - \inte{\vec{v}}{-\vec{q}} = \inte{\vec{v}}{\vec{q}},
	\end{equation*}
	\begin{equation*}
	\vec{(vq\inv)} = \esc{v}(-\vec{q})+\esc{q}\vec{v}+\vec{v} \times (-\vec{q}) = \esc{q}\vec{v}-\vec{v} \times \vec{q},
	\end{equation*}
e, portanto,
	\begin{align*}
	\esc{(qvq\inv)} &= \esc{q}\esc{(vq\inv)} - \inte{\vec{q}}{\vec{(vq\inv)}} \\
		&= \esc{q}\inte{\vec{v}}{\vec{q}} - \inte{\vec{q}}{\esc{q}\vec{v}-\vec{v} \times \vec{q}} \\
		&= \esc{q}\inte{\vec{v}}{\vec{q}}-\esc{q}\inte{\vec{q}}{\vec{v}}+\inte{\vec{q}}{\vec{v} \times \vec{q}} \\
		&= \esc{q}\inte{\vec{v}}{\vec{q}}-\esc{q}\inte{\vec{v}}{\vec{q}} \\
		&= 0,
	\end{align*}
o que mostra que $qvq\inv \in \R^3$, ou seja, é um quatérnio vetorial puro.

Para cada $q \in \S^3$, essa função é linear, pois, para todos $c \in \R$ e $v,v' \in \R^3$, segue da bilinearidade e da associatividade do produto e da comutatividade com escalares que
	\begin{align*}
	R_q(cv+v') &= q(cv+v')q\inv \\
		&= q(cvq\inv+v'q\inv) \\
		&= qcvq\inv + qv'q\inv \\
		&= cqvq\inv + qv'q\inv \\
		&= cR_q(v) + R_q(v').
	\end{align*}
A função $R_q$ é uma isometria, pois
	\begin{equation*}
	\nor{R_q(v)} = \nor{qvq\inv} = \nor{q}\nor{v}\nor{q\inv} = \nor{q}\nor{q}\inv\nor{v} = \nor{v}.
	\end{equation*}

Por fim, notemos que
	\begin{align*}
	qvq\inv &= \vec{(qvq\inv)} \\
		&= \esc{q}\vec{(vq\inv)}+\esc{(vq\inv)}\vec{q}+\vec{q} \times \vec{(vq\inv)} \\
		&= \esc{q}(\esc{q}\vec{v}-\vec{v}\times \vec{q})+\inte{\vec{v}}{\vec{q}}\vec{q}+\vec{q} \times (\esc{q}\vec{v}-\vec{v}\times \vec{q}) \\
		&= \esc{q}\esc{q}\vec{v}-\esc{q}\vec{v}\times \vec{q} + \inte{\vec{v}}{\vec{q}}\vec{q} + \esc{q} \vec{q} \times \vec{v} - \vec{q} \times (\vec{v}\times \vec{q}) \\
		&= \esc{q}\esc{q}\vec{v}+2\esc{q}\vec{q}\times \vec{v} + \inte{\vec{v}}{\vec{q}}\vec{q} - \inte{\vec{q}}{\vec{q}}\vec{v} + \inte{\vec{q}}{\vec{v}}\vec{q} \\
		&= (\esc{q}\esc{q} - \inte{\vec{q}}{\vec{q}})\vec{v}+2\esc{q}\vec{q}\times \vec{v} + 2\inte{\vec{v}}{\vec{q}}\vec{q}.
	\end{align*}

Como $\nor{q}^2 = \esc{q}^2 + \nor{\vec{q}}^2$, podemos tomar $\theta \in \intff{0}{\tau}$ tal que
	\begin{equation*}
	\esc{q} = \cos(\theta \div 2)
	\end{equation*}
e
	\begin{equation*}
	\nor{\vec{q}} = \sin(\theta \div 2).
	\end{equation*}
Definindo
	\begin{equation*}
	u := \frac{\vec{q}}{\sin(\theta \div 2)},
	\end{equation*}
temos $\nor{u}=1$ e
	\begin{equation*}
	q=\cos(\theta \div 2)+\sin(\theta \div 2)u = \e^{(\theta \div 2)u}.
	\end{equation*}

Segue então que
	\begin{align*}
	qvq\inv
%		&= \left(\left(\cos\frac{\theta}{2}\right)^2-\left(\sin\frac{\theta}{2}\right)^2\right)v + 2\cos\frac{\theta}{2}\sin\frac{\theta}{2}{u \times v} + 2\left(\sin\frac{\theta}{2}\right)^2\inte{v}{u}u \\
		&= (\cos(\theta \div 2)^2-\sin(\theta \div 2)^2)v + 2\cos(\theta \div 2)\sin(\theta \div 2){u \times v} + 2\sin(\theta \div 2)^2\inte{v}{u}u \\
		&= \cos(\theta) v + \sin(\theta) u \times v + (1-\cos(\theta))\inte{v}{u}u \\
		&= \inte{v}{u}u + \cos(\theta) (v-\inte{v}{u}u) + \sin(\theta) u \times v \\
		&= \proj_{\parallel u}(v) + \cos(\theta) \proj_{\perp u}(v) + \sin(\theta) u \times v \\
		&= R^\theta_u(v).
	\end{align*}

%	\begin{align*}
%	\nor{qvq\inv} &= \left(\nor{\proj_{\parallel u}(v)}^2 + \nor{\cos(\theta) \proj_{\perp u}(v)}^2 + \nor{\sin(\theta) u \times v}^2\right)^{\frac{1}{2}} \\
%		&= \left( (1-\cos(\theta))^2\cos(\alpha)^2\nor{v}^2 + \cos(\theta)^2 \nor{v}^2 + \sin(\theta)^2 \sin(\alpha)^2 \nor{v}^2 \right) \\
%		&= \nor{v}\left( \cos(\alpha)^2-2\cos(\theta)\cos(\alpha)^2+\cos(\theta)^2\cos(\alpha)^2 + \cos(\theta)^2 + \sin(\theta)^2 \sin(\alpha)^2 \right) \\
%	\end{align*}





%%%%%%%%%%%%%%%%%%%%%%%%%%%%%%%%%%%%%%%%%%%%

%Explicitamente, a matriz de $R_q$ na base canônica é
%	\begin{equation*}
%	[R_q] =
%	\begin{bmatrix}
%		1-2({q_2}^2+{q_3}^2) & 2(q_1q_2-q_3q_0) & 2(q_1q_3+q_2q_0) \\
%		2(q_1q_2+q_3q_0) & 1-2({q_1}^2+{q_3}^2) & 2(q_2q_3-q_1q_0) \\
%		2(q_1q_3-q_2q_0) & 2(q_2q_3+q_1q_0) & 1-2({q_1}^2+{q_2}^2).
%	\end{bmatrix}
%	\end{equation*}


%	\begin{equation*}
%	\begin{bmatrix}
%			1-2\sin(\theta \div 2)^2({u_2}^2+{u_3}^2) & 2(u_1u_2\sin(\theta \div 2)^2-u_3\sin(\theta \div 2)\cos(\theta \div 2)) & 2(u_1u_3\sin(\theta \div 2)^2+u_2\sin(\theta \div 2)\cos(\theta \div 2)) \\
%			2(u_1u_2\sin(\theta \div 2)^2+u_3\sin(\theta \div 2)\cos(\theta \div 2)) & 1-2\sin(\theta \div 2)^2({u_1}^2+{u_3}^2) & 2(u_2u_3\sin(\theta \div 2)^2-u_1\sin(\theta \div 2)\cos(\theta \div 2)) \\
%			2(u_1u_3\sin(\theta \div 2)^2-u_2\sin(\theta \div 2)\cos(\theta \div 2)) & 2(u_2u_3\sin(\theta \div 2)^2+u_1\sin(\theta \div 2)\cos(\theta \div 2)) & 1-2\sin(\theta \div 2)^2({u_1}^2+{u_2}^2)
%	\end{bmatrix}
%	\end{equation*}

%Definindo $c:= \cos(\theta \div 2)$ e $s:=\sin(\theta \div 2)$ e tomando $q=\e^{\frac{\theta}{2}u}=c+su$, temos
%	\begin{equation*}
%	\begin{bmatrix}
%			1-2s^2({u_2}^2+{u_3}^2) & 2(u_1u_2s^2-u_3sc) & 2(u_1u_3s^2+u_2sc) \\
%			2(u_1u_2s^2+u_3sc) & 1-2s^2({u_1}^2+{u_3}^2) & 2(u_2u_3s^2-u_1sc) \\
%			2(u_1u_3s^2-u_2sc) & 2(u_2u_3s^2+u_1sc) & 1-2s^2({u_1}^2+{u_2}^2)
%	\end{bmatrix}
%	\end{equation*}
%Mas ${u_2}^2+{u_3}^2 = 1-{u_1}^2$, etc...
%	\begin{equation*}
%	\begin{bmatrix}
%			1-2s^2(1-{u_1}^2) & 2(u_1u_2s^2-u_3sc) & 2(u_1u_3s^2+u_2sc) \\
%			2(u_1u_2s^2+u_3sc) & 1-2s^2(1-{u_2}^2) & 2(u_2u_3s^2-u_1sc) \\
%			2(u_1u_3s^2-u_2sc) & 2(u_2u_3s^2+u_1sc) & 1-2s^2(1-{u_3}^2)
%	\end{bmatrix}
%	\end{equation*}
%e, definindo $C:= \cos\theta$ e $S:=\sin\theta$, temos $2s^2 = 1-C$ e $2sc=S$, logo
%	\begin{equation*}
%	\begin{bmatrix}
%			1-(1-C)(1-{u_1}^2) & u_1u_2(1-C)-u_3S & u_1u_3(1-C)+u_2S \\
%			u_1u_2(1-C)+u_3S & 1-(1-C)(1-{u_2}^2) & u_2u_3(1-C)-u_1S \\
%			u_1u_3(1-C)-u_2S & u_2u_3(1-C)+u_1S & 1-(1-C)(1-{u_3}^2)
%	\end{bmatrix}
%	\end{equation*}
%	\begin{equation*}
%	\begin{bmatrix}
%			{u_1}^2(1-C)+C & u_1u_2(1-C)-u_3S & u_1u_3(1-C)+u_2S \\
%			u_1u_2(1-C)+u_3S & {u_2}^2(1-C)+C & u_2u_3(1-C)-u_1S \\
%			u_1u_3(1-C)-u_2S & u_2u_3(1-C)+u_1S & {u_3}^2(1-C)+C
%	\end{bmatrix}
%	\end{equation*}

%	\begin{equation*}
%	(1-C)
%	\begin{bmatrix}
%			{u_1}^2 & u_1u_2 & u_1u_3 \\
%			u_1u_2 & {u_2}^2 & u_2u_3 \\
%			u_1u_3 & u_2u_3 & {u_3}^2
%	\end{bmatrix}
%	+
%	C\Id + S
%	\begin{bmatrix}
%			0 & -u_3 & u_2 \\
%			u_3 & 0 & -u_1 \\
%			-u_2 & u_1 & 0
%	\end{bmatrix}
%	\end{equation*}

%%%%%%%%%%%%%%%%%%%%%%%%%%%%%%%%%%%%%%%%%%%%

Matricialmente, $[R_q]$ é dada por
	\begin{equation*}
	\begin{bmatrix}
			{u_1}^2 & u_1u_2 & u_1u_3 \\
			u_1u_2 & {u_2}^2 & u_2u_3 \\
			u_1u_3 & u_2u_3 & {u_3}^2
	\end{bmatrix}
	+\cos(\theta)
	\begin{bmatrix}
		1-{u_1}^2 & -u_1u_2 & -u_1u_3 \\
		-u_1u_2 & 1-{u_2}^2 & -u_2u_3 \\
		-u_1u_3 & -u_2u_3 & 1-{u_3}^2
	\end{bmatrix}
	+ \sin(\theta)
	\begin{bmatrix}
			0 & -u_3 & u_2 \\
			u_3 & 0 & -u_1 \\
			-u_2 & u_1 & 0
	\end{bmatrix}
	\end{equation*}
ou
	\begin{equation*}
	(1-\cos(\theta))
	\begin{bmatrix}
			{u_1}^2 & u_1u_2 & u_1u_3 \\
			u_1u_2 & {u_2}^2 & u_2u_3 \\
			u_1u_3 & u_2u_3 & {u_3}^2
	\end{bmatrix}
	+
	\cos(\theta)\Id + \sin(\theta)
	\begin{bmatrix}
			0 & -u_3 & u_2 \\
			u_3 & 0 & -u_1 \\
			-u_2 & u_1 & 0
	\end{bmatrix}
	\end{equation*}

\end{comment}










\subsubsection{Rotações em \ensuremath{\R^4}}

\begin{definition}
A função
	\begin{align*}
	\func{R}{\S^3 \times \S^3}{\SO(4)}{(u,v)}{
		\begin{aligned}[t]
		\func{R{u,v}}{\R^4}{\R^4}{x}{uxv\inv}
		\end{aligned}
	}
	\end{align*}
é um homomorfismo sobrejetivo de grupos tal que $\nuc(R) = (1,1)\S^0$.
\end{definition}
\begin{proof}
A função $R$ é diferenciável porque $R_u(x)$ é um polinômio em cada entrada. Primeiro, vamos mostrar que $R$ de fato está bem definida, ou seja, que, para todos $u,v \in \S^3$, $R_{u,v} \in \SO(4)$. Sejam $u,v \in \S^3$. A função $R_{u,v}$ é linear, pois, para todos $x,x' \in \vec{\R}^3$ e $c \in \R$, segue da bilinearidade do produto quaterniônico que
	\begin{align*}
	R_{u,v}(cx+x') &= u(cx+x')v\inv \\
		&= (cux+ux')v\inv \\
		&= cuxv\inv + ux'v\inv \\
		&= cR_{u,v}(x) + R_{u,v}(x').
	\end{align*}
Isso mostra que $R_{u,v} \in \GL(4,\R)$. Além disso, $R_{u,v}$ preserva norma, pois
	\begin{equation*}
	\nor{R_{u,v}(x)} = \nor{uxv\inv} = \nor{u}\nor{x}\nor{v}\inv = \nor{x}.
	\end{equation*}
Isso mostra que $R_{u,v} \in \GO(4)$. Agora, como $R$ é diferenciável, em particular é contínua, logo, como $\S^3 \times \S^3$ é conexo, $R(\S^3 \times \S^3)$ é conexo. Isso implica que $R(\S^3 \times \S^3) \subseteq \SO(4)$.

Sejam $u,v \in \S^3$ tais que $R_{u,v} = \Id$. Então, para todo $x \in \R^4$, $uxv\inv = R_{u,v}(x) = x$. Em particular, $u1v\inv = 1$, portanto $u=v$ e $R_{u,v}(x) = uxu\inv$. Suponhamos por absurdo que $\vec{u} \neq 0$. Então existiria $\vec{x} \in \vec{\R}^3$ tal que $\pesc{\vec{u}}{\vec{x}} = 0$, de modo que $\{\vec{u},\vec{x},\pvec{\vec{u}}{\vec{x}}\}$ seria uma base de $\vec{\R}^3$. Pela fórmula de conjugação por quatérnio, seguiria que
	\begin{equation*}
	\vec{u} = u\vec{u}u\inv = 2(\pesc{\vec{u}}{\vec{u}})\vec{u} + \inteq{u}{u}\vec{u} + 2 \esc{u}(\pvec{\vec{u}}{\vec{u}}) = (2+\inteq{u}{u})\vec{u},
	\end{equation*}
portanto $\inteq{u}{u} = -1$, e que
	\begin{equation*}
	\vec{x} = u\vec{x}u\inv = 2(\pesc{\vec{u}}{\vec{x}})\vec{u} + \inteq{u}{u}\vec{x} + 2 \esc{u}(\pvec{\vec{u}}{\vec{x}}) = \inteq{u}{u}\vec{x} + 2\esc{u}(\pvec{\vec{u}}{\vec{x}}).
	\end{equation*}
portanto $\inteq{u}{u}=1$, o que é uma contradição. Isso mostra que $\vec{u} = 0$, portanto que $u = \esc{u} \in \esc{\R}$; como $u \in \S^3$, concluímos que $u \in \S^0$, ou seja, $\nuc{R} = \{(1,1),(-1,-1)\} = (1,1)\S^0$.

Para a sobrejetividade, primeiro notamos que, como $R(\S^3 \times \S^3) \subseteq \SO(4)$ é compacto, pois $\S^3 \times \S^3$ é compacto, e $\SO(4)$ é separado\footnote{Hausdorff}, segue que $R(\S^3 \times \S^3)$ é fechado. Além disso, $R$ é difeomorfismo local, pois o núcleo $(1,1)\S^0$ é um subgrupo discreto (portanto tem álgebra de Lie trivial), logo $R$ é difeomorfismo local na identidade, e portanto em todo $\S^3 \times \S^3$, já que um morfismo de grupos diferenciáveis\footnote{Grupos de lie} tem posto constante. Assim, segue que $R(\S^3 \times \S^3) \subseteq \SO(4)$ é aberto. Como $\SO(4)$ é conexo, segue que $R(\S^3 \times \S^3) = \SO(4)$.
\end{proof}



\begin{comment}

Lembremos que, pela decomposição polar, todo $u \in \S^3$ pode ser escrito como
	\begin{equation*}
	u = \cos(\phi) + \sin(\phi)\hat{u} = \e^{\phi \hat{u}},
	\end{equation*}
com $\phi = \cos(\esc{u}) \in \intff{0}{\tau \div 2}$ e $\hat{u} = \frac{\vec{u}}{\nor{\hat{u}}} \in \vec{\S}^2$.





Definimos a função
	\begin{align*}
	\func{R}{\S^3 \times \S^3}{\SO(3)}{(u,u')}{
		\begin{aligned}[t]
		\func{R^{2\phi}_{\hat{u}}}{\vec{\R}^3}{\vec{\R}^3}{x}{\e^{\phi \hat{u}}x\e^{-\phi \hat{u}}}
		\end{aligned}
	}
	\end{align*}

\end{comment}




\subsection{Classificação das álgebras de divisão reais}

Uma álgebra de divisão associativa real é uma álgebra associativa, unitária e invertível sobre $\R$. Nessa seção usaremos o teorema fundamental da álgebra, o teorema de Cayley-Hamilton e propriedades de produto interno, ou seja, $2$-formas simétricas positivo-definidas, como a existência de uma base ortonormal.

\begin{proposition}
Seja $\bm A$ uma álgebra de divisão associativa sobre $\R$. Então $\bm A$ é isomorfa a $\R$, $\R^2$ ou $\R^4$.
\end{proposition}
\begin{proof}
Primeiro identificamos $\R$ com $\R 1 = \set{c1}{c \in \R}$, em que $1 \in A$ é a unidade de $A$, e, conforme essa identificação, para todos $a,a' \in \R$, consideramos a ordem induzida $a \leq a'$. Mostraremos que $V := \set{a \in A}{a^2 \in \R \land a^2 \leq 0}$ é um subespaço vetorial de $\bm A$ e $\bm A = \R \oplus \bm V$.

Defina $d := \dim(A)$. Para todo $a \in A$, a função multiplicação à esquerda
	\begin{align*}
	\func{a\times}{A}{A}{x}{ax}
	\end{align*}
é uma função linear e, da associatividade de $\times$, segue que, para todos $a,a' \in A$, $(a\times) \circ (a' \times) = (aa')\times$. Seja $p_a$ o polinômio característico de $a\times$. pelo teorema fundamental da álgebra, existem números reais $(c_k)_{k \in [n]}$ e números complexos $(z_k)_{k \in [n']}$ tais que
	\begin{equation*}
	p_a(x) = (x-c_0) \cdots (x-c_{n-1}) (x-z_0)(x-\conju{(z_0)}) \cdots (x-z_{n-1})(x-\conju{(z_{n-1})}),
	\end{equation*}
e, para todo $k \in [n']$, o polinômio $(x-z_k)(x-\conju{(z_k)})$ é irredutível em $\R$. Pelo teorema de Cayley-Hamilton, segue que $p_a(a\times) = 0$. Da linearidade de $a\times$ e de $(a\times) \circ (a' \times) = (aa')\times$ segue que
	\begin{equation*}
	p_a(a\times) = p_a(a)\times.
	\end{equation*}
Como $\bm A$ é álgebra de divisão, segue que $p_a(a) = 0$ e que, para algum $k \in [n]$, $a-c_k = 0$ ou, para algum $k \in [n']$, $(a-z_k)(a-\conju{(z_k)}) = 0$. No primeiro caso, segue que $a=c_k \in \R$ e, no segunda caso,
	\begin{equation*}
	\mu_a(x) := (x-z_k)(x-\conju{(z_k)}) = x^2 - 2\Re(z_k)x + \abs{z_k}^2
	\end{equation*}
é o polinômio minimal de $a\times$. Como $p$ é real e tem as mesmas raízes complexas que $\mu_a$, segue que, para algum $p \in \Z_{>}$,
	\begin{equation*}
	p_a(x) = (\mu_a(x))^p = (x^2 - 2\Re(z_k)x + \abs{z_k}^2)^p.
	\end{equation*}
Como $p_a$ é polinômio característico de $a\times$, o coeficiente de $x^{2p-1}$ em $p_a$ é $-\tr{a\times}$ (inverso aditivo do traço de $a \times$). Mas esse coeficiente é $0$ se, e somente se, $\Re(z_k)=0$, o que por sua vez é equivalente a $a^2 = -\abs{z_k}^2$. Com $-\abs{z_k}^2 < 0$, concluímos que
	\begin{equation*}
	V = \set{a \in A}{\tr{a\times}=0},
	\end{equation*}
portanto é um subespaço linear de $\bm A$ e tem codimensão $1$, pois é o núcleo do traço $\tr{\var}$, um funcional linear não nulo, e conclui-se que $\bm A = \R \oplus \bm V$.

Isso implica que, para todo $v \in V$, $v^2 \in \R$e segue que, para todos $v,v' \in A$, $vv'+v'v = (v+v')^2 - v^2 - (v')^2 \in \R$. Assim, definimos
	\begin{align*}
	\func{\inte{\var}{\var}}{V \times V}{\R}{(v,v')}{-\frac{vv'+v'v}{2}}.
	\end{align*}
Essa função é uma $2$-forma simétrica positiva definida em $V$:
	\begin{itemize}
		\item (Bilinearidade) Segue da bilinearidade da multiplicação;
		\item (Simetria) Para todo $v \in V$,
			\begin{equation*}
			\inte{v}{v'} = -\frac{vv'+v'v}{2} = -\frac{v'v+vv'}{2} = \inte{v'}{v};
			\end{equation*}
		\item (Positiva-definida) Para todo $v \in V$, $v^2 \leq 0$, logo
			\begin{equation*}
			\inte{v}{v} = -\frac{vv+vv}{2} = -v^2 \geq 0.
			\end{equation*}
	\end{itemize}
Seja $V'$ o subespaço de $V$ que gera $A$ como álgebra e seja $(b_k)_{k \in [n]}$ uma base ortonormal de $V'$ com respeito a $\inte{\var}{\var}$. A ortonormalidade implica que, para todo $k \in [n]$,
	\begin{equation*}
	{b_k}^2 = -\inte{b_k}{b_k} = -1
	\end{equation*}
e, para todos $k,k' \in [n]$ distintos,
	\begin{align*}
	b_kb_{k'} &= b_kb_{k'} + 0 \\
		&= b_kb_{k'} + 2\inte{b_k}{b_{k'}} \\
		&= b_kb_{k'} -(b_kb_{k'}+b_{k'}b_k) \\
		&= -b_{k'}b_k.
	\end{align*}
Disso segue que, para todos $k,k' \in [n]$ distintos,
	\begin{equation*}
	(b_kb_{k'})^2 = b_kb_{k'}b_kb_{k'} = -b_{k'}b_kb_kb_{k'} = -b_{k'}(-1)b_{k'} = b_{k'}b_{k'} = -1
	\end{equation*}
e, para todos $k,k',k'' \in [n]$ distintos dois a dois,
	\begin{align*}
	(b_kb_{k'}b_{k''})^2 &= b_kb_{k'}b_{k''}b_kb_{k'}b_{k''} \\
		&= -b_kb_{k'}b_kb_{k''}b_{k'}b_{k''} \\
		&= (b_kb_k)(b_{k'}b_{k''})(b_{k'}b_{k''}) \\
		&= -(b_{k'}b_{k''})^2 \\
		&= 1.
	\end{align*}
Isso implica que $n \in \{0,1,2\}$. Suponhamos, por absurdo, que $n\geq 3$, e seja $u := b_0b_1b_2$. Pelas relações anteriormente calculadas, seguiria que $u^2 = 1$, portanto $0 = u^2 - 1 = (u+1)(u-1)$. Como $A$ é invertível, valeria $u+1=0$ ou $u-1=0$, o que implicaria que $b_2=b_0b_1$ ou $b_2=-b_0b_1$, o que seria uma contradição com a minimalidade de $V'$.
Consideramos por fim cada um dos 3 casos.
\begin{itemize}
	\item Se $n=0$, segue que $V=\{0\}$ e portanto que $A \simeq \R$.
	\item Se $n=1$, $V=\set{c_0b_0}{c_0 \in \R}$ e segue que $A \simeq \R^2$ pelo isomorfismo
		\begin{align*}
		\func{h}{A}{\R^2}{c+c_0b_0}{(c,c_0)}.
		\end{align*}
	\item Se $n=2$, $V=\set{c_0b_0 + c_1b_1 + c_2(b_0b_1)}{c_0,c_1,c_2 \in \R}$ e $A \simeq \R^4$ pelo isomorfismo
		\begin{align*}
		\func{h}{A}{\R^4}{c+c_0b_0 + c_1b_1 + c_2(b_0b_1)}{(c,c_0,c_1,c_2)}.
		\end{align*}
\end{itemize}
\end{proof}





\subsection{Octônios \texorpdfstring{$\R^8$}{}}

\subsubsection{Multiplicação octoniônica}

Consideremos o espaço linear $\R^8$ com a norma $2$
	\begin{align*}
	\func{\nor{\var}}{\R^8}{\R}{x}{({x_0}^2+{x_1}^2+{x_2}^2+{x_3}^2+{x_4}^2+{x_5}^2+{x_6}^2+{x_7}^2)^{\frac{1}{2}}}.
	\end{align*}

A base canônica de $\R^8$ será denotada por
	\begin{align*}
	\ii_0 &:= (1,0,0,0,0,0,0,0) \\
	\ii_1 &:= (0,1,0,0,0,0,0,0) \\
	\ii_2 &:= (0,0,1,0,0,0,0,0) \\
	\ii_3 &:= (0,0,0,1,0,0,0,0) \\
	\ii_4 &:= (0,0,0,0,1,0,0,0) \\
	\ii_5 &:= (0,0,0,0,0,1,0,0) \\
	\ii_6 &:= (0,0,0,0,0,0,1,0) \\
	\ii_7 &:= (0,0,0,0,0,0,0,1).
	\end{align*}

\begin{definition}
A \emph{multiplicação octoniônica} em $\R^8$ é a operação
%	\begin{align*}
%	\fun{\times}{\R^8 \times \R^8}{\R^8}
%	\end{align*}
%definida pela tabela de multiplicação \ref{tab:multiplicacao.octonionica} na base de $\R^8$ e estendida linearmente.
%	\begin{align*}
%	\func{\times}{\R^8 \times \R^8}{\R^8}{(x,y)}{
%		\begin{aligned}[t]
%		xy := (&x_0y_0 - x_1y_1 - x_2y_2 - x_3y_3 - x_4y_4 - x_5y_5 - x_6y_6 - x_7y_7)1 \\
%			+ (&x_0y_1 + x_1y_0 + x_2y_3 - x_3y_2 + x_4y_5 + x_5y_4 - x_6y_7 + x_7y_6)\ii_1 \\
%			+ (&x_0y_2 - x_1y_3 + x_2y_0 + x_3y_1 + x_4y_6 - x_5y_7 - x_6y_4 - x_7y_5)\ii_2 \\
%			+ (&x_0y_3 + x_1y_2 - x_2y_1 + x_3y_0 + x_4y_7 - x_5y_6 + x_6y_5 - x_7y_4)\ii_3 \\
%			+ (&x_0y_4 - x_1y_5 - x_2y_6 - x_3y_7 + x_4y_0 + x_5y_1 + x_6y_2 + x_7y_3)\ii_4 \\
%			+ (&x_0y_5 + x_1y_4 - x_2y_7 + x_3y_6 - x_4y_1 + x_5y_0 - x_6y_3 + x_7y_2)\ii_5 \\
%			+ (&x_0y_6 + x_1y_7 + x_2y_4 - x_3y_5 - x_4y_2 + x_5y_3 + x_6y_0 - x_7y_1)\ii_6 \\
%			+ (&x_0y_7 - x_1y_6 + x_2y_5 + x_3y_4 - x_4y_3 + x_5y_2 + x_6y_1 + x_7y_0)\ii_7 .
%		\end{aligned}
%	}
%	\end{align*}
	\begin{align*}
	\func{\times}{\R^8 \times \R^8}{\R^8}{(x,y)}{
		\begin{aligned}[t]
		xy := (&x_0y_0 - x_1y_1 - x_2y_2 - x_3y_3 - x_4y_4 - x_5y_5 - x_6y_6 - x_7y_7, \\
				&x_0y_1 + x_1y_0 + x_2y_3 - x_3y_2 + x_4y_5 + x_5y_4 - x_6y_7 + x_7y_6, \\
				&x_0y_2 - x_1y_3 + x_2y_0 + x_3y_1 + x_4y_6 - x_5y_7 - x_6y_4 - x_7y_5, \\
				&x_0y_3 + x_1y_2 - x_2y_1 + x_3y_0 + x_4y_7 - x_5y_6 + x_6y_5 - x_7y_4, \\
				&x_0y_4 - x_1y_5 - x_2y_6 - x_3y_7 + x_4y_0 + x_5y_1 + x_6y_2 + x_7y_3, \\
				&x_0y_5 + x_1y_4 - x_2y_7 + x_3y_6 - x_4y_1 + x_5y_0 - x_6y_3 + x_7y_2, \\
				&x_0y_6 + x_1y_7 + x_2y_4 - x_3y_5 - x_4y_2 + x_5y_3 + x_6y_0 - x_7y_1, \\
				&x_0y_7 - x_1y_6 + x_2y_5 + x_3y_4 - x_4y_3 + x_5y_2 + x_6y_1 + x_7y_0) .
		\end{aligned}
	}
	\end{align*}
A \emph{unidade} de $(\R^8,\times)$ é
	\begin{equation*}
	1_{\R^8} := \ii_0 = (1,0,0,0,0,0,0,0).
	\end{equation*}
A \emph{inversão} de $(\R^8,\times,1)$ é a operação
	\begin{align*}
	\func{\div}{\R^8 \setminus \{0\}}{\R^8 \setminus \{0\}}{x}{x\inv := \nor{x}^{-2} (x_0, -x_1, -x_2, -x_3, -x_4, -x_5, -x_6, -x_7)}.
	\end{align*}
\end{definition}

Em geral, por não haver ambiguidade, denotaremos $1_{\R^8}$ simplesmente por $1$. Assim, todo $x \in \R^8$ pode ser escrito
	\begin{equation*}
	x = x_0 1 + x_1\ii_1 + x_2\ii_2 + x_3\ii_3 + x_4\ii_4 + x_5\ii_5 + x_6\ii_6 + x_7\ii_7.
	\end{equation*}
Como a notação acima sugere, omitiremos o símbolo de multiplicação $\times$ sempre que possível pois isso não gerará ambiguidade.

A multiplicação octoniônica, quando restrita à base canônica, é dada pela tabela \ref{tab:multiplicacao.octonionica} e estendida linearmente.

\begin{table}
	\centering

	\begin{tabular}{c | c c c c c c c c}
	\toprule
	$\times$&	$1$		&	$\ii_1$	&	$\ii_2$	&	$\ii_3$	&	$\ii_4$	&	$\ii_5$	&	$\ii_6$	&	$\ii_7$		\\
%	\midrule
	\hline
	$1$		&	$1$		&	$\ii_1$	&	$\ii_2$	&	$\ii_3$	&	$\ii_4$	&	$\ii_5$	&	$\ii_6$	&	$\ii_7$		\\
	$\ii_1$	&	$\ii_1$	&	$-1$	&	$\ii_3$	&	$-\ii_2$&	$\ii_5$	&	$-\ii_4$&	$-\ii_7$&	$\ii_6$		\\
	$\ii_2$	&	$\ii_2$	&	$-\ii_3$&	$-1$	&	$\ii_1$	&	$\ii_6$	&	$\ii_7$	&	$-\ii_4$&	$-\ii_5$	\\
	$\ii_3$	&	$\ii_3$	&	$\ii_2$	&	$-\ii_1$&	$-1$	&	$\ii_7$	&	$-\ii_6$&	$\ii_5$	&	$-\ii_4$	\\
	$\ii_4$	&	$\ii_4$	&	$-\ii_5$&	$-\ii_6$&	$-\ii_7$&	$-1$	&	$\ii_1$	&	$\ii_2$	&	$\ii_3$		\\
	$\ii_5$	&	$\ii_5$	&	$\ii_4$	&	$\ii_7$	&	$\ii_6$	&	$\ii_1$	&	$-1$	&	$-\ii_3$&	$-\ii_2$	\\
	$\ii_6$	&	$\ii_6$	&	$\ii_7$	&	$\ii_4$	&	$-\ii_5$&	$-\ii_2$&	$\ii_3$	&	$-1$	&	$-\ii_1$	\\
	$\ii_7$	&	$\ii_7$	&	$-\ii_6$&	$\ii_5$	&	$\ii_4$	&	$-\ii_3$&	$-\ii_2$&	$\ii_1$	&	$-1$		\\
	\bottomrule
	\end{tabular}

	\caption{Tabela de multiplicação dos octônios.}
	\label{tab:multiplicacao.octonionica}
\end{table}

Pela definição da multiplicação, vale que
%	\begin{equation*}
%	x = (x_0 1 + x_1\ii) + (x_2 + x_3\ii)\ddot{\text{\i}}_1 + (x_4 + x_5\ii)\ddot{\text{\i}} + (x_6 + x_7\ii)\ddot{\text{\i}}.
%	\end{equation*}
	\begin{align*}
	x &= x_0 + x_1\ii_1 + x_2\ii_2 + x_3\ii_3 + x_4\ii_4 + x_5\ii_5 + x_6\ii_6 + x_7\ii_7 \\
		&= (x_0 + x_1\ii_1 + x_2\ii_2 + x_3\ii_3) + (x_4 + x_5\ii_1 + x_6\ii_2 + x_7\ii_3)\ii_4 \\
		&= (x_0 + x_1\ii_1) + (x_2 + x_3\ii_1)\ii_2 + (x_4 + x_5\ii_1)\ii_4 + (x_6 - x_7\ii_1)\ii_6 ,
	\end{align*}
%de modo que valem os seguintes isomorfismos
%	\begin{align*}
%	\func{h}{\R^8}{(\R^4)^2}{x_0 + x_1\ii_1 + x_2\ii_2 + x_3\ii_3 + x_4\ii_4 + x_5\ii_5 + x_6\ii_6 + x_7\ii_7}{(x_0 + x_1\ii_1 + x_2\ii_2 + x_3\ii_3) + (x_4 + x_5\ii_1 + x_6\ii_2 + x_7\ii_3)\ii}
%	\end{align*}
%	\begin{align*}
%	\func{h}{\Oct}{\HH^2}{(x_0, x_1, x_2, x_3, x_4, x_5, x_6, x_7)}{((x_0, x_1, x_2, x_3), (x_4, x_5, x_6, x_7))}
%	\end{align*}
%	\begin{align*}
%	\func{h}{\R^8}{(\R^2)^4}{(x_0, x_1, x_2, x_3, x_4, x_5, x_6, x_7)}{((x_0, x_1), (x_2, x_3), (x_4, x_5), (x_6, -x_7))}
%	\end{align*}
%	\begin{align*}
%	\func{h}{\R^8}{(\R^2)^4}{(x_0, x_1, x_2, x_3, x_4, x_5, x_6, x_7)}{((x_0, x_1), (x_2, x_3), (x_4, x_5), (x_6, -x_7))}
%	\end{align*}

%	\begin{align*}
%	xy &= (x_0 + x_1\ii_1 + x_2\ii_2 + x_3\ii_3 + x_4\ii_4 + x_5\ii_5 + x_6\ii_6 + x_7\ii_7)\\
%		&\times(y_0 + y_1\ii_1 + y_2\ii_2 + y_3\ii_3 + y_4\ii_4 + y_5\ii_5 + y_6\ii_6 + y_7\ii_7) \\
%		&= (x_0y_0 - x_1y_1 - x_2y_2 - x_3y_3 - x_4y_4 - x_5y_5 - x_6y_6 - x_7y_7)1 \\
%		&+ (x_0y_1 + x_1y_0 + x_2y_3 - x_3y_2 + x_4y_5 + x_5y_4 - x_6y_7 + x_7y_6)\ii_1 \\
%		&+ (x_0y_2 - x_1y_3 + x_2y_0 + x_3y_1 + x_4y_6 - x_5y_7 - x_6y_4 - x_7y_5)\ii_2 \\
%		&+ (x_0y_3 + x_1y_2 - x_2y_1 + x_3y_0 + x_4y_7 - x_5y_6 + x_6y_5 - x_7y_4)\ii_3 \\
%		&+ (x_0y_4 - x_1y_5 - x_2y_6 - x_3y_7 + x_4y_0 + x_5y_1 + x_6y_2 + x_7y_3)\ii_4 \\
%		&+ (x_0y_5 + x_1y_4 - x_2y_7 + x_3y_6 - x_4y_1 + x_5y_0 - x_6y_3 + x_7y_2)\ii_5 \\
%		&+ (x_0y_6 + x_1y_7 + x_2y_4 - x_3y_5 - x_4y_2 + x_5y_3 + x_6y_0 - x_7y_1)\ii_6 \\
%		&+ (x_0y_7 - x_1y_6 + x_2y_5 + x_3y_4 - x_4y_3 + x_5y_2 + x_6y_1 + x_7y_0)\ii_7 .
%	\end{align*}


\begin{comment}

\begin{table}
	\centering

	\begin{tabular}{c | c c c c c c c c}
	\toprule
	$\times$&	$1$		&	$\ii_1$	&	$\ii_2$	&	$\ii_3$	&	$\ii_4$	&	$\ii_5$	&	$\ii_6$	&	$\ii_7$		\\
%	\midrule
	\hline
	$1$		&	$1$		&	$\ii_1$	&	$\ii_2$	&	$\ii_3$	&	$\ii_4$	&	$\ii_5$	&	$\ii_6$	&	$\ii_7$		\\
	$\ii_1$	&	$\ii_1$	&	$-1$	&	$\ii_3$	&	$-\ii_2$&	$\ii_5$	&	$-\ii_4$&	$\ii_7$	&	$-\ii_6$	\\
	$\ii_2$	&	$\ii_2$	&	$-\ii_3$&	$-1$	&	$-\ii_1$&	$\ii_6$	&	$-\ii_7$&	$-\ii_4$&	$\ii_5$		\\
	$\ii_3$	&	$\ii_3$	&	$\ii_2$	&	$\ii_1$	&	$-1$	&	$-\ii_7$&	$-\ii_6$&	$\ii_5$	&	$\ii_4$		\\
	$\ii_4$	&	$\ii_4$	&	$-\ii_5$&	$-\ii_6$&	$\ii_7$	&	$-1$	&	$-\ii_1$&	$\ii_2$	&	$-\ii_3$	\\
	$\ii_5$	&	$\ii_5$	&	$\ii_4$	&	$\ii_7$	&	$\ii_6$	&	$\ii_1$	&	$-1$	&	$-\ii_3$&	$-\ii_2$	\\
	$\ii_6$	&	$\ii_6$	&	$-\ii_7$&	$\ii_4$	&	$\ii_5$	&	$-\ii_2$&	$\ii_3$	&	$-1$	&	$-\ii_1$	\\
	$\ii_7$	&	$\ii_7$	&	$\ii_6$	&	$-\ii_5$&	$-\ii_4$&	$\ii_3$	&	$\ii_2$	&	$\ii_1$	&	$-1$		\\
	\bottomrule
	\end{tabular}

	\caption{Tabela de multiplicação dos octônios como $(\R^2)^4$ (tomando a multiplicação em $\R^2 \times (\R^2)^3$ análoga à multiplicação dos quatérnios $\R^4 = \R \times \R^3$, mas adaptando-a para a conjugação complexa de $\R^2$).}
	\label{tab:multiplicacao.octonionica.antiga}
\end{table}


\begin{table}
	\centering

	\begin{tabular}{c | c c c c c c c c}
	\toprule
	$\times$	&	$1$		&	$\ii_1$	&	$\ii_2$	&	$\ii_3$	&	$\ii_4$	&	$\ii_5$	&	$\ii_6$	&	$\ii_7$	\\
%	\midrule
	\hline
	$1$			&	$1$		&	$\ii_1$	&	$\ii_2$	&	$\ii_3$	&	$\ii_4$	&	$\ii_5$	&	$\ii_6$	&	$\ii_7$	\\
	$\ii_1$		&	$\ii_1$	&	$-1$	&	$\ii_?$	&	$\ii_?$	&	$\ii_?$	&	$\ii_?$	&	$\ii_?$	&	$\ii_?$	\\
	$\ii_2$		&	$\ii_2$	&	$\ii_?$	&	$-1$	&	$\ii_?$	&	$\ii_?$	&	$\ii_?$	&	$\ii_?$	&	$\ii_?$	\\
	$\ii_3$		&	$\ii_3$	&	$\ii_?$	&	$\ii_?$	&	$-1$	&	$\ii_?$	&	$\ii_?$	&	$\ii_?$	&	$\ii_?$	\\
	$\ii_4$		&	$\ii_4$	&	$\ii_?$	&	$\ii_?$	&	$\ii_?$	&	$-1$	&	$\ii_?$	&	$\ii_?$	&	$\ii_?$	\\
	$\ii_5$		&	$\ii_5$	&	$\ii_?$	&	$\ii_?$	&	$\ii_?$	&	$\ii_?$	&	$-1$	&	$\ii_?$	&	$\ii_?$	\\
	$\ii_6$		&	$\ii_6$	&	$\ii_?$	&	$\ii_?$	&	$\ii_?$	&	$\ii_?$	&	$\ii_?$	&	$-1$	&	$\ii_?$	\\
	$\ii_7$		&	$\ii_7$	&	$\ii_?$	&	$\ii_?$	&	$\ii_?$	&	$\ii_?$	&	$\ii_?$	&	$\ii_?$	&	$-1$	\\
	\bottomrule
	\end{tabular}

	\caption{Tabela de multiplicação dos octônios.}
\end{table}
\end{comment}

\begin{figure}
	\centering

	\begin{tikzpicture}

	\def \raio{1.5cm}

	\coordinate (O) at (0,0);
	\coordinate (A) at (-90:\raio);
	\coordinate (B) at (30:\raio);
	\coordinate (C) at (150:\raio);

	\coordinate (AA) at (90:2*\raio);
	\coordinate (BB) at (210:2*\raio);
	\coordinate (CC) at (330:2*\raio);

	\foreach \P/\PP in {A/AA, B/BB, C/CC}
		{
			\draw[->] (\P) -- ($ (\P)!.5!(O) $);
			\draw ($ (\P)!.4!(O) $) -- (O);
			\draw[->] (O) -- ($ (O)!.75!(\PP) $);
			\draw ($ (O)!.70!(\PP) $) -- (\PP);
		}

	\foreach \n in {0,1,2,3}
		{
			\draw[->] (-90+120*\n:\raio) arc (-90+120*\n:-29+120*\n:\raio);
			\draw (-30+120*\n:\raio) arc (-30+120*\n:30+120*\n:\raio);
		}

	\foreach \P/\PP in {A/CC, CC/B, B/AA, AA/C, C/BB, BB/A}
		{
			\draw[->] (\P) -- ($ (\P)!.5!(\PP) $);
			\draw ($ (\P)!.4!(\PP) $) -- (\PP);
		}

	\foreach \P/\n in {O/4, A/1, B/2, C/3, AA/5, BB/6, CC/7}
		\node[draw,circle,fill=amarelo,inner sep=1.5pt] at (\P) {$\ii_\n$};

	\end{tikzpicture}

	\caption{Grafo para memorizar multiplicação octoniônica.}

\end{figure}


\begin{comment}


\begin{figure}
	\centering

	\begin{tikzpicture}

	\def \raio{1.5cm}

	\coordinate (O) at (0,0);
	\coordinate (A) at (-90:\raio);
	\coordinate (B) at (30:\raio);
	\coordinate (C) at (150:\raio);

	\coordinate (AA) at (90:2*\raio);
	\coordinate (BB) at (210:2*\raio);
	\coordinate (CC) at (330:2*\raio);

	\foreach \P/\PP in {A/AA, B/BB, C/CC}
		{
			\draw[->] (O) -- ($ (O)!.5!(\P) $);
			\draw ($ (O)!.4!(\P) $) -- (\P);
			\draw[->] (\PP) -- ($ (\PP)!.6!(O) $);
			\draw ($ (\PP)!.5!(O) $) -- (O);
		}

	\foreach \n in {0,1,2,3}
		{
			\draw[->] (-90+120*\n:\raio) arc (-90+120*\n:-29+120*\n:\raio);
			\draw (-30+120*\n:\raio) arc (-30+120*\n:30+120*\n:\raio);
		}

	\foreach \P/\PP in {A/CC, CC/B, B/AA, AA/C, C/BB, BB/A}
		{
			\draw[->] (\P) -- ($ (\P)!.5!(\PP) $);
			\draw ($ (\P)!.4!(\PP) $) -- (\PP);
		}

	\foreach \P/\n in {O/1, A/2, B/4, C/6, AA/3, BB/5, CC/7}
		\node[draw,circle,fill=amarelo,inner sep=1.5pt] at (\P) {$\ii_\n$};

	\end{tikzpicture}

	\caption{Grafo para memorizar multiplicação octoniônica (ANTIGO).}

\end{figure}



% NOTA: Isso a seguir está incompleto!
A multiplicação octoniônica está completamente definida pela relações
	\begin{enumerate}
		\item Para todo $k \in [7]$, $\ii_k \ii_k = -1$;
		\item Para toda permutação $\fun{\pi}{\{1,2,3\}}{\{1,2,3\}}$,
			\begin{equation*}
			\ii_{\pi(1)} \ii_{\pi(2)} = \prd(\pi)\ii_{\pi(3)};
			\end{equation*}
		\item Para toda permutação $\fun{\pi}{\{1,4,5\}}{\{1,4,5\}}$,
			\begin{equation*}
			\ii_{\pi(1)} \ii_{\pi(4)} = \prd(\pi)\ii_{\pi(5)};
			\end{equation*}
		\item Para toda permutação $\fun{\pi}{\{1,6,7\}}{\{1,6,7\}}$,
			\begin{equation*}
			\ii_{\pi(1)} \ii_{\pi(6)} = \prd(\pi)\ii_{\pi(7)};
			\end{equation*}
%		\item Para todo $n \in [3]$ e toda permutação $\fun{\pi}{\{1,2+2n,3+2n\}}{\{1,2+2n,3+2n\}}$,
%			\begin{equation*}
%			\ii_{\pi(1)} \ii_{\pi(2+2n)} = \prd(\pi)\ii_{\pi(3+2n)},
%			\end{equation*}
%		em que o sinal $\prd(\pi)$ da permutação $\pi$ é calculado supondo a ordem usual em $\{1,2+2n,3+2n\}$ induzida de $\N$;
		\item Para toda permutação $\fun{\pi}{\{2,4,6\}}{\{2,4,6\}}$,
			\begin{equation*}
			\ii_{\pi(2)} \ii_{\pi(4)} = \prd(\pi)\ii_{\pi(6)};
			\end{equation*}
%		em que o sinal $\prd(\pi)$ da permutação $\pi$ é calculado supondo a ordem usual em $\{2,4,6\}$ induzida de $\N$;
		\item Para toda permutação $\fun{\pi}{\{2,5,7\}}{\{2,5,7\}}$,
			\begin{equation*}
			\ii_{\pi(2)} \ii_{\pi(5)} = -\prd(\pi)\ii_{\pi(7)};
			\end{equation*}
%		em que o sinal $\prd(\pi)$ da permutação $\pi$ é calculado supondo a ordem usual em $\{2,5,7\}$ induzida de $\N$;
		\item Para toda permutação $\fun{\pi}{\{3,4,7\}}{\{3,4,7\}}$,
			\begin{equation*}
			\ii_{\pi(3)} \ii_{\pi(4)} = -\prd(\pi)\ii_{\pi(7)};
			\end{equation*}
%		em que o sinal $\prd(\pi)$ da permutação $\pi$ é calculado supondo a ordem usual em $\{3,4,7\}$ induzida de $\N$;
		\item Para toda permutação $\fun{\pi}{\{3,5,6\}}{\{3,5,6\}}$,
			\begin{equation*}
			\ii_{\pi(3)} \ii_{\pi(5)} = -\prd(\pi)\ii_{\pi(6)};
			\end{equation*}
%		em que o sinal $\prd(\pi)$ da permutação $\pi$ é calculado supondo a ordem usual em $\{3,5,6\}$ induzida de $\N$.
	\end{enumerate}
e em cada caso, o sinal $\prd(\pi)$ da permutação $\fun{\pi}{S}{S}$ é calculado supondo a ordem usual em $S$ induzida de $\N$.

Isso nos permite calcular que existem $480$ diferentes escolhas de operações $\times$ que fazem de $\R^8$ uma álgebra isomorfa aos octônios.
\end{comment}

\begin{exercise}
A lista
	\begin{equation*}
	((\R^8,+,-,0,\cdot,\times,\div,1),\nor{\var})
	\end{equation*}
é uma álgebra\footnote{A álgebra de $\R^8$ com essa multiplicação costuma ser denotada por $\mathbb{O}$, mas manteremos a notação $\R^8$ por simplicidade de notação, pois nenhuma ambiguidade resultará disso.} alternativa invertível normada sobre $\R$.
\begin{enumerate}
	\item (Bilinearidade) A multiplicação quaterniônica $\fun{\times}{\R^8 \times \R^8}{\R^8}$ é uma função bilinear;
	\item (Alternatividade) Para todos $x,y \in \R^8$,
		\begin{align*}
		(xx)y &= x(xy) \\
		(xy)y &= x(yy);
		\end{align*}
	\item (Identidade) Para todo $x \in \R^8$,
		\begin{equation*}
		x1 = 1x = x;
		\end{equation*}
	\item (Inversa) Para todo $x \in \R^8$,
		\begin{equation*}
		x\inv x = 1 = xx\inv;
		\end{equation*}
	\item (Submultiplicatividade) Para todos $x,y \in \R^8$,
		\begin{equation*}
		\nor{xx'} \leq \nor{x}\nor{x'}.
		\end{equation*}
	\item (Unitariedade) $\nor{1}=1$;
	\item (Invertibilidade) Para todo $x \in \R^8$,
		\begin{equation*}
		\nor{x\inv} = \nor{x}\inv.
		\end{equation*}
\end{enumerate}
\end{exercise}








\begin{table}
	\centering

	\begin{tabular}{c | c c c c c c c}
	\toprule
	$\pvec{}{}$&	$\ii_1$	&	$\ii_2$	&	$\ii_3$	&	$\ii_4$	&	$\ii_5$	&	$\ii_6$	&	$\ii_7$		\\
	\hline
	$\ii_1$	&	$0$		&	$\ii_3$	&	$-\ii_2$&	$\ii_5$	&	$-\ii_4$&	$-\ii_7$&	$\ii_6$		\\
	$\ii_2$	&	$-\ii_3$&	$0$		&	$\ii_1$	&	$\ii_6$	&	$\ii_7$	&	$-\ii_4$&	$-\ii_5$	\\
	$\ii_3$	&	$\ii_2$	&	$-\ii_1$&	$0$		&	$\ii_7$	&	$-\ii_6$&	$\ii_5$	&	$-\ii_4$	\\
	$\ii_4$	&	$-\ii_5$&	$-\ii_6$&	$-\ii_7$&	$0$		&	$\ii_1$	&	$\ii_2$	&	$\ii_3$		\\
	$\ii_5$	&	$\ii_4$	&	$\ii_7$	&	$\ii_6$	&	$\ii_1$	&	$0$		&	$-\ii_3$&	$-\ii_2$	\\
	$\ii_6$	&	$\ii_7$	&	$\ii_4$	&	$-\ii_5$&	$-\ii_2$&	$\ii_3$	&	$0$		&	$-\ii_1$	\\
	$\ii_7$	&	$-\ii_6$&	$\ii_5$	&	$\ii_4$	&	$-\ii_3$&	$-\ii_2$&	$\ii_1$	&	$0$			\\
	\bottomrule
	\end{tabular}

	\caption{Tabela de produto vetorial em $\vec{\R}^7$.}
	\label{tab:multiplicacao.octonionica.vetorial}
\end{table}


















\subsection{$2$-Hiperbólicos \texorpdfstring{$\R^{1,1}$}{}}

Definiremos o `produto hiperbólico' no espaço linear $\R^2$, cuja base canônica denotaremos por
	\begin{align*}
	\bm 1 &:= (1,0) \\
	\jj &:= (0,1).
	\end{align*}
Como de costume, denotaremos o vetor $\bm 1$ simplesmente por $1$, pois isso não gerará ambiguidade, e sempre que possível ele será sempre omitido da notação, de modo que todo $x \in \R^2$ será denotado
	\begin{equation*}
	x = x_0 + x_1 \jj.
	\end{equation*}
Além disso, denotaremos o espaço $\R^2$ por $\R^{1,1}$ para enfatizar que estamos usando um produto diferente do complexo e que a geometria desse espaço será diferente.

\begin{definition}
O \emph{produto hiperbólico} em $\R^{1,1}$ é a operação
	\begin{align*}
	\func{\htimes}{\R^{1,1} \times \R^{1,1}}{\R^{1,1}}{(x,y)}{xy := (x_0y_0 + x_1y_1, x_0y_1+x_1y_0)}
	\end{align*}
A \emph{unidade} de $(\R^{1,1},\htimes)$ é $1 = (1,0)$.
\end{definition}

Escrevendo esse produto explicitamente na base de $\R^{1,1}$, temos
	\begin{equation*}
	xy = (x_0y_0 + x_1y_1) + (x_0y_1+x_1y_0)\jj.
	\end{equation*}

Notemos que
	\begin{equation*}
	\jj^2 = (0,1)(0,1) = (0 \times 0 + 1 \times 1) + (0 \times 1 + 1 \times 0)\jj = 1.
	\end{equation*}

\begin{proposition}
A lista
	\begin{equation*}
	(\R^{1,1},+,-,0,\htimes,1,\cdot)
	\end{equation*}
é uma álgebra associativa, comutativa e unitária sobre $\R$.
\end{proposition}
\begin{proof}
\begin{itemize}
	\item (Comutatividade) Para todos $x,y \in \R^{1,1}$,
%		\begin{equation*}
%		xy = (x_0y_0 + x_1y_1) + (x_0y_1+x_1y_0)\jj = (y_0x_0 + y_1x_1) + (y_0x_1+y_1x_0)\jj = yx.
%		\end{equation*}
		\begin{equation*}
		xy = (x_0y_0 + x_1y_1,x_0y_1+x_1y_0) = (y_0x_0 + y_1x_1, y_0x_1+y_1x_0) = yx.
		\end{equation*}
	\item %(Associatividade) Para todos $x,y,z \in \R^{1,1}$,
%		\begin{align*}
%		(xy)z &= ((x_0y_0 + x_1y_1) + (x_0y_1+x_1y_0)\jj)(z_0+z_1\jj) \\
%				&= ((x_0y_0 + x_1y_1)z_0 + (x_0y_1+x_1y_0)z_1) + ((x_0y_0 + x_1y_1)z_1 + (x_0y_1+x_1y_0)z_0)\jj \\
%				&= (x_0y_0z_0 + x_1y_1z_0 + x_0y_1z_1+x_1y_0z_1) + (x_0y_0z_1 + x_1y_1z_1 + x_0y_1z_0+x_1y_0z_0)\jj \\
%				&= (x_0(y_0z_0+y_1z_1) + x_1(y_0z_1+y_1z_0)) + (x_0(y_0z_1+y_1z_0) + x_1(y_0z_0+y_1z_1))\jj \\
%				&= x(yz).
%		\end{align*}
%		\begin{align*}
%		(xy)z &= (x_0y_0 + x_1y_1, x_0y_1+x_1y_0)(z_0,z_1) \\
%				&= ((x_0y_0 + x_1y_1)z_0 + (x_0y_1+x_1y_0)z_1, (x_0y_0 + x_1y_1)z_1 + (x_0y_1+x_1y_0)z_0) \\
%				&= (x_0y_0z_0 + x_1y_1z_0 + x_0y_1z_1+x_1y_0z_1, x_0y_0z_1 + x_1y_1z_1 + x_0y_1z_0+x_1y_0z_0) \\
%				&= (x_0(y_0z_0+y_1z_1) + x_1(y_0z_1+y_1z_0), x_0(y_0z_1+y_1z_0) + x_1(y_0z_0+y_1z_1)) \\
%				&= x(yz).
%		\end{align*}
		(Associatividade) Faremos as contas por coordenadas. Para todos $x,y,z \in \R^{1,1}$, segue da definição do produto que as coordenadas de $xy$ são
			\begin{align*}
			(xy)_0 &= x_0y_0 + x_1y_1 \\
			(xy)_1 &= x_0y_1+x_1y_0
			\end{align*}
		e que as coordenadas de $yz$ são
			\begin{align*}
			(yz)_0 &= y_0z_0+y_1z_1 \\
			(yz)_1 &= y_0z_1+y_1z_0.
			\end{align*}
		Ainda da definição do produto, as coordenadas de $(xy)z$ são
			\begin{align*}
			((xy)z)_0 &= (x_0y_0 + x_1y_1)z_0 + (x_0y_1+x_1y_0)z_1 \\
			((xy)z)_1 &= (x_0y_0 + x_1y_1)z_1 + (x_0y_1+x_1y_0)z_0)
			\end{align*}
		e as coordenadas de $x(yz)$ são
			\begin{align*}
			(x(yz))_0 &= x_0(y_0z_0+y_1z_1) + x_1(y_0z_1+y_1z_0) \\
			(x(yz))_1 &= x_0(y_0z_1+y_1z_0) + x_1(y_0z_0+y_1z_1).
			\end{align*}
		Para a coordenada $0$, segue que
			\begin{align*}
			((xy)z)_0 &= (x_0y_0 + x_1y_1)z_0 + (x_0y_1+x_1y_0)z_1 \\
				&= x_0y_0z_0 + x_1y_1z_0 + x_0y_1z_1+x_1y_0z_1 \\
				&= x_0(y_0z_0+y_1z_1) + x_1(y_0z_1+y_1z_0) \\
				&= (x(yz))_0
			\end{align*}
		e, para a coordenada $1$, segue que
			\begin{align*}
			((xy)z)_1 &= (x_0y_0 + x_1y_1)z_1 + (x_0y_1+x_1y_0)z_0) \\
				&= x_0y_0z_1 + x_1y_1z_1 + x_0y_1z_0+x_1y_0z_0 \\
				&= x_0(y_0z_1+y_1z_0) + x_1(y_0z_0+y_1z_1) \\
				&= (x(yz))_1.
			\end{align*}
		Portanto
			\begin{equation*}
			(xy)z = ((xy)z)_0 1 + ((xy)z)_1 \jj = (x(yz))_0 1 + (x(yz))_1 \jj = x(yz).
			\end{equation*}



	\item (Bilinearidade) Como o produto é comutativo, basta mostrar a linearidade na primeira entrada. Para todos $x,x',y \in \R^{1,1}$ e $c \in \R$,
%		\begin{align*}
%		(cx+x')y &= ((cx_0+x'_0)y_0 + (cx_1+x'_1)y_1) + ((cx_0+x'_0)y_1 + (cx_1+x'_1)y_0)\jj \\
%			&= (cx_0y_0+x'_0y_0 + cx_1y_1+x'_1y_1) + (cx_0y_1+x'_0y_1 + cx_1y_0+x'_1y_0))\jj \\
%%			&= (c(x_0y_0+x_1y_1) + x'_0y_0 +x'_1y_1) + (c(x_0y_1+x_1y_0) + x'_0y_1+x'_1y_0))\jj \\
%			&= (cx_0y_0+cx_1y_1) + (cx_0y_1+cx_1y_0)\jj \\
%				&\qquad\qquad\qquad\qquad + (x'_0y_0+x'_1y_1) + (x'_0y_1+x'_1y_0)\jj \\
%			&= c((x_0y_0+x_1y_1) + (x_0y_1+x_1y_0)\jj) \\
%				&\qquad\qquad\qquad\qquad + (x'_0y_0+x'_1y_1) + (x'_0y_1+x'_1y_0)\jj \\
%			&= c(xy) + x'y.
%		\end{align*}
		\begin{align*}
		(cx+x')y &= ((cx_0+x'_0)y_0 + (cx_1+x'_1)y_1, (cx_0+x'_0)y_1 + (cx_1+x'_1)y_0) \\
			&= (cx_0y_0+x'_0y_0 + cx_1y_1+x'_1y_1, cx_0y_1+x'_0y_1 + cx_1y_0+x'_1y_0) \\
%			&= (c(x_0y_0+x_1y_1) + x'_0y_0 +x'_1y_1) + (c(x_0y_1+x_1y_0) + x'_0y_1+x'_1y_0))\jj \\
			&= (cx_0y_0+cx_1y_1, cx_0y_1+cx_1y_0) + (x'_0y_0+x'_1y_1, x'_0y_1+x'_1y_0) \\
			&= c(x_0y_0+x_1y_1, x_0y_1+x_1y_0) + (x'_0y_0+x'_1y_1, x'_0y_1+x'_1y_0) \\
			&= c(xy) + x'y.
		\end{align*}
	\item (Identidade) Como o produto é comutativo, basta mostrar a identidade à esquerda. Para todo $x \in \R^{1,1}$,
%		\begin{equation*}
%		1x = (1+0\jj)(x_0+x_1\jj) = (1x_0+0x_1)+(1x_1+0x_0)\jj = x_0+x_1\jj.
%		\qedhere
%		\end{equation*}
		\begin{equation*}
		1x = (1,0)(x_0,x_1) = (1x_0+0x_1, 1x_1+0x_0) = (x_0, x_1).
		\qedhere
		\end{equation*}
\end{itemize}
\end{proof}

\begin{definition}
Seja $x \in \R^{1,1}$. O \emph{conjugado} de $x$ é $\conju{x} := x_0 - x_1\jj$. A
\end{definition}

\begin{exercise}
Seja $x \in \R^{1,1}$. Então $x\conju{x} = {x_0}^2 - {x_1}^2 = \inte{x}{x}_{1,1}$.
\end{exercise}




\subsection{$4$-Hiperbólicos \texorpdfstring{$\R^{1,3}$}{}}

Definiremos o `produto hiperbólico' no espaço linear $\R^4$, cuja base canônica denotaremos por
	\begin{align*}
	\bm 1 &:= (1,0,0,0) \\
	\jj_1 &:= (0,1,0,0) \\
	\jj_2 &:= (0,0,1,0) \\
	\jj_3 &:= (0,0,0,1).
	\end{align*}
Como de costume, denotaremos o vetor $\bm 1$ simplesmente por $1$, pois isso não gerará ambiguidade, e sempre que possível ele será sempre omitido da notação, de modo que todo $x \in \R^4$ será denotado
	\begin{equation*}
	x = x_0 + x_1 \jj_1 + x_2 \jj_2 + x_3 \jj_3.
	\end{equation*}
Além disso, denotaremos o espaço $\R^4$ por $\R^{1,3}$ para enfatizar que estamos usando um produto diferente do quaterniônico e que a geometria desse espaço será diferente.

\begin{definition}
O \emph{produto hiperbólico} em $\R^{1,3}$ é a operação
	\begin{align*}
	\func{\htimes}{\R^{1,3} \times \R^{1,3}}{\R^{1,3}}{(x,y)}{
		\begin{aligned}[t]
		xy := &(x_0y_0 + x_1y_1 + x_2y_2 + x_3y_3) 1 \\
			+ &(x_0y_1 + x_1y_0 + x_2y_3 - x_3y_2) \jj_1 \\
			+ &(x_0y_2 - x_1y_3 + x_2y_0 + x_3y_1) \jj_2 \\
			+ &(x_0y_3 + x_1y_2 - x_2y_1 + x_3y_0) \jj_3 .
		\end{aligned}
	}
	\end{align*}
A \emph{unidade} de $(\R^{1,3},\htimes)$
	\begin{equation*}
	1 := (1,0,0,0).
	\end{equation*}
\end{definition}

Notemos que
	\begin{align*}
	(\jj_1)^2 &= 1 \\
	(\jj_2)^2 &= 1 \\
	(\jj_3)^2 &= 1 \\
	\jj_1\jj_2 &= \jj_3 = -\jj_2\jj_1 \\
	\jj_2\jj_3 &= \jj_1 = -\jj_3\jj_2 \\
	\jj_3\jj_1 &= \jj_2 = -\jj_1\jj_3.
	\end{align*}

\begin{exercise}
A lista
	\begin{equation*}
	(\R^{1,3},+,-,0,\htimes,1,\cdot)
	\end{equation*}
é uma álgebra unitária sobre $\R$.
\end{exercise}