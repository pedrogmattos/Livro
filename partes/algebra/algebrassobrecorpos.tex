\chapter{Álgebras sobre corpos}

\section{Álgebra e ação adjunta}

\begin{definition}
Seja $\bm C$ um corpo. Uma \emph{álgebra} sobre $\bm C$ é um par $(\bm A,\cdot)$ em que $\bm A$ é um espaço vetorial sobre $\bm C$ e $\cdot\colon A \times A \to A$ é uma função bilinear. Uma álgebra é \emph{associativa, comutativa} ou \emph{antissimétrica} conforme a respectiva propriedade do produto $\cdot$, e é unitária se $\cdot$ tem identidade.
\end{definition}

\begin{definition}
Sejam $(\bm A,\cdot)$ uma álgebra sobre um corpo $\bm C$ e $a \in A$. A \emph{ação adjunta} em $\bm A$ baseada em $a$ é a função linear
	\begin{align*}
	\func{\adj_a}{A}{A}{a'}{a \cdot a'.}
	\end{align*}
\end{definition}

\begin{proposition}
Seja $(\bm A,\cdot)$ uma álgebra sobre um corpo $\bm C$. Então $(\lin(A,A),\circ)$ é uma álgebra associativa sobre $\bm C$.
\end{proposition}
\begin{proof}
Sabemos que $\lin(A,A)$ é um espaço linear. Para mostrar que é uma álgebra, devemos mostrar que $\circ$ é bilinear. Sejam $L,L,L'' \in \lin(A,A)$ e $c \in C$. Então, para todo $a \in A$,
	\begin{align*}
	((cL+L') \circ L'')(a) &= (cL+L')(L''(a)) \\
		&= cL(L''(a))+L'(L''(a)) \\
		&= cL \circ L''(a) + L' \circ L''(a) \\
		&= (cL \circ L'' + L' \circ L'')(a).
	\end{align*}
Isso mostra que $(cL+L') \circ L'' = cL \circ L'' + L' \circ L''$. Agora,
	\begin{align*}
	(L \circ (cL'+L''))(a) &= L((cL'+L'')(a)) \\
		&= L(cL'(a)+L''(a)) \\
		&= cL(L'(a))+L(L''(a)) \\
		&= cL \circ L'(a)+L \circ L''(a) \\
		&= (cL \circ L' + L \circ L'')(a).
	\end{align*}
Isso mostra que $L \circ (cL'+L'') = cL \circ L' + L \circ L''$. A composição de funções é associativa, portanto a álgebra é associativa.
\end{proof}

\begin{proposition}
Sejam $(\bm A,\cdot)$ uma álgebra sobre um corpo $\bm C$ e $I$ um conjunto. Então $(A^I,\cdot)$, em que $\cdot\colon A^I \times A^I \to A^I$ é o produto entrada a entrada, é uma álgebra sobre $\bm C$. Se o produto de $A$ é associativo ou comutativo, então o produto de $A^I$ é, respectivamente, associativo ou comutativo, e é se $A$ é unitária, $(1)_{in \in I}$ é identidade do produto de $A^I$.
\end{proposition}
\begin{proof}
Sabemos que $A^I$ é um espaço linear sobre $\bm C$. Basta mostrar que $\cdot$ é um produto bilinear. Sejam $(a_i)_{i \in I},(a'_i)_{i \in I},(a''_i)_{i \in I} \in A^I$ e $c \in C$. Então
	\begin{align*}
	(c(a_i)_{i \in I} + (a'_i)_{i \in I}) \cdot (a''_i)_{i \in I} &= (ca_i + a'_i)_{i \in I} \cdot (a''_i)_{i \in I} \\
	&= ((ca_i + a'_i) \cdot a''_i)_{i \in I} \\
	&= (ca_i \cdot a''_i + a'_i \cdot a''_i)_{i \in I} \\
	&= c(a_i \cdot a''_i)_{i \in I} + (a'_i \cdot a''_i)_{i \in I} \\
	&= c(a_i)_{i \in I} \cdot (a''_i)_{i \in I} + (a'_i)_{i \in I} \cdot (a''_i)_{i \in I}.
	\end{align*}
A demonstração da linearidade na segunda entrada é análoga, e as demonstrações de associatividade e comutatividade e identidade são triviais.
\end{proof}

\begin{proposition}
Seja $(\bm A,\cdot)$ uma álgebra sobre um corpo $\bm C$. A álgebra $\bm A$ é associativa se, e somente se, para todos $a,a' \in A$,
	\begin{equation*}
	\adj_{a \cdot a'} = \adj_a \circ \adj_{a'}.
	\end{equation*}
\end{proposition}

\section{Derivação}

\begin{definition}
Seja $(\bm A,\cdot)$ uma álgebra sobre um corpo $\bm C$. Uma \emph{derivação} em $\bm A$ é uma função linear $D\colon A \to A$ tal que
	\begin{enumerate}
	\item (Regra do produto) Para todos $a,a' \in A$,
		\begin{equation*}
		D(a \cdot a') = D(a) \cdot a' + a \cdot D(a').
		\end{equation*}
	\end{enumerate}
O conjunto dessas derivações é $\Der(A)$.
\end{definition}

Note que a propriedade acima nem sempre é equivalente a
	\begin{equation*}
	D(a \cdot a') = a' \cdot D(a)  + a \cdot D(a'),
	\end{equation*}
pois o produto $\cdot$ nem sempre é comutativo, mas sempre é equivalente a
	\begin{equation*}
	D(a \cdot a') = a \cdot D(a') + D(a) \cdot a',
	\end{equation*}
pois a soma $+$ é comutativa.

\begin{proposition}
Sejam $(\bm A,\cdot)$ uma álgebra sobre um corpo $\bm C$ e $D\colon A \to A$ uma derivação em $\bm A$.
	\begin{enumerate}
	\item (Regra do produto generalizada) Para todos $a_0,\ldots,a_{n-1} \in A$,
		\begin{equation*}
		D(a_0 \cdots a_{n-1}) = \sum_{i \in [n]} a_0 \cdots D(a_i) \cdots a_{n-1};
		\end{equation*}
	\item Se $\cdot$ é comutativo, então, para todos $a \in A$ e $n \in \N^*$,
		\begin{equation*}
		D(a^n) = na^{n-1}D(a);
		\end{equation*}
	\item Se existe identidade $1 \in A$ do produto, então
		\begin{equation*}
		D(1)=0.
		\end{equation*}
	\item (Regra do produto de ordem superior) Para todos $a,a' \in A$ e $n \in \N$,
		\begin{equation*}
		D^n(aa') = \sum_{i \in [n+1]} \binom{n}{i} D^{n-i}(a)D^i(a').
		\end{equation*}
	\end{enumerate}
\end{proposition}

\begin{definition}
Seja $(\bm A,\cdot)$ uma álgebra sobre um corpo $\bm C$. O \emph{colchete comutador} de $(\bm A,\cdot)$ é a função
	\begin{align*}
	\func{\col{\var}{\var}}{A \times A}{A}{(a,a')}{a \cdot a' - a' \cdot a.}
	\end{align*}
\end{definition}

\begin{proposition}
Seja $(\bm A,\cdot)$ uma álgebra sobre um corpo $\bm C$. Então
	\begin{enumerate}
	\item $(A,\col{\var}{\var})$ é uma álgebra antissimétrica sobre $\bm C$;
	\item O produto $\cdot$ é comutativo se, e somente se, $\col{\var}{\var} = 0$;
%	\item Se $(\bm A,\cdot)$ é associativa, então $\mathrm{Der}(A) \subseteq A^A$, com as operações pontuais de espaço linear induzidas de $\bm A$, e com o produto como o comutador $\col{\var}{\var}$ do produto pontual $\cdot$, induzido de $A$, é uma álgebra.
	\end{enumerate}
\end{proposition}
\begin{proof}
	\begin{enumerate}
	\item Primeiro, notemos que, para todos $a,a' \in A$,
		\begin{equation*}
		\col{a}{a'} = a \cdot a' - a' \cdot a = -(a' \cdot a - a \cdot a') = -\col{a'}{a}.
		\end{equation*}	
Sendo assim, para mostrar que $\col{\var}{\var}$ é bilinear antissimétrica, basta mostrar que ela é linear na primeira entrada. Para todos $a,a',a'' \in A$ e $c \in C$,
		\begin{align*}
		\col{ca+a'}{a''} &= (ca+a')a'' - a''(ca+a') \\
			&= caa'' + a'a'' - ca''a - a''a' \\
			&= caa'' - ca''a + a'a'' - a''a' \\
			&= c\col{a}{a''} + \col{a'}{a''}.
		\end{align*}

	\item Suponhamos, primeiro, que $\cdot$ é comutativo. Então, para todos $a,a' \in A$,
		\begin{equation*}
		\col{a}{a'} = aa' - a'a = aa' - aa' = 0.
		\end{equation*}
Reciprocamente, suponhamos que $\col{\var}{\var}=0$. Então, para todos $a,a' \in A$,
		\begin{equation*}
		aa' = aa' + 0 = aa' + \col{a'}{a} = aa' + a'a - aa' = a'a.
		\qedhere
		\end{equation*}

\begin{comment}
	\item Mostremos que $\mathrm{Der}(A)$ é linearmente fechado. A soma e o produto por escalar de derivações são lineares, pois as derivações são funções lineares. Mostremos que essas funções satisfazem a regra do produto. Sejam $D,D' \in \mathrm{Der}(A)$ e $c \in C$. Então, para todos $a,a' \in A$,
		\begin{align*}
		(cD + D')(aa') &= cD(aa') + D'(aa') \\
			&= cD(a)a' + caD(a') + D'(a)a' + aD'(a') \\
			&= cD(a)a' + D'(a)a' + caD(a') + aD'(a') \\
			&= (cD(a) + D'(a))a' + a(cD(a') + D'(a')) \\
			&= (cD + D')(a)a' + a(cD + D')(a').
		\end{align*}
Agora, devemos mostrar que $\mathrm{Der}(A)$ é fechado por $\col{\var}{\var}$. Notaremos que $\mathrm{Der}(A)$ não é fechado pelo produto $\cdot$ induzido pontualmente na demonstração, e por isso precisamos do comutador. Além disso, ainda não usamos a associatividade do produto $\cdot$. Ela será usada agora. Sejam $D,D' \in \mathrm{Der}(A)$. Para todos $a,a' \in A$, primeiro notemos que, pela associatividade, segue que
	\begin{align*}
	(D \cdot D')(a)a' + a(D \cdot D')(a') &= (D(a)D'(a))a' + a(D(a')D'(a')) \\
		&= D(a)D'(a)a' + aD(a')D'(a').
	\end{align*}
Então,
	\begin{align*}
	(D \cdot D')(a,a') &= D(aa')D'(aa') \\
		&= (D(a)a' + aD(a'))(D'(a)a' + aD'(a')) \\
		&= (D(a)a')(D'(a)a') + (aD(a'))(D'(a)a') \\
			&\qquad + (D(a)a')(aD'(a')) + (aD(a'))(aD'(a')) \\
		&= (D(a)a'D'(a))a' + a(D(a')D'(a)a') \\
			&\qquad + (D(a)a')(aD'(a')) + (aD(a'))(aD'(a')) \\
		&= (D \cdot D')(a)a' - 
	\end{align*}
e
	\begin{align*}
	(D' \cdot D)(a,a') &= D'(aa')D(aa') \\
		&= (D'(a)a' + aD'(a'))(D(a)a' + aD(a')) \\
		&= (D'(a)a')(D(a)a') + (aD'(a'))(D(a)a') \\
			&\qquad + (D'(a)a')(aD(a')) + (aD'(a'))(aD(a')) \\
		&= (D'(a)a'D(a))a' + a(D'(a')D(a)a') \\
			&\qquad + (D'(a)a')(aD(a')) + (aD'(a'))(aD(a'))
	\end{align*}
\end{comment}
	\end{enumerate}
\end{proof}




\section{Álgebra de derivação adjunta}

\begin{proposition}
Sejam $(\bm A,\cdot)$ uma álgebra sobre um corpo $\bm C$ e $a \in A$. A função adjunta $\adj_a$ é uma derivação em $\bm A$ se, e somente se, para todos $a',a'' \in A$,
	\begin{equation*}
	a \cdot (a' \cdot a'') = (a \cdot a') \cdot a'' + a' \cdot (a \cdot a'').
	\end{equation*}
\end{proposition}

A demonstração é imediata. Essa propriedade é conhecida às vezes como identidade de Jacobi. No entanto, a identidade mais conhecida como identidade de Jacobi é
	\begin{equation*}
	a \cdot (a' \cdot a'') + a' \cdot (a'' \cdot a) + a'' \cdot (a \cdot a') = 0,
	\end{equation*}
que é equivalente à anterior se o produto é antissimétrico. Na maioria das vezes em que se usa essa identidade o produto é de fato antissimétrico, o que torna as duas propriedades equivalentes.

\begin{definition}
Seja $\bm C$ um corpo. Uma \emph{álgebra de derivação adjunta}\footnote{Essas álgebras são conhecidas como `álgebras de Lie'.} sobre $\bm C$ é um par $(\bm A,\col{\var}{\var})$ em que $\col{\var}{\var}\colon A \times A \to A$ é um produto alternado tal que, para todo $a \in A$, $\adj_a$ é uma derivação em $\bm A$. O produto $\col{\var}{\var}$ é o \emph{colchete de derivação}.
\end{definition}

Como $\col{\var}{\var}$ é alternada, é antissimétrica, portanto a bilinearidade é equivalente à linearidade na segunda entrada de $\col{\var}{\var}$. A alternância é equivalente ao produto de um elemento com ele mesmo ser $0$, o que é o mesmo que a derivação adjunta baseada em um elemento aplicada a esse elemento ser $0$.

As três propriedades de $\fun{\col{\var}{\var}}{A \times A}{A}$ são equivalentes a
	\begin{enumerate}
	\item (Linearidade na 2ª entrada) Para todos $a,a',a'' \in A$ e $c \in C$,
		\begin{equation*}
		\col{a}{ca'+a''} = c\col{a}{a'} + \col{a}{a''};
		\end{equation*}
	\item (Alternância) Para todo $a \in A$,
		\begin{equation*}
		\col{a}{a} = 0;
		\end{equation*}
	\item (Derivação adjunta) Para todo $a \in A$, $\adj_a$ é uma derivação: para todos $a',a'' \in A$,
		\begin{equation*}
		\col{a}{\col{a'}{a''}} = \col{\col{a}{a'}}{a''} + \col{a'}{\col{a}{a''}}.
		\end{equation*}
	\end{enumerate}

Nesse caso em que a função adjunta é sempre uma derivação, pode-se também denotar $\ad{a} := \adj_a$, de modo que as propriedades acima se reduzem a termos: para todo $a \in A$, $\ad{a}$ é uma derivação tal que $\ad{a}(a)=0$. A partir de agora, denotaremos a adjunta de $a \in A$ em álgebras de derivação adjunta como
	\begin{align*}
	\func{\ad{a}}{A}{A}{a'}{\col{a}{a'}}
	\end{align*}
e a \emph{representação adjunta} será a função
	\begin{align*}
	\func{\ad{\var}}{A}{\lin(A)}{a}{
		\begin{aligned}[t]
		\func{\ad{a}}{A}{A}{a'}{\col{a}{a'}},
		\end{aligned}
	}
	\end{align*}
em que $\lin(A) = \lin(A,A)$ é o espaço das funções lineares de $A$ para $A$.

Consideremos, agora, o conjunto $\Der(A)$ das derivações em uma álgebra associativa $(\bm A, \cdot)$. O espaço $\Der(A)$ é um subespaço linear de $\lin(A)$. Para mostrar isso, mostraremos que $\Der(A)$ é fechado pela soma e pelo produto por escalar pontuais.
\begin{itemize}
\item (Soma) Para todas derivações $D,D' \in \Der(A)$, a soma $D+D'$ é uma função linear, pois $D$ e $D'$ são lineares, portanto basta mostrar que ela é uma derivação. Para todos $a,a' \in A$,
	\begin{align*}
	(D+D')(aa') &= D(aa') + D'(aa') \\
		&= D(a)a' + aD(a') + D'(a)a' + aD'(a') \\
		&= (D(a)+D'(a))a' + a(D(a')+D'(a')) \\
		&= (D+D')(a)a' + a(D+D')(a').
	\end{align*}
portanto $D+D'$ é uma derivação.

\item (Produto) Para toda derivação $D \in \Der(A)$ e escalar $c \in C$, o produto $cD$ é linear, pois $D$ é linear, portanto basta mostrar que ele é uma derivação. Para todos $a,a' \in A$,
	\begin{equation*}
	(cD)(aa') = c(D(aa')) = c(D(a)a' + aD(a')) = (cD)(a)a' + a(cD)(a'),
	\end{equation*}
portanto $cD$ é uma derivação. Isso mostra que $\Der(A)$ é subespaço linear de $\lin(A,A)$.
\end{itemize}

No entanto, $\Der(A)$ não é uma subálgebra de $\lin(A)$ com o produto de composição de funções, pois, para todos $D,D' \in \Der(A)$ e $a,a' \in A$,
%	\begin{align*}
%	(DD')(aa') &= D(aa')D'(aa') \\
%		&= (D(a)a'+aD(a'))(D'(a)a'+aD'(a')) \\
%		&= D(a)a'D'(a)a' + D(a)a'aD'(a') + aD(a')D'(a)a' + aD(a')aD'(a'),
%	\end{align*}
%e
%	\begin{align*}
%	(DD')(a)a' + a(DD')(a') &= D(a)D'(a)a' + aD(a')D'(a').
%	\end{align*}
%Se notarmos que
%	\begin{align*}
%	(D'D)(aa') &= D'(aa')D(aa') \\
%		&= (D'(a)a'+aD'(a'))(D(a)a'+aD(a')) \\
%		&= D'(a)a'D(a)a' + D'(a)a'aD(a') + aD'(a')D(a)a' + aD'(a')aD(a').
%	\end{align*}
% EU ESTAVA TENTANDO CONSIDERAR O PRODUTO PONTUAL EM L(A,A), MAS ISSO É UM ERRO, O CORRETO É CONSIDERAR O PRODUTO COMPOSIÇÃO DE FUNÇÕES.
	\begin{align*}
	(D \circ D')(aa') &= D(D'(aa')) = D(D'(a)a' + aD'(a')) \\
		&= D(D'(a))a' + D'(a)D(a') + D(a)D'(a') + a D(D'(a')) \\
		&= (D \circ D')(a)a' + a (D \circ D')(a') + D'(a)D(a') + D(a)D'(a').
	\end{align*}
Notando que invertendo as posições de $D$ e $D'$ obtemos a expressão
	\begin{equation*}
	(D' \circ D)(aa') = (D' \circ D)(a)a' + a (D' \circ D)(a') + D(a)D'(a') + D'(a)D(a'),
	\end{equation*}
podemos definir o produto $\col{D}{D'} := D \circ D' - D' \circ D$ de modo a obter das expressões anteriores que
	\begin{align*}
	\col{D}{D'}(aa') &= (D \circ D')(aa') - (D' \circ D)(aa') \\
		&= (D \circ D')(a)a' + a (D \circ D')(a') - (D' \circ D)(a)a' - a (D' \circ D)(a') \\
		&= \col{D}{D'}(a)a' + a\col{D}{D'}(a').
	\end{align*}

O produto $\col{\var}{\var}$ é bilinear, pois envolve somente diferença e composição de funções lineares. Assim, está demonstrada a seguinte proposição.

\begin{proposition}
\label{alge:prop.algebra.colchete.deriv}
Seja $(\bm A,\cdot)$ uma álgebra sobre um corpo $\bm C$. Então $(\Der(A),\col{\var}{\var})$ é uma subálgebra de $(\lin(A),\col{\var}{\var})$.
\end{proposition}
















\section{Álgebras reais}

Nesta seção, descreveremos algumas álgebras sobre os números reais $\R$. A álgebra real $\R$ é simplesmente o corpo $\R$. Analisaremos com detalhes os casos das álgebras dos complexos $\R^2$ e dos quatérnios $\R^4$. Descreveremos brevemente os números $2$-hiperbólicos $\R^{1,1}$, às vezes chamados de `complexos hiperbólicos', os números $4$-hiperbólicos $\R^{1,3}$.

\subsection{Complexos $\R^2$}

Consideremos o espaço linear $\R^2$ com a norma $2$
	\begin{align*}
	\func{\nor{\var}}{\R^2}{\R}{x}{((x_0)^2 + (x_1)^2)^{\frac{1}{2}}}.
	\end{align*}

Denotemos a base canônica de $\R^2$ por $\bm 1 := (1,0)$ e $\ii := (0,1)$. Em geral, por não haver ambiguidade, denotaremos $\bm 1$ simplesmente por $1$.

\begin{definition}
O \emph{produto complexo} em $\R^2$ é a operação
	\begin{align*}
	\func{\times}{\R^2 \times \R^2}{\R^2}{(x,y)}{xy := (x_0y_0-x_1y_1,x_0y_1+x_1y_0)}.
	\end{align*}
A \emph{unidade} de $(\R^2,\times)$ é
	\begin{equation*}
	1 := (1,0).
	\end{equation*}
A \emph{inversa} de $(\R^2,\times,1)$ é a operação
	\begin{align*}
	\func{\div}{\R^2 \setminus \{0\}}{\R^2 \setminus \{0\}}{x}{x\inv := \nor{x}^{-2}(x_0,-x_1)}.
	\end{align*}
\end{definition}

Como a notação acima sugere, omitiremos o símbolo de multiplicação $\times$ sempre que possível pois isso não gerará ambiguidade. Omitiremos também a unidade $1$ dos produtos, de modo que todo $x \in \R^2$ pode ser denotado
	\begin{equation*}
	x = x_0 + x_1\ii.
	\end{equation*}

Notemos que
	\begin{equation*}
	\ii^2 = (0,1)(0,1) = (0 \times 0 - 1 \times 1, 0 \times 1 + 1 \times 0) = (-1,0) = -1.
	\end{equation*}
Todo $x \in \R^2$ pode ser escrito como $x_0 + x_1 \ii$, de modo que o produto $\times$ nessa notação é o produto usual dos números complexos
	\begin{equation*}
	xy = (x_0 + x_1 \ii)(y_0 + y_1 \ii) = (x_0y_0-x_1y_1) + (x_0y_1+x_1y_0) \ii.
	\end{equation*}

\begin{proposition}
A lista
	\begin{equation*}
	(\R^2,+,-,0,\times,1,\cdot)
	\end{equation*}
é uma álgebra associativa, comutativa e unitária (e invertível) sobre $\R$.
\end{proposition}
\begin{proof}
\begin{itemize}
	\item (Comutatividade) Para todos $x,y \in \R^2$,
		\begin{equation*}
		xy = (x_0y_0-x_1y_1,x_0y_1+x_1y_0) = (y_0x_0-y_1x_1,y_0x_1+y_1x_0) = yx.
		\end{equation*}
	\item (Bilinearidade) Como o produto é comutativo, basta mostrar a linearidade na primeira entrada. Para todos $x,x',y \in \R^2$ e todo $c \in \R$,
		\begin{align*}
		(cx+x')y &= ((cx_0+x'_0)y_0-(cx_1+x'_1)y_1,(cx_0+x'_0)y_1+(cx_1+x'_1)y_0) \\
			&= (cx_0y_0+x'_0y_0-cx_1y_1-x'_1y_1,cx_0y_1+x'_0y_1+cx_1y_0+x'_1y_0) \\
			&= c(x_0y_0-x_1y_1,x_0y_1+x_1y_0) + (x'_0y_0-x'_1y_1,x'_0y_1x'_1y_0) \\
			&= c(xy) + x'y.
		\end{align*}
	\item (Associatividade) Para todos $x,y,z \in \R^2$,
	\begin{align*}
	(xy)z &= (x_0y_0-x_1y_1,x_0y_1+x_1y_0)z \\
		&= ((x_0y_0-x_1y_1)z_0 - (x_0y_1+x_1y_0)z_1 , (x_0y_0-x_1y_1)z_1 + (x_0y_1+x_1y_0)z_0 ) \\
		&= (x_0(y_0z_0-y_1z_1)-x_1(y_0z_1+y_1z_0),x_0(y_0z_1+y_1z_0)+x_1(y_0z_0-y_1z_1)) \\
		&= x(y_0z_0-y_1z_1,y_0z_1+y_1z_0) \\
		&= x(yz).
	\end{align*}
	\item (Identidade) Para todo $x \in \R^2$,
	\begin{equation*}
	1x = (1x_0-0x_1,1x_1+0x_0) = (x_0,x_1) = x.
	\end{equation*}
\end{itemize}
(Não demonstraremos a invertibilidade agora.)
\end{proof}


O produto $\times$ pode também ser visto como uma ação de $\R^2$ aditivo sobre $\R^2$ aditivo. Cada $x \in \R^2$ pode ser identificado com uma transformação linear $x\colon \R^2 \to \R^2$. Os elementos $x \in \S^1 \subseteq \R^2$ são as rotações: se $x \in \S^1$, então existe $\theta \in \intfa{0}{\tau}$ tal que
	\begin{equation*}
	x = \cos(\theta) + \sin(\theta) \ii.
	\end{equation*}

Sendo assim, para todo $\cos(\theta) + \sin(\theta) \ii \in \S^1$, e todo $x \in \R^2$,
	\begin{align*}
	(\cos(\theta) + \sin(\theta)) \ii \times x &= (\cos(\theta)x_0-\sin(\theta)x_1) +(\cos(\theta)x_1+\sin(\theta)x_0) \ii = R_\theta(x).
	\end{align*}

Os elementos $x = x_0 \in \R \subseteq \R^2$ são as expansões e contrações. Os elementos de $\R^2$ podem ser decompostos como
	\begin{equation*}
	x = \nor{x}(\cos(\theta) + \sin(\theta) \ii).
	\end{equation*}

\begin{proposition}[Decomposição polar]
Seja $x \in \R^2 \setminus \{0\}$. Então
	\begin{equation*}
	x = \nor{x}(\cos(\theta) + \sin(\theta) \ii),
	\end{equation*}
em que $\theta \in \intfa{0}{\tau}$ é definido
	\begin{equation*}
	\theta :=	\begin{cases}
				\cos\inv\left(\frac{x_0}{\nor{x}}\right),			& x_1 > 0 \text{\ \ ou\ \ } x_0 = 1, \\
				\tau - \cos\inv\left(\frac{x_0}{\nor{x}}\right)		& x_1 < 0 \text{\ \ ou\ \ } x_0 = -1.
				\end{cases}
	\end{equation*}
\end{proposition}
\begin{proof}
Como $x \neq 0$, $\nor{x} \neq 0$, logo da igualdade $\nor{x}^2 = {x_0}^2 + {x_1}^2$ segue que
	\begin{equation*}
	1 = \left(\frac{x_0}{\nor{x}}\right)^2 + \left(\frac{x_1}{\nor{x}}\right)^2,
	\end{equation*}
portanto $\frac{\abs{x_0}}{\nor{x}} \leq 1$, o que é equivalente a $\frac{x_0}{\nor{x}} \in \intff{-1}{1}$. Disso segue que $\phi := \cos\inv\left(\frac{x_0}{\nor{x}}\right) \in \intff{0}{\tau \div 2}$, de modo que $\frac{\abs{x_1}}{\nor{x}} = (1-(\cos(\phi))^2)^{1 \div 2}$, o que implica que $\frac{\abs{x_1}}{\nor{x}} = \sin(\phi)$ ou $\frac{\abs{x_1}}{\nor{x}} = -\sin(\phi)$. Como $\phi \in \intff{0}{\tau \div 2}$, concluímos que $\sin(\phi) = \frac{\abs{x_1}}{\nor{x}}$.

Da definição de $\theta$ segue que $x_0 = \nor{x}\cos(\theta)$ e $x_1 = \nor{x}\sin(\theta)$. Para a demonstração disso, consideramos dois casos.
	\begin{itemize}
	\item ($x_1 > 0$ ou $x_0 = 1$) Nesse caso, $\abs{x_1}=x_1$ e $\theta = \phi$, portanto
		\begin{equation*}
		\cos(\theta) = \cos(\phi) = \frac{x_0}{\nor{x}}
		\end{equation*}
	e
		\begin{equation*}
		\sin(\theta) = \sin(\phi) = \frac{\abs{x_1}}{\nor{x}} = \frac{x_1}{\nor{x}}.
		\end{equation*}
	
	\item ($x_1 < 0$ ou $x_0 = -1$) Nesse caso, $\abs{x_1}=-x_1$ e $\theta = \tau - \phi$, portanto
		\begin{equation*}
		\cos(\theta) = \cos(\tau - \phi) = \cos(-\phi) = \cos (\phi)= \frac{x_0}{\nor{x}}
		\end{equation*}
	e
		\begin{equation*}
		\sin(\theta) = \sin(\tau - \phi) = \sin(-\phi) = -\sin(\phi) = -\frac{\abs{x_1}}{\nor{x}} = \frac{x_1}{\nor{x}}.
		\end{equation*}
	\end{itemize}

Assim, concluímos que
	\begin{equation*}
	x = x_0 + x_1 \ii = \nor{x}\cos(\theta) + \nor{x}\sin(\theta) \ii = \nor{x}(\cos(\theta) + \sin(\theta) \ii).
	\qedhere
	\end{equation*}
\end{proof}

\begin{comment}

\begin{figure}
\centering
\begin{tikzpicture}[scale=2]
	\draw (-1,0) node[anchor=north] {$-1$} -- (0,0) node[anchor=north] {$0$} -- (1,0) node[anchor=north] {$1$};
	\draw (0,0) -- (0,pi/2) node[anchor=west] {$\displaystyle\frac{\tau}{4}$} -- (0,pi) node[anchor=west] {$\displaystyle\frac{\tau}{2}$};
	\draw[dotted] (1,0) -- (1,pi) -- (-1,pi) -- (-1,0);
	\draw plot [domain=0:pi,smooth] ({cos(\x r)},\x);
\end{tikzpicture}
\caption{Gráfico da função $\cos\inv\colon \intff{-1}{1} \to \intff{0}{\frac{\tau}{2}}$.}
%\label{fig:cossenoinv}
\end{figure}

\begin{figure}
\centering
\begin{tikzpicture}[scale=2]
	\draw (-1,0) node[anchor=east] {$-1$} -- (0,0) node[anchor=east] {$0$} -- (1,0) node[anchor=west] {$1$};
	\draw (0,-pi/2) node[anchor=east] {$-\displaystyle\frac{\tau}{4}$} -- (0,pi/2) node[anchor=east] {$\displaystyle\frac{\tau}{4}$};
	\draw[dotted] (-1,-pi/2) rectangle (1,pi/2);
	\draw plot [domain=-pi/2:pi/2,smooth] ({sin(\x r)},\x);
\end{tikzpicture}
\caption{Gráfico da função $\sin\inv\colon \intff{-1}{1} \to \intff{-\frac{\tau}{4}}{\frac{\tau}{4}}$.}
\label{fig:senoinv}
\end{figure}


\begin{figure}
\centering
\begin{tikzpicture}[scale=1]
%	\draw (-1,0) node[anchor=north] {$-1$} -- (0,0) node[anchor=north] {$0$} -- (1,0) node[anchor=north] {$1$};
%	\draw (0,0) -- (0,pi/2) node[anchor=west] {$\displaystyle\frac{\tau}{4}$} -- (0,pi) node[anchor=west] {$\displaystyle\frac{\tau}{2}$};
%	\draw[dotted] (1,0) -- (1,pi) -- (-1,pi) -- (-1,0);
	\draw plot [domain=-1:1,smooth] (\x,{tan(\x r));
\end{tikzpicture}
\caption{Gráfico da função $\tan\inv\colon \R \to \intaa{\frac{\tau}{2}}{\frac{\tau}{2}}$.}
%\label{fig:cossenoinv}
\end{figure}

\end{comment}




\subsubsection{Rotações em $\R^2$}

\begin{proposition}
A função
	\begin{align*}
	\func{R}{\S^1}{\SO(2)}{u}{
		\begin{aligned}[t]
			\func{R_u}{\R^2}{\R^2}{x}{ux}
		\end{aligned}
	}
	\end{align*}
é um difeomorfismo e isomorfismo de grupos.
\end{proposition}
\begin{proof}
A função $R$ é diferenciável porque $R_u(x)$ é um polinômio em cada entrada. Primeiro, vamos mostrar que $R$ de fato está bem definida, ou seja, que, para todo $u \in \S^1$, $R_u \in \SO(2)$. Seja $u \in \S^1$. A função $R_u$ é linear, pois, para todos $x,x' \in \R^2$ e $c \in \R$, segue da bilinearidade do produto complexo que
	\begin{equation*}
	R_u(cx+c') = u(cx+c') = cux + ux' = cR_u(x) + R_u(x').
	\end{equation*}
Isso mostra que $R_u \in \GL(2,\R)$. Além disso, $R_u$ preserva a norma, pois
	\begin{equation*}
	\nor{R_u(x)} = \nor{ux} = \nor{u}\nor{x} = \nor{x}.
	\end{equation*}
Isso mostra que $R_u \in \OO(2)$. Agora, como $R$ é diferenciável, em particular é contínua, logo, como $\S^1$ é conexo, $R(\S^1)$ é conexo. Isso implica que $R(\S^1) \subseteq \SO(2)$.

Seja $u \in \S^3$ tal que $R_u = \Id$. Então, para todo $x \in \R^2$, $ux = R_u(x) = x$. Em particular, $u = u1 = 1$, o que mostra que $R$ é injetiva.

Para a sobrejetividade, primeiro notamos que, como $R(\S^1) \subseteq \SO(2)$ é compacto, pois $\S^1$ é compacto, e $\SO(2)$ é separado\footnote{Hausdorff}, segue que $R(\S^1)$ é fechado. Além disso, $R$ é difeomorfismo local, pois o núcleo $\{1\}$ é um subgrupo discreto (portanto tem álgebra de Lie trivial), logo $R$ é difeomorfismo local na identidade, e portanto em todo $\S^1$, já que um morfismo de grupos diferenciáveis\footnote{Grupos de lie} tem posto constante. Assim, segue que $R(\S^1) \subseteq \SO(2)$ é aberto. Como $\SO(2)$ é conexo, segue que $R(\S^1) = \SO(2)$ e a inversa de $R$ é diferenciável.
\end{proof}

Para calcular exatamente qual é a rotação que $R_u$ representa em $\SO(2)$, lembremos que, pela decomposição polar de $\R^2$, todo $u \in \S^1$ pode ser escrito como
	\begin{equation*}
	\e^{\theta\ii} = \cos(\theta) + \sin(\theta)\ii,
	\end{equation*}
em que $\theta \in \intff{0}{\tau}$.

A rotação de $x \in \R^2$ por um ângulo $\theta \in \intfa{0}{\tau}$ (em torno da origem) é dada por
	\begin{equation*}
	R^\theta = (\cos(\theta)x_0 - \sin(\theta)x_1) + (\sin(\theta)x_0 + \cos(\theta)x_1) \ii.
	\end{equation*}

Assim segue que, para todo $x \in \R^2$,
	\begin{align*}
	R_{\e^{\theta \ii}}(x) &= (\cos(\theta) + \sin(\theta)\ii)(x_0 + x_1 \ii) \\
		&= (\cos(\theta)x_0 - \sin(\theta)x_1) + (\sin(\theta)x_0 + \cos(\theta)x_1) \ii \\
		&= R^\theta(x),
	\end{align*}
portanto $R(\e^{\theta \ii}) = R^{\theta}$.

Isso mostra que
	\begin{align*}
	\func{R}{\S^1}{\SO(2)}{\e^{\theta \ii}}{
		\begin{aligned}[t]
			\func{R^{\theta}}{\R^2}{\R^2}{x}{\e^{\theta \ii} x}.
		\end{aligned}
	}
	\end{align*}



\subsection{Quatérnios $\R^4$}

\subsubsection{Produto quaterniônico}

Consideremos o espaço linear $\R^4$ com a norma $2$
	\begin{align*}
	\func{\nor{\var}}{\R^4}{\R}{x}{({x_0}^2+{x_1}^2+{x_2}^2+{x_3}^2)^{\frac{1}{2}}}.
	\end{align*}

A base canônica de $\R^4$ será denotada por
	\begin{align*}
	\ii_0 &:= (1,0,0,0) \\
	\ii_1 &:= (0,1,0,0) \\
	\ii_2 &:= (0,0,1,0) \\
	\ii_3 &:= (0,0,0,1).
	\end{align*}

\begin{definition}
O \emph{produto quaterniônico} em $\R^4$ é a operação
%	\begin{align*}
%	\func{\times}{\R^4 \times \R^4}{\R^4}{(x,y)}{
%		\begin{aligned}[t]
%		xy := &(x_0y_0 - x_1y_1 - x_2y_2 - x_3y_3) 1 \\
%			+ &(x_0y_1 + x_1y_0 + x_2y_3 - x_3y_2) \ii_1 \\
%			+ &(x_0y_2 - x_1y_3 + x_2y_0 + x_3y_1) \ii_2 \\
%			+ &(x_0y_3 + x_1y_2 - x_2y_1 + x_3y_0) \ii_3.
%		\end{aligned}
%	}
%	\end{align*}
	\begin{align*}
	\func{\times}{\R^4 \times \R^4}{\R^4}{(x,y)}{
		\begin{aligned}[t]
		xy := (&x_0y_0 - x_1y_1 - x_2y_2 - x_3y_3, \\
			&x_0y_1 + x_1y_0 + x_2y_3 - x_3y_2, \\
			&x_0y_2 - x_1y_3 + x_2y_0 + x_3y_1, \\
			&x_0y_3 + x_1y_2 - x_2y_1 + x_3y_0).
		\end{aligned}
	}
	\end{align*}
A \emph{unidade} de $(\R^4,\times)$ é
	\begin{equation*}
	\bm 1 := \ii_0 = (1,0,0,0).
	\end{equation*}
A \emph{inversa} de $(\R^4,\times,1)$ é a operação
%	\begin{align*}
%	\func{\div}{\R^4 \setminus \{0\}}{\R^4 \setminus \{0\}}{x}{x\inv := \frac{x_0 1 - x_1 \ii_1 - x_2 \ii_2 - x_3 \ii_3}{\nor{x}^2}}.
%	\end{align*}
	\begin{align*}
	\func{\div}{\R^4 \setminus \{0\}}{\R^4 \setminus \{0\}}{x}{x\inv := \nor{x}^{-2} (x_0, -x_1, -x_2, -x_3)}.
	\end{align*}
\end{definition}

Em geral, por não haver ambiguidade, denotaremos $\bm 1$ simplesmente por $1$. Assim, todo $x \in \R^4$ pode ser escrito
	\begin{equation*}
	x = x_0 1 + x_1\ii_1 + x_2\ii_2 + x_3\ii_3.
	\end{equation*}
Como a notação acima sugere, omitiremos o símbolo de multiplicação $\times$ sempre que possível pois isso não gerará ambiguidade.% Omitiremos também a unidade $1$ dos produtos, de modo que todo $x \in \R^4$ pode ser denotado
%	\begin{equation*}
%	x = x_0 + x_1\ii_1 + x_2\ii_2 + x_3\ii_3.
%	\end{equation*}

%Notemos que
%	\begin{align*}
%	{\ii_1}^2 &= (0,1,0,0)(0,1,0,0) = -1, \\
%	{\ii_2}^2 &= (0,0,1,0)(0,0,1,0) = -1, \\
%	{\ii_3}^2 &= (0,0,0,1)(0,0,0,1) = -1.
%	\end{align*}

%Pode-se mostrar que $\times$ é um produto bilinear associativo (mas não comutativo), que $1$ é identidade de $(\R^4,\times)$ e que $\div$ é inversa de $(\R^4,\times,1)$, e que além disso todas operações são preservadas pela norma. As contas são longas, mas diretas, portanto não as apresentaremos aqui. Alguns resultados, como a de que $x\inv$ é a operação inversa, serão provados mais à frente usando a notação de componente escalar, vetorial e conjugado de um quatérnio que simplificará as contas. Ressaltamos esse fato na seguinte proposição.% Isso significa que

\begin{proposition}
A lista
	\begin{equation*}
	((\R^4,+,-,0,\cdot,\times,\div,1),\nor{\var})
	\end{equation*}
é uma álgebra\footnote{A álgebra de $\R^4$ com esse produto costuma ser denotada por $\mathbb{H}$, mas manteremos a notação $\R^4$ por simplicidade de notação, pois nenhuma ambiguidade resultará disso. A letra `H' é usada em homenagem ao matemático irlandês William Rowan Hamilton (1805 -- 1865).} (associativa) invertível normada sobre $\R$.
\begin{enumerate}
	\item (Bilinearidade) O produto quaterniônico $\fun{\times}{\R^4 \times \R^4}{\R^4}$ é uma função bilinear;
	\item (Associatividade) Para todos $x,y,z \in \R^4$,
		\begin{equation*}
		(xy)z = x(yz);
		\end{equation*}
	\item (Identidade) Para todo $x \in \R^4$,
		\begin{equation*}
		x1 = 1x = x;
		\end{equation*}
	\item (Inversa) Para todo $x \in \R^4$,
		\begin{equation*}
		x\inv x = 1 = xx\inv;
		\end{equation*}
	\item (Submultiplicatividade) Para todos $x,y \in \R^4$,
		\begin{equation*}
		\nor{xx'} \leq \nor{x}\nor{x'}.
		\end{equation*}
	\item (Unitariedade) $\nor{1}=1$;
	\item (Invertibilidade) Para todo $x \in \R^4$,
		\begin{equation*}
		\nor{x\inv} = \nor{x}\inv.
		\end{equation*}
\end{enumerate}
\end{proposition}
\begin{proof}
\begin{itemize}
	\item (Identidade) Para todo $x \in \R^4$,
		\begin{align*}
		1x &= (1,0,0,0)(x_0,x_1,x_2,x_3) \\
			&= (1 x_0 - 0 x_1 - 0 x_2 - 0 x_3)1 \\
			&+ (1 x_1 + 0 x_0 + 0 x_3 - 0 x_2)\ii_1 \\
			&+ (1 x_2 - 0 x_3 + 0 x_0 + 0 x_1)\ii_2 \\
			&+ (1 x_3 + 0 x_2 - 0 x_1 + 0 x_0)\ii_3 \\
			&= (x_0,x_1,x_2,x_3) \\
			&= x
		\end{align*}
	e
		\begin{align*}
		x1 &= (x_0,x_1,x_2,x_3)(1,0,0,0) \\
			&= (x_0 1 - x_1 0 - x_2 0 - x_3 0)1 \\
			&+ (x_0 0 + x_1 1 + x_2 0 - x_3 0)\ii_1 \\
			&+ (x_0 0 - x_1 0 + x_2 1 + x_3 0)\ii_2 \\
			&+ (x_0 0 + x_1 0 - x_2 0 + x_3 1)\ii_3 \\
			&= (x_0,x_1,x_2,x_3) \\
			&= x.
		\end{align*}
	
	\item (Bilinearidade) Para mostrar que o produto é bilinear, primeiro mostremos quais são seus valores na base $\{1,\ii_2,\ii_2,\ii_3\}$. Como $1$ é identidade do produto, basta calcular os valores do produto $\ii_1$, $\ii_2$ e $\ii_3$. Cálculos simples mostram que
		\begin{align*}
		\ii_1 \ii_1 &= -1 &\ii_1 \ii_2 = \ii_3 = -\ii_2 \ii_1 \\
		\ii_2 \ii_2 &= -1 &\ii_2 \ii_3 = \ii_1 = -\ii_3 \ii_2 \\
		\ii_3 \ii_3 &= -1 &\ii_3 \ii_1 = \ii_2 = -\ii_1 \ii_3
		\end{align*}
	e, sendo assim, basta mostrar que uma função bilinear $\fun{b}{\R^4 \times \R^4}{\R^4}$ com esses mesmos valores na base canônica coincide com o produto quaterniônico definido. Da bilinearidade segue que
		\begin{align*}
		b(x,y) &= b(x_0 1 + x_1 \ii_1 + x_2 \ii_2 + x_3 \ii_3, y_0 1 + y_1 \ii_1 + y_2 \ii_2 + y_3 \ii_3) \\
			&= x_0 b(1, y_0 1 + y_1 \ii_1 + y_2 \ii_2 + y_3 \ii_3) \\
			&+ x_1 b(\ii_1, y_0 1 + y_1 \ii_1 + y_2 \ii_2 + y_3 \ii_3) \\
			&+ x_2 b(\ii_2, y_0 1 + y_1 \ii_1 + y_2 \ii_2 + y_3 \ii_3) \\
			&+ x_3 b(\ii_3, y_0 1 + y_1 \ii_1 + y_2 \ii_2 + y_3 \ii_3) \\
			&= x_0 (y_0 b(1,1) + y_1 b(1,\ii_1) + y_2 b(1,\ii_2) + y_3 b(1,\ii_3)) \\
			&+ x_1 (y_0 b(\ii_1,1) + y_1 b(\ii_1,\ii_1) + y_2 b(\ii_1,\ii_2) + y_3 b(\ii_1,\ii_3)) \\
			&+ x_2 (y_0 b(\ii_2,1) + y_1 b(\ii_2,\ii_1) + y_2 b(\ii_2,\ii_2) + y_3 b(\ii_2,\ii_3)) \\
			&+ x_3 (y_0 b(\ii_3,1) + y_1 b(\ii_3,\ii_1) + y_2 b(\ii_3,\ii_2) + y_3 b(\ii_3,\ii_3)) \\
			&= x_0 (y_0 1 + y_1 \ii_1 + y_2 \ii_2 + y_3 \ii_3) \\
			&+ x_1 (y_0 \ii_1 + y_1 (-1) + y_2 \ii_3 + y_3 (-\ii_2)) \\
			&+ x_2 (y_0 \ii_2 + y_1 (-\ii_3) + y_2 (-1) + y_3 \ii_1) \\
			&+ x_3 (y_0 \ii_3 + y_1 \ii_2 + y_2 (-\ii_1) + y_3 (-1)) \\
			&= (x_0y_0 - x_1y_1 - x_2y_2 - x_3y_3) 1 \\
			&+ (x_0y_1 + x_1y_0 + x_2y_3 - x_3y_2) \ii_1 \\
			&+ (x_0y_2 - x_1y_3 + x_2y_0 + x_3y_1) \ii_2 \\
			&+ (x_0y_3 + x_1y_2 - x_2y_1 + x_3y_0) \ii_3 \\
			&= xy.
		\end{align*}

	\item (Associatividade) Faremos as contas por coordenadas para que seja tipograficamente possível apresentar os resultados. Para todos $x,y,z \in \R^4$, segue da definição do produto que as coordenadas de $xy$ são
		\begin{align*}
		(xy)_0 &= (x_0y_0 - x_1y_1 - x_2y_2 - x_3y_3) \\
		(xy)_1 &= (x_0y_1 + x_1y_0 + x_2y_3 - x_3y_2) \\
		(xy)_2 &= (x_0y_2 - x_1y_3 + x_2y_0 + x_3y_1) \\
		(xy)_3 &= (x_0y_3 + x_1y_2 - x_2y_1 + x_3y_0)
		\end{align*}
	e que as coordenadas de $yz$ são
		\begin{align*}
		(yz)_0 &= (y_0z_0 - y_1z_1 - y_2z_2 - y_3z_3) \\
		(yz)_1 &= (y_0z_1 + y_1z_0 + y_2z_3 - y_3z_2) \\
		(yz)_2 &= (y_0z_2 - y_1z_3 + y_2z_0 + y_3z_1) \\
		(yz)_3 &= (y_0z_3 + y_1z_2 - y_2z_1 + y_3z_0).
		\end{align*}
	Ainda da definição do produto, as que as coordenadas de $(xy)z$ são
		\begin{align*}
		((xy)z)_0 &= (xy)_0z_0 - (xy)_1z_1 - (xy)_2z_2 - (xy)_3z_3 \\
		((xy)z)_1 &= (xy)_0z_1 + (xy)_1z_0 + (xy)_2z_3 - (xy)_3z_2 \\
		((xy)z)_2 &= (xy)_0z_2 - (xy)_1z_3 + (xy)_2z_0 + (xy)_3z_1 \\
		((xy)z)_3 &= (xy)_0z_3 + (xy)_1z_2 - (xy)_2z_1 + (xy)_3z_0 \\
		\end{align*}
	e as que as coordenadas de $x(yz)$ são
		\begin{align*}
		(x(yz))_0 &= x_0(yz)_0 - x_1(yz)_1 - x_2(yz)_2 - x_3(yz)_3 \\
		(x(yz))_1 &= x_0(yz)_1 + x_1(yz)_0 + x_2(yz)_3 - x_3(yz)_2 \\
		(x(yz))_2 &= x_0(yz)_2 - x_1(yz)_3 + x_2(yz)_0 + x_3(yz)_1 \\
		(x(yz))_3 &= x_0(yz)_3 + x_1(yz)_2 - x_2(yz)_1 + x_3(yz)_0.
		\end{align*}
	Para a coordenada $0$, segue que
		\begin{align*}
		((xy)z)_0 &= (xy)_0z_0 - (xy)_1z_1 - (xy)_2z_2 - (xy)_3z_3 \\
			&= (x_0y_0 - x_1y_1 - x_2y_2 - x_3y_3)z_0 \\
			&+ (x_0y_1 + x_1y_0 + x_2y_3 - x_3y_2)z_1 \\
			&+ (x_0y_2 - x_1y_3 + x_2y_0 + x_3y_1)z_2 \\
			&+ (x_0y_3 + x_1y_2 - x_2y_1 + x_3y_0)z_3 \\
			&= x_0(y_0z_0 - y_1z_1 - y_2z_2 - y_3z_3) \\
			&+ x_1(y_0z_1 + y_1z_0 + y_2z_3 - y_3z_2) \\
			&+ x_2(y_0z_2 - y_1z_3 + y_2z_0 + y_3z_1) \\
			&+ x_3(y_0z_3 + y_1z_2 - y_2z_1 + y_3z_0) \\
			&= x_0(yz)_0 + x_1(yz)_1 + x_2(yz)_2 + x_3(yz)_3 \\
			&= (x(yz))_0;
		\end{align*}
	para a coordenada $1$, segue que
		\begin{align*}
		((xy)z)_1 &= (xy)_0z_1 + (xy)_1z_0 + (xy)_2z_3 - (xy)_3z_2 \\
			&= (x_0y_0 - x_1y_1 - x_2y_2 - x_3y_3)z_1 \\
			&+ (x_0y_1 + x_1y_0 + x_2y_3 - x_3y_2)z_0 \\
			&+ (x_0y_2 - x_1y_3 + x_2y_0 + x_3y_1)z_3 \\
			&- (x_0y_3 + x_1y_2 - x_2y_1 + x_3y_0)z_2 \\
			&= x_0(y_0z_1 + y_1z_0 + y_2z_3 - y_3z_2) \\
			&+ x_1(y_0z_0 - y_1z_1 - y_2z_2 - y_3z_3) \\
			&+ x_2(y_0z_3 + y_1z_2 - y_2z_1 + y_3z_0) \\
			&- x_3(y_0z_2 - y_1z_3 + y_2z_0 + y_3z_1) \\
			&= x_0(yz)_1 + x_1(yz)_0 + x_2(yz)_3 - x_3(yz)_2 \\
			&= (x(yz))_1;
		\end{align*}
	para a coordenada $2$, segue que
		\begin{align*}
		((xy)z)_2 &= (xy)_0z_2 - (xy)_1z_3 + (xy)_2z_0 + (xy)_3z_1 \\
			&= (x_0y_0 - x_1y_1 - x_2y_2 - x_3y_3)z_2 \\
			&- (x_0y_1 + x_1y_0 + x_2y_3 - x_3y_2)z_3 \\
			&+ (x_0y_2 - x_1y_3 + x_2y_0 + x_3y_1)z_0 \\
			&+ (x_0y_3 + x_1y_2 - x_2y_1 + x_3y_0)z_1 \\
			&= x_0(y_0z_2 - y_1z_3 + y_2z_0 + y_3z_1) \\
			&- x_1(y_0z_3 + y_1z_2 - y_2z_1 + y_3z_0) \\
			&+ x_2(y_0z_0 - y_1z_1 - y_2z_2 - y_3z_3) \\
			&+ x_3(y_0z_1 + y_1z_0 + y_2z_3 - y_3z_2) \\
			&= x_0(yz)_2 - x_1(yz)_3 + x_2(yz)_0 + x_3(yz)_1 \\
			&= (x(yz))_2;
		\end{align*}
	e para a coordenada $3$, segue que
		\begin{align*}
		((xy)z)_3 &= (xy)_0z_3 + (xy)_1z_2 - (xy)_2z_1 + (xy)_3z_0 \\
			&= (x_0y_0 - x_1y_1 - x_2y_2 - x_3y_3)z_3 \\
			&+ (x_0y_1 + x_1y_0 + x_2y_3 - x_3y_2)z_2 \\
			&- (x_0y_2 - x_1y_3 + x_2y_0 + x_3y_1)z_1 \\
			&+ (x_0y_3 + x_1y_2 - x_2y_1 + x_3y_0)z_0 \\
			&= x_0(y_0z_3 + y_1z_2 - y_2z_1 + y_3z_0) \\
			&+ x_1(y_0z_2 - y_1z_3 + y_2z_0 + y_3z_1) \\
			&- x_2(y_0z_1 + y_1z_0 + y_2z_3 - y_3z_2) \\
			&+ x_3(y_0z_0 - y_1z_1 - y_2z_2 - y_3z_3) \\
			&= x_0(yz)_3 + x_1(yz)_2 - x_2(yz)_1 + x_3(yz)_0 \\
			&= (x(yz))_3.
		\end{align*}
	Portanto
		\begin{align*}
		((xy)z) &= ((xy)z)_0 1 + ((xy)z)_1 \ii_1 + ((xy)z)_2 \ii_2 + ((xy)z)_3 \ii_3 \\
			&= (x(yz))_0 1 + (x(yz))_1 \ii_1 + (x(yz))_2 \ii_2 + (x(yz))_3 \ii_3 \\
			&= (x(yz)).
		\end{align*}
	
	\item (Inversa) Para todo $x \in \R^4 \setminus \{0\}$,
		\begin{align*}
		x\inv x &= \nor{x}^{-2} (x_0,-x_1,-x_2,-x_3)(x_0,x_1,x_2,x_3) \\
			&= \nor{x}^{-2} (x_0x_0 - (-x_1)x_1 - (-x_2)x_2 - (-x_3)x_3) 1 \\
			&+ \nor{x}^{-2} (x_0x_1 + (-x_1)x_0 + (-x_2)x_3 - (-x_3)x_2) \ii_1 \\
			&+ \nor{x}^{-2} (x_0x_2 - (-x_1)x_3 + (-x_2)x_0 + (-x_3)x_1) \ii_2 \\
			&+ \nor{x}^{-2} (x_0x_3 + (-x_1)x_2 - (-x_2)x_1 + (-x_3)x_0) \ii_3 \\
			&= \nor{x}^{-2} ({x_0}^2 + {x_1}^2 + {x_2}^2 + {x_3}^2) 1 \\
			&= \nor{x}^{-2} \nor{x}^2 1 \\
			&= 1
		\end{align*}
	e
		\begin{align*}
		x x\inv &= (x_0,x_1,x_2,x_3) \nor{x}^{-2}(x_0,-x_1,-x_2,-x_3) \\
			&= \nor{x}^{-2} (x_0x_0 - x_1(-x_1) - x_2(-x_2) - x_3(-x_3)) 1 \\
			&+ \nor{x}^{-2} (x_0(-x_1) + x_1x_0 + x_2(-x_3) - x_3(-x_2)) \ii_1 \\
			&+ \nor{x}^{-2} (x_0(-x_2) - x_1(-x_3) + x_2x_0 + x_3(-x_1)) \ii_2 \\
			&+ \nor{x}^{-2} (x_0(-x_3) + x_1(-x_2) - x_2(-x_1) + x_3x_0) \ii_3 \\
			&= \nor{x}^{-2} ({x_0}^2 + {x_1}^2 + {x_2}^2 + {x_3}^2) 1 \\
			&= \nor{x}^{-2} \nor{x}^2 1 \\
			&= 1.
		\end{align*}
\end{itemize}
\begin{itemize}
	\item (Submultiplicatividade) Demonstramos de fato a multiplicatividade da norma, que implica a submultiplicatividade. Para todos $x,y \in \R^4$,
		\begin{align*}
		\nor{xy}^2 &= (x_0y_0 - x_1y_1 - x_2y_2 - x_3y_3)^2 \\
			&+ (x_0y_1 + x_1y_0 + x_2y_3 - x_3y_2)^2 \\
			&+ (x_0y_2 - x_1y_3 + x_2y_0 + x_3y_1)^2 \\
			&+ (x_0y_3 + x_1y_2 - x_2y_1 + x_3y_0)^2 \\
%			&= (x_0y_0)^2 + (x_1y_1)^2 + (x_2y_2)^2 + (x_3y_3)^2 \\
%			&+ 2 (-x_0y_0x_1y_1 - x_0y_0x_2y_2 - x_0y_0x_3y_3 + x_1y_1x_2y_2 + x_1y_1x_3y_3 + x_2y_2x_3y_3) \\
			&= (x_0y_0)^2 + (x_1y_1)^2 + (x_2y_2)^2 + (x_3y_3)^2 \\
			&- 2 (x_0y_0x_1y_1 + x_0y_0x_2y_2 + x_0y_0x_3y_3 - x_1y_1x_2y_2 - x_1y_1x_3y_3 - x_2y_2x_3y_3) \\
%			&+ (x_0y_1)^2 + (x_1y_0)^2 + (x_2y_3)^2 + (x_3y_2)^2 \\
%			&+ 2 (x_0y_1x_1y_0 + x_0y_1x_2y_3 - x_0y_1x_3y_2 + x_1y_0x_2y_3 - x_1y_0x_3y_2 - x_2y_3x_3y_2) \\
			&+ (x_0y_1)^2 + (x_1y_0)^2 + (x_2y_3)^2 + (x_3y_2)^2 \\
			&+ 2 (x_0y_1x_1y_0 + x_0y_1x_2y_3 - x_0y_1x_3y_2 + x_1y_0x_2y_3 - x_1y_0x_3y_2 - x_2y_3x_3y_2) \\
%			&+ (x_0y_2)^2 + (x_1y_3)^2 + (x_2y_0)^2 + (x_3y_1)^2 \\
%			&+ 2 (-x_0y_2x_1y_3 + x_0y_2x_2y_0 + x_0y_2x_3y_1 - x_1y_3x_2y_0 - x_1y_3x_3y_1 + x_2y_0x_3y_1) \\
			&+ (x_0y_2)^2 + (x_1y_3)^2 + (x_2y_0)^2 + (x_3y_1)^2 \\
			&- 2 (x_0y_2x_1y_3 - x_0y_2x_2y_0 - x_0y_2x_3y_1 + x_1y_3x_2y_0 + x_1y_3x_3y_1 - x_2y_0x_3y_1) \\
%			&+ (x_0y_3)^2 + (x_1y_2)^2 + (x_2y_1)^2 + (x_3y_0)^2 \\
%			&+ 2 (x_0y_3x_1y_2 - x_0y_3x_2y_1 + x_0y_3x_3y_0 - x_1y_2x_2y_1 + x_1y_2x_3y_0 - x_2y_1x_3y_0) \\
			&+ (x_0y_3)^2 + (x_1y_2)^2 + (x_2y_1)^2 + (x_3y_0)^2 \\
			&+ 2 (x_0y_3x_1y_2 - x_0y_3x_2y_1 + x_0y_3x_3y_0 - x_1y_2x_2y_1 + x_1y_2x_3y_0 - x_2y_1x_3y_0) \\
			&= (x_0y_0)^2 + (x_1y_1)^2 + (x_2y_2)^2 + (x_3y_3)^2 \\
			&+ (x_0y_1)^2 + (x_1y_0)^2 + (x_2y_3)^2 + (x_3y_2)^2 \\
			&+ (x_0y_2)^2 + (x_1y_3)^2 + (x_2y_0)^2 + (x_3y_1)^2 \\
			&+ (x_0y_3)^2 + (x_1y_2)^2 + (x_2y_1)^2 + (x_3y_0)^2 \\
			&= {x_0}^2({y_0}^2+{y_1}^2+{y_2}^2+{y_3}^2) \\
			&+ {x_1}^2({y_0}^2+{y_1}^2+{y_2}^2+{y_3}^2) \\
			&+ {x_2}^2({y_0}^2+{y_1}^2+{y_2}^2+{y_3}^2) \\
			&+ {x_3}^2({y_0}^2+{y_1}^2+{y_2}^2+{y_3}^2) \\
			&= ({x_0}^2+{x_1}^2+{x_2}^2+{x_3}^2)({y_0}^2+{y_1}^2+{y_2}^2+{y_3}^2) \\
			&= \nor{x}^2\nor{y}^2.
		\end{align*}
	
	\item (Unitariedade)
		\begin{equation*}
		\nor{1} = (1^2+0^2+0^2+0^2)^{\frac{1}{2}} = 1.
		\end{equation*}
	
	\item (Invertibilidade) Para todo $x \in \R^4 \setminus \{0\}$,
		\begin{align*}
		\nor{x\inv} &= \left(\left(\frac{x_0}{\nor{x}^2}\right)^2 + \left(\frac{-x_1}{\nor{x}^2}\right)^2 + \left(\frac{-x_2}{\nor{x}^2}\right)^2 + \left(\frac{-x_3}{\nor{x}^2}\right)^2\right)^{\frac{1}{2}} \\
			&= \frac{1}{\nor{x}^2}((x_0)^2+(-x_1)^2+(-x_2)^2+(-x_3)^2)^{\frac{1}{2}} \\
			&= \frac{\nor{x}}{\nor{x}^2} \\
			&= \nor{x}\inv.
			\qedhere
		\end{align*}
\end{itemize}
\end{proof}

Demonstramos de fato a multiplicatividade da norma, mas a submultiplicatividade, junto às outras propriedades da norma, implica que, para todos $x,x' \in \R^4$,
	\begin{equation*}
	\nor{xx'} = \nor{x}\nor{x'}.
	\end{equation*}
Como o vetor $1 = (1,0,0,0)$ é uma unidade, em geral o omitiremos da notação da base, denotando $x_0 1$ por $x_0$ e
	\begin{equation*}
	x = x_0 + x_1 \ii_1 + x_2 \ii_2 + x_3 \ii_3.
	\end{equation*}

\begin{comment}
Segue direto da definição do produto que valem as seguintes relações de produto entre os elementos da base:
	\begin{equation*}
	\ii_1\ii_2\ii_3 = {\ii_1}^2 = {\ii_2}^2 = {\ii_3}^2 = -1.
	\end{equation*}
Dessas relações, deduzem-se
	\begin{equation*}
	\ii_1\ii_2 = -\ii_2\ii_1 = \ii_3, \qquad \ii_2\ii_3 = -\ii_3\ii_2 = \ii_1, \qquad \ii_3\ii_1 = -\ii_1\ii_3 = \ii_2.
	\end{equation*}

De fato, em vez da expressão explícita para $\times$ em termos das entradas de $x$ e $y$, poderíamos somente definir o produto entre os elementos de uma base, no caso $\{1,\ii_1, \ii_2, \ii_3\}$, e o produto se estenderia linearmente. Essa definição seria dada como nas relações acima. Com essas relações e representando os elementos $x \in \R^4$ como
	\begin{equation*}
	x = x_0 + x_1 \ii_1 + x_2 \ii_2 + x_3 \ii_3,
	\end{equation*}
pode-se calcular o produto quaterniônico $\times$ em $\R^4$ usando linearidade.


Por linearidade, deve valer
	\begin{align*}
	xy &= (x_0 1  + x_1 \ii_1 + x_2 \ii_2 + x_3 \ii_3)(y_0 1  + y_1 \ii_1 + y_2 \ii_2 + y_3 \ii_3) \\
		&= x_0 y_0 1  1  + x_0 y_1 1  \ii_1 + x_0 y_2 1  \ii_2 + x_0 y_3 1  \ii_3 \\
		&+ x_1 y_0 \ii_1  1  + x_1 y_1 \ii_1  \ii_1 + x_1y_2 \ii_1  \ii_2 + x_1 y_3 \ii_1  \ii_3 \\
		&+ x_2 y_0 \ii_2 1  + x_2 y_1 \ii_2 \ii_1 + x_2 y_2 \ii_2 \ii_2 + x_2 y_3 \ii_2 \ii_3 \\
		&+ x_3 y_0 \ii_3 1  + x_3 y_1 \ii_3 \ii_1 + x_3 y_2 \ii_3 \ii_2 + x_3 y_3 \ii_3 \ii_3
	\end{align*}

Portanto basta definir os valores dos produtos entre $1,i,j,k$. Escolhendo $1$ para ser a unidade, já que deve haver uma unidade (e essa notação não foi escolhida por acaso para $(1,0,0,0)$), basta determinarmos
	\begin{equation*}
	\ii_1\ii_1,\ii_1\ii_2,\ii_1\ii_3,\ii_2\ii_1,\ii_2\ij,\ii_2\ii_3,\ii_3\ii_1,\ii_3\ii_2,\ii_3\ii_3.
	\end{equation*}

\end{comment}

\begin{definition}
Seja $x \in \R^4$. A \emph{componente escalar} de $x$ é (o número) $\esc{x} := x_0 1$ e a \emph{componente vetorial} de $x$ é (o vetor)
	\begin{equation*}
	\vec{x} := x_1 \ii_1 + x_2 \ii_2 + x_3 \ii_3.
	\end{equation*}

O \emph{subespaço de escalares} de $\R^4$ é o subespaço
	\begin{equation*}
	\esc{\R} := \set{x \in \R^4}{\vec{x}=0}
	\end{equation*}
e o \emph{subespaço de vetores} de $\R^4$ é o subespaço
	\begin{equation*}
	\vec{\R}^3 := \set{x \in \R^4}{\esc{x}=0}.
	\end{equation*}
A \emph{esfera de vetores} de $\R^4$ é o conjunto
	\begin{equation*}
	\vec{\S}^2 := \set{x \in \S^3}{\esc{x}=0} = \S^3 \cap \vec{\R}^3.
	\end{equation*}
\end{definition}

Isso nos permite escrever todo quatérnio $x \in \R^4$ unicamente como
	\begin{equation*}
	x = \esc{x} + \vec{x}
	\end{equation*}
e decompor $\R^4$ como a soma direta
	\begin{equation*}
	\R^4 = \esc{\R} + \vec{\R}^3.
	\end{equation*}

\begin{definition}
Seja $x \in \R^4$. O \emph{conjugado} de $x$ é
	\begin{equation*}
	\conju{x} := \esc{x}-\vec{x}.
	\end{equation*}
%Um \emph{versor} é um quatérnio $u \in \HH$ tal que $\esc{u}=0$ e $\nor{\vec{u}}=1$.
\end{definition}

\begin{exercise}
Seja $x \in \R^4$.
	\begin{enumerate}
	\item $\nor{x}^2 = \abs{\esc{x}}^2 + \nor{\vec{x}}^2$;
	\item $\nor{x}^2 = x\conju{x}$;
	\item $\conju{x} = \frac{1}{2}(1x(-1) + \ii_1 x (-\ii_1) + \ii_2 x (-\ii_2) + \ii_3 x (-\ii_3))$.
	\end{enumerate}
\end{exercise}

\begin{proposition}[Decomposição polar]
Seja $x \in \R^4 \setminus \{0\}$. Existem únicos $\phi_0 \in \intff{0}{\tau \div 2}$ e $\hat{x} \in \vec{\S}^2$ tais que
	\begin{equation*}
	x = \nor{x}(\cos(\phi_0) + \sin(\phi_0) \hat{x}).
	\end{equation*}
Eles são dados por $\phi_0 := \cos\inv\left( \frac{\esc{x}}{\nor{x}} \right) \in \intff{0}{\tau \div 2}$ e $\hat{x} := \frac{\vec{x}}{\nor{\vec{x}}} \in \vec{\S}^2$.
\end{proposition}
%\begin{proposition}[Decomposição polar]
%	Seja $x \in \R^4 \setminus \{0\}$. Então % existem $\phi_0 \in \intff{0}{\tau \div 2}$ e $\hat{x} \in \vec{\S}^2$ tais que
%		\begin{equation*}
%		x=\nor{x}(\cos(\phi_0) + \sin(\phi_0) \hat{x}),
%		\end{equation*}
%	em que $\phi_0 := \cos\inv\left( \frac{\esc{x}}{\nor{x}} \right) \in \intff{0}{\tau \div 2}$ e $\hat{x} := \frac{\vec{x}}{\nor{\vec{x}}} \in \vec{\S}^2$.
%\end{proposition}
\begin{proof}
Mostraremos primeiro a decomposição. Como $x \neq 0$, $\nor{x} \neq 0$, logo da igualdade $\nor{x}^2 = \abs{\esc{x}}^2 + \nor{\vec{q}}^2$ segue que
	\begin{equation*}
	1 = \left( \frac{\abs{\esc{x}}}{\nor{x}} \right)^2 + \left( \frac{\nor{\vec{x}}}{\nor{x}} \right)^2,
	\end{equation*}
portanto $\frac{\abs{\esc{x}}}{\nor{x}} \leq 1$, o que é equivalente a $\frac{\esc{x}}{\nor{x}} \in \intff{-1}{1}$. Disso segue que $\phi_0 = \cos\inv\left( \frac{\esc{x}}{\nor{x}} \right) \in \intff{0}{\tau \div 2}$, de modo que $\frac{\nor{\vec{x}}}{\nor{x}} = (1-(\cos(\phi_0))^2)^{1 \div 2}$, o que implica $\frac{\nor{\vec{x}}}{\nor{x}}=\sin(\phi_0)$ ou $\frac{\nor{\vec{x}}}{\nor{x}}=-\sin(\phi_0)$. Como para $\phi_0 \in \intff{0}{\tau \div 2}$ vale $\sin(\phi_0) \geq 0$, concluímos que $\frac{\nor{\vec{x}}}{\nor{x}}=\sin(\phi_0)$.

Das definições de $\phi_0$ e $\hat{x}$, segue que $\esc{x} = \nor{x}\cos(\phi_0)$ e $\vec{x} = \nor{x}\sin(\phi_0)\hat{x}$, portanto
	\begin{align*}
	x &= \esc{x} + \vec{x} \\
		&= \nor{x}\cos(\phi_0) + \nor{x}\sin(\phi_0)\hat{x} \\
		&= \cos(\phi_0) + \sin(\phi_0) \frac{\vec{x}}{\nor{\vec{x}}} \\
		&= \nor{x} (\cos(\phi_0) + \sin(\phi_0) \hat{x}).
		\qedhere
	\end{align*}

Para a unicidade, sejam $\phi_0' \in \intff{0}{\tau \div 2}$ e $\hat{x}' \in \vec{\S}^2$ tais que
	\begin{equation*}
	x = \nor{x}(\cos(\phi_0) + \sin(\phi_0) \hat{x}) = \nor{x}(\cos(\phi_0') + \sin(\phi_0') \hat{x}').
	\end{equation*}
Então $\cos(\phi_0) = \cos(\phi_0')$ e $\sin(\phi_0) \hat{x} = \sin(\phi_0') \hat{x}'$. Da primeira igualdade segue que existe $n \in \Z$ tal que $\phi_0' = \phi_0 + n\tau$ ou $\phi_0' = -\phi_0 + n\tau$. Como $\phi_0,\phi_0' \in \intff{0}{\tau \div 2}$, segue que $\phi_0 = \phi_0'$. Isso implica que $\sin(\phi_0) = \sin(\phi_0')$, portanto da segunda igualdade segue que $\hat{x} = \hat{x}'$.
\end{proof}

\subsubsection{Produtos interno, escalar e vetorial}

\begin{definition}
O \emph{produto interno quaterniônico} em $\R^4$ é a função
	\begin{align*}
	\func{\inteq{\var}{\var}}{\R^4 \times \R^4}{\R}{(x,y)}{\inteq{x}{y} := x_0y_0 - (x_1y_1 + x_2y_2 + x_3y_3)}.
	\end{align*}

	O \emph{produto escalar} em $\vec{\R}^3$ é a função
	\begin{align*}
	\func{\pesc{}{}}{\vec{\R}^3 \times \vec{\R}^3}{\R}{(\vec{x},\vec{y})}{\pesc{\vec{x}}{\vec{y}} := x_1y_1 + x_2y_2 + x_3y_3}.
	\end{align*}

O \emph{produto vetorial} em $\vec{\R}^3$ é a função
	\begin{align*}
	\func{\pvec{}{}}{\vec{\R}^3 \times \vec{\R}^3}{\vec{\R}^3}{(\vec{x},\vec{y})}{
		\begin{aligned}[t]
			\pvec{\vec{x}}{\vec{y}} := &(x_2y_3-x_3y_2) \ii_1 \\
			+ &(-x_1y_3+x_3y_1) \ii_2 \\
			+ &(x_1y_2-x_2y_1) \ii_3.
		\end{aligned}
	}
	\end{align*}
\end{definition}

O produto interno quaterniônico difere do produto interno usual em $\R^4$, mas os outros dois produtos são o produto interno e o produto vetorial usuais em $\R^3$.

\begin{exercise}
	\begin{enumerate}
	\item Para todos $x,y \in \R^4$,
		\begin{equation*}
		\inteq{x}{y} = \esc{x}\esc{y}-\pesc{\vec{x}}{\vec{y}}.
		\end{equation*}

	\item O produto escalar $\bm{\cdot}\colon \vec{\R}^3 \times \vec{\R}^3 \to \R$ é um produto interno no subespaço vetorial $\vec{\R}^3$;
	
	\item O produto vetorial $\bm{\times}\colon \vec{\R}^3 \times \vec{\R}^3 \to \vec{\R}^3$ é uma função bilinear alternada tal que, para todos $\vec{x},\vec{y} \in \vec{\R}^3$,
		\begin{enumerate}
		\item $\pesc{(\pvec{\vec{x}}{\vec{y}})}{\vec{x}} = \pesc{(\pvec{\vec{x}}{\vec{y}})}{\vec{y}} = 0$;
		
		\item $\pesc{(\pvec{\vec{x}}{\vec{y}})}{(\pvec{\vec{x}}{\vec{y}})} = \det
				\begin{bmatrix}
				\pesc{\vec{x}}{\vec{x}} & \pesc{\vec{x}}{\vec{y}} \\
				\pesc{\vec{y}}{\vec{x}} & \pesc{\vec{y}}{\vec{y}}
				\end{bmatrix}$.
		\end{enumerate}
	
	\item Para todos $\vec{x},\vec{y},\vec{z} \in \vec{\R}^3$,
		\begin{equation*}
		\pvec{(\pvec{\vec{x}}{\vec{y}})}{\vec{z}} = (\pesc{\vec{x}}{\vec{z}})\vec{y} - (\pesc{\vec{y}}{\vec{z}})\vec{x};
		\end{equation*}
	
	\item A função
		\begin{align*}
		\func{\vol_{\Yup}}{\vec{\R}^3 \times \vec{\R}^3 \times \vec{\R}^3}{\vec{\R}^3}{(\vec{x},\vec{y},\vec{z})}{\pesc{(\pvec{\vec{x}}{\vec{y}})}{\vec{z}}}
		\end{align*}
é uma função trilinear alternada.

	\end{enumerate}
\end{exercise}


\begin{proposition}
	\begin{enumerate}
	\item Para todos $x,y \in \vec{\R}^3$ ($\esc{x} = \esc{y} = 0$),
		\begin{equation*}
		xy = - \pesc{\vec{x}}{\vec{y}} + \pvec{\vec{x}}{\vec{y}};
		\end{equation*}
	\item Para todos $x,y \in \R^4$,
		\begin{equation*}
		xy = \esc{x}\esc{y} - \pesc{\vec{x}}{\vec{y}} + \esc{x}\vec{y} + \esc{y}\vec{x} + \pvec{\vec{x}}{\vec{y}},
		\end{equation*}
sendo
		\begin{align*}	
		\esc{(xy)} &= \esc{x}\esc{y} - \pesc{\vec{x}}{\vec{y}}, \\
		\vec{(xy)} &= \esc{x}\vec{y} + \esc{y}\vec{x} + \pvec{\vec{x}}{\vec{y}}.
		\end{align*}
	
	\item Para todos $x,y \in \R^4$, $xy=yx$ se, e somente se, $\pvec{\vec{x}}{\vec{y}}=0$;% (ou seja, $\vec{x} \parallel \vec{y}$);
	
	\item Para todo $x \in \R^4 \setminus \{0\}$,
		\begin{equation*}
		x\inv = \frac{\conju{x}}{\nor{x}^2};
		\end{equation*}
	
	\item $\vec{\S}^2 = \set{x \in \R^4}{x^2 = -1}$;

	\item Para todos $x \in \R^4 \setminus \{0\}$ e $y \in \R^4$,
		\begin{equation*}
		xyx\inv = \esc{y} + 2\frac{\pesc{\vec{x}}{\vec{y}}}{\nor{x}^2}\vec{x} + \frac{\inteq{x}{x}}{\nor{x}^2}\vec{y} + 2\frac{\esc{x}}{\nor{x}^2}(\pvec{\vec{x}}{\vec{y}}).
		\end{equation*}

	\item Para todos $x,y \in \R^4 \setminus \{0\}$ tais que $xy \in \esc{\R}$, % vale que $\vec{x}=0$ ou $\vec{y}=0$ ou
	 $\vec{x} \parallel \vec{y}$.
%	  e
%		\begin{equation*}
%		xy = \esc{x}\esc{y} - \nor{\vec{x}}\nor{\vec{y}}.
%		\end{equation*}
	\end{enumerate}
\end{proposition}
\begin{proof}
\begin{enumerate}
	\item Como $x_0=y_0=0$,
	\begin{align*}
	xy =& -(x_1y_1 + x_2y_2 + x_3y_3) \\
			&+ (x_2y_3-x_3y_2) \ii_1 \\
			&+ (-x_1y_3+x_3y_1) \ii_2 \\
			&+ (x_1y_2-x_2y_1) \ii_3 \\
		=& - \pesc{\vec{x}}{\vec{y}} + \pvec{\vec{x}}{\vec{y}}.
	\end{align*}

	\item Segue da bilinearidade do produto e de $\vec{x}$ e $\vec{y}$ serem puramente vetoriais que
	\begin{align*}
	xy &= (\esc{x}+\vec{x})(\esc{y}+\vec{y}) \\
		&= \esc{x}\esc{y}+\esc{x}\vec{y}+\esc{y}\vec{x}+\vec{x}\vec{y} \\
		&= \esc{x}\esc{y}+\esc{x}\vec{y}+\esc{y}\vec{x}+(-\pesc{\vec{x}}{\vec{y}} + \pvec{\vec{x}}{\vec{y}}) \\
		&= (\esc{x}\esc{y} - \pesc{\vec{x}}{\vec{y}}) + (\esc{x}\vec{y}+\esc{y}\vec{x}+\pvec{\vec{x}}{\vec{y}}).
	\end{align*}
	
	\item Para todos $x,y \in \R^4$,
		\begin{align*}
		xy - yx &= \esc{x}\esc{y} - \pesc{\vec{x}}{\vec{y}}+\esc{x}\vec{y}+\esc{y}\vec{x}+\pvec{\vec{x}}{\vec{y}} \\
			&- \esc{y}\esc{x} + \pesc{\vec{y}}{\vec{x}}-\esc{y}\vec{x}-\esc{x}\vec{y}-\pvec{\vec{y}}{\vec{x}} \\
			&= \pvec{\vec{x}}{\vec{y}} - \pvec{\vec{y}}{\vec{x}} \\
			&= 2\pvec{\vec{x}}{\vec{y}}.
		\end{align*}
Isso implica que $xy-yx=0$ se, e somente se, $\pvec{\vec{x}}{\vec{y}} - \pvec{\vec{y}}{\vec{x}}=0$, o que ocorre se, e somente se, $\pvec{\vec{x}}{\vec{y}}=0$.%, que por sua vez é equivalente a $\vec{x} \parallel \vec{y}$.

	\item Para todo $x \in \R^4 \setminus \{0\}$,
	\begin{align*}
	xx\inv &= (\esc{x}+\vec{x})\nor{x}^{-2}(\esc{x}-\vec{x}) \\
		&= \nor{x}^{-2}(\esc{x}^2-\pesc{\vec{x}}{(-\vec{x})}+\esc{x}(-\vec{x})+\esc{x}\vec{x}+\pvec{\vec{x}}{(-\vec{x})}) \\
		&= \nor{x}^{-2}(\esc{x}^2+\pesc{\vec{x}}{\vec{x}}-\esc{x}\vec{x}+\esc{x}\vec{x}-\pvec{\vec{x}}{\vec{x}}) \\
		&= \nor{x}^{-2}\nor{x}^2 \\
		&= 1
	\end{align*}
e, como $\pvec{\vec{x}}{(-\vec{x})} = 0$, $x\inv x = xx\inv = 1$.

	\item Para todo $x \in \R^4$,
		\begin{align*}
		x^2 &= \esc{x}\esc{x} - \pesc{\vec{x}}{\vec{x}}+\esc{x}\vec{x}+\esc{x}\vec{x}+\pvec{\vec{x}}{\vec{x}} \\
			&= \esc{x}^2 - \nor{\vec{x}}^2 + 2\esc{x}\vec{x}.
		\end{align*}
Segue que $\esc{x}=0$ e $\nor{\vec{x}}=1$ se, e somente se, $x^2=-1$.

	\item Para todos $x,y \in \R^4$,
		\begin{align*}
		\esc{(xyx\inv)} &= \esc{(xy)}\esc{(x\inv)} - \pesc{\vec{(xy)}}{\vec{(x\inv)}} \\
			&= (\esc{x}\esc{y} - \pesc{\vec{x}}{\vec{y}})\frac{\esc{x}}{\nor{x}^2} - \pesc{(\esc{x}\vec{y} + \esc{y}\vec{x} + \pvec{\vec{x}}{\vec{y}})}{\frac{(-\vec{x})}{\nor{x}^2}} \\
			&= \frac{\esc{x}^2\esc{y} - \esc{x}(\pesc{\vec{x}}{\vec{y}}) + \esc{x}\pesc{\vec{y}}{\vec{x}} + \esc{y}\pesc{\vec{x}}{\vec{x}} + \pesc{(\pvec{\vec{x}}{\vec{y}})}{\vec{x}}}{\nor{x}^2} \\
			&= \frac{\esc{x}^2\esc{y} + \nor{\vec{x}}^2\esc{y}}{\nor{x}^2} \\
			&= \esc{y}
		\end{align*}
e
		\begin{align*}
		\vec{(xyx\inv)} &= \esc{(xy)}\vec{(x\inv)} + \esc{(x\inv)}\vec{(xy)} + \pvec{\vec{(xy)}}{\vec{(x\inv)}} \\
			&= (\esc{x}\esc{y} - \pesc{\vec{x}}{\vec{y}})\frac{(-\vec{x})}{\nor{x}^2} 
			+ \frac{\esc{x}}{\nor{x}^2}(\esc{x}\vec{y} + \esc{y}\vec{x} + \pvec{\vec{x}}{\vec{y}}) \\
				&\qquad + \pvec{(\esc{x}\vec{y} + \esc{y}\vec{x} + \pvec{\vec{x}}{\vec{y}})}{\frac{(-\vec{x})}{\nor{x}^2}} \\
			&= \frac{(\pesc{\vec{x}}{\vec{y}})\vec{x} + \esc{x}^2\vec{y} + \esc{x}(\pvec{\vec{x}}{\vec{y}})}{\nor{x}^2} \\
				&\qquad - \frac{\esc{x}(\pvec{\vec{y}}{\vec{x})} + \pvec{(\pvec{\vec{x}}{\vec{y}})}{\vec{x}}}{\nor{x}^2} \\
			&= \frac{(\pesc{\vec{x}}{\vec{y}})\vec{x}+ \esc{x}^2\vec{y} + 2\esc{x}(\pvec{\vec{x}}{\vec{y}}) - (\pesc{\vec{x}}{\vec{x}})\vec{y} + (\pesc{\vec{y}}{\vec{x}})\vec{x}}{\nor{x}^2} \\
			&= \frac{2(\pesc{\vec{x}}{\vec{y}})\vec{x} + (\esc{x}^2-\nor{\vec{x}}^2)\vec{y} + 2\esc{x}(\pvec{\vec{x}}{\vec{y}})}{\nor{x}^2} \\
			&= 2\frac{\pesc{\vec{x}}{\vec{y}}}{\nor{x}^2}\vec{x} + \frac{\inteq{x}{x}}{\nor{x}^2}\vec{y} + 2\frac{\esc{x}}{\nor{x}^2}(\pvec{\vec{x}}{\vec{y}}).
		\end{align*}

Portanto
		\begin{equation*}
		xyx\inv = \esc{y} + 2\frac{\pesc{\vec{x}}{\vec{y}}}{\nor{x}^2}\vec{x} + \frac{\inteq{x}{x}}{\nor{x}^2}\vec{y} + 2\frac{\esc{x}}{\nor{x}^2}(\pvec{\vec{x}}{\vec{y}}).
		\end{equation*}

	\item Sejam $x,y \in \R^4 \setminus \{0\}$ tais que $xy \in \esc{\R}$. Então
		\begin{equation*}
		0 = \esc{x}\vec{y}+\esc{y}\vec{x}+\pvec{\vec{x}}{\vec{y}}.
		\end{equation*}
	%Se $\vec{x} \parallel \vec{y}$, então $\vec{x} \times \vec{y}=0$ e existe $c \in \R$ tal que $\vec{y}=c\vec{x}$ ou $\vec{x}=c\vec{y}$. Isso significa que a condição anterior se reduz a
	%	\begin{equation*}
	%	0 = \esc{x}\vec{y}+\esc{y}\vec{x} = (\esc{x}c+\esc{y})\vec{x}
	%	\end{equation*}
	%ou
	%	\begin{equation*}
	%	0 = \esc{x}\vec{y}+\esc{y}\vec{x} = (\esc{x}+\esc{y}c)\vec{y},
	%	\end{equation*}
	%o que implica que $\vec{x}=\vec{y}=0$ ou $c=-\esc{y}/\esc{x}$ ou $c=-\esc{x}/\esc{y}$.
	%
	%Se $\vec{x} \perp \vec{y}$, então $(\vec{x},\vec{y},\vec{x} \times \vec{y})$ é uma base de $\vec{\HH}$
	%
	Isso implica que $(\vec{x},\vec{y},\pvec{\vec{x}}{\vec{y}})$ não é uma base de $\vec{\R}^3$, pois caso contrário
		\begin{equation*}
		\esc{x}\vec{y}+\esc{y}\vec{x}+\vec{x} \times \vec{y} \neq 0,
		\end{equation*}
	já que $(\esc{x},\esc{y},1) \neq (0,0,0)$. Mas a tripla não é base se, e somente se, $\vec{x}=0$ ou $\vec{y}=0$ ou $\pvec{\vec{x}}{\vec{y}}=0$, porque sempre vale $\pesc{\vec{x}}{(\pvec{\vec{x}}{\vec{y}})} = \pesc{\vec{y}}{(\pvec{\vec{x}}{\vec{y}})}=0$. Consideramos cada caso.
		\begin{itemize}
		\item ($\vec{x}=0$) Nesse caso, segue que $\esc{x} \neq 0$, pois $x \neq 0$, e que $\vec{x} \times \vec{y}=0$, logo
			\begin{equation*}
			0 = \esc{x}\vec{y}.
			\end{equation*}
		o que implica que $\vec{y}=0$, portanto $\vec{x} \parallel \vec{y}$ e $x,y \in \esc{\R}$.

		\item ($\vec{y}=0$) Nesse caso, segue que $\esc{y} \neq 0$, pois $y \neq 0$, e que $\vec{x} \times \vec{y}=0$, logo
			\begin{equation*}
			0 = \esc{y}\vec{x},
			\end{equation*}
		o que implica que $\vec{x}=0$, portanto $\vec{x} \parallel \vec{y}$ e $x,y \in \esc{\R}$.

		\item ($\vec{x} \neq 0$, $\vec{y} \neq 0$ e $\pvec{\vec{x}}{\vec{y}}=0$) Nesse caso, segue que $\vec{x} \parallel \vec{y}$.
%		 e que
%			\begin{equation*}
%			\esc{y}\vec{x} = -\esc{x}\vec{y}.
%			\end{equation*}
%		Se $\esc{x} = 0$, então $\esc{y}\vec{x} = 0$, logo $\esc{y} = 0$. Se $\esc{x} \neq 0$, então
%			\begin{equation*}
%			\frac{\esc{y}}{\esc{x}}\vec{x} = -\vec{y} \neq 0,
%			\end{equation*}
%		logo $\esc{y} \neq 0$.
		\end{itemize}

%	Nos primeiros dois casos, $\pesc{\vec{x}}{\vec{y}} = 0 = \nor{\vec{x}}\nor{\vec{y}}$, e no terceiro caso, $\pesc{\vec{x}}{\vec{y}}=\nor{\vec{x}}\nor{\vec{y}}$, portanto
%		\begin{equation*}
%		xy = \esc{(xy)} = \esc{x}\esc{y} - \nor{\vec{x}}\nor{\vec{y}}.
%		\qedhere
%		\end{equation*}
\end{enumerate}
\end{proof}



\subsubsection{Rotações em \ensuremath{\R^3}}

\begin{proposition}
A função
	\begin{align*}
	\func{R}{\S^3}{\SO(3)}{u}{
		\begin{aligned}[t]
		\func{R_u}{\vec{\R}^3}{\vec{\R}^3}{x}{uxu\inv}.
		\end{aligned}
	}
	\end{align*}
é um homomorfismo sobrejetivo de grupos diferenciável e $\nuc(R)=\S^0$.
\end{proposition}
\begin{proof}
A função $R$ é diferenciável porque $R_u(x)$ é um polinômio em cada entrada. Primeiro, vamos mostrar que $R$ de fato está bem definida, ou seja, que, para todo $u \in \S^3$, $R_u \in \SO(3)$. Seja $u \in \S^3$. Pela fórmula de conjugação por quatérnio, temos que, para todo $x \in \vec{\R}^3$,
	\begin{equation*}
	\esc{(R_u(x))} = \esc{(uxu\inv)} = \esc{x} = 0,
	\end{equation*}
portanto $R_u(x) \in \vec{\R}^3$ e então $\fun{R_u}{\vec{\R}^3}{\vec{\R}^3}$. A função $R_u$ é linear, pois, para todos $x,x' \in \vec{\R}^3$ e $c \in \R$, segue da bilinearidade do produto quaterniônico que
	\begin{equation*}
	R_u(cx+x') = u(cx+x')u\inv = (cux+ux')u\inv = cuxu\inv + ux'u\inv = cR_u(x) + R_u(x').
	\end{equation*}
Isso mostra que $R_u \in \GL(3,\R)$. Além disso, $R_u$ preserva norma, pois
	\begin{equation*}
	\nor{R_u(x)} = \nor{uxu\inv} = \nor{u}\nor{x}\nor{u}\inv = \nor{x}.
	\end{equation*}
Isso mostra que $R_u \in \OO(3)$. Agora, como $R$ é diferenciável, em particular é contínua, logo, como $\S^3$ é conexo, $R(\S^3)$ é conexo. Isso implica que $R(\S^3) \subseteq \SO(3)$.

Seja $u \in \S^3$ tal que $R_u=\Id$. Então, para todo $x \in \vec{\R}^3$, $uxu\inv = R_u(x) = x$, logo $ux=xu$. Isso ocorre se, e somente se, $\pvec{\vec{u}}{x} = 0$. Como isso vale para todo $x \in \vec{\R}^3$, segue que $\vec{u}=0$, portanto $u=\esc{u} \in \esc{\R}$; como $u \in \S^3$, concluímos que $u \in \S^0 = \{1,-1\}$, ou $\nuc(R) = \S^0$.

%A sobrejetividade se dá a seguir mostrando qual é a rotação que $R_u$ representa em $\SO(3)$.
Para a sobrejetividade, primeiro notamos que, como $R(\S^3) \subseteq \SO(3)$ é compacto, pois $\S^3$ é compacto, e $\SO(3)$ é separado\footnote{Hausdorff}, segue que $R(\S^3)$ é fechado. Além disso, $R$ é difeomorfismo local, pois o núcleo $\S^0$ é um subgrupo discreto (portanto tem álgebra de Lie trivial), logo $R$ é difeomorfismo local na identidade, e portanto em todo $\S^3$, já que um morfismo de grupos diferenciáveis\footnote{Grupos de lie} tem posto constante. Assim, segue que $R(\S^3) \subseteq \SO(3)$ é aberto. Como $\SO(3)$ é conexo, segue que $R(\S^3) = \SO(3)$.
\end{proof}

Em particular, obtemos desse recobrimento duplo que
	\begin{equation*}
	\quo{\S^3}{\S^0} \simeq \SO(3).
	\end{equation*}

Para calcular exatamente qual é a rotação que $R_u$ representa em $\SO(3)$, lembremos que, pela decomposição polar, todo $u \in \S^3$ pode ser escrito como
	\begin{equation*}
	u = \cos(\phi) + \sin(\phi)\hat{u} = \e^{\phi \hat{u}},
	\end{equation*}
com $\phi = \cos(\esc{u}) \in \intff{0}{\tau \div 2}$ e $\hat{u} = \frac{\vec{u}}{\nor{\hat{u}}} \in \vec{\S}^2$.

A rotação de $x \in \R^3$ por um ângulo $\theta \in \intff{0}{\tau}$ em torno de um vetor unitário $\hat{u} \in \S^2$ é dada por
	\begin{equation*}
	R^{\theta}_{\hat{u}}(x) = \proj_{\parallel \hat{u}}(x) + \cos(\theta) \proj_{\perp \hat{u}}(x) + \sin(\theta) \pvec{\hat{u}}{x}.
	\end{equation*}
%Note que genericamente $(\proj_{\parallel u}(x),\proj_{\perp u}(x),u \times x)$ é uma base de $\R^3$. %se $u$ e $x$ são ...
Isso ocorre pois, se rotacionamos um vetor $x$ em torno de um eixo $\hat{u}$, a componente de $x$ que é paralela a $\hat{u}$, dada pela projeção $\proj_{\parallel \hat{u}}(x)$ de $x$ em $\hat{u}$, fica constante, não rotaciona pois é paralela ao eixo de rotação, enquanto que a componente de $x$ que é perpendicular a $\hat{u}$, dada pela projeção $\pro_{\perp \hat{u}}(x)$, rotaciona no plano perpendicular a $\hat{u}$, dado pela base $\pro_{\perp \hat{u}}(x)$ e $\pvec{\hat{u}}{x}$, e essa rotação é por um ângulo $\theta$, portanto resultará em $\cos(\theta) \proj_{\perp \hat{u}}(x) + \sin(\theta) \pvec{\hat{u}}{x}$. Um detalhe é que o plano perpendicular é dado por essa base se, e somente se, $\hat{u}$ e $x$ não são paralelos, mas quando eles são paralelos vale $\proj_{\parallel \hat{u}}(x)=x$ e $\proj_{\perp \hat{u}}(x) = \pvec{\hat{u}}{x} = 0$, portanto a fórmula acima ainda vale.

Isso significa que
	\begin{equation*}
	R^{\theta}_{\hat{u}} = \proj_{\parallel \hat{u}} + \cos(\theta) \proj_{\perp \hat{u}} + \sin(\theta){ \pvec{\hat{u}}{}} .
	\end{equation*}

Como
	\begin{align*}
	2\sin(\phi)^2 &= 1-\cos(2\phi) \\
	\cos(\phi)^2 - \sin(\phi)^2 &= \cos(2\phi) \\
	2\cos(\phi)\sin(\phi) &= \sin(2\phi)
	\end{align*}
segue que da fórmula de conjugação por quatérnio que
	\begin{align*}
	\e^{\phi \hat{u}}x\e^{-\phi \hat{u}} &= 2(\pesc{\vec{u}}{x})\vec{u} + \inteq{u}{u}x + 2\esc{u}(\pvec{\vec{u}}{x}) \\
		&= 2\sin(\phi)(\pesc{\hat{u}}{x})(\sin(\phi)\hat{u}) + (\cos(\phi)^2-\sin(\phi)^2)x + 2\cos(\phi)\sin(\phi)(\pvec{\hat{u}}{x}) \\
		&= (1-\cos(2\phi))(\pesc{x}{\hat{u}})\hat{u} + \cos(2\phi)x + \sin(2\phi)(\pvec{\hat{u}}{x}) \\
		&= (\pesc{x}{\hat{u}})\hat{u} + \cos(2\phi)(x-(\pesc{x}{\hat{u}})\hat{u}) + \sin(2\phi)(\pvec{\hat{u}}{x}) \\
		&= \proj_{\parallel \hat{u}}(x) + \cos(2\phi)\proj_{\perp \hat{u}}(x) + \sin(2\phi)(x) \\
		&= R^{2\phi}_{\hat{u}}(x).
	\end{align*}

Isso mostra que
	\begin{align*}
	\func{R}{\S^3}{\SO(3)}{\e^{\phi \hat{u}}}{
		\begin{aligned}[t]
		\func{R^{2\phi}_{\hat{u}}}{\vec{\R}^3}{\vec{\R}^3}{x}{\e^{\phi \hat{u}}x\e^{-\phi \hat{u}}}.
		\end{aligned}
	}
	\end{align*}

Como $\nuc(R) = \S^0$, sabemos que $R\inv(R^{\theta}_{\hat{u}}) = \{\e^{\frac{\theta}{2}\hat{u}},-\e^{\frac{\theta}{2}\hat{u}}\}$. Podemos achar $-\e^{\frac{\theta}{2}\hat{u}}$ explicitamente em função de seu ângulo em $\intff{0}{\tau \div 2}$ e eixo de rotação em $\vec{\S}^2$. Como para todo $\phi \in \intff{0}{\tau \div 2}$ vale
	\begin{align*}
	-\cos(\phi) &= \cos((\tau \div 2) - \phi) \\
	\sin(\phi) &= \sin((\tau \div 2) - \phi),
	\end{align*}
segue que
	\begin{align*}
	-\e^{\frac{\theta}{2}\hat{u}} &= -(\cos(\theta \div 2) + \sin(\theta \div 2) \hat{u}) \\
		&= -\cos(\theta \div 2) - \sin(\theta \div 2) \hat{u} \\
		&= \cos((\tau-\theta) \div 2) + \sin((\tau - \theta) \div 2)(-\hat{u}) \\
		&= \e^{\frac{\tau - \theta}{2}(-\hat{u})}.
	\end{align*}
Notemos que como $(\tau \div 2) - \phi \in \intff{0}{\tau \div 2}$ e $-\hat{u} \in \vec{\S}^2$, $\e^{\frac{\tau - \theta}{2}(-\hat{u})}$ é a decomposição polar de $-\e^{\frac{\theta}{2}\hat{u}}$. Obtemos assim
	\begin{equation*}
	R\inv(R^{\theta}_{\hat{u}}) = \{\e^{\frac{\theta}{2}\hat{u}},-\e^{\frac{\tau - \theta}{2}(-\hat{u}})\}.
	\end{equation*}

%Por outro lado, como
%	\begin{align*}
%	-\cos(\theta) &= \cos(\theta + (\tau \div 2)) \\
%	-\sin(\theta) &= \sin(\theta + (\theta \div 2)),
%	\end{align*}
%segue que
%	\begin{align*}
%	-\e^{\frac{\theta}{2}\hat{u}} &= -(\cos(\theta \div 2) + \sin(\theta \div 2) \hat{u}) \\
%		&= -\cos(\theta \div 2) - \sin(\theta \div 2) \hat{u} \\
%		&= \cos((\theta \div 2) + (\tau \div 2)) + \sin((\theta \div 2) + (\theta \div 2))\hat{u} \\
%		&= \e^{\frac{\theta + \tau}{2}\hat{u}},
%	\end{align*}
%portanto temos que
%	\begin{equation*}
%	R\inv(R^{\theta}_{\hat{u}}) = \{\e^{\frac{\theta}{2}\hat{u}},-\e^{\frac{\theta + \tau}{2}\hat{u}}\}.
%	\end{equation*}
% MAS nesse caso $(\theta + \tau) \div 2 \in \intff{0}{\tau \div 2}$





As funções $\proj_{\parallel u}$, $\proj_{\perp u}$ e $\pvec{u\!}{}$ são funções lineares e, na base canônica de $\R^3$, são dadas pelas matrizes
	\begin{equation*}
	[\proj_{\parallel u}] =
		\begin{bmatrix}
			{u_1}^2 & u_1u_2 & u_1u_3 \\
			u_1u_2 & {u_2}^2 & u_2u_3 \\
			u_1u_3 & u_2u_3 & {u_3}^2
		\end{bmatrix},
	\end{equation*}
	\begin{equation*}
	[\proj_{\perp u}] =
		\begin{bmatrix}
			1-{u_1}^2 & -u_1u_2 & -u_1u_3 \\
			-u_1u_2 & 1-{u_2}^2 & -u_2u_3 \\
			-u_1u_3 & -u_2u_3 & 1-{u_3}^2
		\end{bmatrix}
	\end{equation*}
e
	\begin{equation*}
	[\pvec{u\!}{}] =
		\begin{bmatrix}
			0 & -u_3 & u_2 \\
			u_3 & 0 & -u_1 \\
			-u_2 & u_1 & 0
		\end{bmatrix}.
	\end{equation*}

Matricialmente, $[R_u]$ é dada por
	\begin{equation*}
	\begin{bmatrix}
			{u_1}^2 & u_1u_2 & u_1u_3 \\ 
			u_1u_2 & {u_2}^2 & u_2u_3 \\ 
			u_1u_3 & u_2u_3 & {u_3}^2
	\end{bmatrix}
	+\cos(\theta)
	\begin{bmatrix}
		1-{u_1}^2 & -u_1u_2 & -u_1u_3 \\ 
		-u_1u_2 & 1-{u_2}^2 & -u_2u_3 \\ 
		-u_1u_3 & -u_2u_3 & 1-{u_3}^2
	\end{bmatrix}
	+ \sin(\theta)
	\begin{bmatrix}
			0 & -u_3 & u_2 \\ 
			u_3 & 0 & -u_1 \\ 
			-u_2 & u_1 & 0
	\end{bmatrix}
	\end{equation*}
ou
	\begin{equation*}
	(1-\cos(\theta))
	\begin{bmatrix}
			{u_1}^2 & u_1u_2 & u_1u_3 \\ 
			u_1u_2 & {u_2}^2 & u_2u_3 \\ 
			u_1u_3 & u_2u_3 & {u_3}^2
	\end{bmatrix}
	+
	\cos(\theta)\Id + \sin(\theta)
	\begin{bmatrix}
			0 & -u_3 & u_2 \\ 
			u_3 & 0 & -u_1 \\ 
			-u_2 & u_1 & 0
	\end{bmatrix}
	\end{equation*}








\begin{comment}

\subsubsection{Rotações em \ensuremath{\R^3} por quatérnios [Antiga]}

A rotação de $v \in \R^3$ por um ângulo $\theta$ em torno de um vetor unitário $u \in \R^3$ é dada por
	\begin{equation*}
	R^\theta_u(v) = \proj_{\parallel u}(v) + \cos(\theta) \proj_{\perp u}(v) + \sin(\theta) u \times v.
	\end{equation*}
%ou
%	\begin{equation*}
%	R^\theta_u(v) =(1-\cos(\theta)) \proj_{\parallel u}(v) + \cos(\theta) \Id + \sin(\theta) u \times v.
%	\end{equation*}

Note que genericamente $(\proj_{\parallel u}(v),\proj_{\perp u}(v),u \times v)$ é uma base de $\R^3$. %se $u$ e $v$ são ...
Isso significa que
	\begin{equation*}
	R^\theta_u = \proj_{\parallel u} + \cos(\theta) \proj_{\perp u} + \sin(\theta){ u \times} .
	\end{equation*}
As funções $\proj_{\parallel u}$, $\proj_{\perp u}$ e $u \times$ são funções lineares e, na base canônica de $\R^3$, são dadas pelas matrizes
	\begin{equation*}
	[\proj_{\parallel u}] = 
	\begin{bmatrix}
		{u_1}^2 & u_1u_2 & u_1u_3 \\ 
		u_1u_2 & {u_2}^2 & u_2u_3 \\ 
		u_1u_3 & u_2u_3 & {u_3}^2
	\end{bmatrix}
	\end{equation*}
	\begin{equation*}
	[\proj_{\perp u}] = 
	\begin{bmatrix}
		1-{u_1}^2 & -u_1u_2 & -u_1u_3 \\ 
		-u_1u_2 & 1-{u_2}^2 & -u_2u_3 \\ 
		-u_1u_3 & -u_2u_3 & 1-{u_3}^2
	\end{bmatrix}
	\end{equation*}
e
	\begin{equation*}
	[u \times] = 
	\begin{bmatrix}
			0 & -u_3 & u_2 \\ 
			u_3 & 0 & -u_1 \\ 
			-u_2 & u_1 & 0
	\end{bmatrix}
	\end{equation*}




Os quatérnios unitários são os elementos de $\S^3 \subseteq \R^4$ e $\SO(3)$ é o grupo de rotações de $\R^3$. Definimos a função
	\begin{align*}
	\func{R}{\S^3}{\SO(3)}{q}{
		\begin{aligned}[t]
		\func{R_q}{\R^3}{\R^3}{v}{qvq\inv}.
		\end{aligned}
		}
	\end{align*}
Mostremos que essa função está bem definida. Precisamos mostrar que $qvq\inv \in \R^3$ e que $R_q$ é uma rotação. Primeiro, seja $q \in \S^3$. Então, como $\nor{q}=1$,
	\begin{equation*}
	q\inv = \esc{q}-\vec{q}.
	\end{equation*}
Como $v \in \R^3$ é um quatérnio vetorial puro, temos
	\begin{equation*}
	\esc{(vq\inv)} = \esc{v}\esc{q} - \inte{\vec{v}}{-\vec{q}} = \inte{\vec{v}}{\vec{q}},
	\end{equation*}
	\begin{equation*}
	\vec{(vq\inv)} = \esc{v}(-\vec{q})+\esc{q}\vec{v}+\vec{v} \times (-\vec{q}) = \esc{q}\vec{v}-\vec{v} \times \vec{q},
	\end{equation*}
e, portanto,
	\begin{align*}
	\esc{(qvq\inv)} &= \esc{q}\esc{(vq\inv)} - \inte{\vec{q}}{\vec{(vq\inv)}} \\
		&= \esc{q}\inte{\vec{v}}{\vec{q}} - \inte{\vec{q}}{\esc{q}\vec{v}-\vec{v} \times \vec{q}} \\
		&= \esc{q}\inte{\vec{v}}{\vec{q}}-\esc{q}\inte{\vec{q}}{\vec{v}}+\inte{\vec{q}}{\vec{v} \times \vec{q}} \\
		&= \esc{q}\inte{\vec{v}}{\vec{q}}-\esc{q}\inte{\vec{v}}{\vec{q}} \\
		&= 0,
	\end{align*}
o que mostra que $qvq\inv \in \R^3$, ou seja, é um quatérnio vetorial puro.

Para cada $q \in \S^3$, essa função é linear, pois, para todos $c \in \R$ e $v,v' \in \R^3$, segue da bilinearidade e da associatividade do produto e da comutatividade com escalares que
	\begin{align*}
	R_q(cv+v') &= q(cv+v')q\inv \\
		&= q(cvq\inv+v'q\inv) \\
		&= qcvq\inv + qv'q\inv \\
		&= cqvq\inv + qv'q\inv \\
		&= cR_q(v) + R_q(v').
	\end{align*}
A função $R_q$ é uma isometria, pois
	\begin{equation*}
	\nor{R_q(v)} = \nor{qvq\inv} = \nor{q}\nor{v}\nor{q\inv} = \nor{q}\nor{q}\inv\nor{v} = \nor{v}.
	\end{equation*}

Por fim, notemos que
	\begin{align*}
	qvq\inv &= \vec{(qvq\inv)} \\
		&= \esc{q}\vec{(vq\inv)}+\esc{(vq\inv)}\vec{q}+\vec{q} \times \vec{(vq\inv)} \\
		&= \esc{q}(\esc{q}\vec{v}-\vec{v}\times \vec{q})+\inte{\vec{v}}{\vec{q}}\vec{q}+\vec{q} \times (\esc{q}\vec{v}-\vec{v}\times \vec{q}) \\
		&= \esc{q}\esc{q}\vec{v}-\esc{q}\vec{v}\times \vec{q} + \inte{\vec{v}}{\vec{q}}\vec{q} + \esc{q} \vec{q} \times \vec{v} - \vec{q} \times (\vec{v}\times \vec{q}) \\
		&= \esc{q}\esc{q}\vec{v}+2\esc{q}\vec{q}\times \vec{v} + \inte{\vec{v}}{\vec{q}}\vec{q} - \inte{\vec{q}}{\vec{q}}\vec{v} + \inte{\vec{q}}{\vec{v}}\vec{q} \\
		&= (\esc{q}\esc{q} - \inte{\vec{q}}{\vec{q}})\vec{v}+2\esc{q}\vec{q}\times \vec{v} + 2\inte{\vec{v}}{\vec{q}}\vec{q}.
	\end{align*}
	
Como $\nor{q}^2 = \esc{q}^2 + \nor{\vec{q}}^2$, podemos tomar $\theta \in \intff{0}{\tau}$ tal que
	\begin{equation*}
	\esc{q} = \cos(\theta \div 2)
	\end{equation*}
e
	\begin{equation*}
	\nor{\vec{q}} = \sin(\theta \div 2).
	\end{equation*}
Definindo
	\begin{equation*}
	u := \frac{\vec{q}}{\sin(\theta \div 2)},
	\end{equation*}
temos $\nor{u}=1$ e
	\begin{equation*}
	q=\cos(\theta \div 2)+\sin(\theta \div 2)u = \e^{(\theta \div 2)u}.
	\end{equation*}

Segue então que
	\begin{align*}
	qvq\inv 
%		&= \left(\left(\cos\frac{\theta}{2}\right)^2-\left(\sin\frac{\theta}{2}\right)^2\right)v + 2\cos\frac{\theta}{2}\sin\frac{\theta}{2}{u \times v} + 2\left(\sin\frac{\theta}{2}\right)^2\inte{v}{u}u \\
		&= (\cos(\theta \div 2)^2-\sin(\theta \div 2)^2)v + 2\cos(\theta \div 2)\sin(\theta \div 2){u \times v} + 2\sin(\theta \div 2)^2\inte{v}{u}u \\
		&= \cos(\theta) v + \sin(\theta) u \times v + (1-\cos(\theta))\inte{v}{u}u \\
		&= \inte{v}{u}u + \cos(\theta) (v-\inte{v}{u}u) + \sin(\theta) u \times v \\
		&= \proj_{\parallel u}(v) + \cos(\theta) \proj_{\perp u}(v) + \sin(\theta) u \times v \\
		&= R^\theta_u(v).
	\end{align*}

%	\begin{align*}
%	\nor{qvq\inv} &= \left(\nor{\proj_{\parallel u}(v)}^2 + \nor{\cos(\theta) \proj_{\perp u}(v)}^2 + \nor{\sin(\theta) u \times v}^2\right)^{\frac{1}{2}} \\
%		&= \left( (1-\cos(\theta))^2\cos(\alpha)^2\nor{v}^2 + \cos(\theta)^2 \nor{v}^2 + \sin(\theta)^2 \sin(\alpha)^2 \nor{v}^2 \right) \\
%		&= \nor{v}\left( \cos(\alpha)^2-2\cos(\theta)\cos(\alpha)^2+\cos(\theta)^2\cos(\alpha)^2 + \cos(\theta)^2 + \sin(\theta)^2 \sin(\alpha)^2 \right) \\
%	\end{align*}





%%%%%%%%%%%%%%%%%%%%%%%%%%%%%%%%%%%%%%%%%%%%

%Explicitamente, a matriz de $R_q$ na base canônica é
%	\begin{equation*}
%	[R_q] = 
%	\begin{bmatrix}
%		1-2({q_2}^2+{q_3}^2) & 2(q_1q_2-q_3q_0) & 2(q_1q_3+q_2q_0) \\ 
%		2(q_1q_2+q_3q_0) & 1-2({q_1}^2+{q_3}^2) & 2(q_2q_3-q_1q_0) \\ 
%		2(q_1q_3-q_2q_0) & 2(q_2q_3+q_1q_0) & 1-2({q_1}^2+{q_2}^2).
%	\end{bmatrix}
%	\end{equation*}


%	\begin{equation*}
%	\begin{bmatrix}
%			1-2\sin(\theta \div 2)^2({u_2}^2+{u_3}^2) & 2(u_1u_2\sin(\theta \div 2)^2-u_3\sin(\theta \div 2)\cos(\theta \div 2)) & 2(u_1u_3\sin(\theta \div 2)^2+u_2\sin(\theta \div 2)\cos(\theta \div 2)) \\ 
%			2(u_1u_2\sin(\theta \div 2)^2+u_3\sin(\theta \div 2)\cos(\theta \div 2)) & 1-2\sin(\theta \div 2)^2({u_1}^2+{u_3}^2) & 2(u_2u_3\sin(\theta \div 2)^2-u_1\sin(\theta \div 2)\cos(\theta \div 2)) \\ 
%			2(u_1u_3\sin(\theta \div 2)^2-u_2\sin(\theta \div 2)\cos(\theta \div 2)) & 2(u_2u_3\sin(\theta \div 2)^2+u_1\sin(\theta \div 2)\cos(\theta \div 2)) & 1-2\sin(\theta \div 2)^2({u_1}^2+{u_2}^2)
%	\end{bmatrix}
%	\end{equation*}

%Definindo $c:= \cos(\theta \div 2)$ e $s:=\sin(\theta \div 2)$ e tomando $q=\e^{\frac{\theta}{2}u}=c+su$, temos
%	\begin{equation*}
%	\begin{bmatrix}
%			1-2s^2({u_2}^2+{u_3}^2) & 2(u_1u_2s^2-u_3sc) & 2(u_1u_3s^2+u_2sc) \\ 
%			2(u_1u_2s^2+u_3sc) & 1-2s^2({u_1}^2+{u_3}^2) & 2(u_2u_3s^2-u_1sc) \\ 
%			2(u_1u_3s^2-u_2sc) & 2(u_2u_3s^2+u_1sc) & 1-2s^2({u_1}^2+{u_2}^2)
%	\end{bmatrix}
%	\end{equation*}
%Mas ${u_2}^2+{u_3}^2 = 1-{u_1}^2$, etc...
%	\begin{equation*}
%	\begin{bmatrix}
%			1-2s^2(1-{u_1}^2) & 2(u_1u_2s^2-u_3sc) & 2(u_1u_3s^2+u_2sc) \\ 
%			2(u_1u_2s^2+u_3sc) & 1-2s^2(1-{u_2}^2) & 2(u_2u_3s^2-u_1sc) \\ 
%			2(u_1u_3s^2-u_2sc) & 2(u_2u_3s^2+u_1sc) & 1-2s^2(1-{u_3}^2)
%	\end{bmatrix}
%	\end{equation*}
%e, definindo $C:= \cos\theta$ e $S:=\sin\theta$, temos $2s^2 = 1-C$ e $2sc=S$, logo
%	\begin{equation*}
%	\begin{bmatrix}
%			1-(1-C)(1-{u_1}^2) & u_1u_2(1-C)-u_3S & u_1u_3(1-C)+u_2S \\ 
%			u_1u_2(1-C)+u_3S & 1-(1-C)(1-{u_2}^2) & u_2u_3(1-C)-u_1S \\ 
%			u_1u_3(1-C)-u_2S & u_2u_3(1-C)+u_1S & 1-(1-C)(1-{u_3}^2)
%	\end{bmatrix}
%	\end{equation*}
%	\begin{equation*}
%	\begin{bmatrix}
%			{u_1}^2(1-C)+C & u_1u_2(1-C)-u_3S & u_1u_3(1-C)+u_2S \\ 
%			u_1u_2(1-C)+u_3S & {u_2}^2(1-C)+C & u_2u_3(1-C)-u_1S \\ 
%			u_1u_3(1-C)-u_2S & u_2u_3(1-C)+u_1S & {u_3}^2(1-C)+C
%	\end{bmatrix}
%	\end{equation*}

%	\begin{equation*}
%	(1-C)
%	\begin{bmatrix}
%			{u_1}^2 & u_1u_2 & u_1u_3 \\ 
%			u_1u_2 & {u_2}^2 & u_2u_3 \\ 
%			u_1u_3 & u_2u_3 & {u_3}^2
%	\end{bmatrix}
%	+
%	C\Id + S
%	\begin{bmatrix}
%			0 & -u_3 & u_2 \\ 
%			u_3 & 0 & -u_1 \\ 
%			-u_2 & u_1 & 0
%	\end{bmatrix}
%	\end{equation*}

%%%%%%%%%%%%%%%%%%%%%%%%%%%%%%%%%%%%%%%%%%%%

Matricialmente, $[R_q]$ é dada por
	\begin{equation*}
	\begin{bmatrix}
			{u_1}^2 & u_1u_2 & u_1u_3 \\ 
			u_1u_2 & {u_2}^2 & u_2u_3 \\ 
			u_1u_3 & u_2u_3 & {u_3}^2
	\end{bmatrix}
	+\cos(\theta)
	\begin{bmatrix}
		1-{u_1}^2 & -u_1u_2 & -u_1u_3 \\ 
		-u_1u_2 & 1-{u_2}^2 & -u_2u_3 \\ 
		-u_1u_3 & -u_2u_3 & 1-{u_3}^2
	\end{bmatrix}
	+ \sin(\theta)
	\begin{bmatrix}
			0 & -u_3 & u_2 \\ 
			u_3 & 0 & -u_1 \\ 
			-u_2 & u_1 & 0
	\end{bmatrix}
	\end{equation*}
ou
	\begin{equation*}
	(1-\cos(\theta))
	\begin{bmatrix}
			{u_1}^2 & u_1u_2 & u_1u_3 \\ 
			u_1u_2 & {u_2}^2 & u_2u_3 \\ 
			u_1u_3 & u_2u_3 & {u_3}^2
	\end{bmatrix}
	+
	\cos(\theta)\Id + \sin(\theta)
	\begin{bmatrix}
			0 & -u_3 & u_2 \\ 
			u_3 & 0 & -u_1 \\ 
			-u_2 & u_1 & 0
	\end{bmatrix}
	\end{equation*}

\end{comment}










\subsubsection{Rotações em $\R^4$}

\begin{definition}
A função
	\begin{align*}
	\func{R}{\S^3 \times \S^3}{\SO(4)}{(u,v)}{
		\begin{aligned}[t]
		\func{R{u,v}}{\R^4}{\R^4}{x}{uxv\inv}
		\end{aligned}
	}
	\end{align*}
é um homomorfismo sobrejetivo de grupos tal que $\nuc(R) = (1,1)\S^0$.
\end{definition}
\begin{proof}
A função $R$ é diferenciável porque $R_u(x)$ é um polinômio em cada entrada. Primeiro, vamos mostrar que $R$ de fato está bem definida, ou seja, que, para todos $u,v \in \S^3$, $R_{u,v} \in \SO(4)$. Sejam $u,v \in \S^3$. A função $R_{u,v}$ é linear, pois, para todos $x,x' \in \vec{\R}^3$ e $c \in \R$, segue da bilinearidade do produto quaterniônico que
	\begin{equation*}
	R_{u,v}(cx+x') = u(cx+x')v\inv = (cux+ux')v\inv = cuxv\inv + ux'v\inv = cR_{u,v}(x) + R_{u,v}(x').
	\end{equation*}
Isso mostra que $R_{u,v} \in \GL(4,\R)$. Além disso, $R_{u,v}$ preserva norma, pois
	\begin{equation*}
	\nor{R_{u,v}(x)} = \nor{uxv\inv} = \nor{u}\nor{x}\nor{v}\inv = \nor{x}.
	\end{equation*}
Isso mostra que $R_{u,v} \in \OO(4)$. Agora, como $R$ é diferenciável, em particular é contínua, logo, como $\S^3 \times \S^3$ é conexo, $R(\S^3 \times \S^3)$ é conexo. Isso implica que $R(\S^3 \times \S^3) \subseteq \SO(4)$.

Sejam $u,v \in \S^3$ tais que $R_{u,v} = \Id$. Então, para todo $x \in \R^4$, $uxv\inv = R_{u,v}(x) = x$. Em particular, $u1v\inv = 1$, portanto $u=v$ e $R_{u,v}(x) = uxu\inv$. Suponhamos por absurdo que $\vec{u} \neq 0$. Então existiria $\vec{x} \in \vec{\R}^3$ tal que $\pesc{\vec{u}}{\vec{x}} = 0$, de modo que $\{\vec{u},\vec{x},\pvec{\vec{u}}{\vec{x}}\}$ seria uma base de $\vec{\R}^3$. Pela fórmula de conjugação por quatérnio, seguiria que
	\begin{equation*}
	\vec{u} = u\vec{u}u\inv = 2(\pesc{\vec{u}}{\vec{u}})\vec{u} + \inteq{u}{u}\vec{u} + 2 \esc{u}(\pvec{\vec{u}}{\vec{u}}) = (2+\inteq{u}{u})\vec{u},
	\end{equation*}
portanto $\inteq{u}{u} = -1$, e que
	\begin{equation*}
	\vec{x} = u\vec{x}u\inv = 2(\pesc{\vec{u}}{\vec{x}})\vec{u} + \inteq{u}{u}\vec{x} + 2 \esc{u}(\pvec{\vec{u}}{\vec{x}}) = \inteq{u}{u}\vec{x} + 2\esc{u}(\pvec{\vec{u}}{\vec{x}}).
	\end{equation*}
portanto $\inteq{u}{u}=1$, o que é uma contradição. Isso mostra que $\vec{u} = 0$, portanto que $u = \esc{u} \in \esc{\R}$; como $u \in \S^3$, concluímos que $u \in \S^0$, ou seja, $\nuc{R} = \{(1,1),(-1,-1)\} = (1,1)\S^0$.

Para a sobrejetividade, primeiro notamos que, como $R(\S^3 \times \S^3) \subseteq \SO(4)$ é compacto, pois $\S^3 \times \S^3$ é compacto, e $\SO(4)$ é separado\footnote{Hausdorff}, segue que $R(\S^3 \times \S^3)$ é fechado. Além disso, $R$ é difeomorfismo local, pois o núcleo $(1,1)\S^0$ é um subgrupo discreto (portanto tem álgebra de Lie trivial), logo $R$ é difeomorfismo local na identidade, e portanto em todo $\S^3 \times \S^3$, já que um morfismo de grupos diferenciáveis\footnote{Grupos de lie} tem posto constante. Assim, segue que $R(\S^3 \times \S^3) \subseteq \SO(4)$ é aberto. Como $\SO(4)$ é conexo, segue que $R(\S^3 \times \S^3) = \SO(4)$.
\end{proof}



\begin{comment}

Lembremos que, pela decomposição polar, todo $u \in \S^3$ pode ser escrito como
	\begin{equation*}
	u = \cos(\phi) + \sin(\phi)\hat{u} = \e^{\phi \hat{u}},
	\end{equation*}
com $\phi = \cos(\esc{u}) \in \intff{0}{\tau \div 2}$ e $\hat{u} = \frac{\vec{u}}{\nor{\hat{u}}} \in \vec{\S}^2$.





Definimos a função
	\begin{align*}
	\func{R}{\S^3 \times \S^3}{\SO(3)}{(u,u')}{
		\begin{aligned}[t]
		\func{R^{2\phi}_{\hat{u}}}{\vec{\R}^3}{\vec{\R}^3}{x}{\e^{\phi \hat{u}}x\e^{-\phi \hat{u}}}
		\end{aligned}
	}
	\end{align*}

\end{comment}















\subsection{$2$-Hiperbólicos $\R^{1,1}$}

Definiremos o `produto hiperbólico' no espaço linear $\R^2$, cuja base canônica denotaremos por
	\begin{align*}
	\bm 1 &:= (1,0) \\
	\jj &:= (0,1).
	\end{align*}
Como de costume, denotaremos o vetor $\bm 1$ simplesmente por $1$, pois isso não gerará ambiguidade, e sempre que possível ele será sempre omitido da notação, de modo que todo $x \in \R^2$ será denotado
	\begin{equation*}
	x = x_0 + x_1 \jj.
	\end{equation*}
Além disso, denotaremos o espaço $\R^2$ por $\R^{1,1}$ para enfatizar que estamos usando um produto diferente do complexo e que a geometria desse espaço será diferente.

\begin{definition}
O \emph{produto hiperbólico} em $\R^{1,1}$ é a operação
	\begin{align*}
	\func{\htimes}{\R^{1,1} \times \R^{1,1}}{\R^{1,1}}{(x,y)}{xy := (x_0y_0 + x_1y_1, x_0y_1+x_1y_0)}
	\end{align*}
A \emph{unidade} de $(\R^{1,1},\htimes)$ é $1 = (1,0)$.
\end{definition}

Escrevendo esse produto explicitamente na base de $\R^{1,1}$, temos
	\begin{equation*}
	xy = (x_0y_0 + x_1y_1) + (x_0y_1+x_1y_0)\jj.
	\end{equation*}

Notemos que
	\begin{equation*}
	\jj^2 = (0,1)(0,1) = (0 \times 0 + 1 \times 1) + (0 \times 1 + 1 \times 0)\jj = 1.
	\end{equation*}

\begin{proposition}
A lista
	\begin{equation*}
	(\R^{1,1},+,-,0,\htimes,1,\cdot)
	\end{equation*}
é uma álgebra associativa, comutativa e unitária sobre $\R$.
\end{proposition}
\begin{proof}
\begin{itemize}
	\item (Comutatividade) Para todos $x,y \in \R^{1,1}$,
%		\begin{equation*}
%		xy = (x_0y_0 + x_1y_1) + (x_0y_1+x_1y_0)\jj = (y_0x_0 + y_1x_1) + (y_0x_1+y_1x_0)\jj = yx.
%		\end{equation*}
		\begin{equation*}
		xy = (x_0y_0 + x_1y_1,x_0y_1+x_1y_0) = (y_0x_0 + y_1x_1, y_0x_1+y_1x_0) = yx.
		\end{equation*}
	\item (Associatividade) Para todos $x,y,z \in \R^{1,1}$,
%		\begin{align*}
%		(xy)z &= ((x_0y_0 + x_1y_1) + (x_0y_1+x_1y_0)\jj)(z_0+z_1\jj) \\
%				&= ((x_0y_0 + x_1y_1)z_0 + (x_0y_1+x_1y_0)z_1) + ((x_0y_0 + x_1y_1)z_1 + (x_0y_1+x_1y_0)z_0)\jj \\
%				&= (x_0y_0z_0 + x_1y_1z_0 + x_0y_1z_1+x_1y_0z_1) + (x_0y_0z_1 + x_1y_1z_1 + x_0y_1z_0+x_1y_0z_0)\jj \\
%				&= (x_0(y_0z_0+y_1z_1) + x_1(y_0z_1+y_1z_0)) + (x_0(y_0z_1+y_1z_0) + x_1(y_0z_0+y_1z_1))\jj \\
%				&= x(yz).
%		\end{align*}
		\begin{align*}
		(xy)z &= (x_0y_0 + x_1y_1, x_0y_1+x_1y_0)(z_0,z_1) \\
				&= ((x_0y_0 + x_1y_1)z_0 + (x_0y_1+x_1y_0)z_1, (x_0y_0 + x_1y_1)z_1 + (x_0y_1+x_1y_0)z_0) \\
				&= (x_0y_0z_0 + x_1y_1z_0 + x_0y_1z_1+x_1y_0z_1, x_0y_0z_1 + x_1y_1z_1 + x_0y_1z_0+x_1y_0z_0) \\
				&= (x_0(y_0z_0+y_1z_1) + x_1(y_0z_1+y_1z_0), x_0(y_0z_1+y_1z_0) + x_1(y_0z_0+y_1z_1)) \\
				&= x(yz).
		\end{align*}
	\item (Bilinearidade) Como o produto é comutativo, basta mostrar a linearidade na primeira entrada. Para todos $x,x',y \in \R^{1,1}$ e $c \in \R$,
%		\begin{align*}
%		(cx+x')y &= ((cx_0+x'_0)y_0 + (cx_1+x'_1)y_1) + ((cx_0+x'_0)y_1 + (cx_1+x'_1)y_0)\jj \\
%			&= (cx_0y_0+x'_0y_0 + cx_1y_1+x'_1y_1) + (cx_0y_1+x'_0y_1 + cx_1y_0+x'_1y_0))\jj \\
%%			&= (c(x_0y_0+x_1y_1) + x'_0y_0 +x'_1y_1) + (c(x_0y_1+x_1y_0) + x'_0y_1+x'_1y_0))\jj \\
%			&= (cx_0y_0+cx_1y_1) + (cx_0y_1+cx_1y_0)\jj \\
%				&\qquad\qquad\qquad\qquad + (x'_0y_0+x'_1y_1) + (x'_0y_1+x'_1y_0)\jj \\
%			&= c((x_0y_0+x_1y_1) + (x_0y_1+x_1y_0)\jj) \\
%				&\qquad\qquad\qquad\qquad + (x'_0y_0+x'_1y_1) + (x'_0y_1+x'_1y_0)\jj \\
%			&= c(xy) + x'y.
%		\end{align*}
		\begin{align*}
		(cx+x')y &= ((cx_0+x'_0)y_0 + (cx_1+x'_1)y_1, (cx_0+x'_0)y_1 + (cx_1+x'_1)y_0) \\
			&= (cx_0y_0+x'_0y_0 + cx_1y_1+x'_1y_1, cx_0y_1+x'_0y_1 + cx_1y_0+x'_1y_0) \\
%			&= (c(x_0y_0+x_1y_1) + x'_0y_0 +x'_1y_1) + (c(x_0y_1+x_1y_0) + x'_0y_1+x'_1y_0))\jj \\
			&= (cx_0y_0+cx_1y_1, cx_0y_1+cx_1y_0) + (x'_0y_0+x'_1y_1, x'_0y_1+x'_1y_0) \\
			&= c(x_0y_0+x_1y_1, x_0y_1+x_1y_0) + (x'_0y_0+x'_1y_1, x'_0y_1+x'_1y_0) \\
			&= c(xy) + x'y.
		\end{align*}
	\item (Identidade) Como o produto é comutativo, basta mostrar a identidade à esquerda. Para todo $x \in \R^{1,1}$,
%		\begin{equation*}
%		1x = (1+0\jj)(x_0+x_1\jj) = (1x_0+0x_1)+(1x_1+0x_0)\jj = x_0+x_1\jj.
%		\qedhere
%		\end{equation*}
		\begin{equation*}
		1x = (1,0)(x_0,x_1) = (1x_0+0x_1, 1x_1+0x_0) = (x_0, x_1).
		\qedhere
		\end{equation*}
\end{itemize}
\end{proof}

\begin{definition}
Seja $x \in \R^{1,1}$. O \emph{conjugado} de $x$ é $\conju{x} := x_0 - x_1\jj$. A 
\end{definition}




\subsection{$4$-Hiperbólicos $\R^{1,3}$}

Definiremos o `produto hiperbólico' no espaço linear $\R^4$, cuja base canônica denotaremos por
	\begin{align*}
	\bm 1 &:= (1,0,0,0) \\
	\jj_1 &:= (0,1,0,0) \\
	\jj_2 &:= (0,0,1,0) \\
	\jj_3 &:= (0,0,0,1).
	\end{align*}
Como de costume, denotaremos o vetor $\bm 1$ simplesmente por $1$, pois isso não gerará ambiguidade, e sempre que possível ele será sempre omitido da notação, de modo que todo $x \in \R^4$ será denotado
	\begin{equation*}
	x = x_0 + x_1 \jj_1 + x_2 \jj_2 + x_3 \jj_3.
	\end{equation*}
Além disso, denotaremos o espaço $\R^4$ por $\R^{1,3}$ para enfatizar que estamos usando um produto diferente do quaterniônico e que a geometria desse espaço será diferente.

\begin{definition}
O \emph{produto hiperbólico} em $\R^{1,3}$ é a operação
	\begin{align*}
	\func{\htimes}{\R^{1,3} \times \R^{1,3}}{\R^{1,3}}{(x,y)}{
		\begin{aligned}[t]
		xy := &(x_0y_0 + x_1y_1 + x_2y_2 + x_3y_3) 1 \\
			+ &(x_0y_1 + x_1y_0 + x_2y_3 - x_3y_2) \jj_1 \\
			+ &(x_0y_2 - x_1y_3 + x_2y_0 + x_3y_1) \jj_2 \\
			+ &(x_0y_3 + x_1y_2 - x_2y_1 + x_3y_0) \jj_3 .
		\end{aligned}
	}
	\end{align*}
A \emph{unidade} de $(\R^{1,3},\htimes)$
	\begin{equation*}
	1 := (1,0,0,0).
	\end{equation*}
\end{definition}

Notemos que
	\begin{align*}
	(\jj_1)^2 &= 1 \\
	(\jj_2)^2 &= 1 \\
	(\jj_3)^2 &= 1 \\
	\jj_1\jj_2 &= \jj_3 = -\jj_2\jj_1 \\
	\jj_2\jj_3 &= \jj_1 = -\jj_3\jj_2 \\
	\jj_3\jj_1 &= \jj_2 = -\jj_1\jj_3.
	\end{align*}

\begin{exercise}
A lista
	\begin{equation*}
	(\R^{1,3},+,-,0,\htimes,1,\cdot)
	\end{equation*}
é uma álgebra unitária sobre $\R$.
\end{exercise}