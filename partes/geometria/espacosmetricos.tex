\chapter{Espaços métricos}

\section{Espaço métrico}

\subsection{Métricas}

\begin{definition}
Seja $M$ um conjunto. Uma \emph{métrica} (ou \emph{função distância}) em $M$ é uma função
	\begin{align*}
	\func{\dist{\var}{\var}}{M \times M}{\R}{(p,p')}{\dist{p}{p'}}
	\end{align*}
que satisfaz
	\begin{enumerate}
	\item (Separação) Para todos $p,p' \in M$,
		\begin{equation*}
		\dist{p}{p'} = 0 \sse p = p';
		\end{equation*}
	\item (Simetria) Para todos $p,p' \in M$
		\begin{equation*}
		\dist{p}{p'} = \dist{p'}{p};
		\end{equation*}
	\item (Desigualdade Triangular) Para todos $p,p',p'' \in M$,
		\begin{equation*}
		\dist{p}{p''} \leq \dist{p}{p'} + \dist{p'}{p''}.
		\end{equation*}
	\end{enumerate}
A \emph{distância} entre $p$ e $p'$ é o número real $\dist{p}{p'}$.
\end{definition}

Na definição da função distância geralmente se assume que o contradomínio é $\intfa{0}{\infty}$. No entanto, pode-se mostrar que qualquer função real que satisfaz separação, simetria e desigualdade triangular é positiva. Por isso a proposição seguinte.

\begin{definition}
Um \emph{espaço métrico} é um par $\bm M = (M,\dist{\var}{\var})$ em que $M$ é um conjunto e $\dist{\var}{\var}$ é uma métrica em $M$. Os elementos de $M$ são \emph{pontos}. Um \emph{subespaço métrico} de $\bm M$ é o par $\bm S=(S,\dist{\var}{\var}|_{S \times S})$.
\end{definition}

\begin{proposition}
Seja $\bm M$ um espaço métrico. Então
	\begin{enumerate}
	\item (Positividade) Para todos $p,p' \in M$, $\dist{p}{p'} \geq 0$.
	\item (Desigualdade triangular generalizada) Para todos $p_0,\ldots,p_n \in M$,
		\begin{equation*}
		\dist{p_0}{p_n} \leq \sum_{i=1}^{n-1} \dist{p_i}{p_{i+1}}
		\end{equation*}
	\end{enumerate}
\end{proposition}
\begin{proof}
	\begin{enumerate}
	\item Sejam $p,p' \in M$. Da separação, desigualdade triangular e da simetria de $\dist{\var}{\var}$, segue que
	\begin{equation*}
	0 = \frac{\dist{p}{p}}{2} \leq \frac{\dist{p}{p'} + \dist{p'}{p}}{2}= \dist{p}{p'}.
	\end{equation*}

	\item Para $n=1$, seja $p_1 \in M$; então $\dist{p_1}{p_1}=0$ e $\sum_{i=1}^{0} \dist{p_i}{p_{i+1}}=0$, pois a soma é vazia. Para $n=2$, sejam $p_1,p_2 \in M$; então $\dist{p_1}{p_2}$ e $\sum_{i=1}^{1} \dist{p_i}{p_{i+1}}=\dist{p_1}{p_2}$, e vale a propriedade. Para $n=3$, sejam $p_1,p_2,p_3 \in M$; então a propriedade é a desigualdade triangular. Agora, sejam $n \geq 4$, $p_1,\ldots,p_n \in M$ e assumamos que a propriedade vale para todo $k \in \N$, tal que $3 \leq k \leq n-1$. Então
	\begin{equation*}
	\dist{p_1}{p_n} \leq \sum_{i=1}^{n-3} \dist{p_i}{p_{i+1}}+\dist{p_{n-2}}{p_n},
	\end{equation*}
	pois essa soma tem $n-1$ termos e vale a hipótese de indução. Pela desigualdade triangular, vale que $\dist{p_{n-2}}{p_n} \leq \dist{p_{n-2}}{p_{n-1}}+\dist{p_{n-1}}{p_n}$, e, portanto,
	\begin{align*}
	\dist{p_1}{p_n} &\leq \sum_{i=1}^{n-3} \dist{p_i}{p_{i+1}}+\dist{p_{n-2}}{p_n} \\
			&\leq \sum_{i=1}^{n-3} \dist{p_i}{p_{i+1}} + \dist{p_{n-2}}{p_{n-1}}+\dist{p_{n-1}}{p_n} \\
			&= \sum_{i=1}^{n-1} \dist{p_i}{p_{i+1}}.
	\end{align*}
	\end{enumerate}
\end{proof}

Alguns exemplos de métricas seguem. Podemos definir distâncias a partir de distâncias já conhecidas no espaço.

\begin{exercise}
Seja $M$ um conjunto.
	\begin{enumerate}
	\item A \emph{métrica discreta}
		\begin{align*}
		\func{\dist{\var}{\var}_d}{M \times M}{\R}{(p,p')}{
			\begin{cases}
				0,& p = p' \\
				1,& p \neq p'
			\end{cases}
		}
		\end{align*}
é uma métrica sobre $M$.

	\item Se $\dist{\var}{\var}$ é uma métrica em $M$, então
		\begin{align*}
		\func{\dist{\var}{\var}'}{M \times M}{\intfa{0}{\infty}}{(p,p')}{\frac{\dist{p}{p'}}{1+\dist{p}{p'}}}
		\end{align*}
	é uma métrica sobre $M$.
	\end{enumerate}
\end{exercise}


\begin{proposition}
Sejam $M$ um conjunto e $\dist{\var}{\var}_0, \ldots, \dist{\var}{\var}_{n-1}$ métricas em $M$. Então a função
	\begin{align*}
		\func{\dist{\var}{\var}}{M \times M}{\R}{(p,p')}{\sum_{i=0}^{n-1} \dist{p}{p'}_i}
	\end{align*}
é uma métrica em $M$.
\end{proposition}
\begin{proof}
	\begin{enumerate}
	\item (Separação) Sejam $p,p' \in M$. Suponhamos que
	\begin{equation*}
	\dist{p}{p'} = \sum_{i=0}^n \dist{p}{p'}_i = 0.
	\end{equation*}
Como, para todo $i \in [n]$, $\dist{p}{p'}_i \geq 0$, então, para todo $i \in [n]$, $\dist{p}{p'}_i = 0$. Logo $p=p'$. Reciprocamente, suponhamos $p=p'$. Então, para todo $i \in [n]$, $\dist{p}{p'}_i=0$, o que implica
	\begin{equation*}
	\dist{p}{p'} = \sum_{i=0}^{n-1} 0 = 0.
	\end{equation*}
	
	\item (Simetria) Sejam $p,p' \in M$. Então, pela simetria de $\dist{\var}{\var}_i$ para todo $i \in [n]$,
	\begin{equation*}
	\dist{p}{p'} = \sum_{i=0}^{n-1} \dist{p}{p'}_i = \sum_{i=1}^{n-1} \dist{p}{p'}_i = \dist{p}{p'}.
	\end{equation*}
	
	\item (Desigualdade Triangular) Sejam $p,p',p'' \in M$. Então, para todo $i \in [n]$, vale $\dist{p}{p''}_i \leq \dist{p}{p'}_i+\dist{p'}{p''}_i$ pela desigualdade triangular de $\dist{\var}{\var}_i$, e segue que
	\begin{align*}
	\dist{p}{p''} &= \sum_{i=0}^{n-1} \dist{p}{p''}_i \\
				&\leq \sum_{i=0}^{n-1} (\dist{p}{p'}_i+\dist{p'}{p''}_i) \\
				&= \sum_{i=0}^{n-1} \dist{p}{p'}_i + \sum_{i=0}^{n-1} \dist{p'}{p''}_i \\
				&= \dist{p}{p'}+\dist{p'}{p''}. \qedhere
	\end{align*}
	\end{enumerate}
\end{proof}

\begin{proposition}
Sejam $M$ um conjunto não vazio e $\dist{\var}{\var}_0, \ldots,\dist{\var}{\var}_{n-1}$ métricas em $M$. Então a função
	\begin{align*}
	\func{\dist{\var}{\var}}{M \times M}{\R}{(p,p')}{\max\set{\dist{p}{p'}_i}{i \in [n]}}
	\end{align*}
é uma métrica sobre $M$.
\end{proposition}
\begin{proof}
Demonstraremos para $n=2$, pois o caso geral é análogo.
	\begin{enumerate}
	\item (Separação) Sejam $p,p' \in M$. Suponhamos que
	\begin{equation*}
	\dist{p}{p'}=\max\{\dist{p}{p'}_1,\dist{p}{p'}_2\}=0.
	\end{equation*}
Então $\dist{p}{p'}_1=0$ ou $\dist{p}{p'}_2=0$. Em ambos os casos, temos $p=p'$. Reciprocamente, suponhamos que $p=p'$. Então $\dist{p}{p'}_1=0$ e $\dist{p}{p'}_2=0$, o que implica $\dist{p}{p'}=\max\{\dist{p}{p'}_1,\dist{p}{p'}_2\}=0$.
	
	\item (Simetria) Sejam $p,p' \in M$. Então
	\begin{equation*}
	\dist{p}{p'} = \max\{\dist{p}{p'}_1,\dist{p}{p'}_2\} = \max\{\dist{p'}{p}_1,\dist{p'}{p}_2\} = \dist{p'}{p}.
	\end{equation*}
	
	\item (Desigualdade Triangular) Sejam $p,p',p'' \in M$. Então $\dist{p}{p''}=\dist{p}{p''}_1$ ou $\dist{p}{p''}=\dist{p}{p''}_2$. No primeiro caso, segue que
	\begin{equation*}
	\dist{p}{p''} = \dist{p}{p''}_1 \leq \dist{p}{p'}_1 + \dist{p'}{p''}_1 \leq \dist{p}{p'} + \dist{p'}{p''}.
	\end{equation*}
	No segundo caso, segue que
	\begin{equation*}
	\dist{p}{p''} = \dist{p}{p''}_2 \leq \dist{p}{p'}_2 + \dist{p'}{p''}_2 \leq \dist{p}{p'} + \dist{p'}{p''}. \qedhere
	\end{equation*}
	\end{enumerate}
\end{proof}

\begin{exercise}[Métrica Limitada]
Seja $(M,\dist{\var}{\var})$ um espaço métrico. A função
	\begin{align*}
	\func{\dist{\var}{\var} \opmin 1}{M \times M}{\R}{(p,p')}{\dist{p}{p'} \opmin 1}
	\end{align*}
é uma métrica sobre $M$, a \emph{métrica limitada induzida por $\dist{\var}{\var}$}.
\end{exercise}

\begin{exercise}[Métricas $p$]
Sejam $\bm M$ e $\bm M'$ espaços métricos e $p \in \intfa{1}{\infty}$.
A função
	\begin{align*}
	\func{\dist{\var}{\var}_p}{(M \times M') \times (M \times M')}{\R}{((x,x'),(y,y'))}{(\dist{x}{y}^p + \dist{x'}{y'}'^p)^{\frac{1}{p}}}
	\end{align*}
é uma métrica sobre $M \times M'$.
\end{exercise}

\begin{definition}
Seja $d \in \N$. A \emph{métrica reta} sobre $\R^d$ é a função
	\begin{align*}
	\func{\dist{\var}{\var}_{\R^d}}{\R^d}{\R^d}{(x,x')}{\left( \sum_{i \in [d]} \abs{x_i-x'_i}^2 \right)^{\frac{1}{2}}}.
	\end{align*}
\end{definition}


\subsection{Diâmetro, bolas e conjuntos e funções limitadas}

\begin{definition}
Sejam $\bm M$ um espaço métrico e $C \subseteq M$. O \emph{diâmetro} de $C$ é
	\begin{equation*}
	\diam(C) := \sup\set{\dist{p}{p'}}{p,p' \in C}
	\end{equation*}
se $\set{\dist{p}{p'}}{p,p' \in C}$ é limitado superiormente, e $\infty$, caso contrário. Um \emph{conjunto limitado em $\bm M$} é um conjunto $C \subseteq M$ tal que $\diam(C) < \infty$.
\end{definition}

Na definição, adotamos a convenção de que $\sup \emptyset = 0$ e $\sup \intfa{0}{\infty} = \infty$. Isso é só parcialmente uma convenção, pois a ambiguidade não está em que valor atribuir a $\sup \emptyset$, mas sim em qual conjunto parcialmente ordenado está sendo considerado. Por exemplo, quando o conjunto ordenado é $\intff{0}{\infty}$, $\sup_{\intff{0}{\infty}} \emptyset = 0$, e quando o conjunto ordenado é $\intff{-\infty}{+\infty}$, $\sup_{\intff{-\infty}{+\infty}} \emptyset = -\infty$.

Isso define uma função
	\begin{align*}
	\func{\diam}{\p(M)}{\intff{0}{\infty}}{C}{\diam(C)}.
	\end{align*}

\begin{definition}
Sejam $\bm M$ um espaço métrico, $c \in M$ e $r \in \intfa{0}{\infty}$. A \emph{bola aberta} de centro $c$ e raio $r$ em $M$ é o conjunto
	\begin{equation*}
	\bola{c}{r} := \set{p \in M}{\dist{c}{p} < r}.
	\end{equation*}
A \emph{bola fechada} de centro $c$ e raio $r$ em $M$ é o conjunto
	\begin{equation*}
	\bolafec{c}{r} := \set{p \in M}{\dist{c}{p} \leq r}.
	\end{equation*}
\end{definition}

\begin{figure}
\centering
\begin{tikzpicture}[scale=2]
	\draw[dashed] (0,0) circle (1);
	\draw[dotted] (0,0) node {$\bullet$} node[anchor=east] {$c$} -- (1/2,0) node[anchor=south] {$r$} -- (1,0);
\end{tikzpicture}\hspace{3cm}
\begin{tikzpicture}[scale=2]
	\draw (0,0) circle (1);
	\draw[dotted] (0,0) node {$\bullet$} node[anchor=east] {$c$} -- (1/2,0) node[anchor=south] {$r$} -- (1,0);
\end{tikzpicture}
\caption{Bolas aberta e fechada de centro $c$ e raio $r$, respectivamente.}
\end{figure}

\begin{proposition}
Sejam $\bm M$ um espaço métrico e $C \subseteq M$ um conjunto. Então $C$ é um conjunto limitado se, e somente se, existe bola $\bolafec{c}{r}$ tal que $C \subseteq \bolafec{c}{r}$.
\end{proposition}
\begin{proof}
Se $C$ é limitado, basta tomar $r=\diam(C)$ e $c \in C$. Reciprocamente, se existe bola $\bolafec{c}{r}$ tal que $C \subseteq \bolafec{c}{r}$, então para todos $p,p' \in C$, segue da desigualdade triangular que
	\begin{equation*}
	\dist{p}{p'} \leq \dist{p}{c}+\dist{c}{p'} \leq r+r=2r,
	\end{equation*}
portanto $\diam(C) \leq 2r \in \intfa{0}{\infty}$.
\end{proof}

\begin{exercise}
Seja $\bm M$ um espaço métrico. 
	\begin{enumerate}
	\item Para todos $C,C' \subseteq M$ tais que $C \subseteq C'$,
		\begin{equation*}
		\diam(C) \leq \diam (C').
		\end{equation*}
	
	\item Para todos $C, C' \subseteq M$,
		\begin{equation*}
		\diam(C \cup C') = \max \{ \diam(C),\diam(C'),\sup\set{\dist{p}{p'}}{p \in C, p' \in C'} \}.
		\end{equation*}
	
%	\item Para todos $C, C' \subseteq M$,
%		\begin{equation*}
%		\diam(C \cup C') \leq \min \{ \diam(C),\diam(C') \};
%		\end{equation*}
	
	\item Para todo $r \in \intfa{0}{\infty}$ e todo $p \in M$.
	\begin{equation*}
	\diam(\bola{p}{r}) \leq 2r.
	\end{equation*}
	\end{enumerate}
\end{exercise}

Note que o diâmetro da bola de raio $r$ não necessariamente é $2r$, por exemplo para $r=2$ na métrica discreta.

\begin{exercise}
Sejam $M$ um conjunto, $\dist{\var}{\var}$ uma métrica em $M$ e $\dist{\var}{\var} \opmin 1$ a métrica limitada sobre $M$. Todo subconjunto de $M$ é limitado com respeito a $\dist{\var}{\var} \opmin 1$.
\end{exercise}

\begin{definition}
Seja $\bm M$ um espaço métrico. Uma função \emph{limitada} em $\bm M$ é uma função $f\colon X \to M$ de um conjunto $X$ para $M$ cuja imagem $f(X)$ é limitada.
\end{definition}

\section{Topologia dos espaços métricos}

\subsection{Interior e pontos interiores}

\begin{definition}
Sejam $\bm M$ um espaço métrico e $C \subseteq M$ um conjunto. Um \emph{ponto interior} de $C$ é um ponto $p \in C$ para o qual existe um número real $r > 0$ tal que $\bola{p}{r} \subseteq C$. O \emph{interior} de $C$ é o conjunto $\Int{C}$ de todos pontos interiores de $C$. Um \emph{conjunto aberto} de $\bm M$ é um conjunto $A \subseteq M$ tal que $A = A^\circ$. O conjunto dos conjuntos abertos de $\bm M$ é denotado $\topo_{\bm M}$.
\end{definition}

\begin{proposition}
Seja $\bm M = (M,\dist{\var}{\var})$ um espaço métrico. Então
	\begin{enumerate}
	\item Para todo $c \in M$ e para todo número real $r > 0$, a bola aberta $\bola{c}{r}$ é um conjunto aberto;
	\item O conjunto $\topo_{\bm M}$ é uma topologia de $M$.
	\end{enumerate}
\end{proposition}
\begin{proof}
	\begin{enumerate}
	\item Sejam $c \in M$ e $r \in \intaa{0}{\infty}$. Queremos mostrar que $\bola{c}{r}$ é aberto. Para isso, seja $p \in \bola{c}{r}$. Então segue que $d := \dist{c}{p} < r$, pela definição de bola aberta, e, portanto, $r-d \in \intaa{0}{\infty}$. Para mostrar que essa bola centrada em $p$ está contida na bola maior centrada em $c$, seja $p' \in \bola{p}{r-d}$. Então $\dist{p}{p'}<r-d$ e, pela desigualdade triangular, segue que
	\begin{equation*}
	\dist{c}{p'} \leq \dist{c}{p} + \dist{p}{p'} < d + (r-d) = r,
	\end{equation*}
o que mostra que $p' \in \bola{c}{r}$ e que, portanto, $\bola{p}{s} \subseteq \bola{c}{r}$. Assim, mostramos que $\bola{p}{r}$ é aberta.
	
	\item
		\begin{enumerate}
		\item Podemos notar que $\emptyset$ é aberto por vacuidade, pois, se não fosse, existiria $p \in \emptyset$ para o qual não há $r \in \intaa{0}{\infty}$ satisfazendo $\bola{p}{r} \subseteq A$, o que é absurdo.
	Para mostrar que $M$ é aberto, sejam $p \in M$ e $r \in \intaa{0}{\infty}$. Então $\bola{p}{r} \subseteq M$, pois qualquer bola aberta é subconjunto de $M$. Portanto $M$ é aberto.
	
		\item Seja $(A_i)_{i \in I}$ uma família de abertos em $\bm M$ e seja $p \in (A_i)_{i \in I}$. Então existe $k \in I$ tal que $p \in A_k$. Como $A_k$ é aberto, então existe $r \in \intaa{0}{\infty}$ tal que $\bola{p}{r} \subseteq A_k$. Como $A_k \subseteq (A_i)_{i \in I}$, segue que $\bola{p}{r} \subseteq (A_i)_{i \in I}$ e que, portanto, $(A_i)_{i \in I}$ é aberto.
	
		\item Seja $(A_i)_{i \in [n]}$ uma sequência de abertos em $\bm M$ e seja $p \in (A_i)_{i \in [n]}$. Então, para todo $k \in [n]$, $p \in A_k$. Como, para todo $k \in [n]$, $A_k$ é aberto, segue que existe $r_k \in \intaa{0}{\infty}$ tal que $\bola{p}{r-k} \subseteq A_k$, Seja $r := \min \{r_k : k \in [n]\}$. Então, para todo $k \in [n]$, vale $\bola{p}{r} \subseteq \bola{p}{r-k}$, e segue que $\bola{p}{r} \subseteq A_k$ e, portanto, $\bola{p}{r} \subseteq (A_i)_{i \in [n]}$, o que mostra que $(A_i)_{i \in [n]}$ é aberto.
		\end{enumerate}		
	\end{enumerate}
\end{proof}

\begin{exercise}
Sejam $M$ um conjunto e $\dist{\var}{\var}$ uma métrica em $M$. A métrica limitada $\dist{\var}{\var} \opmin 1$ sobre $M$ induz a mesma topologia que $\dist{\var}{\var}$ sobre $M$.
\end{exercise}

\subsection{Limites e convergência de sequências}

\begin{definition}
Sejam $\bm M$ um espaço métrico, $(p_n)_{n \in \N}$ uma sequência de pontos em $M$ e $p \in M$. A sequência $(p_n)_{n \in \N}$  \emph{converge} para o ponto $p$ se, e somente se, para todo número real $\varepsilon > 0$, existe um número natural $N$ tal que
	\begin{equation*}
	\forall n \in \N \qquad n \geq N \Rightarrow p_n \in \bola{p}{\varepsilon}.
	\end{equation*}
Denota-se $(p_n)_{n \in \N} \conv p$. O ponto $p$ é um \emph{limite} da sequência.  Caso contrário, a sequência não converge para $p$. Uma \emph{sequência convergente} é uma sequência que tem limite. Uma sequência \emph{divergente} é uma sequência que não tem limite.
\end{definition}

\begin{proposition}
Todo espaço métrico $\bm M$ é um espaço topológico separado.
\end{proposition}
\begin{proof}
Sejam $p,p' \in M$ pontos distintos. Mostraremos que existe um número real $r$ tal que $0 < r \leq \frac{1}{2} \dist{p}{p'}$, e que isso implica que $\bola{p}{r} \cap \bola{p'}{r} = \emptyset$. Como $p \neq p'$, então $\dist{p}{p'} > 0$, portanto existe $r \in \R$ tal que $0 < r \leq \frac{1}{2} \dist{p}{p'}$. Suponhamos que existe $p'' \in \bola{p}{r} \cap \bola{p'}{r}$. Então $\dist{p}{p''}<r$ e $\dist{p'}{p''}<r$. Mas, pela desigualdade triangular, segue que
	\begin{equation*}
	\dist{p}{p'} \leq \dist{p}{p''} + \dist{p''}{p'} < r + r \leq \dist{p}{p'},
	\end{equation*}
o que é absurdo. Portanto $\bola{p}{r} \cap \bola{p'}{r} = \emptyset$.
\end{proof}

\begin{corollary}
Toda sequência convergente em um espaço métrico $\bm M$ tem limite único.
\end{corollary}
\begin{proof}
Suponhamos que $p,p'$ são limites de $(p_n)_{n \in \N}$. Se $p \neq p'$, então $\dist{p}{p'}>0$. Seja $\varepsilon \in \R$ tal que $0 < \varepsilon \leq \frac{1}{2} \dist{p}{p'}$. Então existe $N_1 \in \N$ tal que, para todo $n \in \N$, se $n \geq N_1$, então $p_n \in \bola{p}{\varepsilon}$, e existe $N_2 \in \N$ tal que, para todo $n \in \N$, se $n \geq N_2$, então $p_n \in \bola{p'}{\varepsilon}$. Assim, definindo $N := \max \{N_1,N_2\}$, segue que, se $n \geq N$, então $n \geq N_1$ e $n \geq N_2$, e, portanto, que $p_n \in \bola{p}{\varepsilon}$ e $p_n \in \bola{p'}{\varepsilon}$; ou seja, $p_n \in \bola{p}{\varepsilon} \cap \bola{p'}{\varepsilon}$, mas isso é absurdo, pois $\bola{p}{\varepsilon} \cap \bola{p'}{\varepsilon}=\emptyset$. Portanto $p=p'$.
\end{proof}

Essa proposição nos permite tratar o limite de uma sequência como um número único e, por isso, podemos usar a notação $\displaystyle\lim_{n \in \N} p_n = p$ para quando $(p_n)_{n \in \N} \conv p$.

\begin{proposition}
Uma sequência de em um espaço métrico $\bm M$ é convergente se, e somente se, todas suas subsequências são convergentes.
\end{proposition}
\begin{proof}
	Suponhamos que $(p_n) \conv p$ e seja $(p_{n_k})_{k \in \N}$ uma subsequência de $(p_n)_{n \in \N}$. Seja $\varepsilon \in \R$ tal que $\varepsilon > 0$. Como $(p_n) \conv p$, existe $N \in \N$ tal que, para todo $n \in \N$, se $n \geq N$, então $p_n \in \bola{p}{\varepsilon}$; como $(n_k)_{k \in \N}$ é estritamente crescente, existe $K \in \N$ tal que, para todo $k \in \N$, se $k \geq K$, então $n_k \geq N$. Mas então
	\begin{equation*}
	k \geq K \Rightarrow n_k \geq N \Rightarrow p_{n_k} \in \bola{p}{\varepsilon}
	\end{equation*}
e, portanto, $(p_{n_k}) \to p$.	Reciprocamente, se toda subsequência de $(p_n)_{n \in \N}$ converge para $p$, $(p_n)_{n \in \N}$, em particular, é uma dessas subsequências e, portanto, $(p_n) \conv p$.
\end{proof}

\begin{proposition}
Toda sequência convergente em um espaço métrico $\bm M$ é limitada.
\end{proposition}
\begin{proof}
	Seja $(p_n)_{n \in \N}$ uma sequência de pontos em $M$ tal que $(p_n) \conv p$. Então, para $\varepsilon = 1$, existe $N \in \N$ tal que, para todo $n \in \N$, se $n \geq N$, então $p_n \in \bola{p}{1}$. Assim, seja $l \in \R$ tal que
	\begin{equation*}
	l > \max(\{1\} \cup \{\dist{p}{p_n} : n \in [N]\}),
	\end{equation*}
seque que, para todo $n \in \N$, $p_n \in \bola{p}{l}$ pois, se $0 \leq n \leq N$, $\dist{p}{p_n} < l$ pela definição de $l$ e, se $n \geq N$, então $p_n \in \bola{p}{1} \subseteq B_l(p)$, pois $1 < l$. Logo $(p_n)_{n \in \N}$ é limitada.
\end{proof}

\begin{proposition}
Sejam $\bm M$ um espaço métrico, $C \subseteq M$ um conjunto e $p \in M$. Então existe uma sequência de pontos em $C$ que converge para $p$ se, e somente se, para todo número real $\varepsilon > 0$, $C \cap \bola{p}{\varepsilon} \neq \emptyset$.
\end{proposition}
\begin{proof}
	Suponhamos que exista uma sequência $(p_n)_{n \in \N}$ de pontos em $C$ tal que $(p_n) \conv p$. Então, para todo número real $\varepsilon > 0$, existe $N \in \N$ tal que, para todo $n \in \N$, se $n \geq N$, então $p_n \in \bola{p}{\varepsilon}$. Mas isso implica que $p_n \in C \cap \bola{p}{\varepsilon}$. Reciprocamente, suponhamos que, para todo número real $\varepsilon > 0$, $C \cap \bola{p}{\varepsilon} \neq \emptyset$. Então, em particular, para todo $n \in \N$, escolhamos $p_n \in C \cap \bola{p}{\frac{1}{n}}$. Assim, temos a sequência $(p_n)_{n \in \N}$. Para mostrar que $(p_n) \conv p$, seja $\varepsilon \in \R$ tal que $\varepsilon > 0$. Então existe $N \in \N$ tal que $\frac{1}{N} \leq \varepsilon$. Mas isso implica que, para todo número natural $n \geq N$, $\frac{1}{n} \leq \frac{1}{N}$, e segue que
	\begin{equation*}
	\dist{p}{p_n} < \frac{1}{n} \leq \frac{1}{N} \leq \varepsilon
	\end{equation*}
e, portanto, $(p_n) \conv p$.
\end{proof}

\begin{proposition}
Sejam $\bm M$ um espaço métrico, $p,q \in M$ e $(p_n)_{n \in \N}$ e $(q_n)_{n \in \N}$ sequências em $M$ que convergem para $p$ e $q$ respectivamente. Então a sequência $(\dist{p_n}{q_n})_{n \in \N}$ em $\R$ converge para $\dist{p}{q}$.
\end{proposition}
\begin{proof}
	Para todo $n \in \N$, segue da desigualdade triangular que
	\begin{equation*}
	\dist{p_n}{q_n} \leq \dist{p_n}{p} + \dist{p}{q} + \dist{q}{q_n}.
	\end{equation*}
Seja $\varepsilon > 0$ um número real. Então existem $N_1,_2 \in \N$ tais que
	\begin{equation*}
	\forall n \in \N \qquad n \geq N_1 \Rightarrow \dist{p}{p_n} < \frac{\varepsilon}{2}
	\end{equation*}
e
	\begin{equation*}
	\forall n \in \N \qquad n \geq N_2 \Rightarrow \dist{q}{q_n} < \frac{\varepsilon}{2}.
	\end{equation*}
Fazendo $N_3 := \max\{N_1,N_2\}$, segue que
	\begin{equation*}
	\forall n \in \N \qquad n \geq N_3 \Rightarrow \dist{p_n}{q_n} \leq \dist{p_n}{p} + \dist{p}{q} + \dist{q}{q_n} < \dist{p}{q} + \varepsilon;
	\end{equation*}
ou seja, $\dist{p_n}{q_n} - \dist{p}{q} < \varepsilon$. Analogamente, achamos $N_6 \in \N$ tal que
	\begin{equation*}
	\forall n \in \N \qquad n \geq N_3 \Rightarrow \dist{p_n}{q_n} \leq \dist{p_n}{p} + \dist{p}{q} + \dist{q}{q_n} < \dist{p}{q} + \varepsilon
	\end{equation*}
e fazendo $n := \max\{N_3,N_6\}$, segue que
	\begin{equation*}
	\forall n \in \N \qquad n \geq N \Rightarrow |\dist{p}{q} - \dist{p_n}{q_n}| < \varepsilon,
	\end{equation*}
o que mostra que $(\dist{p_n}{q_n}) \conv \dist{p}{q}$ em $\R$.
	
\end{proof}

\subsection{Fecho e pontos aderentes}

\begin{definition}
Sejam $\bm M$ um espaço métrico e $C \subseteq M$ um conjunto. Um \emph{ponto aderente} a $C$ é um ponto $p \in M$ para o qual existe uma sequência $(p_n)_{n \in \N}$ de pontos de $C$ que converge para $p$. O \emph{fecho} de $C$ é o conjunto $\Fec{C}$ de todos os pontos aderentes a $C$. Um \emph{conjunto fechado} de $\bm M$ é um conjunto $F \subseteq M$ tal que $F = \Fec{F}$.
\end{definition}

\begin{proposition}
Sejam $\bm M$ um espaço métrico e $F \subseteq M$. Então $F$ é um conjunto fechado se, e somente se, $F^\complement$ é um conjunto aberto.
\end{proposition}
\begin{proof}
Suponhamos que $F$ é um conjunto fechado. Se $F^\complement = \emptyset$, Mas $\emptyset$ é aberto pois, caso contrário, existe $p \in \emptyset$ para o qual não há número real $\varepsilon > 0$ tal que $\bola{p}{\varepsilon} \subseteq \emptyset$, mas isso é absurdo. Se $F^\complement \neq \emptyset$, seja $p \in F^\complement$. Se não existe número real $\varepsilon > 0$ tal que $\bola{p}{\varepsilon} \subseteq F^\complement$, então, para todo número real $\varepsilon > 0$, $F \cap \bola{p}{\varepsilon} \neq \emptyset$. Mas isso implica que existe uma sequência $(p_n)_{n \in \N}$ de pontos em $F$ tal que $(p_n) \conv p$. Como $F$ é fechado, isso implica $p \in F$, o que é uma contradição. Então existe número real $\varepsilon > 0$ tal que $\bola{p}{\varepsilon} \subseteq F^\complement$, e isso mostra que $F^\complement$ é aberto.
	
Reciprocamente, suponhamos que $F^\complement$ é aberto. Se $F = \emptyset$, então $F$ é fechado. Se $F \neq \emptyset$, seja $(p_n)_{n \in \N}$ uma sequência em $F$ que converge para $p \in M$. Suponhamos que $p \notin F$. Então $p \in F^\complement$ e, como $F^\complement$ é aberto, existe um número real $\varepsilon > 0$ tal que $\bola{p}{\varepsilon} \subseteq F^\complement$. Como $(p_n) \conv p$, existe $N \in \N$ tal que, para todo $n \in \N$, se $n \geq N$, então $p_n \in \bola{p}{\varepsilon}$. Mas isso implica que $p_N \in \bola{p}{\varepsilon} \subseteq F^\complement$, o que é absurdo, pois $p_n \in F$. Portanto $p \in F$ e isso mostra que $F$ é fechado.
\end{proof}

\begin{proposition}
Seja $\bm M$ um espaço métrico. Então, para todo $c \in M$ e para todo número real $r > 0$, a bola fechada $\bolafec{c}{r}$ é um conjunto fechado.
\end{proposition}
\begin{proof}
Basta notar que $\bolafec{c}{r}^\complement$ é aberto.
\end{proof}

\subsection{Conjuntos densos}

\begin{definition}
	Sejam $\bm M$ um espaço métrico e $C \subseteq M$ um conjunto. Um conjunto \emph{denso em $C$} é um conjunto $D \subseteq M$ tal que $C \subseteq \overline D$.
\end{definition}

	Isso que dizer que, para todo ponto de $C$, existe uma sequência em $D$ que converge para esse ponto.
	
\begin{proposition}
	Sejam $\bm M = (M,\dist{\var}{\var})$ um espaço métrico e $C,D \subseteq M$ conjuntos. Então $D$ é denso em $C$ se, e somente se, para todo conjunto aberto $A$ de $\bm M$, $A \cap C \neq \emptyset$ implica $A \cap D \neq \emptyset$.
\end{proposition}
\begin{proof}
	Suponhamos que $D$ é denso em $C$. Sejam $A$ um conjunto aberto de $\bm M$ tal que $A \cap C \neq \emptyset$ e seja $p \in A \cap C$. Como $D$ é denso em $C$ e $p \in C$, existe uma sequência $(p_n)_{n \in \N}$ em $D$ que converge para $p$. Como $A$ é aberto e $p \in A$, existe um número real $\varepsilon>0$ tal que $\bola{p}{\varepsilon} \subseteq A$. Então, como $(p_n) \conv p$, existe um número natural $N$ tal que, para todo natural $n \geq N$, $p_n \in \bola{p}{\varepsilon}$. Mas isso implica que $p_n \in A \cap D$.
	
	Reciprocamente, suponhamos que, para todo conjunto aberto $A$ de $\bm M$, $A \cap C \neq \emptyset$ implica $A \cap D \neq \emptyset$. Se $C=\emptyset$, então $C \subseteq \overline D$. Se $C \neq \emptyset$, seja $p \in C$. Para todo $n \in \N$, o conjunto $\bola{p}{\frac{1}{n}}$ é um conjunto aberto que contém $p$. Mas então $\bola{p}{\frac{1}{n}} \cap C \neq \emptyset$, o que implica $\bola{p}{\frac{1}{n}} \cap D \neq \emptyset$. Para cada $n \in \N$, escolhamos $p_n \in B_\frac{1}{n}(p) \cap D$. Assim, temos uma sequência $(p_n)_{n \in \N}$ de pontos em $D$ que converge para $p$, pois, para todo número real $\varepsilon>0$, existe um natural $N$ tal que $\frac{1}{N} \leq \varepsilon$ e, então
	\begin{equation*}
	\forall n \in \N \qquad n \geq N \Rightarrow \frac{1}{n} \leq \frac{1}{N} \leq \varepsilon \Rightarrow p_n \in \bola{p}{\frac{1}{n}} \subseteq \bola{p}{\frac{1}{N}} \subseteq \bola{p}{\varepsilon}.
	\end{equation*}
	Isso mostra que $p \in \overline D$ e, portanto, que $C \subseteq \overline D$.
\end{proof}

\begin{proposition}
	Sejam $\bm M_1$ e $\bm M_2$ espaços métricos e $f,g\colon M_1 \to M_2$ funções contínuas. Então o conjunto 
	\begin{equation*}
	F := \set{p \in M_1}{f(p)=g(p)}
	\end{equation*}
é um conjunto fechado.
\end{proposition}
\begin{proof}
	Se $F=\emptyset$, então $F$ é fechado. Se $F \neq \emptyset$, seja $(p_n)_{n \in \N}$ uma sequência em $F$ que converge para $p \in M_1$. Mostraremos que $p \in F$. Como $f$ e $g$ são contínuas em $p$, segue que
	\begin{equation*}
	(f(p_n)) \conv f(p) \text{\ \ e\ \ } (g(p_n)) \conv g(p).
	\end{equation*}
	Como $(p_n)_{n \in \N}$ é uma sequência em $F$, as sequências $(f(p_n))_{n \in \N}$ e $(g(p_n))_{n \in \N}$ são a mesma sequência e segue da unicidade do limite que $f(p)=g(p)$, o que mostra que $p \in F$ e que, portanto, $F$ é um conjunto fechado.
\end{proof}

\begin{proposition}
	Sejam $\bm M_1$ e $\bm M_2$ espaços métricos, $f,g: M_1 \to M_2$ funções contínuas e $C,D \subseteq M_1$ conjuntos tais que $D$ é denso em $C$. Se $f|_D = g|_D$, então $f|_C = g|_C$.
\end{proposition}
\begin{proof}
	Pela proposição anterior, sabemos que $F := \{p \in M_1 : f(p)=g(p)\}$ é um conjunto fechado. Como $f|_D = g|_D$, então $D \subseteq F$. Mas isso significa que $\overline D \subseteq \overline F = F$ e, como $D$ é denso em $C$, segue que $C \subseteq \overline D \subseteq F$ e, portanto, que $f|_C = g|_C$. 
\end{proof}

\subsection{Conjuntos compactos}

\begin{proposition}
Sejam $\bm M$ um espaço métrico e $C \subseteq M$. Se $C$ é compacto, então é limitado.
\end{proposition}
\begin{proof}
Seja $p \in M$ e consideremos a cobertura $\set{\bola{p}{r}}{r \in \intaa{0}{\infty}}$ de $C$. Pela compacidade, existe subcobertura finita $\{\bola{p}{r_0},\ldots,\bola{p}{r_{n-1}}\}$ de $C$. Tomando $r := \max\set{r_i}{i \in [n]}$, segue que $\bola{p}{r_i} \subseteq \bola{p}{r}$ para todo $i \in [n]$, logo $C \subseteq \bola{p}{r}$, o que implica que $\diam(C) \leq 2r < \infty$. 
\end{proof}

A recíproca nem sempre é verdade. Nos espaços $\R^d$, $ d \in \N$, vale que um conjunto é compacto se, e somente se, é fechado e limitado. Esse resultado é conhecido como Teorema de Heine-Borel. No entanto, isso não vale em qualquer espaço métrico\footnote{Para mais detalhes, conferir \url{https://math.stackexchange.com/questions/674982/difference-between-closed-bounded-and-compact-sets}.}.

\subsection{Continuidade}

\begin{definition}
Sejam $\bm M$ e $\bm M'$ espaços métricos e $p \in M$. Uma função \emph{contínua em $p$} é uma função $\fun{f}{M}{M'}$ que satisfaz: para todo $\varepsilon \in \intaa{0}{\infty}$, existe $\delta \intaa{0}{\infty}$ tal que, para todo $x \in M$
%	\begin{equation*}
%	\qquad \dist{p}{x}_1 < \delta \Rightarrow \dist{f(p)}{f(x)}_2 < \varepsilon.
%	\end{equation*}
	\begin{equation*}
	x \in \bola{p}{\delta} \Rightarrow f(x) \in \bola{f(p)}{\varepsilon}.
	\end{equation*}
Uma função \emph{descontínua} em $p$ é uma função que não é contínua em $p$.
\end{definition}

Denotamos as bolas abertas em $\bm M$ e em $\bm M'$ por $\bola{\var}{\var}$, mas deve-se perceber que elas são relativas a métricas possivelmente diferentes.

\begin{proposition}
Sejam $\bm M_1$ e $\bm M_2$ espaços métricos, $f: M_1 \to M_2$ uma função e $p \in M_1$. Então $f$ é contínua em $p$ se, e somente se, para toda sequência $(p_n)_{n \in \N}$ de pontos em $M_1$ que converge para $p$, a sequência $(f(p_n))_{n \in \N}$ de pontos em $M_2$ converge para $f(p)$; ou seja
	\begin{equation*}
	\lim f(p_n) = f(\lim p_n).
	\end{equation*}
\end{proposition}
\begin{proof}
	Suponhamos que $f$ é contínua em $p$. Seja $(p_n)_{n \in \N}$ uma sequência de pontos em $M_1$ que converge para $p$. Seja um número real $\varepsilon > 0$. Como $f$ é contínua, existe um número real $\delta > 0$ tal que $p_n \in \bola{p}{\delta}$ implica $f(p_n) \in \bola{f(p)}{\varepsilon}$. Mas, como $(p_n) \conv p$, existe $N \in \N$ tal que
	\begin{equation*}
	\forall n \in \N \qquad n \geq N \Rightarrow p_n \in \bola{p}{\delta} \Rightarrow f(p_n) \in \bola{f(p)}{\varepsilon}
	\end{equation*}
o que mostra que $(f(p_n)) \conv f(p)$.
	
	Reciprocamente, suponhamos que, para toda sequência $(p_n)_{n \in \N}$ em $M_1$ que converge para $p$, a sequência $(f(p_n))_{n \in \N}$ converge para $f(p)$. Suponhamos, por absurdo, que $f$ não é contínua em $p$. Então existe um número real $\varepsilon > 0$ tal que, para todo número real $\delta > 0$, existe $x \in M_1$ tal que $x \in \bola{p}{\delta}$, mas $f(x) \notin \bola{f(p)}{\varepsilon}$. Vamos mostrar que isso implica que existe uma sequência $(p_n)_{n \in \N}$ em $M_1$ que converge para $p$, mas que a sequência $(f(p_n))_{n \in \N}$ não converge para $f(p)$; ou seja, que existe um número real $\varepsilon > 0$ tal que, para todo número natural $N$, existe $n \in \N$ tal que $n \geq N$, mas $f(p_n) \notin \bola{f(p)}{\varepsilon}$. Seja $n \in \N$ e tomemos $\delta = \frac{1}{n}$. Então existe $x \in M_1$ tal que $x \in \bola{p}{\frac{1}{n}}$, mas $f(x) \notin \bola{f(p)}{\varepsilon}$. Nomeando esse $x \in M_1$ de $p_n$, obtemos uma sequência $(p_n)_{n \in \N}$ que converge para $p$ pois, para todo número real $\varepsilon' > 0$, existe um número natural $N \in \N$ tal que $\frac{1}{N} \leq \varepsilon'$ e isso implica que
\begin{equation*}
	\forall n \in \N \qquad n \geq N \Rightarrow \frac{1}{n} \leq \frac{1}{N} \leq \varepsilon' \Rightarrow p_n \in \bola{p}{\frac{1}{n}} \subseteq \bola{p}{\frac{1}{N}} \subseteq \bola{p}{\varepsilon'}.
	\end{equation*}
	No entanto, $(f(p_n))_{n \in \N}$ é uma sequência que não converge para $f(p)$ pois, considerando o $\varepsilon$ original tomado da descontinuidade de $f$, para todo número natural $N$, $f(p_N) \notin \bola{f(p)}{\varepsilon}$ e isso contradiz a hipótese de que, para toda sequência $(p_n)_{n \in \N}$ em $M_1$ que converge para $p$, a sequência $(f(p_n))_{n \in \N}$ converge para $f(p)$. Portanto $f$ é contínua.	
\end{proof}

\begin{definition}
Sejam $\bm M_1$ e $\bm M_2$ espaços métricos, $D \subseteq M_1$ e $f: D \to M_2$ uma função. A função $f$ é \emph{contínua} em $D$ se ela é contínua em todo ponto de $D$. Caso contrário, a função $f$ é \emph{descontínua} em $D$. Para $D=M_1$, dizemos simplesmente que $f$ é contínua ou descontínua.
\end{definition}

\subsection{Ponto limite e conjunto derivado}

\begin{definition}
Sejam $\bm M$ um espaço métrico e $C \subseteq M$ um conjunto. Um \emph{ponto limite} (ou \emph{ponto de acumulação}) de $C$ é um ponto $p \in M$ para o qual existe uma sequência $(p_n)_{n \in \N}$ de pontos de $C \setminus \{p\}$ que converge para $p$. O \emph{derivado} de $C$ é o conjunto de todos os pontos limites de $C$. 
\end{definition}

Da definição, segue que $C' \subseteq \overline C$. A inclusão contrária caracteriza a seção a seguir.

\begin{definition}
Sejam $\bm M$ um espaço métrico e $C \subseteq M$ um conjunto. Um \emph{ponto isolado} de $C$ é um ponto $p \in M$ que é um ponto aderente a $C$ mas que não é um ponto limite de $C$.
\end{definition}

Um ponto isolado de $C$ é um ponto $p \in \overline C \setminus C'$.

\subsection{Distância e bolas de conjuntos e separação métrica}

\begin{definition}
Sejam $\bm M$ um espaço métrico e $C,C' \subseteq M$. A \emph{distância} entre $C$ e $C'$ é
	\begin{equation*}
	\dist{C}{C'} := \inf \set{\dist{c}{c'}}{c \in C, c' \in C'}.
	\end{equation*}
Para todo $p \in M$, denotam-se $\dist{C}{p} := \dist{C}{\{p\}}$ e $\dist{p}{C} := \dist{\{p\}}{C}$.
\end{definition}

Essa definição, em particular, estabelece a distância de pontos para conjuntos também. Existem outras definições de distâncias entre conjuntos, mas elas não serão tratadas aqui.

\begin{definition}
Sejam $\bm M$ um espaço métrico e $C \subseteq M$ e $r \in \intaa{0}{\infty}$. A $r$-\emph{vizinhança} de $C$ é o conjunto
%	\begin{equation*}
%	\bola{C}{r} := \set{p \in M}{\exists_{c \in C} \dist{c}{p} < r}.
%	\end{equation*}
	\begin{equation*}
	\bola{C}{r} := \set{p \in M}{\dist{C}{p} < r}.
	\end{equation*}
 A $r$-\emph{vizinhança fechada} de $C$ é o conjunto
%	\begin{equation*}
%	\bolafec{C}{r} := \set{p \in M}{\exists_{c \in C} \dist{c}{p} \leq r}.
%	\end{equation*}
	\begin{equation*}
	\bolafec{C}{r} := \set{p \in M}{\dist{C}{p} \leq r}.
	\end{equation*}
\end{definition}

Note que excluímos $r=0$ da definição pois basicamente teríamos $\bola{C}{0} = \emptyset$ e $\bolafec{C}{0} = \Fec{C}$.
	
\begin{exercise}
Sejam $\bm M$ um espaço métrico e $C \subseteq M$ e $r \in \intaa{0}{\infty}$.
	\begin{enumerate}
	\item A vizinhança $\bola{C}{r} $ é um conjunto aberto e
		\begin{equation*}
		\bola{C}{r} = \bigcup_{c \in C} \bola{c}{r};
		\end{equation*}
	\item  Para todo $r' \in \intaa{0}{\infty}$ tais que $r \leq r'$,
		\begin{equation*}
		C \subseteq \bola{C}{r} \subseteq \bola{C}{r'};
		\end{equation*}
	\item A vizinhança fechada $\bolafec{C}{r} $ é um conjunto fechado e
		\begin{equation*}
		\bolafec{C}{r} \supseteq \bigcup_{c \in C} \bolafec{c}{r};
		\end{equation*}
	\item  Para todo $r' \in \intaa{0}{\infty}$ tais que $r \leq r'$,
		\begin{equation*}
		\Fec{C} \subseteq \bolafec{C}{r} \subseteq \bolafec{C}{r'};
		\end{equation*}
	\item Para $r \in \intaa{0}{\infty}$,
		\begin{equation*}
		\bolafec{C}{r} = \Fec{\bola{C}{r}}
		\end{equation*}
	\end{enumerate}
\end{exercise}
\begin{proof}
	\begin{enumerate}
	\item Primeiro, mostramos a igualdade dos conjuntos. ($\supseteq$) Seja $p \in \bigcup_{c \in C} \bola{c}{r}$. Então existe $c \in C$ tal que $\dist{c}{p} < r$, o que implica que
	\begin{equation*}
	\dist{C}{p} = \inf \set{\dist{c}{p}}{c \in C} \leq \dist{c}{p} < r,
	\end{equation*}
logo $p \in \bola{C}{r}$.

($\subseteq$) Seja $p \in \bola{C}{r}$. Então
	\begin{equation*}
	\dist{C}{p} = \inf \set{\dist{c}{p}}{c \in C} < r.
	\end{equation*}
Isso significa que existe sequência $(c_n)_{n \in \N}$ em $C$ tal que $\lim_{n \conv \infty} \dist{c}{p} < r$, o que implica existe $n \in \N$ tal que $\dist{c_n}{p}<r$, logo $p \in \bigcup_{c \in C} \bola{c}{r}$.

Como as bolas $\bola{c}{r}$ são abertas, segue que $\bola{C}{r} = \bigcup_{c \in C} \bola{c}{r}$ é aberto.
	\end{enumerate}
\end{proof}

\begin{definition}
Seja $\bm M$ um espaço métrico. Conjuntos \emph{metricamente separados} são conjuntos $C,C' \subseteq M$ tais que
	\begin{equation*}
	\dist{C}{C'} > 0.
	\end{equation*}
\end{definition}

\begin{proposition}
Sejam $\bm M$ um espaço métrico e $C,C' \subseteq M$ conjuntos metricamente separados. Então $C$ e $C'$ são separados por vizinhanças.
\end{proposition}
\begin{proof}
Seja $\delta := \dist{C}{C'}$. Pela separação métrica, $\delta > 0$. Então $C \subseteq \bola{C}{\frac{\delta}{2}}$ e $C' \subseteq \bola{C'}{\frac{\delta}{2}}$ e $\bola{C}{\frac{\delta}{2}} \cap \bola{C'}{\frac{\delta}{2}} = \emptyset$, já que, se existe $p \in \bola{C}{\frac{\delta}{2}} \cap \bola{C'}{\frac{\delta}{2}}$, então existem $c \in C$ e $c' \in C'$ tais que $\dist{c}{p} < \frac{\delta}{2}$ e $\dist{c'}{p} < \frac{\delta}{2}$, portanto
	\begin{equation*}
	\dist{c}{c'} \leq \dist{c}{p} + \dist{c'}{p} < \frac{\delta}{2}+\frac{\delta}{2} = \delta,
	\end{equation*}
o que implica que $\dist{C}{C'}<\delta$, contradição.
\end{proof}

Isso implica, em particular, que conjuntos metricamente separados são, além de separados por vizinhanças, separados, disjuntos e, claro, distintos.

\begin{proposition}
Todo espaço métrico $\bm M$ é um espaço topológico normal.
\end{proposition}
\begin{proof}
Sejam $F,F' \subseteq M$ fechados disjuntos. Mostraremos que $F$ e $F'$ são
% metricamente separados e portanto 
separados por vizinhanças, o que mostrará que o espaço é normal.

Para todo $f \in F$, existe $\delta_f \in \intaa{0}{\infty}$ tal que
	\begin{equation*}
	\bola{f}{\delta_f} \cap F' = \emptyset,
	\end{equation*}
pois $F'$ é fechado e $f \notin F'$. Definimos
	\begin{equation*}
	V := \bigcup_{f \in F} \bola{f}{\frac{\delta_f}{2}}.
	\end{equation*}
Esse conjunto é uma vizinhança aberta de $F$. Analogamente, definimos $V'$ uma vizinhança aberta de $F'$. Claramente $V \cap V'=\empty$, pois caso contrário, se existe $p \in V \cap V'$, então existem $f \in F$e $f' \in F'$ tais que
	\begin{equation*}
	p \in \bola{f}{\frac{\delta_f}{2}} \cap \bola{f'}{\frac{\delta_{f'}}{2}},
	\end{equation*}
portanto se $\delta_f \leq \delta_{f'}$, então $f \in \bola{f'}{\frac{\delta_{f'}}{2}}$ e, se $\delta_{f'} \leq \delta_f$, então $f' \in \bola{f}{\frac{\delta_f}{2}}$, ambos contradições.
%
%Outra demonstração usa os conjuntos:
%Defina
%	\begin{equation*}
%	V := \set{p \in M}{\dist{F}{p} < \dist{F'}{p}}
%	\end{equation*}
%e
%	\begin{equation*}
%	V := \set{p \in M}{\dist{F'}{p} < \dist{F}{p}}.
%	\end{equation*}
%Então claramente $V \cap V' = \emptyset$. Ainda, $F \subseteq V$
\end{proof}

%%%%%%%%%%%%%%%%%%%%%%%%%%%%%%%%%%%%%%%%%%%%
\begin{comment}
\begin{proposition}
Seja $\bm M$ um espaço métrico e $K,K' \subseteq M$ compactos disjuntos. Então $F$ e $F'$ são metricamente separados.
\end{proposition}
\begin{proof}
% FALTA MOSTRAR QUE REALMENTE AS SEQUÊNCIAS f_n f'_n convergem.
Sejam $F,F' \subseteq M$ fechados disjuntos. Mostraremos que $F$ e $F'$ são metricamente separados.
 Queremos mostrar que
	\begin{equation*}
	\dist{F}{F'} = \inf \set{\dist{f}{f'}}{f \in F, f' \in F'} > 0.
	\end{equation*}
Existem sequências $(f_n)_{n \in \N}$ em $F$ e $(f'_n)_{n \in \N}$ em $F'$ tais que
	\begin{equation*}
	\dist{F}{F'} = \lim_{n \conv \infty} \dist{f_n}{f'_n}.
	\end{equation*}
Mas como $F$ e $F'$ são fechados, existem $f \in F$ e $f' \in F'$ tais que $f = \lim_{n \conv \infty} f_n$ e $f' = \lim_{n \conv \infty} f'_n$. Da continuidade da distância,
	\begin{equation*}
	\dist{F}{F'} = \lim_{n \conv \infty} \dist{f_n}{f'_n} = \dist{f}{f'}.
	\end{equation*}
Como $F \cap F' = \emptyset$, então $f \neq f'$, portanto $\dist{f}{f'} > 0$, o que implica que
	\begin{equation*}
	\dist{F}{F'} = \dist{f}{f'} > 0.
	\end{equation*}
\end{proof}

\end{comment}
%%%%%%%%%%%%%%%%%%%%%%%%%%%%%%%%%%%%%%%%%%%%








\section{Estrutura uniforme}

\subsection{Sequências aproximantes}

\begin{definition}
Seja $\bm M$ um espaço métrico. Uma sequência \emph{aproximante} em $\bm M$ é uma sequência $(p_n)_{n \in \N}$ de pontos em $M$ tal que, para todo número real $\varepsilon > 0$, existe um número natural $N$ satisfazendo
	\begin{equation*}
	\forall n,m \in \N \qquad n,m \geq N \Rightarrow \dist{p_n}{p_m} < \varepsilon.
	\end{equation*}
\end{definition}

Essa sequências são conhecidas como \emph{sequências de Cauchy}. O nome aproximante se dá pelo fato de que os termos da sequência ficam cada vez mais próximos entre si, e será adotado por ser mais intuitivo, embora não seja a nomenclatura padrão.

\begin{proposition}
Toda sequência convergente em um espaço métrico $\bm M$ é aproximante.
\end{proposition}
\begin{proof}
Seja $(p_n)_{n \in \N}$ uma sequência em $M$ que converge para $p$. Seja $\varepsilon \in \R$ tal que $\varepsilon > 0$. Então $\frac{1}{2}\varepsilon > 0$ é um número real e segue que existe $N \in \N$ tal que, para todo número natural $n \geq N$, $p_n \in \bola{p}{\frac{1}{2}\varepsilon}$. Assim, segue que	
	\begin{equation*}
	\forall n,m \in \N \qquad n,m \geq N \Rightarrow \dist{p_n}{p_m} \leq \dist{p_n}{p} + \dist{p}{p_n} < \frac{\varepsilon}{2} + \frac{\varepsilon}{2} = \varepsilon,
	\end{equation*}
o que mostra que $(p_n)_{n \in \N}$ é uma sequência aproximante.
\end{proof}

\begin{proposition}
Toda sequência aproximante em um espaço métrico $\bm M$ que tem uma subsequência convergente é convergente.
\end{proposition}
\begin{proof}
	Seja $(p_{n_k})_{k \in \N}$ uma subsequência de $(p_n)_{n \in \N}$ que converge  para $p$. Seja $\varepsilon > 0$ um número real. Como $(p_n)_{n \in \N}$ é uma sequência de Cauchy e $\frac{1}{2}\varepsilon > 0$ é um número real, existe um número natural $N$ tal que
	\begin{equation*}
	\forall n,m \in \N \qquad n,m \geq N \Rightarrow \dist{p_n}{p_m} < \frac{\varepsilon}{2}.
	\end{equation*}
Como $(p_{n_k})_{k \in \N}$ é uma subsequência convergente, existe $K_1 \in \N$ tal que
	\begin{equation*}
	\forall k \in \N \qquad k \geq K_1 \Rightarrow \dist{p}{p_{n_k}} < \frac{\varepsilon}{2}.
	\end{equation*}
Como $(n_k)_{k \in \N}$ é uma sequência estritamente crescente, existe $K_2 \in \N$ tal que, para todo número natural $k \geq K_2$, $n_k \geq N$. Assim, tomando $K := \max\{K_1,K_2\}$, segue que, para todo número natural $n \in \N$, existe $k \in \N$ tal que $n_k \geq N$ e,  pela desigualdade triangular, que
	\begin{equation*}
	\forall n \in \N \qquad n \geq N \Rightarrow \dist{p_n}{p} \leq \dist{p_n}{p_{n_k}} + \dist{p_{n_k}}{p} < \frac{\varepsilon}{2}+\frac{\varepsilon}{2}=\varepsilon.
	\end{equation*}
\end{proof}

\begin{proposition}
Toda sequência aproximante em um espaço métrico $\bm M$ é limitada.
\end{proposition}
\begin{proof}
Seja $(p_n)_{n \in \N}$ uma sequência aproximante em $\bm M$. Então, para $\varepsilon=1$, existe $N \in \N$ tal que
	\begin{equation*}
	\forall n,m \in \N \qquad n,m \geq N \Rightarrow \dist{p_n}{p_m}<1.
	\end{equation*}
	Definamos $P := \{p_n : n \in \N\}$. Então segue que
	\begin{align*}
	\diam(P) &= \sup \set{\dist{p_n}{p_m}}{n,m \in \N} \\
		&= \max\{1 \cup \set{\dist{p_n}{p_m}}{0 \leq n,m \leq N}\} \in \R,
	\end{align*}
o que mostra que $(p_n)_{n \in \N}$ é limitada.
\end{proof}

\subsection{Continuidade uniforme}

\begin{definition}
Sejam $\bm M_1$ e $\bm M_2$ espaços métricos. Uma função \emph{uniformemente contínua} é uma função $f: M_1 \to M_2$ tal que, para todo número real $\varepsilon > 0$, existe um número real $\delta > 0$ tal que
	\begin{equation*}
	\forall p_1,p_2 \in M_1 \qquad \dist{p_1}{p_2}_1 < \delta \Rightarrow \dist{f(p_1)}{f(p_2)}_2 < \varepsilon.
	\end{equation*}
\end{definition}

\begin{proposition}
Sejam $\bm M_1$ e $\bm M_2$ espaços métricos, $f: M_1 \to M_2$ uma função uniformemente contínua e $(p_n)_{n \in \N}$ uma sequência aproximante em $M_1$. Então a sequência $(f(p_n))_{n \in \N}$ em $M_2$ é aproximante.
\end{proposition}
\begin{proof}
Seja $\varepsilon > 0$ um número real. Da continuidade uniforme de $f$, existe um número real $\delta > 0$ tal que, para todo $p,p' \in M_1$, $\dist{p}{p'}_1 < \delta$ implica $\dist{f(p)}{f(p')}_2 < \varepsilon$. Como $(p_n)_{n \in \N}$ é sequência aproximante, existe $N \in \N$ tal que
	\begin{equation*}
	\forall n,m \in \N \qquad n,m \geq N \Rightarrow \dist{p_n}{p_m}_1 < \delta.
	\end{equation*}
Mas, da continuidade uniforme de $f$, isso implica que
	\begin{equation*}
	\forall n,m \in \N \qquad n,m \geq N \Rightarrow \dist{p_n}{p_m}_1 < \delta \Rightarrow \dist{f(p_n)}{f(p_m)}_2 < \varepsilon,
	\end{equation*}
e isso mostra que $(f(p_n))_{n \in \N}$ é uma sequência aproximante.
\end{proof}

\subsection{Espaços métricos completos}

\begin{definition}
Um espaço métrico \emph{completo} é um espaço métrico em que todas sequências aproximantes convergem.
\end{definition}

\begin{proposition}
Seja $\bm M$ um espaço métrico. Todo subespaço completo de $\bm M$ é um conjunto fechado em $\bm M$.
\end{proposition}
\begin{proof}
Sejam $\bm C \subseteq \bm M$ subespaço métrico completo e $(p_n)_{n \in \N}$ uma sequência convergente em $C$. Então $(p_n)_{n \in \N}$ é aproximante e, como $C$ é completo, converge para um ponto em $C$, o que significa que $C$ é fechado.
\end{proof}

\begin{proposition}
Sejam $\bm M$ um espaço métrico, $\bm C \subseteq \bm M$ um subespaço completo e $F \subseteq C$ um conjunto fechado em $\bm M$. Então $\bm F$ é completo.
\end{proposition}
\begin{proof}
Seja $(p_n)_{n \in \N}$ uma sequência aproximante em $F$. Então $(p_n)_{n \in \N}$ é uma sequência aproximante em $C$ e, como $C$ é completo, $(p_n)_{n \in \N}$ converge. Porém, como $F$ é fechado, então $(p_n)_{n \in \N}$ converge para um ponto em $F$, o que mostra que $F$ é completo.
\end{proof}

\begin{theorem}
Seja $\bm M$ um espaço métrico. Então $\bm M$ é completo se, e somente se, para toda sequência decrescente $(F_n)_{n \in \N}$ de conjuntos não vazios e fechados em $\bm M$ tais que $(\diam(F_n))_{n \in \N} \conv 0$ em $\R$, vale que
	\begin{equation*}
	\bigcap_{n \in \N} A_n \neq \emptyset.
	\end{equation*}
\end{theorem}

\begin{theorem}
Sejam $\bm M_1$  um espaço métrico, $\bm M_2$ espaço métrico completo, $D \subseteq M_1$ um conjunto denso em $M_1$ e $f: D \to M_2$ uma função uniformemente contínua. Então $f$ tem uma única extensão para uma função uniformemente contínua $f^*: M_1 \to M_2$. Ainda, se $f$ é uma isometria, então $f^*$ é uma isometria.
\end{theorem}
\begin{proof}
	Seja $p \in M_1$. Como $D$ é denso em $M_1$, existe uma sequência $(p_n)_{n \in \N}$ em $D$ que converge para $p$. Como $(p_n)_{n \in \N}$ é convergente, é uma sequência de Cauchy e, como $f$ é uniformemente contínua em $D$, segue que $(f(p_n))_{n \in \N}$ é uma sequência de Cauchy em $M_2$. Mas $M_2$ é completo, o que implica que $(f(p_n))_{n \in \N}$ converge para um ponto $p' \in M_2$. Definimos, portanto, a função $f^*$ em $p$ como $f^*(p)=p'$. Precisamos mostrar que $f^*$ independe da escolha da sequência em $D$ que converge para $p$. Se $(q_n)_{n \in \N}$ é uma sequência em $D$ que converge para $p$, definamos a sequência $(r_n)_{n \in \N}$ em $D$ por
	\begin{equation*}
	r_n :=
			\begin{cases}
			p_n &\text{se $n=2k$}\\
			q_n &\text{se $n=2k+1$}.
			\end{cases}
	\end{equation*}
A sequência $(r_n)_{n \in \N}$ converge para $p$ e, portanto, é uma sequência de Cauchy. A continuidade uniforme de $f$ implica que a sequência $(f(r_n))_{n \in \N}$ é de Cauchy e, portanto, como $(f(p_n))_{n \in \N}=(f(r_{2k}))_{k \in \N}$ é uma subsequência que converge para $p'$, a sequência $(f(r_n))_{n \in \N}$ converge para $p'$, o que implica que a subsequência $(f(q_n))_{n \in \N}=(f(r_{2k+1}))_{k \in \N}$ converge para $p'$. Assim, mostramos que $f^*$ está bem definida. Claramente, se $p \in D$, então $f(p)=f^*(p)$, pois, como $D$ é denso em $M_1$, se $(p_n)_{n \in \N}$ é uma sequência em $D$ que converge para $p$, então, como $f$ é contínua, segue que $f(p_n) \conv f(p)$, o que mostra que $f^*(p)=f(p)$.

	Agora, devemos mostrar que $f^*$ é uniformemente contínua. Seja $\varepsilon > 0$ um número real, então $\frac{1}{2}\varepsilon > 0$ é um número real e, como $f$ é uniformemente contínua, existe número real $\delta > 0$ tal que
	\begin{equation*}
	\forall p,p' \in M_1 \qquad \dist{p}{p'}_1 < \delta \Rightarrow \dist{f(p)}{f(p')}_2 < \frac{\varepsilon}{2}.
	\end{equation*}
Assim, sejam $p,q \in M_1$ tais que $\dist{p}{q}_1 < \delta$. Queremos mostrar que $\dist{f(p)}{f(p')}_2 < \varepsilon$. Sejam $(p_n)_{n \in \N}$ e $(q_n)_{n \in \N}$ sequências que convergem para $p$ e $q$, respectivamente. Então $\dist{p_n}{q_n}_1 \conv \dist{p}{q}_1$ em $\R$.

...

	A unicidade de $f^*$ ocorre pois, se existem $f^*$ e $f'^*$ uniformemente contínuas que estendem $f$, como $D$ é denso em $M_1$ e $f^*|_D = f'^*|_D$, segue que $f^* = f'^*$ .
	
	Por fim, mostramos que a isometria se preserva...
\end{proof}

\begin{definition}
Seja $\bm M_1$  um espaço métrico. Um \emph{completamento} de $\bm M$ é um espaço métrico $\bm M_2$ completo tal que $M_1$ é denso em $M_2$.
\end{definition}

\begin{exercise}
Seja $\bm M$ um espaço métrico e $\bm M_1$ e $\bm M_2$ completamentos de $\bm M$. Então existe uma isometria entre $\bm M_1$ e $\bm M_2$ que é a função identidade quando restrita a $M$.
\end{exercise}
%\begin{proof}
%	Seja $f$ a função identidade em $M$. Pela proposição anterior, existe uma única extensão uniformemente contínua de $f^*$ em $M_1$ ...
%	
%	...
%\end{proof}


\begin{proposition}
Sejam $K \subseteq M$ compacto e $f: M \to \bar M$ contínua. Então $f$ é uniformemente contínua.
\end{proposition}
\begin{proof}
Suponhamos, por absurdo, que $f$ não é uniformemente contínua. Então existem $\varepsilon > 0$ e $(x_n)_{n \in \N},(y_n)_{n \in \N}$ sequências em $K$ tais que
	\begin{equation*}
	\dist{x_n}{y_n} < \frac{1}{n} \text{\ \ e\ \ } \dist{f(x_n)}{f(y_n)} \geq \varepsilon.
	\end{equation*}
Como $K$ é compacto, existem subsequências $(x_{n_k})_{k \in \N}$  e $(y_{n_k})_{k \in \N}$ convergindo a $x \in K$ com $\dist{f(x_{n_k})}{f(y_{n_k})} \geq \varepsilon$. Por continuidade de $f$, existe $\delta > 0$ tal que, se $x_{n_k},y_{n_k} \in B(x,\delta)$, então $\dist{f(x_{n_k})}{f(x)} < \frac{\varepsilon}{2}$ e $\dist{f(y_{n_k})}{f(x)} < \frac{\varepsilon}{2}$. Pela desigualdade triangular, temos um absurdo.
\end{proof}


\subsection{Limitação uniforme (ou total)}

Sejam $(M,\dist{\var}{\var})$ um espaço métrico e $\topo$ a topologia induzida por $\dist{\var}{\var}$. A métrica limitada $\dist{\var}{\var} \opmin 1$ induz a mesma topologia $\topo$ que $\dist{\var}{\var}$ sobre $M$, mas com respeito a $\dist{\var}{\var} \opmin 1$ todos conjuntos são limitados, enquanto que com respeito a $\dist{\var}{\var}$ isso nem sempre é verdade (somente nos casos em que o espaço inteiro é limitado). Isso mostra que o conceito de limitação não está sempre relacionado à topologia do espaço.

A convergência de sequências em $(M,\topo)$ independe da métrica que a gera e, portanto, concluímos que não pode existir um análogo ao teorema de Bolzano - Weierstrass usando o conceito de limitação, pois todas sequências em $(M,d \wedge 1)$ são limitadas, mas não necessariamente têm subsequência convergente. Definiremos a seguir um conceito distinto de limitação em espaços métricos que está mais relacionado à estrutura uniforme do espaço.

\begin{definition}
Um espaço métrico \emph{uniformemente limitado} (ou \emph{totalmente limitado}) é um espaço métrico $\bm M$ em que, para todo $\varepsilon \in \intaa{0}{\infty}$, existe uma $\varepsilon$-cobertura finita de $M$. Um conjunto \emph{uniformemente limitado} é um conjunto que é uniformemente limitado com a estrutura métrica induzida.
\end{definition}

\begin{proposition}
Seja $\bm M$ um espaço métrico.
	\begin{enumerate}
	\item $\bm M$ é uniformemente limitado se, e somente se, para todo $\varepsilon \in \intaa{0}{\infty}$, existem $p_0,\cdots,p_{n-1}$ tais que
		\begin{equation*}
		M \subseteq \bigcup_{i \in [n]} \bola{p}{\varepsilon};
		\end{equation*}
	\item Se $\bm M$ é uniformemente limitado, então é limitado.
	\end{enumerate}
\end{proposition}
\begin{proof}
	\begin{enumerate}
	\item ($\Rightarrow$) Seja $\varepsilon \in \intaa{0}{\infty}$. Como $M$ é uniformemente limitado, existe uma $\frac{\varepsilon}{2}$-cobertura finita $\{C_i\}_{i \in [n]}$ de $M$. Tome, para cada $i \in [n]$, $c_i \in C_i$. Então $\diam(\bola{c_i}{\frac{\varepsilon}{2}}) \leq \varepsilon$ e $C_i \subseteq \bola{c_i}{\frac{\varepsilon}{2}}$, portanto
	\begin{equation*}
	M \subseteq \bigcup_{i \in [n]} C_i \subseteq \bigcup_{i \in [n]} \bola{c_i}{\frac{\varepsilon}{2}}.
	\end{equation*}

($\Leftarrow$) Reciprocamente, seja $\varepsilon \in \intaa{0}{\infty}$. Existem existem $p_0,\cdots,p_{n-1}$ tais que
		\begin{equation*}
		M \subseteq \bigcup_{i \in [n]} \bola{p_i}{\frac{\varepsilon}{2}};
		\end{equation*}
Como, para todo $i \in [n]$, $\diam(\bola{p}{\frac{\varepsilon}{2}}) \leq \varepsilon$, segue que $(\bola{p_i}{\frac{\varepsilon}{2}})_{i \in [n]}$ é uma $\varepsilon$-cobertura de $M$, portanto $M$ é uniformemente limitado.
	
	\item Seja $(C_i)_{i \in [n]}$ uma cobertura $1$-precisa de $M$. Então, como para todo $i \in [n]$, $\diam(C_i) < \infty$, e a cobertura é finita, $\diam(M) < \infty$.
	\end{enumerate}
\end{proof}

\begin{exercise}
Seja $d \in \N$ e consideremos $\R^d$ com a métrica reta usual. Um subconjunto de $\R^d$ é limitado se, e somente se, é uniformemente limitado.
\end{exercise}

\begin{lemma}
Sejam $\bm M$ um espaço métrico e $(x_n)_{n \in \N}$ uma sequência em $M$.
	\begin{enumerate}
	\item Se $(x_n)_{n \in \N}$ é aproximante, então sua imagem $x(\N) \subseteq M$ é uniformemente limitada;
	\item Se $x(\N)$ é uniformemente limitada, $(x_n)_{n \in \N}$ tem subsequência aproximante.
	\end{enumerate}
\end{lemma}
\begin{proof}
	\begin{enumerate}
	\item Seja $\varepsilon \in \intaa{0}{\infty}$. Como $(x_n)_{n \in \N}$ é aproximante, existe $N \in \N$ tal que, para todos $n,n' \in \N$, se $n \geq N$ e $n' \geq N$, então $\dist{x_n}{x_{n'}} < \varepsilon$, portanto
		\begin{equation*}
		\diam \set{x_n}{n \in \N, n \geq N} \leq \varepsilon,
		\end{equation*}
portanto
		\begin{equation*}
		(\{x_0\},\cdots,\{x_{N-1}\}, \set{x_n}{n \in \N, n \geq N})
		\end{equation*}
é uma cobertura $\varepsilon$-precisa finita de $x(\N)$.

	\item Se $x(\N)$ for finito, então $(x_n)_{n \in \N}$ tem subsequência constante, logo aproximante. Suponhamos o caso em que $x(\N)$ é infinito.

	\end{enumerate}
\end{proof}


\begin{proposition}
Seja $\bm M$ um espaço métrico. O espaço $\bm M$ é uniformemente limitado se, e somente se, toda sequência tem subsequência aproximante.
\end{proposition}
\begin{proof}
($\Rightarrow$) Suponha que $\bm M$ é uniformemente limitado. Seja $•$
\end{proof}

\begin{proposition}
Seja $\bm M$ um espaço métrico. O espaço $\bm M$ é compacto se, e somente se, é completo e uniformemente limitado. 
\end{proposition}






\section{Funções métricas}

\subsection{Funções controladas}

Neste seção, definiremos funções de um espaço métrico para outro que de certa forma preserva alguma informação da estrutura métrica do espaço. Essas funções não preservam o valor da distância entre dois pontos, mas são limitadas de em um sentido específico, o que garante sua continuidade (uniforme) e traduz alguma propriedades de um espaço para o outro. No que segue, não distinguiremos a notação das distâncias em espaços métricos diferentes.

\begin{definition}
Sejam $\bm M$ e $\bm M'$ espaços métricos. Uma \emph{função controlada}% subsemelhança, métrica...
	\footnote{Essas funções são conhecidas na literatura como funções `Lipschitz contínuas'.} %
de $\bm M$ para $\bm M'$ é uma função $\fun{f}{M}{M'}$ tal que, para algum $c \in \intfa{0}{\infty}$, $f$ satisfaz que, para todos $p,p' \in M$,
	\begin{equation*}
	\dist{f(p)}{f(p')} \leq c\dist{p}{p'}.
	\end{equation*}
A \emph{distorção} (ou \emph{norma de distorção}) de $f$ é
	\begin{equation*}
	\distor{f} := \inf\set{c \in \intfa{0}{\infty}}{\forall_{p,p' \in M} \dist{f(p)}{f(p')} \leq c\dist{p}{p'}}.
	\end{equation*}
\end{definition}

A distorção de uma função controlada $f$ é portanto o menor valor tal que vale a desigualdade entre as distâncias da definição. Vale que
	\begin{equation*}
	\dist{f(p)}{f(p')} \leq \distor{f}\dist{p}{p'}.
	\end{equation*}
Além disso, a distorção de $f$ satisfaz a seguinte definição em termos do supremo:
	\begin{equation*}
	\distor{f} = \sup \set{\frac{\dist{f(p)}{f(p')}}{\dist{p}{p'}}}{p,p' \in M \land p \neq p'}.
	\end{equation*}
Note que, caso $\card{M} \leq 1$, o conjunto nesse supremos é vazio, e nesse caso devemos ressaltar que o supremo é tomado sobre $\intfa{0}{\infty}$, portanto o supremo de $\emptyset$ é $0$.

%\begin{exercise}
%Sejam $\bm M$ e $\bm M'$ espaços métricos e $\fun{f}{M}{M'}$ uma função controlada.
%	\begin{enumerate}
%	\item Para todos $p,p' \in M$,
%		\begin{equation*}
%		\dist{f(p)}{f(p')} \leq \distor{f}\dist{p}{p'};
%		\end{equation*}
%		
%	\item Valem a seguinte igualdade:
%		\begin{equation*}
%		\distor{L} = \sup \set{\frac{\dist{f(p)}{f(p')}}{\dist{p}{p'}}}{\forall_{p,p' \in M} p \neq p'}.
%		\end{equation*}
%	\end{enumerate}
%\end{exercise}

\begin{exercise}
Seja $\bm M$ um espaço métrico. A identidade $\fun{\Id}{M}{M}$ é função controlada e $\distor{\Id}=1$.
\end{exercise}

\begin{proposition}
\label{prop:composicao.controlada}
Sejam $\bm M, \bm M'$ e $\bm M''$ espaços métricos e $\fun{f}{M}{M'}$ e $\fun{f'}{M'}{M''}$ funções controladas. Então $\fun{f' \circ f}{M}{M''}$ é uma função controlada e
	\begin{equation*}
		\distor{f' \circ f} \leq \distor{f'}\distor{f}.
	\end{equation*}
\end{proposition}
\begin{proof}
Para todos $p,p' \in M$,
	\begin{equation*}
	\dist{f' \circ f(p)}{f' \circ f(p'))} \leq \distor{f'} \dist{f(p)}{f(p')} \leq \distor{f'}\distor{f}\dist{p}{p'},
	\end{equation*}
o que mostra que $f' \circ f$ é controlada e $\distor{f' \circ f} \leq \distor{f'}\distor{f}$.
\end{proof}

\begin{proposition}
\label{prop:continuidade.controlada}
Sejam $\bm M$ e $\bm M'$ espaços métricos e $\fun{f}{M}{M'}$ uma função controlada. Então $f$ é uniformemente contínua.
\end{proposition}
\begin{proof}
Se $\distor{f}=0$, a demonstração é óbvia. Se $\distor{f} \neq 0$, seja $\varepsilon>0$. Tomando $\delta=\frac{\varepsilon}{\distor{f}}$, segue que, para todos $p,p' \in M$, se $\dist{p}{p'}_0 \leq \delta$, então
	\begin{equation*}
	\dist{f(p)}{f(p')} \leq \distor{f}\dist{p}{p'} \leq \distor{f}\frac{\varepsilon}{\distor{f}} = \varepsilon.
	\qedhere
	\end{equation*}
\end{proof}

\begin{proposition}
\label{prop:distorcao.injetiva.sobrejetiva}
Sejam $\bm M$ e $\bm M'$ espaços métricos e $\fun{f}{M}{M'}$ uma função controlada.
	\begin{enumerate}
	\item $f$ é constante se, e somente se, $\distor{f} = 0$;
	\item Se $\card{M} > 1$ e $f$ é injetiva, $\distor{f} > 0$;
	\item Se $\card{M'} > 1$ e $f$ é sobrejetiva, $\distor{f} > 0$.
	\end{enumerate}
\end{proposition}
\begin{proof}
	\begin{enumerate}
	\item Se $f$ é constante, então para todos $p,p' \in M$ vale $f(p)=f(p')$, portanto
		\begin{equation*}
		\dist{f(p)}{f(p')} = 0 \leq 0 \dist{p}{p'},
		\end{equation*}
	o que mostra que $\distor{f} = 0$.
	
	Reciprocamente, se $\distor{f}=0$, então para todos $p,p' \in M$ vale
		\begin{equation*}
		\dist{f(p)}{f(p')} \leq \distor{f}\dist{p}{p'} = 0,
		\end{equation*}
	portanto $f(p) = f(p')$.

	\item Como $\card{M} > 1$, existem $p,p' \in M$ tais que $p \neq p'$; como $f$ é injetiva, segue que $f(p) \neq f(p')$, portanto $f$ não é constante; do primeiro item segue que $\distor{f} > 0$.

	\item Como $\card{M'} > 1$, existem $q,q' \in M'$ tais que $q \neq q'$; como $f$ é sobrejetiva, $f(M) = M'$, logo $\card{f(M)} > 1$, o que significa que $f$ não é constante; do primeiro item segue que $\distor{f} > 0$.
	\end{enumerate}
\end{proof}

\begin{proposition}
Sejam $\bm M$ e $\bm M'$ espaços métricos e $\fun{f}{M}{M}$ função controlada.
	\begin{enumerate}
		\item Se $\card{M} > 1$ e $f$ é injetiva com inversa à esquerda $\fun{f\inv}{f(M)}{M}$ controlada, então
			\begin{equation*}
\textstyle	1 \leq \distor{f}\distor{f\inv}.
			\end{equation*}
		\item Se $\card{M'} > 1$ e $f$ é sobrejetiva com inversa à direita $\fun{f\inv}{M'}{M}$ controlada, então
			\begin{equation*}
\textstyle	1 \leq \distor{f}\distor{f\inv}.
			\end{equation*}
	\end{enumerate}
\end{proposition}
\begin{proof}
	\begin{enumerate}
	\item Sejam $p,p' \in M$ tais que $p \neq p'$ e $q := f(p)$ e $q' := f(p')$. Como $f$ e $f\inv$ são controladas e $f\inv(q) = p$ e $f\inv(q')=p'$, segue que
		\begin{align*}
		\dist{p}{p'} &= \dist{f\inv(q)}{f\inv(q')} \\
			&\leq \distor{f\inv}\dist{q}{q'} \\
			&= \distor{f\inv}\dist{f(p)}{f(p')} \\
			&\leq \distor{f\inv}\distor{f}\dist{p}{p'}.
		\end{align*}
	Como $\dist{p}{p'} \neq 0$, segue que $1 \leq \distor{f\inv}\distor{f}$.
	
	\item Segue do item anterior, pois $f\inv$ é injetiva com inversa à esquerda $f$.
	\qedhere
%	\item Sejam $q,q' \in M'$ tais que $q \neq q'$ e, da sobrejetividade de $f$, sejam $p,p' \in M$ tais que $q = f(p)$ e $q' = f(p')$. Como $f$ e $f\inv$ são controladas e $f\inv(q) = p$ e $f\inv(q')=p'$, segue que
%		\begin{align*}
%		\dist{q}{q'}' &= \dist{f(p)}{f(p')}' \\
%			&\leq \distor{f}\dist{p}{p'} \\
%			&= \distor{f}\dist{f\inv(q)}{f\inv(q')} \\
%			&\leq \distor{f}\distor{f\inv}\dist{q}{q'}'.
%		\end{align*}
%	Como $\dist{q}{q'} \neq 0$, segue que $1 \leq \distor{f\inv}\distor{f}$.
	\end{enumerate}
\end{proof}

\begin{definition}
Sejam $\bm M$ e $\bm M'$ espaços métricos. Um \emph{mergulho métrico} de $\bm M$ para $\bm M'$ é uma função $\fun{f}{M}{M'}$ tal que, para algum $c \in \intaa{0}{\infty}$, $f$ satisfaz que, para todos $p,p' \in M$,
	\begin{equation*}
	c\inv \dist{p}{p'} \leq \dist{f(p)}{f(p')} \leq c\dist{p}{p'}.
	\end{equation*}
\end{definition}

A proposição a seguir dá uma caracterização alternativa de mergulho métrico.

\begin{proposition}
\label{prop:mergulho.metrico}
Sejam $\bm M$ e $\bm M'$ espaços métricos e $\fun{f}{M}{M'}$. Então $f$ é mergulho métrico se, e somente se, $f$ é função controlada injetiva e sua inversa à esquerda $\fun{f\inv}{f(M)}{M}$ é controlada.
\end{proposition}
\begin{proof}
Notemos que se $\card{M} \leq 1$, a demonstração é trivial, portanto supomos $\card{M} > 1$.
	\begin{itemize}
	\item ($\Rightarrow$) Seja $c \in \intaa{0}{\infty}$ tal que valem as desigualdades na definição de mergulho métrico. Segue direto que $f$ é controlada. Para todos $p,p' \in M$ tais que $p \neq p'$, vale $\dist{p}{p'} > 0$, logo de $0 < c\inv \dist{p}{p'} \leq \dist{f(p)}{f(p')}$ segue que $\dist{f(p)}{f(p')}>0$, o que implica $f(p) \neq f(p')$, portanto $f$ é injetiva. Ainda, temos que $\dist{p}{p'} \leq c \dist{f(p)}{f(p')}$, logo para todos $q,q' \in f(M)$, existem $p,p' \in M$ tais que $q=f(p)$ e $q'=f(p')$, portanto
		\begin{align*}
		\dist{f\inv(q)}{f\inv(q')} &= \dist{f\inv(f(p))}{f\inv(f(p'))} \\
			&= \dist{p}{p'} \\
			&\leq c \dist{f(p)}{f(p')} \\
			&= c \dist{q}{q'},
		\end{align*}
	o que mostra que $f\inv$ é controlada.

	\item ($\Leftarrow$) Como $f$ é função controlada injetiva e $\card{M}>1$, segue de \ref{prop:distorcao.injetiva.sobrejetiva} que $\distor{f}>0$. A injetividade de $f$ implica que $f\inv$ é sobrejetiva e que $\card{f(M)}>0$, segue de \ref{prop:distorcao.injetiva.sobrejetiva} que $\distor{f\inv}>0$. Definindo $c := \distor{f} \opmax \distor{f\inv}$, segue que $c > 0$, $c \geq \distor{f}$ e $c\inv \leq \distor{f\inv}\inv$.
	
	Como $f$ é função controlada, então, para todos $p,p' \in M$,
		\begin{equation*}
		\dist{f(p)}{f(p')} \leq \distor{f} \dist{p}{p'} \leq c \dist{p}{p'}.
		\end{equation*}
	
	Como a inversa à esquerda $f\inv$ é função controlada, então, para todos $p,p' \in M$,
		\begin{align*}
		c\inv \dist{p}{p'} &= c\inv \dist{f\inv(f(p))}{f\inv(f(p'))} \\
			&\leq c\inv \distor{f\inv} \dist{f(p)}{f(p')} \\
			&\leq c\inv c\dist{f(p)}{f(p')} \\
			&= \dist{f(p)}{f(p')}.
		\end{align*}
	Portanto
		\begin{equation*}
		c\inv \dist{p}{p'} \leq \dist{f(p)}{f(p')} \leq c \dist{p}{p'}.
		\qedhere
		\end{equation*}
	\end{itemize}
\end{proof}

\begin{proposition}
Sejam $\bm M$ e $\bm M'$ espaços métricos e $\fun{f}{M}{M'}$ um mergulho métrico. Então $f$ é mergulho topológico.
\end{proposition}
\begin{proof}
Segue direto de \ref{prop:mergulho.metrico} que $f$ é contínua, injetiva e sua inversa à esquerda é contínua, portanto $\fun{f}{M}{f(M)}$ é homeomorfismo.
\end{proof}

\subsection{Homometrias e isometrias}

\begin{definition}
Sejam $\bm M$ e $\bm M'$ espaços métricos. Uma \emph{homometria}%
	\footnote{Essas funções são também conhecidas como funções métricas, funções não expansoras, entre outros. O nome homometria faz referência a uma relação que essas funções têm com as isometrias, pois elas representam morfismos de espaços métricos cujos respectivos isomorfismos são as isometrias.} %
de $\bm M$ para $\bm M'$ é uma função controlada $\fun{f}{M}{M'}$ tal que $\distor{f} \leq 1$.
O conjunto das homometrias de $\bm M$ para $\bm M'$ é denotado $\Met(\bm M,\bm M')$.
\end{definition}

Uma homometria $f$ satisfaz portanto
	\begin{equation*}
	\dist{f(p)}{f(p')} \leq \dist{p}{p'}.
	\end{equation*}

\begin{proposition}
\label{prop:composicao.homometria}
Sejam $\bm M, \bm M'$ e $\bm M''$ espaços métricos e $\fun{f}{M}{M'}$ e $\fun{f'}{M'}{M''}$ homometrias. Então $\fun{f' \circ f}{M}{M''}$ é uma homometria.
\end{proposition}
\begin{proof}
Segue de \ref{prop:composicao.controlada} que se $\distor{f} \leq 1$ e $\distor{f'} \leq 1$, então $\distor{f' \circ f} \leq 1$, portanto $f' \circ f$ é uma homometria.
\end{proof}

\begin{definition}
Sejam $\bm{M}$ e $\bm{M'}$ espaços métricos. Uma \emph{isometria local} (ou \emph{imersão isométrica}) de $\bm{M}$ para $\bm{M'}$ é uma função $\fun{f}{M}{M'}$ que satisfaz, para todos $p,p' \in M$,
	\begin{equation*}
	\dist{f(p)}{f(p')} = \dist{p}{p'}.
	\end{equation*}
Uma \emph{isometria} é isometria local bijetiva.
\end{definition}

\begin{proposition}
Sejam $\bm{M}$ e $\bm{M'}$ espaços métricos e $\fun{f}{M}{M'}$ uma isometria local. Então $f$ é injetiva.
\end{proposition}
\begin{proof}
Sejam $p,p' \in M$ tais que $p \neq p'$. Então $\dist{p}{p'}_0 \neq 0$, logo
	\begin{equation*}
	\dist{f(p)}{f(p')} = \dist{p}{p'} \neq 0,
	\end{equation*}
o que implica $f(p) \neq f(p')$.
\end{proof}

\begin{proposition}
Sejam $\bm{M}$ e $\bm{M'}$ espaços métricos e $\fun{f}{M}{M'}$ uma homometria injetiva cuja inversa à esquerda é homometria. Então $f$ é uma isometria local.
\end{proposition}
\begin{proof}
Sejam $p,p' \in M$. Então, como $f$ é homometria,
	\begin{equation*}
	\dist{f(p)}{f(p')} \leq \dist{p}{p'}
	\end{equation*}
e, como $f\inv$ é homometria,
	\begin{equation*}
	\dist{p}{p'} = \dist{f\inv \circ f(p)}{f\inv \circ f(p')} \leq \dist{f(p)}{f(p')};
	\end{equation*}
portanto  $\dist{p}{p'} = \dist{f(p)}{f(p')}$.
\end{proof}

\begin{proposition}
Sejam $\bm{M}$ e $\bm{M'}$ espaços métricos e $\fun{f}{M}{M'}$ homometria. A função $f$ é isometria se, e somente se, é invertível e sua inversa é homometria.
\end{proposition}
\begin{proof}
Suponhamos que $f$ é isometria. Então $f$ é bijetiva e, portanto, invertível. Sua inversa satisfaz, para todos $p,p' \in M'$,
	\begin{equation*}
	\dist{f\inv(p)}{f\inv(p')} = \dist{f(f\inv(p))}{f(f\inv(p'))} = \dist{p}{p'}.
	\end{equation*}
Portanto $f\inv$ é isometria local, logo homometria.

Reciprocamente, suponhamos que $f$ é invertível e sua inversa é homometria. Segue da proposição anterior que $f$ é isometria local e, como é bijetiva, é isometria.
\end{proof}

\subsection{Contrações}

\begin{definition}
Sejam $\bm M$ e $\bm M'$ espaços métricos. Uma \emph{contração} de $\bm M$ para $\bm M'$ é uma função controlada $\fun{f}{M}{M'}$ tal que $\distor{f} < 1$.
\end{definition}

Uma contração $f$ satisfaz portanto
	\begin{equation*}
	\dist{f(p)}{f(p')} < \dist{p}{p'}.
	\end{equation*}
Em particular, uma contração é uma homometria.

\begin{proposition}
\label{prop:composicao.contracao}
Sejam $\bm M, \bm M'$ e $\bm M''$ espaços métricos e $\fun{f}{M}{M'}$ e $\fun{f'}{M'}{M''}$ contrações. Então $\fun{f' \circ f}{M}{M''}$ é uma contração.
\end{proposition}
\begin{proof}
Segue de \ref{prop:composicao.controlada} que se $\distor{f} < 1$ e $\distor{f'} < 1$, então $\distor{f' \circ f} < 1$, portanto $f' \circ f$ é uma contração.
\end{proof}

\begin{proposition}[Ponto fixo para contrações]
\label{prop:ponto.fixo.contracao}
Sejam $\bm M$ um espaço métrico completo não vazio e $\fun{f}{M}{M}$ uma contração. Existe único ponto fixo $\bar p \in M$ para $f$ e, para todo $p \in M$,
	\begin{equation*}
	\lim_{n \to \infty} f^n(p) = \bar p.
	\end{equation*}
\end{proposition}
\begin{proof}
Mostremos por indução que, para todos $p \in M$ e $n \in \N$,
	\begin{equation*}
	\dist{f^n(p)}{f^{n+1}(p)} \leq \distor{f}^n \dist{p}{f(p)}.
	\end{equation*}
Claramente, para $n=0$ isso claramente vale. Agora, suponhamos que a desigualdade valha para $n=k$ e mostremos que ela vale para $n=k+1$. Como $f$ é contração,
	\begin{equation*}
	\dist{f^k(p)}{f^{k+1}(p)} \leq \distor{f} \dist{f^{k-1}(p)}{f^k(p)} \leq \distor{f} \distor{f}^{k-1} \dist{p}{f(p)} = \distor{f}^k \dist{p}{f(p)}.
	\end{equation*}
Agora, notemos que, para todos $n,p \in \N$, segue da desigualdade triangular generalizada que
	\begin{align*}
	\dist{f^n(p)}{f^{n+p}(p)} &\leq \sum_{i=0}^{p-1} \dist{f^{n+i}(p)}{f^{n+i+1}(p)} \\
		&\leq \sum_{i=0}^{p-1} \distor{f}^{n+i} \dist{p}{f(p)} \\
		&= \distor{f}^n\frac{1-\distor{f}^p}{1-\distor{f}} \dist{p}{f(p)} \\
		&\leq \frac{\distor{f}^n}{1-\distor{f}} \dist{p}{f(p)},
	\end{align*}
pois $\distor{f}\geq 0$ implica $1-\distor{f}^p<1$. Como $\distor{f}<1$, então $\lim_{n \to \infty} \frac{\distor{f}^n}{1-\distor{f}} = 0$, portanto, para todos $n,p \in \N$,
	\begin{equation*}
	\lim_{n \to \infty} \dist{f^n(p)}{f^{n+p}(p)} = 0,
	\end{equation*}
o que mostra que $(f^n(p))_{n \in \N}$ é uma sequência aproximante e, como $\bm M$ é completo, converge para $\bar p \in M$. Como $f$ é contínua,
	\begin{equation*}
	f(\bar p) = f\left(\lim_{n \to \infty} f^n(p)\right) = \lim_{n \to \infty} f^{n+1}(p) = \bar p,
	\end{equation*}
esse ponto $\bar p$ é um ponto fixo. Para mostrarmos que $\bar p$ é único, suponhamos que $p$ é ponto fixo de $f$. Então
	\begin{equation*}
	\dist{\bar p}{p} = \dist{f(\bar p)}{f(p)} \leq \distor{f} \dist{\bar p}{p}
	\end{equation*}
e como $\distor{f}<1$ isso implica $\dist{\bar p}{p}=0$, logo $\bar p=p$.
\end{proof}

Esse resultado pode parecer trivial por sua demonstração, mas suas consequências na matemática são enormes. Como consequência dessa proposição seguem por exemplo o teorema da função inversa do cálculo diferencial e o teorema de existência e unicidade de equações diferenciais.



















\subsection{Semelhanças}

\begin{definition}
Sejam $\bm{M}$ e $\bm{M'}$ espaços métricos. Uma \emph{semelhança local} ou (\emph{imersão semelhante}) de $\bm{M}$ para $\bm{M'}$ é uma função $\fun{f}{M}{M'}$ tal que, para algum $c \in \intfa{0}{\infty}$, $f$ satisfaz que, para todos $p,p' \in M$,
	\begin{equation*}
	\dist{f(p)}{f(p')} = c \dist{p}{p'}.
	\end{equation*}
Uma \emph{semelhança} é semelhança local bijetiva.
\end{definition}

\begin{proposition}
Sejam $\bm{M}$ e $\bm{M'}$ espaços métricos e $\fun{f}{M}{M'}$ uma semelhança local. Então, para todos $p,p' \in M$,
	\begin{equation*}
	\dist{f(p)}{f(p')} = \distor{f} \dist{p}{p'}.
	\end{equation*}
\end{proposition}
\begin{proof}
Para algum $c \in \intfa{0}{\infty}$, $f$ satisfaz que, para todos $p,p' \in M$,
	\begin{equation*}
	\dist{f(p)}{f(p')} = c \dist{p}{p'},
	\end{equation*}
o que implica $\distor{f} \leq c$. Se valesse $\distor{f}<c$, então, para todos $p,p' \in M$ tais que $p \neq p'$, seguiria que
	\begin{equation*}
	c \dist{p}{p'} = \dist{f(p)}{f(p')} \leq \distor{f}\dist{p}{p'} < c \dist{p}{p'},
	\end{equation*}
o que seria uma contradição. Assim $\distor{f} \geq c$, portanto $\distor{f}=c$ e concluímos que
	\begin{equation*}
	\dist{f(p)}{f(p')} = \distor{f} \dist{p}{p'}.
	\qedhere
	\end{equation*}
\end{proof}

\begin{proposition}
\label{prop:}
Sejam $\bm{M}$ e $\bm{M'}$ espaços métricos e $\fun{f}{M}{M'}$ uma função controlada com $\distor{f}>0$. Então $f$ é semelhança local se, e somente se, $f$ é mergulho métrico e $\distor{f\inv}=\distor{f}\inv$.
\end{proposition}
\begin{proof}
	\begin{itemize}
	\item ($\Rightarrow$) Suponhamos que $f$ é semelhança local. Mostremos que $f$ é injetiva, sua inversa à esquerda $\fun{f\inv}{f(M)}{M}$ é semelhança e $\distor{f\inv}=\distor{f}\inv$, e seguirá de \ref{prop:mergulho.metrico} que $f$ é mergulho métrico.

	Sejam $p,p' \in M$ tais que $p \neq p'$. Então
		\begin{equation*}
		\dist{f(p)}{f(p')} = \distor{f}\dist{p}{p'} > 0
		\end{equation*}
	portanto $f(p) \neq f(p')$, o que mostra que $f$ é injetiva.

	Para todos $q,q' \in f(M)$, sejam $p,p' \in M$ tais que $f(p)=q$ e $f(p')=q'$. Então
		\begin{equation*}
		\dist{f\inv(q)}{f\inv(q')} = \dist{p}{p'} = \distor{f}\inv\dist{f(p)}{f(p')} = \distor{f}\inv\dist{q}{q'},
		\end{equation*}
	o que mostra que $f\inv$ é semelhança e $\distor{f\inv}=\distor{f}\inv$.

	\item ($\Leftarrow$) Se $f$ é mergulho métrico, segue de \ref{prop:mergulho.metrico} que $f$ é função controlada injetiva e sua inversa à esquerda $\fun{f\inv}{f(M)}{M}$ é controlada. Para todos $p,p' \in M$, vale
		\begin{equation*}
		\dist{f(p)}{f(p')} \leq \distor{f}\dist{p}{p'}.
		\end{equation*}
	Como $\distor{f\inv} = \distor{f}\inv$, vale
		\begin{equation*}
			\distor{f}\dist{f\inv(f(p))}{f\inv(f(p'))} \leq \distor{f}\distor{f}\inv\dist{f(p)}{f(p')} = \inv\dist{f(p)}{f(p')}.
		\end{equation*}
	Disso concluímos que $\dist{f(p)}{f(p')} = \distor{f}\dist{p}{p'}$.
	\qedhere
	\end{itemize}
\end{proof}




\section{Medida em espaços métricos}

\subsection{Medidas exteriores métricas}

\begin{definition}
Seja $\bm M$ um espaço métrico. Uma medida exterior \emph{métrica} em $\bm M$ é uma medida exterior $\med\colon \p(M) \to \intff{0}{\infty}$ sobre $M$ tal que, para todos $C,C' \subseteq M$ metricamente separados,
	\begin{equation*}
	\med(C \cup C') = \med(C) + \med(C').
	\end{equation*}
\end{definition}

\begin{proposition}
\label{prop:criterio.mensur.med.metrica}
Sejam $\bm M$ um espaço métrico, com $\sigma$-álgebra topológica $\mens_\topo$ e $\med$ uma medida exterior métrica em $\bm M$. Então todo $M \in \mens_\topo$ é $\med$-mensurável.
\end{proposition}
\begin{proof}
Para mostrar isso, mostraremos que todo conjunto fechado é $\med$-mensurável. Basta mostrar que, para todo $C \subseteq M$ com $\med(C) < \infty$ e todo fechado $F \subseteq M$,
	\begin{equation*}
	\med(C) \geq \med(C \cap F) + \med(C \cap F^\complement),
	\end{equation*}
pois a desigualdade contrária sempre vale por subaditividade e a igualdade vale trivialmente se $\med(C) = \infty$. Consideremos as vizinhanças fechadas
	\begin{equation*}
	F_j := \bolafec{F}{\frac{1}{j}} = \set{p \in M}{\dist{F}{p} \leq \frac{1}{j}}.
	\end{equation*}
Vale que $\dist{C \cap F}{C \cap F_j^\complement} > 0$, portanto
	\begin{equation*}
	\med(C) \geq \med\left( (C \cap F) \cup (C \cap F_j^\complement) \right) = \med(C \cap F) + \med(C \cap F_j^\complement).
	\end{equation*}
Resta mostrar agora que $\lim_{j \conv \infty} \med(C \cap F_j^\complement) = \med(C \cap F^\complement)$. Como $F$ é fechado, podemos escrever, para todo $j \in \N^*$,
	\begin{equation*}
	C \cap F^\complement = \set{p \in C}{\dist{F}{p} > 0} = (C \cap F_j^\complement) \cup \bigcup_{k=j}^\infty R_k,
	\end{equation*}
em que $R_k := C \cap \bolafec{F}{\frac{1}{k}} \setminus \bolafec{F}{\frac{1}{k+1}} = \set{p \in C}{\frac{1}{k+1} < \dist{F}{p} \leq \frac{1}{k}}$. Pela subaditividade de $\med$, segue que
	\begin{equation*}
	\med(C \cap F_j^\complement) \leq \med(C \cap F^\complement) \leq \med(C \cap F_j^\complement) + \sum_{k=j}^\infty \med(R_k).
	\end{equation*}
Mas note que $\sum_{k=1}^\infty \med(R_k) < \infty$. Isso ocorre pois, para todo $j \geq i+2$, $\dist{F_i}{F_j} > 0$, portanto por indução em $N$ segue que
	\begin{equation*}
	\sum_{k=1}^N \med(R_{2k}) = \med\left( \bigcup_{k=1}^N R_{2k} \right) \leq \med(C) < \infty
	\end{equation*}
e
	\begin{equation*}
	\sum_{k=1}^N \med(R_{2k-1}) = \med\left( \bigcup_{k=1}^N R_{2k-1} \right) \leq \med(C) < \infty.
	\end{equation*}
Portanto $\sum_{k=1}^\infty \med(R_k) < \infty$, o que implica que $\lim_{j \conv \infty} \med(C \cap F_j^\complement) = \med(C \cap F^\complement)$, e concluímos que $F$ é $\med$-mensurável, resultando que todo $M \in \mens_\topo$ é $\med$-mensurável.
\end{proof}

%A recíproca também vale.

\subsection{Medidas por coberturas métricas}

Nesta seção, definiremos uma família de medidas exteriores em um espaço métrico e usaremos essas medidas para definir a dimensão do espaço métrico e de seus subconjuntos mensuráveis. No entanto, é importante ressaltar que existem diferentes definições de medida e de dimensão em espaços métricos e aqui abordaremos somente uma delas, a medida por coberturas métricas. Uma outra abordagem considera, em vez de coberturas métricas, empacotamentos, e essa abordagem chega a resultados semelhantes, mas às vezes distintos dos que chegaremos aqui. Essa abordagem por empacotamentos é, de certa forma, a noção dual da abordagem que estudaremos usando coberturas. No entanto, um paradigma que é em geral seguido é que as dimensões definidas coincidam com as dimensões de espaços lineares como a reta, o plano, e de variedades.

Consideremos a função diâmetro em $\bm M$
	\begin{align*}
	\func{\diam}{\p(M)}{[0,\infty]}{C}{\diam(C)}.
	\end{align*}

Essa função $\diam$ não é uma medida exterior em $M$. Ela satisfaz (1) $\diam(\emptyset)=0$ e (2) Para todos $C,D \subseteq M$ tais que $C \subseteq D$, então $C \times C \subseteq D \times D$, portanto $d(C \times C) \subseteq d(D \times D)$, o que implica
	\begin{equation*}
	\diam(C) \leq \diam(D);
	\end{equation*}
No entanto, não satisfaz 
(3) Para todos $(C_i)_{i \in \N}$ subconjuntos de $M$,
	\begin{equation*}
	\diam\left( \bigcup_{i \in \N} C_i \right) \leq \sum_{i \in \N} \diam\left( C_i \right).
	\end{equation*}
Para ver isso, considere o intervalo $[0,1]$ e os conjuntos $C_i$ como os intervalos de tamanho $\frac{1}{2^{i+2}}$ e que tocam pela direita nos pontos $\frac{1}{2^i}$ do intervalo $[0,1]$. Facilmente nota-se que
	\begin{equation*}
	\diam\left( \bigcup_{i \in \N} C_i \right) = 1 > \frac{1}{2} = \sum_{i \in \N} \diam\left( C_i \right).
	\end{equation*}

Isso ocorre porque a distância entre os conjuntos $C_i$ não é considerada na soma dos diâmetros individuais, mas é considerada na união, e essa distância resulta em $\frac{1}{2}$ nesse caso. Pode-se fazer com que essa diferença seja tão grande quanto se queira.

Definiremos uma família de medidas exteriores em $\bm M$ com um parâmetro $d$, que representa a `dimensão' da medida utilizando a função diâmetro $\diam$. Mas antes brevemente comentamos a definição de cobertura que usaremos.

\begin{definition}
Sejam $\bm M$ um espaço métrico, $C \subseteq M$ e $\delta \in \intaa{0}{\infty}$. Uma \emph{$\delta$-cobertura} de $C$ é uma cobertura $(C_i)_{i \in I}$ de $C$ tal que, para todo $i \in I$, $\diam(C_i) \leq \delta$. O conjunto de $\delta$-coberturas de $C$ é $\mathcal C_\delta (C)$.
\end{definition}

\begin{definition}
Sejam $\bm M$ um espaço métrico, $C \subseteq M$, $d \in \intfa{0}{\infty}$ e $\delta \in \intaa{0}{\infty}$. A \emph{medida $d$-dimensional $\delta$-precisa} de $C$ em $\bm M$ é
%	\begin{equation*}
%	H^d_\delta(C) := \inf\set{\sum_{i \in \N} \diam(U_i)^d}{C \subseteq \bigcup_{i \in \N} U_i, \forall i \in \N\ \diam(U_i) \leq \delta}.
%	\end{equation*}
	\begin{equation*}
	H^d_\delta(C) := \inf\set{\sum_{i \in \N} \diam(U_i)^d}{(U_i)_{i \in \N} \in \mathcal C_\delta(C)}.
	\end{equation*}
 A \emph{medida $d$-dimensional $\delta$-precisa} em $\bm M$ é a função
 	\begin{align*}
 	\func{H^d_\delta}{\p(M)}{\intff{0}{\infty}}{C}{H^d_\delta(C)}.
 	\end{align*}
\end{definition}

A sequência $(U_i)_{i \in \N}$ é uma $\delta$-\emph{cobertura} de $C$. Mostremos que $H^d_\delta$ é uma medida exterior em $M$.

\begin{proposition}
Sejam $\bm M$ um espaço métrico, $d \in \intfa{0}{\infty}$ e $\delta \in \intaa{0}{\infty}$. A função $H^d_\delta\colon \p(M) \to \intff{0}{\infty}$ é uma medida exterior sobre $M$.
\end{proposition}
\begin{proof}
(Conjunto vazio) $H^d_\delta(\emptyset)=0$, pois se tomamos a cobertura vazia $(\emptyset)_{i \in \N}$, temos que $\emptyset \subseteq \bigcup_{i \in \N} \emptyset$ e $\diam(\emptyset)\leq \delta$; (Monotonicidade) Para todos $C,C' \subseteq M$ tais que $C \subseteq C'$, temos que uma $\delta$-cobertura de $C'$ é uma $\delta$-cobertura de $C$, logo $H^d_\delta(C) \leq H^d_\delta(C')$; (Subaditividade contável) Seja $(C_i)_{i \in \N}$ uma sequência de subconjuntos de $M$. Para todos $i \in \N$ e $\varepsilon \in \left]0,\infty\right[$, seja $U^i=(U_{i,j})_{j \in \N}$ é uma cobertura de $C_i$ tal que
	\begin{equation*}
	\sum_{j \in \N} \diam(U_{i,j})^d \leq H^d_\delta(C_i) + \frac{\varepsilon}{2^{i+1}}.
	\end{equation*}
Essa cobertura existe porque $H^d_\delta(C_i)$ é um ínfimo. Então $(U_{i,j})_{(i,j) \in \N^2}$ é uma cobertura de $\bigcup_{i \in \N} C_i$ e segue que
	\begin{align*}
	H^d_\delta\left( \bigcup_{i \in \N} C_i \right) &\leq H^d_\delta\left( \bigcup_{(i,j) \in \N^2} U_{i,j} \right) \\
		&\leq \sum_{(i,j) \in \N^2} \diam(U_{i,j})^d \\
		&\leq \sum_{i \in \N} \left( H^d_\delta(C_i) + \frac{\varepsilon}{2^{i+1}} \right) \\
		&= \sum_{i \in \N} \left( H^d_\delta(C_i)\right) + \sum_{i \in \N}\frac{\varepsilon}{2^{i+1}} \\
		&= \sum_{i \in \N} \left( H^d_\delta(C_i)\right) + \varepsilon.
	\end{align*}
A primeira desigualdade vem da monotonicidade de $H^d_\delta$, a segunda de $H^d_\delta$ ser ínfimo, e a terceira vem da condição para as coberturas $(U_{i,j})_{j \in \N}$. Como isso vale para qualquer $\varepsilon$, segue que
	\begin{equation*}
	H^d_\delta\left( \bigcup_{i \in \N} C_i \right) \leq \sum_{i \in \N} H^d_\delta(C_i). \qedhere
	\end{equation*}
\end{proof}


Definimos agora a medida $H^d$ que independe de $\delta$. Notemos que, se $\delta \leq \delta'$, então $H^d_{\delta'}(C) \leq H^d_\delta(C)$, pois toda cobertura de $C$ com diâmetro $\delta$ é uma cobertura com diâmetro $\delta'$. Isso implica que existe em $\intff{0}{\infty}$ o limite
	\begin{equation*}
	\lim_{\delta \conv 0} H^d_\delta(C) = \sup_{\delta \in \intaa{0}{\infty}} H^d_\delta(C).
	\end{equation*}

\begin{definition}
Sejam $\bm M$ um espaço métrico e $d \in \intfa{0}{\infty}$. A \emph{medida $d$-dimensional} em $\bm M$ é a função
 	\begin{align*}
 	\func{H^d}{\p(M)}{\intff{0}{\infty}}{C}{H^d(C) := \sup_{\delta \in \intaa{0}{\infty}} H^d_\delta(C).}
 	\end{align*}
\end{definition}

\begin{proposition}
Sejam $\bm M$ um espaço métrico e $d \in \intfa{0}{\infty}$. A função
	\begin{equation*}
	H^d\colon \p(M) \to \intff{0}{\infty}
	\end{equation*}
é uma medida exterior métrica em $M$.
\end{proposition}
\begin{proof}
(Conjunto vazio) $H^d(\emptyset)=0$, pois $H^d_\delta(\emptyset)=0$ para todo $\delta \in \intaa{0}{\infty}$; (Monotonicidade) Para todos $C,C' \subseteq M$ tais que $C \subseteq C'$, temos que $H^d_\delta(C) \leq H^d_\delta(C')$ para todo $\delta \in \intaa{0}{\infty}$, logo $H^d(C) \leq H^d(C')$; (Subaditividade contável) Seja $(C_i)_{i \in \N}$ uma sequência de subconjuntos de $M$. Como para todo $i \in \N$ e $\delta \in \intaa{0}{\infty}$ vale por definição que $H^d_\delta(C_i) \leq H^d(C_i)$, segue que
	\begin{equation*}
	H^d_\delta\left( \bigcup_{i \in \N} C_i \right) \leq \sum_{i \in \N} H^d_\delta(C_i) \leq  \sum_{i \in \N} H^d(C_i).
	\end{equation*}
Como isso vale para todo $\delta$, segue que
	\begin{equation*}
	H^d\left( \bigcup_{i \in \N} C_i \right) \leq \sum_{i \in \N} H^d\left( C_i \right).
	\end{equation*}

Por fim, pode-se mostrar que, para todos conjuntos $C,C' \in M$ e para todo $\delta < d(C,C')$, tem-se
	\begin{equation*}
	H^d_\delta(C \cup C') = H^d_\delta(C) + H^d_\delta(C'),
	\end{equation*}
portanto, se $d(C,C') > 0$, tem-se
	\begin{equation*}
	H^d(C \cup C') = H^d(C) + H^d(C'). \qedhere
	\end{equation*}
\end{proof}

Essa medida é comumente chamada de \emph{medida de Hausdorff} $d$-dimensional. Definem-se os conjuntos mensuráveis como usual para medidas exteriores. Um conjunto $E \subseteq M$ é mensurável se, e somente se, para todo conjunto $C \in M$,
	\begin{equation*}
	H^d(C) = H^d(C \cap E) + H^d(C \cap E^\complement).
	\end{equation*}

A proposição mostra que $H^d$ é uma medida exterior métrica, todos os conjuntos da $\sigma$-álgebra topológica (conjuntos de Borel) são mensuráveis pela medida exterior $H^d$ (\ref{prop:criterio.mensur.med.metrica}) e $H^d$ pode ser restringida para uma medida em $M$. Em geral, não se pode garantir o mesmo para as medidas exteriores\footnote{\url{https://web.stanford.edu/class/math285/ts-gmt.pdf}} $H^d_\delta$. Pode-se ainda mostrar a seguinte proposição.

\begin{proposition}
Seja $n \in \N$ e $\R^n$ o espaço métrico real $n$-dimensional. A medida $H^n$ é um múltiplo da medida de volume  $\vol^n$ em $\R^n$ (Lebesgue):
	\begin{equation*}
	H^n = \frac{2^n}{\vol^n(\mathbb B^n)}\vol^n.
	\end{equation*}
\end{proposition}

Lembrando que
	\begin{equation*}
	\vol^n(\mathbb B^n) = \frac{(\quo{\tau}{2})^{\frac{n}{2}}}{\left( \quo{n}{2} \right)!}
	\end{equation*}
em que $\tau = 6,28\ldots$ é a constante do círculo (razão da circunferência pelo raio) e a função fatorial $!$ é entendida como a extensão dada pela função $\Gamma$, definida $x! = \Gamma(x+1)$, ou mais explicitamente por
	\begin{align*}
	\func{!}{\intff{0}{\infty}}{\intff{0}{\infty}}{x}{\int_0^\infty t^x \e^{-t} \dd t.}
	\end{align*}
Alguns casos particulares são
	\begin{align*}
	H^0 &= \vol^0 = \# &&\qquad	H^1 = \vol^1 \\
	H^2 &= \frac{8}{\tau}\vol^2 &&\qquad H^3 = \frac{12}{\tau}\vol^3
	\end{align*}
O fator $\vol^n(\mathbb{B}^n)$ poderia ser evitado multiplicando-o na definição de $H^n_\delta$, e o fator $2^n$ poderia ser evitado avaliando a soma de $\left (\frac{\diam(U_i)}{2}\right )^n$ em vez de somente $\diam(U_i)^n$. O número $\frac{\diam(C)}{2}$ pode ser naturalmente entendido como o \emph{raio} do conjunto $C$.

\subsection{Dimensão métrica e fractais}

Seja $C \subseteq M$ um conjunto. A função
	\begin{align*}
	\func{H^{(\var)}(C)}{\intfa{0}{\infty}}{\intff{0}{\infty}}{d}{H^d(C)}
	\end{align*}
é uma função com uma propriedade interessante. Ela admite no máximo três valores. Pode-se notar que, se $d \leq d'$, então $H^{d'}(C) \leq H^d(C)$. Além disso, existe $d \in \intfa{0}{\infty}$ tal que $H^d(C) = 0$. Portanto podemos definir a \emph{dimensão métrica} de $C$ como
	\begin{equation*}
	\dim(C) := \inf\set{d \in \intfa{0}{\infty}}{H^d(C)=0}.
	\end{equation*}
Nesse caso, pode-se mostrar que, para todo $d>\dim(C)$, $H^d(C) = 0$ e, para todo $d < \dim(C)$, $H^d(C) = \infty$. No entanto, o valor $H^{\dim(C)}(C)$ pode ser qualquer número em $\intff{0}{\infty}$. O valor de $H^{\dim(C)}(C)$ pode ser qualquer valor na linha tracejada do gráfico da figura~\ref{fig:dimensaometrica}. Exitem subconjuntos de $\R^d$ que têm dimensões não inteiras. Esses conjuntos são conhecidos como \emph{fractais}.

\begin{figure}
\centering
\begin{tikzpicture}
	\draw[<->] (0,3) node[anchor=east] {$H^{(\var)}(C)$} -- (0,0) node[anchor=east] {$0$} -- (12,0);
	\draw (0,2) node[anchor=east] {$\infty$} -- (6,2);
	\draw[dotted] (6,2) -- (6,0) node[anchor=north] {$\dim(C)$};
\end{tikzpicture}
\caption{Gráfico de $H^d(C)$ em função de $d$.}
\label{fig:dimensaometrica}
\end{figure}




\subsection{Comprimento de caminhos}

Seja $\bm M$ um espaço métrico. Lembremos que um caminho em $\bm M$ é uma função contínua $\fun{c}{\intff{a}{a'}}{M}$, em que $\intff{a}{a'} \subseteq \R$ é um intervalo compacto não degenerado, e o ponto inicial de $c$ é $c(a)$ e o ponto final de $c$ é $c(a')$. Uma reparametrização de $c$ é um caminho $\fun{c'}{\intff{b}{b'}}{X}$ tal que, para algum homeomorfismo $\fun{h}{\intff{b}{b'}}{\intff{a}{a'}}$,
	\begin{equation*}
	c' = c \circ h.
	\end{equation*}
Além disso, a concatenação de um caminho $\fun{c}{\intff{a}{a'}}{M}$ com um caminho $\fun{c'}{\intff{a'}{a''}}{M}$ é o caminho
	\begin{align*}
	\func{c \conca c'}{\intff{a}{a''}}{M}{t}{
		\begin{cases}
		c(t),	& t \in \intff{a}{a'} \\
		c'(t),	& t \in \intff{a'}{a''}.
		\end{cases}
	}
	\end{align*}

Para a definição do comprimento de um caminho, usamos o conceito de uma partição por intervalos de $\intff{a}{a'}$.

\begin{definition}
Sejam $\intff{a}{a'} \subseteq \R$ um intervalo compacto não degenerado e $n \in \N \setminus \{0\}$. Uma \emph{$n$-partição por intervalos} (ou \emph{partição por $n$ intervalos}) de $\intff{a}{a'}$ é uma lista $(t_i)_{i \in [n+1]} \in \R^{n+1}$ tal que
	\begin{equation*}
	a = t_0 < \cdots < t_n = a'.
	\end{equation*}
O conjunto dessas partições é denotado $\Part_{\intff{a}{a'}}$.
\end{definition}

Claramente, o conjunto $\{\intfa{t_0}{t_1},\ldots,\intfa{t_{n-2}}{t_{n-1}},\intff{t_{n-1}}{t_{n}}\}$ é uma partição de $\intff{a}{a'}$ e qualquer partição desse tipo define uma $n$-partição por intervalos de $\intff{a}{a'}$.

\begin{definition}
Sejam $\bm M$ um espaço métrico e $\fun{c}{\intff{a}{a'}}{M}$ um caminho em $\bm M$. O \emph{comprimento} de $c$ é
	\begin{equation*}
	\compr{c} := \sup\set{\sum_{i \in [n]} \dist{c(t_i)}{c(t_{i+1})}}{(t_i)_{i \in [n+1]} \in \Part_{\intff{a}{a'}}}.
	\end{equation*}
Um caminho \emph{retificável} é um caminho cujo comprimento é finito.
\end{definition}

O comprimento de uma curva é um número real positivo ou $\infty$.

\begin{proposition}
Sejam $\bm M$ um espaço métrico e $\fun{c}{\intff{a}{a'}}{M}$, $\fun{c'}{\intff{a'}{a''}}{M}$ caminhos em $\bm M$ tais que $c(a') = c'(a')$. Então
	\begin{equation*}
	\compr{c \conca c'} = \compr{c} + \compr{c'}.
	\end{equation*}
\end{proposition}
\begin{proof}
	\begin{itemize}
	\item ($\leq$) Seja $(t''_i)_{i \in [n''+1]} \in \Part_{\intff{a}{a''}}$ uma partição por intervalos. Defina $n := \min\set{i \in [n'']}{a' \leq t_i}$, de modo que valha $t_{n-1} < a' \leq t_n$. Defina a partição por intervalos $(t_i)_{i \in [n+1]}$ de $\intff{a}{a'}$ por
		\begin{equation*}
		t_i := \begin{cases}
				t''_i,	& 0 \leq i \leq n-1 \\
				a',		& i = n.
				\end{cases}
		\end{equation*}
	Se $a' = t_n$, defina $n' := n''-n$ e a partição por intervalos $(t'_i)_{i \in [n'+1]} := (t''_{i-n})_{i \in [n'+1]}$ de $\intff{a'}{a''}$; se $a' < t_n$, defina $n' := n''+1-n$ e a partição por intervalos $(t'_i)_{i \in [n'+1]}$ de $\intff{a'}{a''}$ por
		\begin{equation*}
		t_i := \begin{cases}
				a',				& i=0 \\
				t''_{i+1-n},	& 1 \leq i \leq n'.
				\end{cases}
		\end{equation*}
	Em ambos os casos, vale
		\begin{align*}
		\dist{c \conca c'(t''_{n-1})}{c \conca c'(t''_n)} &\leq \dist{c \conca c'(t''_{n-1})}{c \conca c'(a')} + \dist{c \conca c'(a')}{c \conca c'(t''_n)} \\
			&= \dist{c(t''_{n-1})}{c(a')} + \dist{c'(a')}{c'(t''_n)},
		\end{align*}
	portanto
		\begin{align*}
		\sum_{i \in [n'']} \dist{c \conca c'(t''_i)}{c \conca c'(t''_{i+1})} &= \sum_{i=0}^{n-2} \dist{c \conca c'(t''_i)}{c \conca c'(t''_{i+1})} \\
			&+ \dist{c \conca c'(t''_{n-1})}{c \conca c'(t''_n)} \\
			&+ \sum_{i=n}^{n''-1} \dist{c \conca c'(t''_i)}{c \conca c'(t''_{i+1})} \\
			&\leq \sum_{i=0}^{n-2} \dist{c(t''_i)}{c(t''_{i+1})} \\
			&+ \dist{c(t_{n-1})}{c(a')} + \dist{c'(a')}{c'(t_n)} \\
			&+ \sum_{i=n}^{n''-1} \dist{c'(t''_i)}{c'(t''_{i+1})} \\
			&= \sum_{i=0}^{n-1} \dist{c(t_i)}{c(t_{i+1})} + \sum_{i=0}^{n'-1} \dist{c'(t'_i)}{c'(t'_{i+1})}.
		\end{align*}
	Disso concluímos que $\compr{c \conca c'} \leq \compr{c} + \compr{c'}$.

	\item ($\geq$) Sejam $(t_i)_{i \in [n+1]} \in \Part_{\intff{a}{a'}}$ e $(t'_i)_{i \in [n'+1]} \in \Part_{\intff{a'}{a''}}$ partições por intervalos. Defina $n'' := n+n'$ e a partição por intervalos $(t''_i)_{i \in [n''+1]}$ de $\intff{a}{a''}$ por
		\begin{equation*}
		t''_i := \begin{cases}
				t_i,	& 0 \leq i \leq n \\
				t'_i,	& 1 \leq i \leq n'.
				\end{cases}
		\end{equation*}
	Segue então que
		\begin{align*}
		\sum_{i=0}^{n-1}& \dist{c(t_i)}{c(t_{i+1})} + \sum_{i=0}^{n'-1} \dist{c'(t'_i)}{c'(t'_{i+1})} \\
			&= \sum_{i=0}^{n-1} \dist{c \conca c'(t''_i)}{c \conca c'(t''_{i+1})} + \sum_{i=n}^{n''-1} \dist{c \conca c'(t''_i)}{c \conca c'(t''_{i+1})} \\
			&= \sum_{i \in [n'']} \dist{c \conca c'(t''_i)}{c \conca c'(t''_{i+1})}.
		\end{align*}
	Disso concluímos que $\compr{c} + \compr{c'} \leq \compr{c \conca c'}$.
	\end{itemize}
\end{proof}

\begin{proposition}
Sejam $\bm M$ um espaço métrico, $\fun{c}{\intff{a}{a'}}{M}$ um caminho e $\fun{h}{\intff{b}{b'}}{\intff{a}{a'}}$ um homeomorfismo. Então
	\begin{equation*}
	\compr{c \circ h} = \compr{c}.
	\end{equation*}
\end{proposition}
\begin{proof}
Basta mostrar que $\compr{c \circ h} \leq \compr{c}$ pois, como $h$ é homeomorfismo, sua inversa $\fun{h\inv}{\intff{a}{a'}}{\intff{b}{b'}}$ é homeomorfismo e $(c \circ h) \circ h\inv = c$, portanto seguirá que
	\begin{equation*}
	\compr{c} = \compr{(c \circ h) \circ h\inv} \leq \compr{c \circ h},
	\end{equation*}
logo $\compr{c \circ h} = \compr{c}$.

Seja $(t_i)_{i \in [n+1]} \in \Part_{\intff{b}{b'}}$ uma partição por intervalos. Como $h$ é homeomorfismo, existem duas possibilidades: $h(b)=a$ ou $h(b)=a'$. No primeiro caso, em que $h(b)=a$, vale
	\begin{equation*}
	a = h(t_0) < \cdots < h(t_n) = a',
	\end{equation*}
portanto, definindo a partição por intervalos $(t'_i)_{i \in [n+1]} := (h(t_i))_{i \in [n+1]}$ de $\intff{a}{a'}$, segue que
	\begin{equation*}
	\sum_{i=0}^{n-1} \dist{c\circ h(t_i)}{c \circ h(t_{i+1})} = \sum_{i=0}^{n-1} \dist{c(t'_i)}{c(t'_{i+1})}.
	\end{equation*}
Disso concluímos que $\compr{c \circ h} \leq \compr{c}$. No segundo caso, em que $h(b)=a'$, vale
	\begin{equation*}
	a = h(t_n) < \cdots < h(t_0) = a',
	\end{equation*}
portanto, definindo a partição por intervalos $(t'_i)_{i \in [n+1]} := (h(t_{n-i}))_{i \in [n+1]}$ de $\intff{a}{a'}$, segue que
	\begin{equation*}
	\sum_{i=0}^{n-1} \dist{c\circ h(t_i)}{c \circ h(t_{i+1})} = \sum_{i=0}^{n-1} \dist{c(t'_{n-i})}{c(t'_{n-i+1})} = \sum_{i=0}^{n-1} \dist{c(t'_{i})}{c(t'_{i+1})}.
	\end{equation*}
Disso concluímos que $\compr{c \circ h} \leq \compr{c}$.
\end{proof}