\chapter{Grupos diferenciais}

\section{Definições básicas}

Lembremos que denotamos um grupo pela notação multiplicativa: o grupo é denotado $(G,\times,\div,1)$, sendo
	\begin{align*}
	\func{\times}{G^2}{G}{(g,g')}{g \times g = gg'}
	\end{align*}
a operação binária, $1 \in G$ a identidade e
	\begin{align*}
	\func{\div}{G}{G}{g}{\div g = g\inv}
	\end{align*}
a inversa.

No caso de um grupo comutativo, às vezes usa-se a notação aditiva $(G,+,-,0)$, sendo
	\begin{align*}
	\func{+}{G^2}{G}{(g,g')}{g+g'}
	\end{align*}
a operação binária, $0 \in G$ a identidade e
	\begin{align*}
	\func{-}{G}{G}{g}{-g}
	\end{align*}
a inversa. Para a teoria geral, usaremos a notação multiplicativa.

Denotamos as translações à esquerda e à direita e a conjugação por um elemento $g \in G$ respectivamente por
	\begin{align*}
	\func{g \times}{G}{G}{h}{gh}
	\end{align*}
	\begin{align*}
	\func{\times g}{G}{G}{h}{hg}
	\end{align*}
	\begin{align*}
	\func{g\con}{G}{G}{h}{ghg\inv}.
	\end{align*}

\begin{definition}
Um \emph{grupo diferencial}\footnote{Grupos diferenciais são comumente chamados de grupos de Lie, em homenagem ao matemático Sophus Lie.} é uma lista $\bm G = ((G,\atlas),\times,\div,1)$ em que $(G,\atlas)$ é uma variedade diferencial, $(G,\times,\div,1)$ é um grupo e
	\begin{equation*}
	\fun{\times}{G^2}{G} \qquad\text{e}\qquad \fun{\div}{G}{G}
	\end{equation*}
são diferenciáveis\footnote{Assumiremos aqui, como de costume, que o grau de diferenciabilidade é $\Cont^\infty$, mas na maioria dos casos só é necessário $\Cont^2$.}.
\end{definition}

Diferentemente do caso de grupos topológicos, em que é necessário exigir que a operação binária e a inversa do grupo sejam contínuas, para que um grupo com estrutura de variedade seja um grupo diferencial basta que sua operação binária seja diferenciável, não é necessário assumir a diferenciabilidade da inversa $\fun{\div}{G}{G}$. Para demonstrar isso, primeiro deve-se notar que as translações à esquerda e à direita e a conjugação são difeomorfismos se a operação binária é diferenciável. A seguir, a diferencial parcial de $\fun{\times}{G \times G}{G}$ com relação à entrada $0$ é denotada $\D_0$ e a diferencial parcial com relação à entrada $1$ é denotada $\D_1$.

\begin{exercise}
Seja $\bm G = ((G,\atlas),\times,\div,1)$ uma lista em que $(G,\atlas)$ é uma variedade diferencial, $(G,\times,\div,1)$ é um grupo e a operação binária $\fun{\times}{G^2}{G}$ é diferenciável. Então, para todo $g \in G$, as translações à esquerda e à direita $g\times$, $\times g$ e a conjugação $g\con$ são difeomorfismos e
	\begin{enumerate}
	\item $\D_0(\times)|_{(g,g')} = \D (\times g')|_{g}$;
	\item $\D_1(\times)|_{(g,g')} = \D (g \times)|_{g'}$;
	\end{enumerate}
\end{exercise}

%A demonstração pode ser achada em qualquer livro de introdução à teoria de grupos diferenciais.

\begin{proposition}
Seja $\bm G = ((G,\atlas),\times,\div,1)$ uma lista em que $(G,\atlas)$ é uma variedade diferencial, $(G,\times,\div,1)$ é um grupo e a operação binária $\fun{\times}{G^2}{G}$ é diferenciável. Então $\bm G$ é um grupo diferenciável, a inversa $\fun{\div}{G}{G}$ é um difeomorfismo e, para todo $g \in G$,
%	\begin{equation*}
%	\D (\div)|_g = - \D E_{g\inv}|_{1} \circ \D D_{g\inv}|_g.
%	\end{equation*}
	\begin{equation*}
	\D (\div)|_g = - \D (g\inv \times)|_{1} \circ \D (\times g\inv)|_g.
	\end{equation*}
\end{proposition}
\begin{proof}
Para todo $(g,g') \in G \times G$, a diferencial parcial de $\times$ relativa à entrada $1$ é
	\begin{equation*}
	\D_1(\times)|_{(g,g')} = \D (g \times)|_{g'}.
	\end{equation*}
Como $g\times$ é difeomorfismo, $\D (g \times)|_{g'}$ é bijeção, portanto sobrejetiva. Como $g \times g\inv = 1$, pelo teorema da função implícita existem vizinhanças abertas $A \subseteq G$ de $g$ e $A' \subseteq G$ de $g\inv$ e função diferenciável $\fun{f}{A}{A'}$ tal que $g \times f(g) = 1$. Da unicidade da inversa segue que $f=\div$, portanto que $\div$ é diferenciável em $g$, logo diferenciável. Como $\div$ é sua própria função inversa ($\div \circ \div = \Id$), segue que é difeomorfismo.

O teorema da função implícita implica ainda que
	\begin{equation*}
	\D(\div)|_g = - \D_1(\times)|_{(g,g\inv)} \circ \D_0(\times)|_{(g,g\inv)}.
	\end{equation*}
Como $(\D (g \times)|_{g\inv})\inv = \D (g\inv \times)|_{1}$, segue que
	\begin{align*}
	\D(\div)|_g &= - \D_1(\times)|_{(g,g\inv)} \circ \D_0(\times)|_{(g,g\inv)} \\
		&= - (\D (g \times)|_{g\inv})\inv \circ \D (\times g\inv)|_{g} \\
		&= - \D (g\inv \times)|_{1} \circ \D (\times g\inv)|_g.
	\end{align*}
\end{proof}

Em particular, isso implica que $\D(\div)|_1 = \fun{-\Id|_1}{\Tg G|_1}{\Tg G|_1}$, pois $1\inv=1$ e $1\times = \times 1 = \fun{\Id}{G}{G}$, logo
	\begin{equation*}
	\D (\div)|_1 = - \D (1\inv \times)|_{1} \circ \D (\times 1\inv)|_1 = -\Id|_1 \circ \Id|-1 = - \Id|_1.
	\end{equation*}




Denotamos o fibrado tangente de $G$ por $\Tg G$ e a projeção canônica por $\fun{\proj}{\Tg G}{G}$.

\begin{proposition}
Seja $\bm G$ um grupo diferencial. Os fibrados tangente $\Tg G$ e cotangente $\Tg^* G$ de $G$ são triviais.
\end{proposition}
\begin{proof}
\begin{itemize}
\item ($\Tg G$) Basta mostrar que a função
	\begin{align*}
	\func{\Phi}{G \times \Tg G|_1}{\Tg G}{(g,v)}{\D(g \times)|_1 v}
	\end{align*}
é um difeomorfismo. Como, para todo $g \in G$, $\fun{g\times}{G}{G}$ é difeomorfismo, sua diferencial $\fun{\D(g\times)|_1}{\Tg G|_1}{\Tg G|_g}$ é um isomorfismo linear, portanto $\Phi$ é bijeção. Notemos que $\Phi(g,v) = \D_1(\times)|_{(g,1)}v$, portanto é diferenciável. A inversa de $\Phi$ é dada por
	\begin{align*}
		\func{\Phi\inv}{\Tg G}{G \times \Tg G|_1}{v}{(\proj(v),\D(g \times)|_{\proj(v)\inv}v)}
	\end{align*}
que também é diferenciável, portanto $\Phi$ é um difeomorfismo.

\item ($\Tg^* G$) Analogamente, basta mostrar que a função
	\begin{align*}
	\func{\Psi}{G \times \Tg^* G|_1}{\Tg G}{(g,\alpha)}{\D(g \times)|_1^* \alpha}
	\end{align*}
é um difeomorfismo.
\end{itemize}
\end{proof}

\subsection{Álgebras de derivação adjunta}

\subsubsection{Campos vetoriais invariantes}

\begin{definition}
Seja $\bm G$ um grupo diferencial. Um \emph{campo vetorial invariante à direita} em $\bm G$ é um campo vetorial $v \in \tens^1((G,\atlas))$ tal que, para todo $g \in G$,
	\begin{equation*}
	\D(\times g)\emp v = v,
	\end{equation*}
ou seja, para todo $g,g' \in G$,
	\begin{equation*}
	\D(\times g)|_{g'} v|_{g'} = v|_{g'g}.
	\end{equation*}
O conjunto desses campos é denotado $\tens^1_D(\bm G)$.

Um \emph{campo vetorial invariante à esquerda} em $\bm G$ é um campo vetorial $v \in \tens^1((G,\atlas))$ tal que, para todo $g \in G$,
	\begin{equation*}
	\D(g \times)\emp v = v,
	\end{equation*}
ou seja, para todo $g,g' \in G$,
	\begin{equation*}
	\D(g \times)|_{g'} v|_{g'} = v|_{gg'}.
	\end{equation*}
O conjunto desses campos é denotado $\tens^1_E(\bm G)$.
\end{definition}

Na teoria geral, adotaremos campos invariantes à direita, mas definimos campos invariantes à esquerda por completude.

