\chapter{Grupos diferenciais}

\section{Definições básicas}

\begin{definition}
Um \emph{grupo diferencial}\footnote{Grupos diferenciais são comumente chamados de grupos de Lie, em homenagem ao matemático Sophus Lie.} é uma sequência $(\bm G,\times,\div,\id)$ em que $\bm G$ é uma variedade diferencial, $(G,\times,\div,\id)$ é um grupo e a operação
	\begin{equation*}
	\fun{\times}{G^2}{G}
	\end{equation*}
é diferenciável.
\end{definition}

Assumiremos aqui, como de costume, que o grau de diferenciabilidade é $\Cont^\infty$, mas na maioria dos casos só é necessário $\Cont^2$. Diferentemente do caso de grupos contínuos, não é necessário assumir a diferenciabilidade da inversa $\fun{\div}{G}{G}$.

\begin{proposition}
Seja $(\bm G,\times,\div,\id)$ um grupo diferencial. A operação inversa
	\begin{equation*}
		\fun{\div}{G}{G}
	\end{equation*}
é um difeomorfismo.
\end{proposition}