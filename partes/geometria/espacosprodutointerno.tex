\chapter{Espaços lineares com produto interno}

\section{Produto interno}

\begin{definition}
Seja $\bm V$ um espaço linear sobre um corpo $\bm C \subseteq \C$ invariante por conjugação complexa\footnote{Um corpo $\bm C \subseteq \C$ tal que, para todo $c \in C$, temos $\conju{c} \in C$.}. Um \emph{produto interno} em $\bm V$ é uma função $\inte{\var}{\var}\colon V \times V \to C$ que satisfaz
	\begin{enumerate}
	\item (Linearidade na primeira entrada)
		\begin{enumerate}
		\item Para todos $v_0,v_1,v \in V$,
			\begin{equation*}
			\inte{v_0+v_1}{v} = \inte{v_0}{v} + \inte{v_1}{v};
			\end{equation*}
		\item Para todos $v,v' \in V$ e $c \in C$,
			\begin{equation*}
			\inte{cv}{v'} = c\inte{v}{v'};
			\end{equation*}
		\end{enumerate}
	\item (Simetria conjugada) Para todos $v,v' \in V$,
		\begin{equation*}
		\inte{v}{v'} = \conju{\inte{v'}{v}};
		\end{equation*}
	\item (Positividade) Para todo $v \in V$, $\inte{v}{v} \in \intfa{0}{\infty}$;
	\item (Definição positiva) Para todo $v \in V$, se $\inte{v}{v}=0$, então $v=0$.
		\end{enumerate}
\end{definition}

\begin{definition}
Um \emph{espaço com produto interno} é um par $(\bm V,\inte{\var}{\var})$ em que $\bm V$ é um espaço linear sobre um corpo $\bm C \subseteq \C$ invariante por conjugação complexa e $\inte{\var}{\var}\colon \bm V \times \bm V \to C$ é um produto interno em $\bm V$.
\end{definition}

\begin{proposition}[Propriedades de Produto Interno]
Seja $(\bm V,\inte{\var}{\var})$ um espaço com produto interno sobre um corpo $\bm C \subseteq \C$. Então
	\begin{enumerate}
	\item (Linearidade conjugada na segunda entrada)
		\begin{enumerate}
		\item Para todos $v_0,v_1,v \in V$,
			\begin{equation*}
			\inte{v}{v_0+v_1} = \inte{v}{v_0} + \inte{v}{v_1};
			\end{equation*}
		\item Para todos $v,v' \in V$ e $c \in C$,
			\begin{equation*}
			\inte{v'}{cv} = \conju{c}\inte{v'}{v};
			\end{equation*}
		\end{enumerate}
	\item Para todos $v_0,\ldots,v_n,v \in V$ e $c_0,\ldots,c_n \in C$,
		\begin{equation*}
		\inte{\sum_{i=0}^n c_iv_i}{v} = \sum_{i=0}^n c_i\inte{v_i}{v}
		\end{equation*}
e
		\begin{equation*}
		\inte{v}{\sum_{i=0}^n c_iv_i} = \sum_{i=0}^n \conju{c_i}\inte{v}{v_i}.
		\end{equation*}
	\item (Desigualdade de Cauchy-Schwarz) Para todos $v,v' \in V$,
		\begin{equation*}
		\abs{\inte{v}{v'}}^2 \leq \inte{v}{v}\inte{v'}{v'}
		\end{equation*}
		e a igualdade ocorre se, e somente se, um vetor é múltiplo do outro.
		\end{enumerate}
\end{proposition}
\begin{proof}
	\begin{enumerate}
	\item Exercício simples.
	\item Exercício simples.
	\item Se existe $c \in C$ tal que $v'=cv$, então
		\begin{equation*}
		\abs{\inte{v}{v'}}^2 =\abs{\inte{v}{cv}}^2 = \abs{c}^2\abs{\inte{v}{v}}^2 = c\conju{c}\abs{\inte{v}{v}}^2 = \inte{v}{v}\inte{cv}{cv} = \inte{v}{v}\inte{v'}{v'}.
		\end{equation*}
	\end{enumerate}

Caso contrário, se $v'-cv \neq 0$ para todo $c \in C$, então para $c=\frac{\inte{v'}{v}}{\inte{v}{v}}$ segue que
	\begin{align*}
	0 &< \inte{v'-cv}{v'-cv} \\
		&= \inte{v'}{v'} - c\inte{v}{v'} - \conju{c}\inte{v'}{v} + \abs{c}^2\inte{v}{v} \\
		&= \inte{v'}{v'} - c\conju{\inte{v'}{v}} - \conju{c}\inte{v'}{v} + \abs{c}^2\inte{v}{v} \\
		&= \inte{v'}{v'} - \frac{\abs{\inte{v}{v'}}^2}{\inte{v}{v}} - \frac{\abs{\inte{v}{v'}}^2}{\inte{v}{v}} + \frac{\inte{v}{v'}^2}{\inte{v}{v}} \\
		&=  \inte{v'}{v'} - \frac{\abs{\inte{v}{v'}}^2}{\inte{v}{v}},
	\end{align*}
o que implica
	\begin{equation*}
	\abs{\inte{v}{v'}}^2 < \inte{v}{v}\inte{v'}{v'}.
	\end{equation*}
\end{proof}

\begin{proposition}
Sejam $\bm V$ um espaço linear sobre um corpo $\bm C \subseteq \C$ invariante por conjugação complexa e $(b_i)_{i \in I}$ uma base ordenada de $\bm V$. Existe único produto interno $\inte{\var}{\var}$ em $\bm V$ tal que, para todos $i,j \in I$, $\inte{b_i}{b_j}=\delta_{i,j}$.
\end{proposition}

\section{Norma induzida, ortogonalidade e ângulo}

\subsection{Norma}

\begin{definition}
Seja $(\bm V,\inte{\var}{\var})$ um espaço com produto interno sobre um corpo $\bm C \subseteq \C$. A \emph{norma} (induzida pelo produto interno) de $\bm V$ é a função
	\begin{align*}
	\func{\nor{\var}}{V}{\R}{v}{\inte{v}{v}^{\frac{1}{2}}}.
	\end{align*}
\end{definition}

Em termos da norma, a desigualdade de Cauchy-Schwarz fica
	\begin{equation*}
	\abs{\inte{v}{v'}} \leq \nor{v}\nor{v'}.
	\end{equation*}

\begin{proposition}
Seja $(\bm V,\inte{\var}{\var})$ um espaço com produto interno sobre um corpo $\bm C \subseteq \C$. A função $\nor{\var}$ é uma norma em $\bm V$.
\end{proposition}
\begin{proof}
	\begin{enumerate}
	\item (Separação) Seja $v \in V$ tal que $\nor{v}=0$. Então $\inte{v}{v}^\frac{1}{2}=0$, portanto $\inte{v}{v}=0$, o que implica $v=0$.

	\item (Homogeneidade absoluta) Sejam $c \in C$ e $v \in V$. Então
	\begin{equation*}
	\nor{cv} = \inte{cv}{cv}^{\frac{1}{2}} = \left(c\conju{c}\inte{v}{v}\right)^{\frac{1}{2}} = \abs{c}\nor{v}.
	\end{equation*}
	
	\item (Subaditividade) Para todos $v,v' \in V$,
	\begin{align*}
	\nor{v+v'}^2 &= \inte{v+v'}{v+v'} \\
		&= \inte{v}{v}+\inte{v}{v'}+\inte{v'}{v}+\inte{v'}{v'} \\
		&=  \nor{v}^2+\inte{v}{v'}+\conju{\inte{v}{v'}}+\nor{v'}\\
		&= \nor{v}^2+2\Re(\inte{v}{v'})+\nor{v'}\\
		&\leq \nor{v}^2+2\abs{\inte{v}{v'}}+\nor{v'}\\
		&= \nor{v}^2 + 2\nor{v}\nor{v'}+\nor{v'}^2 \\
		&= (\nor{v} + \nor{v'})^2. \qedhere
	\end{align*}
	\end{enumerate}
\end{proof}

\begin{proposition}
Seja $(\bm V,\inte{\var}{\var})$ um espaço com produto interno.
	\begin{enumerate}
	\item (Regra do Paralelogramo) Para todos $v,v' \in V$,
		\begin{equation*}
			2\left( \nor{v}^2 + \nor{v'}^2 \right) = \nor{v+v'}^2+\nor{v-v'}^2;
		\end{equation*}

	\item (Polarização) Para todos $v,v' \in V$, valem\footnote{Se a característica de $\bm C$ é diferente de $2$.}
		\begin{equation*}
		\Re(\inte{v}{v'}) = \frac{1}{4}\left(\nor{v+v'}^2 - \nor{v-v'}^2\right),
		\end{equation*}
		\begin{equation*}
		\Im(\inte{v}{v'}) = \frac{\ii}{4}\left(\nor{v+\ii v'}^2 - \nor{v-\ii v'}^2\right),
		\end{equation*}
e
		\begin{equation*}
		\inte{v}{v'} = \frac{1}{4} \left( \left(\nor{v+v'}^2 - \nor{v-v'}^2\right) + \ii \left(\nor{v+\ii v'}^2 - \nor{v-\ii v'}^2\right) \right).
		\end{equation*}
	\end{enumerate}
\end{proposition}

De certa forma, vale uma recíproca da polarização quando uma norma tem a propriedade de paralelogramo. De fato, a igualdade do paralelogramo pode ser enfraquecida para uma desigualdade.

\begin{proposition}
Seja $(\bm V,\nor{\var})$ um espaço normado tal que, para todos $v,v' \in V$,
	\begin{equation*}
	2\left( \nor{v}^2 + \nor{v'}^2 \right) \leq \nor{v+v'}^2+\nor{v-v'}^2.
	\end{equation*}
Então existe produto interno $\inte{\var}{\var}$ em $\bm V$ tal que $\nor{v}^2=\inte{v}{v}$ para todo $v \in V$.
\end{proposition}
\begin{proof}
O primeiro passo é mostrar que vale a regra do paralelogramo. Segue da desigualdade que
	\begin{equation*}
		2\left( \nor{\frac{v+v'}{2}}^2 + \nor{\frac{v-v'}{2}}^2 \right) \leq \nor{\frac{v+v'}{2} + \frac{v-v'}{2}}^2+\nor{\frac{v+v'}{2} - \frac{v-v'}{2}}^2
	\end{equation*}
e disso segue que
	\begin{align*}
		\nor{v+v'}^2 + \nor{v-v'}^2 &= 4\left( \nor{\frac{v+v'}{2}}^2 + \nor{\frac{v-v'}{2}}^2 \right) \\
			&\leq 2\left( \nor{\frac{v+v'}{2} + \frac{v-v'}{2}}^2+\nor{\frac{v+v'}{2} - \frac{v-v'}{2}}^2 \right) \\
			&= 2(\nor{v}^2 + \nor{v'}^2).
	\end{align*}
Assim, temos a desigualdade oposta e segue a regra do paralelogramo.

Consideremos o caso real. Definimos agora o produto interno por
	\begin{equation*}
	\inte{v}{v'} := \frac{\nor{v+v'}^2 - \nor{v-v'}^2}{4}.
	\end{equation*}

Segue direto da definição que
	\begin{equation*}
		\inte{v}{v} = \frac{\nor{v+v}^2-\nor{v-v}^2}{4} = \frac{\nor{2v}^2}{4} = \nor{v}^2.
	\end{equation*}

	\begin{enumerate}
		\item (Linearidade na primeira entrada)
			\begin{enumerate}
				\item (Aditividade) Sejam $v,v',v'' \in V$. Pela propriedade do paralelogramo,
					\begin{equation*}
						\nor{v+v'+v''}^2 = 2\nor{v+v''}^2 + 2\nor{v'}^2 - \nor{v-v'+v''}^2
					\end{equation*}
					e
					\begin{equation*}
						\nor{v+v'+v''}^2 = 2\nor{v'+v''}^2 + 2\nor{v}^2 - \nor{v'-v+v''}^2.
					\end{equation*}
				Somando as duas igualdades divididas por $2$ segue que
					\begin{align*}
						\nor{v+v'+v''}^2 =& \nor{v}^2 + \nor{v'}^2 + \nor{v+v''}^2 + \nor{v'+v''}^2 \\
							&- \frac{\nor{v-v'+v''}^2 + \nor{v'-v+v''}^2}{2}
					\end{align*}
				e, substituindo $v''$ por $-v''$, obtemos também
					\begin{align*}
						\nor{v+v'-v''}^2 =& \nor{v}^2 + \nor{v'}^2 + \nor{v-v''}^2 + \nor{v'-v''}^2 \\
							&- \frac{\nor{v-v'-v''}^2 + \nor{v'-v-v''}^2}{2}.
					\end{align*}
				Como
					\begin{align*}
						\nor{v-v'+v''}^2 + \nor{v'-v+v''}^2 &= \nor{-(v'-v-v'')}^2 + \nor{-(v-v'-v'')}^2 \\
							&= \nor{v-v'-v''}^2 + \nor{v'-v-v''}^2,
					\end{align*}
				segue que
					\begin{align*}
						\inte{v+v'}{v''} &= \frac{\nor{v+v'+v''}^2 - \nor{v+v'-v''}^2}{4} \\
							&= \frac{\nor{v+v''}^2 + \nor{v'+v''}^2 - \nor{v-v''}^2 - \nor{v'-v''}^2}{4} \\
							&= \frac{\nor{v+v''}^2 - \nor{v-v''}^2}{4} + \frac{\nor{v'+v''}^2 - \nor{v'-v''}^2}{4} \\
							&= \inte{v}{v''} + \inte{v'}{v''}.
					\end{align*}

				\item (Homogeneidade) Sejam $v,v' \in V$. A homogeneidade vale para $-1$, pois
					\begin{align*}
						\inte{-v}{v} = \frac{\nor{-v+v'}^2 - \nor{-v-v'}^2}{4} = \frac{\nor{v'-v}^2 - \nor{v'+v}^2}{4} = -\inte{v}{v'}.
					\end{align*}
				
				 A homogeneidade vale para números naturais. Mostremos por indução. Para $n=0$, basta notar que
					\begin{equation*}
						\inte{0v}{v'} = \frac{\nor{0v+v'}^2 - \nor{0v-v'}^2}{4} = \frac{\nor{v'}^2 - \nor{-v'}^2}{4} = 0 = 0\inte{v}{v'}.
					\end{equation*}
				Supondo que vale para $n$, segue da aditividade que
					\begin{align*}
						\inte{(n+1)v}{v'} &= \inte{nv+v}{v'} \\
							&= \inte{nv}{v'} + \inte{v}{v'} \\
							&= n\inte{v}{v'} + \inte{v}{v'} \\
							&= (n+1)\inte{v}{v'}.
					\end{align*}
				
				A homogeneidade vale para inteiros negativos, pois segue da homogeneidade para $-1$ que
					\begin{align*}
						\inte{(-n)v}{v'} = \inte{-(nv)}{n'} = -\inte{nv}{v'} = -n\inte{v}{v'}.
					\end{align*}
				
				A homogeneidade vale para números racionais. Seja $\frac{n}{d} \in \Q$. Segue da homogeneidade para números inteiros que
					\begin{align*}
						\inte{\frac{n}{d}v}{v'} = \frac{d}{d}\inte{\frac{n}{d}v}{v'} = \frac{n}{d}\inte{\frac{d}{d}v}{v'} = \frac{n}{d}\inte{v}{v'}.
					\end{align*}

				Por fim, notemos que $\inte{\var}{\var}$ é contínua, pois é composição de adição, subtração, multiplicação por escalar e norma, que são contínuas. Sendo assim, podemos mostrar que a homogeneidade vale para números reais. Sejam $c \in \R$ e $(c_n)_{n \in \N}$ uma sequência de números racionais tal que $c = \lim_{n \to \infty} c_n$. Da continuidade de $\inte{\var}{\var}$ e da homogeneidade para números racionais, segue que
					\begin{equation*}
						\inte{cv}{v'} = \inte{\lim_{n \to \infty} c_n v}{v'} = \lim_{n \to \infty} c_n \inte{v}{v'} = c \inte{v}{v'}.
					\end{equation*}
			\end{enumerate}

		\item (Simetria) Sejam $v,v' \in V$. Segue direto da simetria da adição que
			\begin{equation*}
				\inte{v}{v'} = \frac{\nor{v+v'}^2 - \nor{v-v'}^2}{4} = \frac{\nor{v'+v}^2 - \nor{v'-v}^2}{4} = \inte{v'}{v}
			\end{equation*}

		\item (Positividade) Seja $v \in V$. Segue da positividade da norma que
			\begin{equation*}
				\inte{v}{v} = \nor{v}^2 \geq 0.
			\end{equation*}

		\item (Definição positiva) Seja $v \in V$ tal que $\inte{v}{v}=0$. Então
			\begin{equation*}
				0 = \inte{v}{v} = \nor{v}^2,
			\end{equation*}
		e segue da separação da norma que $v=0$.
	\end{enumerate}

Para o caso complexo, definimos
	\begin{equation*}
	\inte{v}{v'} := \frac{1}{4} \sum_{k=0}^{3} \ii^k \nor{v+\ii^kv'}^2 = \frac{\left(\nor{v+v'}^2 - \nor{v-v'}^2\right) + \ii \left(\nor{v+\ii v'}^2 - \nor{v-\ii v'}^2\right)}{4}.
	\end{equation*}
A demonstração fica como exercício\footnote{Os detalhes podem ser conferidos em \href{https://math.stackexchange.com/questions/21792/norms-induced-by-inner-products-and-the-parallelogram-law}{https://math.stackexchange.com/questions/21792/norms-induced-by-inner-products-and-the-parallelogram-law}}.
\end{proof}

A identidade do caso complexo se reduz à do caso real quando $v,v'$ são reais.

\subsection{Perpendicularidade e paralelismo}

\begin{definition}
Seja $(\bm V,\inte{\var}{\var})$ um espaço com produto interno sobre um corpo $\bm C \subseteq \C$. Vetores \emph{paralelos} são vetores $v,v' \in V$ para os quais existe $c \in C \setminus \{0\}$ satisfazendo $v'=cv$. Denota-se $v \parallel v'$.

Vetores \emph{perpendiculares} (ou \emph{ortogonais}) são vetores $v,v' \in V$ que satisfazem $\inte{v}{v'}=0$. Denota-se $v \perp v'$.

Um conjunto \emph{perpendicular} (ou \emph{ortogonal}) é um conjunto $U \subseteq V$ tal que, para todos $u,u' \in U$, $u \perp u'$, e um conjunto \emph{ortonormal} é um conjunto ortogonal $U$ tal que, para todo $u \in U$, $\nor{v}=1$. O \emph{complemento perpendicular} (ou \emph{ortogonal}) de um conjunto $U \subseteq V$ é o conjunto
	\begin{equation*}
	U^\perp := \set{v \in V}{\forall_{u \in U}\ v \perp u}.
	\end{equation*}
\end{definition}

\begin{proposition}[Propriedades de $\parallel$ e $\perp$]
Seja $(\bm V,\inte{\var}{\var})$ um espaço com produto interno sobre um corpo $\bm C \subseteq \C$.
	\begin{enumerate}
	\item A relação de paralelismo $\parallel$ é uma equivalência;
	\item A relação de perpendicularidade $\perp$ é simétrica;
	\item Para todos $u,v,v' \in V \setminus \{0\}$, se $v \parallel v'$ e $v \perp u$, então $v' \perp u$.
	\item Para todos $v,v' \in V \setminus \{0\}$, $v \parallel v'$ se, e somente se, $\{v,v'\}$ é linearmente dependente.
	\item Para todos $v,v' \in V \setminus \{0\}$, se $v \perp v'$, então $\{v,v'\}$ é linearmente independente.
	\item Para todo $v \in V$, se $v \parallel 0$ então $v=0$; para todo $v \in V$, $v \perp 0$.
	\end{enumerate}
\end{proposition}

\begin{proposition}[Propriedades de complemento perpendicular]
Seja $(\bm V,\inte{\var}{\var})$ um espaço com produto interno sobre um corpo $\bm C \subseteq \C$.
	\begin{enumerate}
	\item Para todo $U \subseteq V$, $U^\perp$ é um subespaço linear de $V$.
	\item $V^\perp = \{0\}$.
	\item Para todo subespaço linear $U \subseteq V$, $(U^\perp)^\perp = U$.
	\end{enumerate}
% e
%	\begin{equation*}
%	V = \ger{U} + U^\perp.
%	\end{equation*}
\end{proposition}
\begin{proof}
Primeiro notamos que $0 \in U^\perp$, pois para todo $u \in U$, $0 \perp u$. Segundo, sejam $v,v' \in U^\perp$ e $c \in C$. Então, para todo $u \in U$, $\inte{v}{u} = \inte{v'}{u} = 0$, logo
	\begin{equation*}
	\inte{cv+v'}{u} = c\inte{v}{u} + \inte{v'}{u} = 0,
	\end{equation*}
o que mostra que $cv+v' \in U^\perp$.
\end{proof}

\begin{definition}
Seja $(\bm V,\inte{\var}{\var})$ um espaço com produto interno. \emph{Subespaços perpendiculares} em $\bm V$ são subespaços lineares $U,U' \subseteq V$ tais, para todos $u \in U$ e $u' \in U'$, $u \perp u'$. Denota-se $U \perp U'$.
\end{definition}

\subsection{Projeções paralela e perpendicular}

\begin{definition}
Sejam $(\bm V,\inte{\var}{\var})$ um espaço com produto interno sobre um corpo $\bm C \subseteq \C$ e $u \in V$. A \emph{projeção paralela}\footnote{Essa projeção é conhecida como \emph{projeção ortogonal} de $V$ sobre $u$, mas aqui adotaremos as nomenclatura de paralela, já que definimos também a projeção perpendicular, e essa pode ser confundida com a ortogonal.} de $V$ sobre $u$ é
	\begin{align*}
	\func{\proj_{\parallel u}}{V}{V}{v}{
		\begin{cases}
			\displaystyle\frac{\inte{v}{u}}{\nor{u}^2}u,& u \neq 0 \\
			0,& u=0.
		\end{cases}
	}
	\end{align*}
A \emph{projeção perpendicular} de $V$ sobre $u$ é
	\begin{align*}
	\func{\proj_{\perp u}}{V}{V}{v}{v - \proj_{\parallel u}(v).}
	\end{align*}
\end{definition}

\begin{proposition}[Propriedades das projeções]
Seja $(\bm V,\inte{\var}{\var})$ um espaço com produto interno sobre um corpo $\bm C \subseteq \C$.
	\begin{enumerate}
	\item $\proj_{\parallel 0} = 0$ e $\proj_{\perp 0} = \Id$;
	\item Para todo $u \in V$, as projeções $\proj_{\parallel u}\colon V \to V$ e $\proj_{\perp u}\colon V \to V$ são projeções lineares.
	\item Para todos $u, v \in V$,
		\begin{enumerate}
		\item se $u \not\perp v$, então $\proj_{\parallel u}(v) \parallel u$;
		\item $\proj_{\perp u}(v) \perp u$;
		\item se $v \parallel u$, então $\proj_{\parallel u}(v) = v$ (ou seja, $\proj_{\parallel u}|_{\ger{u}} = \Id$);
		\item se $v \perp u$, então $\proj_{\perp u}(v) = v$ (ou seja, $\proj_{\perp u}|_{\ger{u}^\perp} = \Id$);
		\end{enumerate}
	\item Para todos $u,v \in V \setminus \{0\}$, $\{u,v\}$ é linearmente independente se, e somente se, $\proj_{\perp u}(v) \neq 0$.
	\end{enumerate}
\end{proposition}
\begin{proof}
	\begin{enumerate}
	\item Direto da definição.
	
	\item O caso em que $u = 0$ é consequência do item anterior, pois $0$ e $\Id$ são lineares e idempotentes. Consideremos $u \in V \setminus \{0\}$.

	(Linearidade) Sejam $v,v' \in V$ e $c \in C$.
	\begin{equation*}
	\proj_{\parallel u}(cv+v') = \frac{\inte{cv+v'}{u}}{\nor{u}^2}u = \frac{c\inte{v}{u} + \inte{v'}{u}}{\nor{u}^2}u = c\proj_{\parallel u}(v) + \proj_{\parallel u}(v').
	\end{equation*}	
	
(Idempotência) Seja $v \in V$.
		\begin{equation*}
		\proj_{\parallel u}\left( \proj_{\parallel u}(v) \right) = \frac{\inte{\frac{\inte{v}{u}}{\nor{u}^2}u}{u}}{\nor{u}^2}u = \frac{\inte{v}{u}}{\nor{u}^2}\frac{\inte{u}{u}}{\nor{u}^2}u = \frac{\inte{v}{u}}{\nor{u}^2}u = \proj_{\parallel u}(v).
		\end{equation*}

	Segue direto da definição $\proj_{\perp u} = \Id-\proj_{\parallel u}$ e de $\proj_{\parallel u}$ ser projeção linear.

%%%%%%%%%%%%%%%%%%%%%%%%%%%%%%%%%%%%%%%%%%%%
\begin{comment}		
(Linearidade) Sejam $v,v' \in V$ e $c \in C$.
	\begin{equation*}
	\proj_{\perp u}(cv+v') = cv+v' - \proj_{\parallel u}(cv+v') = cv+v' - c\proj_{\parallel u}(v) + \proj_{\parallel u}(v') = c\proj_{\perp u}(v) + \proj_{\perp u}(v').
	\end{equation*}

(Idempotência) Seja $v \in V$.
	\begin{align*}
	\proj_{\perp u}\left( \proj_{\perp u}(v) \right) &= v - \proj_{\parallel u}(v) - \proj_{\parallel u}\left( v - \proj_{\parallel u}(v) \right) \\
		&= v - \proj_{\parallel u}(v) - \proj_{\parallel u}(v) + \proj_{\parallel u}\left( \proj_{\parallel u}(v) \right) \\
		&= v - \proj_{\parallel u}(v) - \proj_{\parallel u}(v) + \proj_{\parallel u}(v) \\
		&= v - \proj_{\parallel u}(v) \\
		&= \proj_{\perp u}(v).
	\end{align*}

\end{comment}
%%%%%%%%%%%%%%%%%%%%%%%%%%%%%%%%%%%%%%%%%%%%


	\item	
		\begin{enumerate}
		\item Se $u \not\perp v$, $\inte{u}{v} \neq 0$, o que implica que $u \neq 0$ e $\frac{\inte{v}{u}}{\nor{u}^2} \neq 0$. Como por definição $\proj_{\parallel u}(v) = \frac{\inte{v}{u}}{\nor{u}^2}u$, segue que $\proj_{\parallel u}(v) \parallel u$.
		
		\item Se $u = 0$, $\proj_{\perp u}(v) \perp u$. Se $u \neq 0$,
	\begin{align*}
	\inte{\proj_{\parallel u}(v)}{u} = \inte{\frac{\inte{v}{u}}{\nor{u}^2}u}{u} = \frac{\inte{v}{u}}{\nor{u}^2}\inte{u}{u} = \inte{v}{u},
	\end{align*}
portanto
	\begin{align*}
	\inte{\proj_{\perp u}(v)}{u} = \inte{v - \proj_{\parallel u}(v)}{u} = \inte{v}{u} - \inte{\proj_{\parallel u}(v)}{u} = 0,
	\end{align*}
o que mostra que $\proj_{\perp u}(v) \perp u$.
		
		\item Se $v \parallel u$, existe $c \in C \setminus \{0\}$ tal que $v = cu$. Se $u = 0$, então $v=0$, logo $\proj_{\parallel u}(v) = \proj_{\parallel 0}(0) = 0 = v$. Se $u \neq 0$, então
		\begin{equation*}
		\proj_{\parallel u}(v) = \frac{\inte{v}{u}}{\nor{u}^2}u = \frac{\inte{cu}{u}}{\nor{u}^2}u = cu = v.
		\end{equation*}
		
		\item Se $v \perp u$, então $\inte{u}{v}=0$. Se $u=0$, então $\proj_{\perp u} = \Id$, logo $\proj_{\perp u}(v) = v$. Se $u \neq 0$,
	\begin{equation*}
	\proj_{\perp u}(v) = v - \frac{\inte{v}{u}}{\nor{u}^2} = v.
	\end{equation*}
		\end{enumerate}
	\end{enumerate}
\end{proof}

Todo $v$ pode ser decomposto como $v = \proj_{\parallel u}(v) + \proj_{\perp u}(v)$.

%%%%%%%%%%%%%%%%%%%%%%%%%%%%%%%%%%%%%%%%%%%%
\begin{comment}
Podemos também decompor um vetor $v$ com relação a um conjunto finito $U$ como
	\begin{equation*}
	v = \proj_{\parallel U}(v) + \proj_{\perp U}(v).
	\end{equation*}
em que
	\begin{align*}
	\func{\proj_{\parallel U}}{V}{V}{v}{\sum_{u \in U}\proj_{\parallel u}(v)}
	\end{align*}
	\begin{align*}
	\func{\proj_{\perp U}}{V}{V}{v}{v-\proj_{\parallel U}(v)}.
	\end{align*}

Isso é o mesmo que dizer que
	\begin{equation*}
	\proj_{\parallel U} = \sum_{u \in U} \proj_{\parallel u}
	\end{equation*}
e
	\begin{equation*}
	\proj_{\perp U} = \Id - \proj_{\parallel U}.
	\end{equation*}

Notemos que, para todo $v \in V$ e todo $u \in U$,
	\begin{equation*}
	\proj_{\perp U}(v) \perp u
	\end{equation*}
pois
	\begin{align*}
	\inte{\proj_{\perp U} (v)}{u} &= \inte{v-\proj_{\parallel U}(v)}{u} \\
		&= \inte{v - \sum_{u \in U}\proj_{\parallel u}(v)}{u} \\
		&= \inte{v}{u} - \sum_{u \in U} \inte{\proj_{\parallel u}(v)}{u} \\
	\end{align*}

\end{comment}
%%%%%%%%%%%%%%%%%%%%%%%%%%%%%%%%%%%%%%%%%%%%


\subsubsection{Processo de ortogonalização}

Seja $(\bm V,\inte{\var}{\var})$ um espaço com produto interno sobre um corpo $\bm C \subseteq \C$ e $B = (v_i)_{i \in [d]}$ uma base finita de $\bm V$. A base $(u_i)_{i \in [d]}$, definida recursivamente por
	\begin{equation*}
	u_i := \proj_{\perp u_{i-1}} \circ \cdots \circ \proj_{\perp u_0} (v_i) =  v_i - \sum_{j \in [i]} \proj_{\parallel u_j} (v_i)
	\end{equation*}
é uma base ortogonal (perpendicular) de $\bm V$. A base $(e_i)_{i \in [d]}$ definida por $e_i := \frac{u_i}{\nor{u_i}}$ é uma base ortonormal (ou perpendicular unitária) de $\bm V$.


\subsection{Projeções ortogonais}

Lembremos que, para toda projeção linear $p\colon V \to V$, vale que $(\Id-p)$ é projeção linear, $p\inv(0) = (\Id-p)(V)$ e
	\begin{equation*}
	V = p(V) \oplus (\Id-p)(V).
	\end{equation*}

\begin{definition}
Seja $(\bm V, \inte{\var}{\var})$ um espaço com produto interno. Uma \emph{projeção ortogonal} em $\bm V$ é uma projeção linear $p\colon V \to V$ tal que $p(V) \perp (\Id-p)(V)$.
\end{definition}

\begin{proposition}
Sejam $(\bm V, \inte{\var}{\var})$ um espaço com produto interno (completo) e $p\colon V \to V$ uma projeção ortogonal em $\bm V$.
	\begin{enumerate}
	\item Para todos $v,v' \in V$,
		\begin{equation*}
		\inte{p(v)}{v' - p(v')} = \inte{v-p(v)}{p(v')} = 0;
		\end{equation*}
	
	\item Para todos $v,v' \in V$,
		\begin{equation*}
		\inte{p(v)}{v'} = \inte{p(v)}{p(v')} = \inte{v}{p(v')};
		\end{equation*}
	\end{enumerate}
\end{proposition}

\begin{proposition}
Sejam $(\bm V, \inte{\var}{\var})$ um espaço com produto interno completo e $U \subseteq V$ um subespaço linear fechado. Existe projeção ortogonal $p\colon V \to V$ tal que $p(V)=U$.
\end{proposition}
\begin{proof}
Como $U$ é fechado, é um subespaço completo. Seja $v \in V$. O conjunto
	\begin{equation*}
	N := \set{\nor{v-u}}{u \in U}
	\end{equation*}
tem ínfimo e, como $U$ é completo, $N$ tem mínimo. Definimos $p(v)$ como o ponto $u \in U$ tal que $\nor{v-u} = \min N$. %Mostrar que é único... deve seguir da completude, usando limite.
Mostremos que $p$ é uma projeção ortogonal.

(Idempotência) Seja $v \in V$. Então, como $p(v) \in U$, segue que $p(p(v)) = p(v) \in U$, pois $\nor{p(v) - p(v)}=0$, logo $p^2=p$.

(Ortogonalidade) Sejam $u \in U=p(V)$ e $v \in (\Id-p)(V)$. Se $u=0$, então $\inte{v}{u} = \inte{v}{0}=0$. Suponhamos $u \neq 0$. Temos que
	\begin{align*}
	\nor{v - \proj_{\parallel u}(v)}^2 &= \nor{v}^2 - 2\inte{v}{\proj_{\parallel}(v) u} + \nor{\proj_{\parallel u}(v)}^2 \\
		&= \nor{v}^2 - 2\frac{\inte{v}{u}}{\nor{u}^2}\inte{v}{u} + \frac{\inte{v}{u}^2}{\nor{u}^4}\nor{u}^2 \\
		&=  \nor{v}^2 - 2\frac{\inte{v}{u}^2}{\nor{u}^2} + \frac{\inte{v}{u}^2}{\nor{u}^2} \\
		&= \nor{v}^2 - \frac{\inte{v}{u}^2}{\nor{u}^2}.
	\end{align*}

Seja $v' \in V$ tal que $(\Id-p)(v')=v$. Então $(p(v') + \proj_{\parallel u}(v)) \in U$, logo pela minimalidade de $p(v')$ segue que
	\begin{align*}
	\nor{v}^2 &= \nor{v'-p(v')}^2 \\
		&\leq \nor{v'-(p(v') + \proj_{\parallel u}(v))}^2 \\
		&= \nor{v - \proj_{\parallel u}(v)}^2 \\
		&= \nor{v}^2 - \frac{\inte{v}{u}^2}{\nor{u}^2},
	\end{align*}
logo $\frac{\inte{v}{u}^2}{\nor{u}^2} = 0$, o que é equivalente a $\inte{v}{u}=0$.

(Linearidade) Sejam $v,v' \in V$ e $c \in C$. Então $(cv+v')-p(cv+v')$, $cv-p(cv)$ e $v'-p(v') \in (\Id-p)(V)$. Da ortogonalidade de $p$, para todo $u \in U=p(V)$ temos
	\begin{equation*}
	0 = \inte{(cv+v')-p(cv+v')}{u} = \inte{cv-cp(v)}{u} = \inte{v'-p(v')}{u},
	\end{equation*}
portanto
	\begin{align*}
	0 &= \inte{(cv+v')-p(cv+v')}{u} - \inte{cv-cp(v)}{u} - \inte{v'-p(v')}{u} \\
		&= \inte{cp(v) + p(v') - p(cv+v')}{u}.
	\end{align*}
Tomando $u=cp(v) + p(v') - p(cv+v')$, temos que $u \in U$ % PORQUE?
e então
 	\begin{equation*}
 	0 = \inte{p(cv) + p(v') - p(cv+v')}{cp(v) + p(v') - p(cv+v')},
 	\end{equation*}
o que implica $p(cv+v') = cp(v) + p(v')$.
\end{proof}

\begin{proposition}
Seja $(\bm V, \inte{\var}{\var})$ um espaço com produto interno. Toda projeção ortogonal $p\colon V \to V$ em $\bm V$ é contínua.
\end{proposition}
\begin{proof}
Basta notar que, pela desigualdade do produto interno, para todo $v \in V$,
	\begin{equation*}
	\nor{p(v)}^2 = \inte{p(v)}{p(v)} = \inte{p(v)}{v} \leq \nor{p(v)}\nor{v},
	\end{equation*}
logo $\nor{p(v)} \leq \nor{v}$. Isso mostra que $p$ é limitada, o que é equivalente a ser contínua.
\end{proof}




\subsection{Ângulo}

Definiremos agora a noção de ângulo induzida pelo produto interno. Pela desiguadade de Cauchy-Schwarz, sabemos que, para todos $v,v' \in V$,
	\begin{equation*}
	\abs{\inte{v}{v'}} \leq \nor{v}\nor{v'}.
	\end{equation*}
Disso segue que, para todos $v,v' \in V\setminus\{0\}$,
	\begin{equation*}
	\frac{\abs{\inte{v}{v'}}}{\nor{v}\nor{v'}} \leq 1.
	\end{equation*}
Se o produto interno for real, então
	\begin{equation*}
	-1 \leq \frac{\inte{v}{v'}}{\nor{v}\nor{v'}} \leq 1.
	\end{equation*}
Isso significa que a função $\cos\inv\colon \intff{-1}{1} \to \intff{0}{\tau \div 2}$ está definida para esses valores, portanto podemos definir a função ângulo como a seguir. No caso em que o produto interno não é reaal, ainda se pode definir o ângulo considerando o valor de $\cos\inv$ para $\frac{\abs{\inte{v}{v'}}}{\nor{v}\nor{v'}}\in \intff{0}{1}$, e com imagem $\intff{0}{\tau \div 4}$, mas não estudaremos esse caso aqui.

%%%%%%%%%%%%%%%%%%%%%%%%%%%%%%%%%%%%%%%%%%%%%%%%
\begin{comment}

\begin{figure}
\centering
\begin{tikzpicture}[scale=2]
	\draw (0,-1) node[anchor=east] {$-1$} -- (0,0) node[anchor=east] {$0$} -- (0,1) node[anchor=east] {$1$};
	\draw (0,0) -- (pi,0) node[anchor=west] {$\displaystyle\frac{\tau}{2}$};
	\draw[dotted] (0,1) -- (pi,1) -- (pi,-1) -- (0,-1);
%	\draw plot [domain=0:pi,smooth] ({\x},{cos(\x r)}) node[right] {$\cos$};
	\draw plot [domain=0:pi,smooth] (\x,{cos(\x r)});
\end{tikzpicture}
\caption{Gráfico da função $\cos\colon \intff{0}{\frac{\tau}{2}} \to \intff{-1}{1}$.}
\label{fig:cosseno}
\end{figure}

\end{comment}
%%%%%%%%%%%%%%%%%%%%%%%%%%%%%%%%%%%%%%%%%%%%%%%%

\begin{figure}
\centering
\begin{tikzpicture}[scale=2]
	\draw (-1,0) node[anchor=north] {$-1$} -- (0,0) node[anchor=north] {$0$} -- (1,0) node[anchor=north] {$1$};
	\draw (0,0) -- (0,pi/2) node[anchor=west] {$\tau \div 4$} -- (0,pi) node[anchor=west] {$\tau \div 2$};
	\draw[dotted] (1,0) -- (1,pi) -- (-1,pi) -- (-1,0);
	\draw plot [domain=0:pi,smooth] ({cos(\x r)},\x);
\end{tikzpicture}
\caption{Gráfico da função $\cos\inv\colon \intff{-1}{1} \to \intff{0}{\frac{\tau}{2}}$.}
\label{fig:cossenoinv}
\end{figure}

\begin{definition}
Sejam $(\bm V,\inte{\var}{\var})$ um espaço com produto interno real e $v,v' \in V \setminus \{0\}$. O \emph{ângulo} entre $v$ e $v'$ é
	\begin{equation*}
	\ang{v}{v'} := \cos\inv\left(\frac{\inte{v}{v'}}{\nor{v}\nor{v'}}\right).
	\end{equation*}
A função \emph{ângulo} de $\bm V$ é a função
	\begin{align*}
	\func{\angf}{V\setminus\{0\} \times V\setminus\{0\}}{\intff{0}{\frac{\tau}{2}}}{(v,v')}{\ang{v}{v'}}.
	\end{align*}
\end{definition}

\begin{figure}
\centering
\begin{tikzpicture}[scale=3]
	\draw[<->] (0.866025404,-0.5) node[anchor=north] {$v'$} -- (0,0) -- (0.866025404,0.5) node[anchor=south] {$v$};
	\draw (0.288675135,-0.166666667) arc (-30:0:1/3) node[anchor=west] {$\ang{v}{v'}$} arc (0:30:1/3);
\end{tikzpicture}
\caption{Representação de vetores e o ângulo entre eles.}
\label{fig:angulovetores}
\end{figure}

\begin{proposition}[Propriedades de Ângulo]
Seja $(\bm V,\inte{\var}{\var})$ um espaço com produto interno real.
	\begin{enumerate}
	\item Para todos $v,v' \in V \setminus \{0\}$,
		\begin{equation*}
		\inte{v}{v'} = \nor{v}\nor{v'}\cos(\ang{v}{v'});
		\end{equation*}
	\item Para todos $v,v' \in V \setminus \{0\}$ e $cc' \in \R \setminus \{0\}$,
	\begin{equation*}
	\ang{cv}{c'v'} = \begin{cases}
		\ang{v}{v'},& cc'>0 \\
		\displaystyle\frac{\tau}{2} - \ang{v}{v'},& cc'<0;
	\end{cases}
	\end{equation*}
	
	\item Para todos $v,v' \in V \setminus \{0\}$,
	\begin{equation*}
	v \parallel v' \sse \ang{v}{v'} \in \{0,\tau \div 2\}.
	\end{equation*}
Se existe $c \in \intaa{0}{\infty}$ tal que $v' =cv$, então $\ang{v}{v'} = 0$, e se existe $c \in \intaa{-\infty}{0}$ tal que $v' =cv$, então $\ang{v}{v'} = \frac{\tau}{2}$;
	
	\item Para todos $v,v' \in V \setminus \{0\}$,
	\begin{equation*}
	v \perp  v' \sse \ang{v}{v'} = \frac{\tau}{4}.
	\end{equation*}
	\end{enumerate}
\end{proposition}
\begin{proof}
	\begin{enumerate}
	\item Segue diretamente da definição.
	\item Da igualdade
	\begin{equation*}
	\ang{v}{v'} = \cos\inv\left(\frac{\inte{v}{v'}}{\nor{v}\nor{v'}}\right)
	\end{equation*}
segue que, para todos $c,c' \in \R\setminus\{0\}$
	\begin{align*}
	\ang{cv}{c'v'} &= \cos\inv\left(\frac{\inte{cv}{c'v'}}{\nor{cv}\nor{c'v'}}\right) \\
		&= \cos\inv\left(\frac{cc'\inte{v}{v'}}{\abs{c}\abs{c'}\nor{v}\nor{v'}}\right) \\
		&= \cos\inv\left(\frac{cc'}{\abs{cc'}}\frac{\inte{v}{v'}}{\nor{v}\nor{v'}}\right).
	\end{align*}
Nesse caso, se $cc' > 0$, então
	\begin{equation*}
	\ang{cv}{c'v'} = \cos\inv\left(1\frac{\inte{v}{v'}}{\nor{v}\nor{v'}}\right) = \ang{v}{v'},
	\end{equation*}
e, se $cc' < 0$, então
	\begin{equation*}
	\ang{cv}{c'v'} = \cos\inv\left(-1\frac{\inte{v}{v'}}{\nor{v}\nor{v'}}\right) = \frac{\tau}{2} - \ang{v}{v'}.
	\end{equation*}
	
	\item Notemos que, para todo $v \in V \setminus \{0\}$,
		\begin{equation*}
			\ang{v}{v} = \cos\inv\left(\frac{\inte{v}{v}}{\nor{v}\nor{v}}\right) = \cos\inv(1) = 0.
		\end{equation*}
Notemos também que $\cos\inv\colon \intff{-1}{1} \to \intff{0}{\frac{\tau}{2}}$ é uma bijeção tal que $\cos\inv(1)=0$ e $\cos\inv(-1) = \frac{\tau}{2}$.

Suponhamos agora que existe $c \in \R \setminus \{0\}$ tal que $v'=cv$. Nesse caso, pelo item anterior segue que, relacionando $\ang{v}{v'}$ com $\ang{v}{v}$,
		\begin{equation*}
		\ang{v}{v'} = \ang{v}{cv} = \begin{cases}
		0,& c>0 \\
		\displaystyle\frac{\tau}{2},& c<0.
	\end{cases}
		\end{equation*}

Reciprocamente, suponhamos que $\ang{v}{v'}=0$. Da bijetividade de $\cos\inv$ segue que $\frac{\inte{v}{v'}}{\nor{v}\nor{v'}} = 1$, logo $\abs{\inte{v}{v'}} = \nor{v}\nor{v'}$ e pela desigualdade de Cauchy-Schwarz existe $c \in \R \setminus \{0\}$ tal que $v'=cv$. Como $0 = \ang{v}{v'} = \ang{v}{cv}$, $c \in \intaa{0}{\infty}$. Suponhamos então que $\ang{v}{v'}=\frac{\tau}{2}$. Da bijetividade de $\cos\inv$ segue que $\frac{\inte{v}{v'}}{\nor{v}\nor{v'}} = -1$, logo $\abs{\inte{v}{v'}} = \nor{v}\nor{v'}$ e pela desigualdade de Cauchy-Schwarz existe $c \in \R \setminus \{0\}$ tal que $v'=cv$. Como $\frac{\tau}{2} = \ang{v}{v'} = \ang{v}{cv}$, $c \in \intaa{-\infty}{0}$.

	\item Segue diretamente da bijetividade de $\cos\inv$, pois $\ang{v}{v'} = \frac{\tau}{4}$ se, e somente se, $\frac{\inte{v}{v}}{\nor{v}\nor{v}}=0$, o que ocorre se, e somente se, $\inte{v}{v'}=0$. \qedhere
\end{enumerate}
\end{proof}

%Se quocientamos o intervalo $\intff{0}{\frac{\tau}{2}}$ identificando suas extremidades de modo a obter $\frac{\tau}{2}\T^1$, temos que
%	\begin{equation*}
%	\ang{cv}{v'} = \pm_c\ang{v}{v'},
%	\end{equation*}
%pois em $\frac{\tau}{2}\T^1$ vale a igualdade $\frac{\tau}{2} - \ang{v}{v'} = - \ang{v}{v'}$. Isso é uma curiosidade interessante.



\subsubsection{Definição de uma função ângulo}

Tentamos, nesta seção, definir uma função ângulo em um espaço normado de forma que possamos, a partir dela e da norma, definir um produto interno. A investigação consiste em destacar as propriedades que uma função ângulo deve satisfazer. A ideia é que ela seja uma função invariante pela semirreta positiva gerada por um vetor, ou seja, que não mude de valor se multiplicarmos um vetor por um número positivo. Além disso, a função deve também ser uma distância nesse espaço de semirretas. Assumiremos aqui que o espaço normado é real.

\paragraph{Primeira definição: uma definição ad hoc}

\begin{definition}
Seja $(\bm V, \nor{\var})$ um espaço normado. Uma \emph{função ângulo} em $(\bm V, \nor{\var})$ é uma função contínua $\fun{\angf}{V \setminus \{0\} \times V \setminus \{0\}}{\intff{0}{\tau \div 2}}$ tal que
	\begin{enumerate}
		\item (Nulidade) Para todos $v \in V \setminus \{0\}$,
			\begin{equation*}
				\ang{v}{v} = 0;
			\end{equation*}
		\item (Simetria) Para todos $v,v' \in V \setminus \{0\}$,
			\begin{equation*}
				\ang{v}{v'} = \ang{v'}{v};
			\end{equation*}
		\item (Compatibilidade) Para todos $v,v',v'' \in V \setminus \{0\}$ tais que $v' \neq -v$,
			\begin{equation*}
				\nor{v+v'}\cos(\ang{v+v'}{v''}) = \nor{v}\cos(\ang{v}{v''}) + \nor{v'}\cos(\ang{v'}{v''});
			\end{equation*}
	\end{enumerate}
Um \emph{espaço angulado} é uma tripla $(\bm V, \nor{\var},\angf)$ em que $(\bm V, \nor{\var})$ é um espaço normado e $\fun{\angf}{V \setminus \{0\} \times V \setminus \{0\}}{\intff{0}{\tau \div 2}}$ uma função ângulo em $(\bm V, \nor{\var})$.
\end{definition}

\begin{proposition}
Seja $(\bm V, \nor{\var},\angf)$ um espaço angulado.
	\begin{enumerate}
		\item (Homogeneidade) Para todos $v,v' \in V \setminus \{0\}$ e $c \in \intaa{0}{\infty}$,
			\begin{equation*}
				\ang{cv}{v'} = \ang{v}{v'};
			\end{equation*}
		\item (Suplementação) Para todos $v,v' \in V \setminus \{0\}$,
			\begin{equation*}
				\ang{v}{v'} + \ang{-v}{v'} = \tau \div 2;
			\end{equation*}
	\end{enumerate}
\end{proposition}
\begin{proof}
	\begin{enumerate}
		\item A homogeneidade vale para $\N \setminus \{0\}$. Mostremos por indução. Para $n=1$, claramente vale. Suponhamos que vale para $n \in \N \setminus \{0\}$. Da compatibilidade segue que
			\begin{align*}
				\nor{(n+1)v}\cos(\ang{(n+1)v}{v'}) &= \nor{nv}\cos(\ang{nv}{v'}) + \nor{v}\cos(\ang{v}{v'}) \\	
					&=n\nor{v}\cos(\ang{v}{v'}) + \nor{v}\cos(\ang{v}{v'}) \\
					&= (n+1)\nor{v}\cos(\ang{v}{v'}) \\
					&= \nor{(n+1)v}\cos(\ang{v}{v'}),
			\end{align*}
		portanto
		\begin{equation*}
			\cos(\ang{(n+1)v}{v'}) = \cos(\ang{v}{v'});
		\end{equation*}
		como $\ang{(n+1)v}{v'}, \ang{v}{v'} \in \intff{0}{\tau \div 2}$, segue que
			\begin{equation*}
				\ang{(n+1)v}{v'} = \ang{v}{v'}.
			\end{equation*}
		
		A homogeneidade vale para os números racionais estritamente positivos. Seja $\frac{n}{d} \in \Q \cap \intaa{0}{\infty}$. Da homogeneidade para números naturais positivos segue que
			\begin{equation*}
				\ang{\frac{n}{d}v}{v'} = \ang{d\frac{n}{d}v}{v'} = \ang{nv}{v'} = \ang{v}{v'}.
			\end{equation*}
		
		A homogeneidade vale para os números reais positivos. Seja $c \in \intaa{0}{\infty}$ e $(c_n)_{n \in \N}$ uma sequência de números racionais estritamente positivos tais que $\lim_{n \to \infty} c_n = c$. Então, da continuidade do ângulo, segue que
			\begin{equation*}
				\ang{cv}{v'} = \ang{\lim_{n \to \infty} c_n v}{v'} = \lim_{n \to \infty} \ang{c_n v}{v'} = \ang{v}{v'}.
			\end{equation*}

		\item Da compatibilidade e homogeneidade segue que
			\begin{align*}
				\nor{v}\cos(\ang{v}{v'}) &= \nor{2v-v}\cos(\ang{2v-v}{v'}) \\
					&= \nor{2v}\cos(\ang{2v}{v'}) + \nor{-v}\cos(\ang{-v}{v'}) \\
					&= 2\nor{v}\cos(\ang{v}{v'}) + \nor{v}\cos(\ang{-v}{v'});
			\end{align*}
		disso e da fórmula do cosseno de ângulo suplementar segue que
			\begin{equation*}
				\cos(\ang{-v}{v'}) = -\cos(\ang{v}{v'}) = \cos(\tau \div 2 - \ang{v}{v'}),
			\end{equation*}
		portanto
			\begin{equation*}
				\ang{-v}{v'} = \tau \div 2 - \ang{v}{v'}.
			\end{equation*}
	\end{enumerate}
\end{proof}

\begin{proposition}
Seja $(\bm V, \nor{\var},\angf)$ um espaço angulado. A função
	\begin{align*}
		\func{\inte{\var}{\var}}{V \times V}{\R}{(v,v')}{
			\begin{cases}
				0, &\text{$v=0$ ou $v'=0$} \\
				\nor{v}\nor{v'}\cos(\ang{v}{v'}), &\text{$v \neq 0$ e $v' \neq 0$}.
			\end{cases}
		}
	\end{align*}
é um produto interno em $\bm V$.
\end{proposition}
\begin{proof}
Os casos em que $v=0$ ou $v'=0$ são todos imediatos, portanto assumiremos que os vetores são sempre não nulos.
	\begin{enumerate}
		\item (Linearidade na primeira entrada)
			\begin{enumerate}
				\item (Aditividade) Sejam $v,v',v'' \in V$. Queremos mostrar que
					\begin{equation*}
						\inte{v+v'}{v''} = \inte{v}{v''} + \inte{v'}{v''}.
					\end{equation*}
				Como supomos que $v''$ é não nulo, isso é equivalente à compatibilidade
					\begin{equation*}
						\nor{v+v'}\cos(\ang{v+v'}{v''}) = \nor{v}\cos(\ang{v}{v''}) + \nor{v'}\cos(\ang{v'}{v''}).
					\end{equation*}
				
				\item (Homogeneidade) Sejam $v,v' \in V$ e $c \in \R$. Separamos em caos.
					\begin{enumerate}
						\item ($c=0$) Esse caso é trivial.
						\item ($c>0$) Segue da homogeneidade do ângulo que
							\begin{equation*}
								\inte{cv}{v'} = \nor{cv}\nor{v'}\cos(\ang{cv}{v'}) = c\nor{v}\nor{v'}\cos(\ang{v}{v'}) = c\inte{v}{v'}.
							\end{equation*}
						\item ($c<0$) Segue das propriedades de homogeneidade e suplementação do ângulo e da fórmula do cosseno de ângulo suplementar que
							\begin{align*}
								\inte{cv}{v'} &= \nor{cv}\nor{v'}\cos(\ang{cv}{v'}) \\
									&= -c\nor{v}\nor{v'}\cos(\ang{-\abs{c}v}{v'}) \\
									&= c\nor{v}\nor{v'}(-\cos(\ang{-v}{v'})) \\
									&= c\nor{v}\nor{v'}(-\cos(\tau \div 2 - \ang{v}{v'})) \\
									&= c\nor{v}\nor{v'}\cos(\ang{v}{v'}) \\
									&= c\inte{v}{v'}.
							\end{align*}
					\end{enumerate}
			\end{enumerate}
	
		\item (Simetria) Sejam $v,v' \in V$. Segue da simetria do produto e da simetria do ângulo que
			\begin{equation*}
				\inte{v}{v'} = \nor{v}\nor{v'}\cos(\ang{v}{v'}) = \nor{v'}\nor{v}\cos(\ang{v'}{v}) = \inte{v'}{v}.
			\end{equation*}

		\item (Positividade) Seja $v \in V$. Segue da nulidade do ângulo e da positividade da norma que
			\begin{equation*}
				\inte{v}{v} = \nor{v}\nor{v}\cos(\ang{v}{v}) = \nor{v}^2\cos(0) = \nor{v}^2 \in \intfa{0}{\infty}.
			\end{equation*}

		\item (Definição positiva) Seja $v \in V$. Se $v \neq 0$, então $\nor{v} \neq 0$ e, pelo item anterior, $\inte{v}{v} = \nor{v}^2$, portanto $\inte{v}{v} \neq 0$.

	\end{enumerate}
\end{proof}




\paragraph{Segunda definição: investigação de mais propriedades}

Para começar a discussão, listamos algumas propriedades interessantes e esperadas de uma função ângulo, tanto por si própria como em relação à norma. Consideramos um espaço normado $(\bm V, \nor{\var})$ e uma função $\fun{\angf}{V \setminus \{0\} \times V \setminus \{0\}}{\intff{0}{\tau \div 2}}$
	\begin{enumerate}
		\item ($0$-Homogeneidade) Para todos $v,v' \in V$ e $c,c' \in \intaa{0}{\infty}$,
			\begin{equation*}
				\ang{cv}{c'v'} = \ang{v}{v'};
			\end{equation*}
		\item (Separação) Para todos $v,v' \in V \setminus \{0\}$,
			\begin{equation*}
				\ang{v}{v'} = 0
			\end{equation*}
		se, e somente se, existe $c \in \intaa{0}{\infty}$ tal que $v'=cv$;
		\item (Simetria) Para todos $v,v' \in V \setminus \{0\}$,
			\begin{equation*}
				\ang{v}{v'} = \ang{v'}{v};
			\end{equation*}
		\item (Desigualdade Triangular) Para todos $v,v',v'' \in V \setminus \{0\}$,
			\begin{equation*}
				\ang{v}{v''} \leq \ang{v}{v'} + \ang{v'}{v''};
			\end{equation*}
		
		\item (Aditividade) Para todos $v,v' \in V \setminus \{0\}$ e $c,c' \in \intfa{0}{\infty}$ tais que $cv+c'v' \neq 0$,
			\begin{equation*}
				\ang{v}{v'} = \ang{v}{cv+c'v'} + \ang{cv+c'v'}{v'};
			\end{equation*}
		\item (Aditividade limite) Para todos $v,v' \in V \setminus \{0\}$,
		\begin{equation*}
			\ang{v}{-v} = \ang{v}{v'} + \ang{v'}{-v};
		\end{equation*}
		
		\item (Isotropia) Para todos $v,v' \in V \setminus \{0\}$,
			\begin{equation*}
				\ang{v}{-v} = \ang{v'}{-v'}
			\end{equation*}
		\item (Radiano) Para todo $v \in V \setminus \{0\}$ 
			\begin{equation*}
				\ang{v}{-v} = \tau \div 2;
			\end{equation*}
		\item (Retificação) Para todos $v,v',w,w' \in V \setminus \{0\}$ tais que $\ang{v}{v'} = \ang{-v}{v'}$ e $\ang{w}{w'} = \ang{-w}{w'}$,
			\begin{equation*}
				\ang{v}{v'} = \ang{w}{w'};
			\end{equation*}
		\item (Suplementação) Para todos $v,v' \in V \setminus \{0\}$,
			\begin{equation*}
				\ang{v}{v'} + \ang{-v}{v'} = \tau \div 2;
			\end{equation*}
		
		\item (Lado-ângulo-lado) Para todos $v,v',w,w' \in V \setminus \{0\}$ tais que $\nor{v}=\nor{w}$, $\nor{v'}=\nor{w'}$ e $\ang{v}{v'}=\ang{w}{w'}$, valem
			\begin{enumerate}
				\item $\nor{v+v'}=\nor{w+w'}$;
				\item $\ang{v}{v+v'}=\ang{w}{w+w'}$.
			\end{enumerate}
		\item (Lei dos cossenos) Para todos $v,v' \in V \setminus \{0\}$,
			\begin{equation*}
				\ang{v}{v'} = \cos\inv \left( \frac{\nor{v+v'}^2 - \nor{v}^2-\nor{v'}^2}{2\nor{v}\nor{v'}} \right)
			\end{equation*}
		\item (Pitágoras) Para todos $v,v' \in V \setminus \{0\}$, $\ang{v}{v'}=\tau \div 4$ se, e somente se,
			\begin{equation*}
				\nor{v+v'}^2 = \nor{v}^2 + \nor{v'}^2;
			\end{equation*}
		\item (Tales) Para todos $v,v' \in V \setminus \{0\}$ tais que $\nor{v}=\nor{v'}$,
			\begin{equation*}
				\ang{v+v'}{v-v'} = \tau \div 4.
			\end{equation*}
		\item (Compatibilidade) Para todos $v,v',v'' \in V \setminus \{0\}$ tais que $v' \neq -v$,
			\begin{equation*}
				\nor{v+v'}\cos(\ang{v+v'}{v''}) = \nor{v}\cos(\ang{v}{v''}) + \nor{v'}\cos(\ang{v'}{v''});
			\end{equation*}
	\end{enumerate}

A propriedade de retificação é o postulado de Euclides de que todos ângulos retos são iguais, e a isotropia é similar, pois assume que todo ângulo de meia volta é igual. Como em Euclides, essas propriedades não assumem um valor específico de ângulo. Podemos escolher um valor e \textit{normalizar} o ângulo, ou seja, dividir o ângulo por esse valor para que o valor do ângulo de meia volta seja $\tau \div 2$ (sendo $\tau$ o período\footnote{O número $\tau$ vale aproximadamente $6,\!28$ e é o dobro de $\pi$, como geralmente é denotada metade do período de $\cos$.} de $\cos$, uma volta completa), para que depois possamos calcular o cosseno desse ângulo. A propriedade do radiano é uma forma de normalização e já garante isso, definindo que para ser uma função ângulo o valor de meia volta deve ser $\tau \div 2$. Assumindo a aditividade (limite) e a isotropia, podemos mostrar a propriedade de retificação, ou seja, que todos ângulos retos são iguais entre si. Assumindo a propriedade do radiano, podemos mostrar que todo ângulo reto vale $\tau \div 4$, um quarto de volta, pois se os ângulos suplementares são iguais e somam $\tau \div 2$, cada um deles vale $\tau \div 4$. Ainda, da aditividade (limite) e da propriedade do radiano, segue a suplementação. A aditividade limite segue da aditividade assumindo que a função é contínua e homogênea.

\begin{proposition}
Sejam $(\bm V, \nor{\var})$ um espaço normado e
	\begin{equation*}
	\fun{\angf}{V \setminus \{0\} \times V \setminus \{0\}}{\intfa{0}{\infty}}
	\end{equation*}
uma função contínua que satisfaz $0$-homogeneidade, aditividade e isotropia.
	\begin{enumerate}
	\item (Aditividade limite) Para todos $v,v' \in V \setminus \{0\}$,
		\begin{equation*}
		\ang{v}{-v} = \ang{v}{v'} + \ang{v'}{-v};
		\end{equation*}
	\item (Maximalidade) Existe único $\pi \in \intfa{0}{\infty}$ tal que, para todos $v,v' \in V \setminus \{0\}$,
		\begin{equation*}
		\ang{v}{v'} \leq \ang{v}{-v} = \pi;
		\end{equation*}
	\item (Retificação) Para todos $v,v',w,w' \in V \setminus \{0\}$ tais que $\ang{v}{v'} = \ang{v'}{-v}$ e $\ang{w}{w'} = \ang{w'}{-w}$,
		\begin{equation*}
		\ang{v}{v'} = \ang{w}{w'}.
		\end{equation*}
	\item (Suplementação) Para todos $v,v' \in V \setminus \{0\}$,
		\begin{equation*}
		\ang{v}{v'} + \ang{v'}{-v} = \pi;
		\end{equation*}
	\end{enumerate}
\end{proposition}
\begin{proof}
	\begin{enumerate}
	\item Seja $c \in \intaa{0}{\infty}$ tal que $cv+v' \neq 0$ e $v'-cv \neq 0$. Como $v' = \frac{1}{2}(cv+v') + \frac{1}{2}(v'-cv)$, segue da aditividade do ângulo que
		\begin{equation*}
		\ang{cv+v'}{v'-cv} = \ang{cv+v'}{v'} + \ang{v'}{v'-cv}.
		\end{equation*}
	Como
		\begin{equation*}
		\frac{v}{\nor{v}} = \lim_{c \to \infty} \frac{cv+v'}{\nor{cv+v'}}
		\end{equation*}
	e
		\begin{equation*}
		\frac{-v}{\nor{v}} = \lim_{c \to \infty} \frac{v'-cv}{\nor{v'-cv}},
		\end{equation*}
	segue da $0$-homogeneidade e da continuidade do ângulo que
		\begin{align*}
		\ang{v}{-v} &= \ang{\frac{v}{\nor{v}}}{\frac{-v}{\nor{v}}} \\
			&= \ang{\lim_{c \to \infty} \frac{cv+v'}{\nor{cv+v'}}}{\lim_{c \to \infty} \frac{v'-cv}{\nor{v'-cv}}} \\
			&= \lim_{c \to \infty} \ang{\frac{cv+v'}{\nor{cv+v'}}}{\frac{v'-cv}{\nor{v'-cv}}} \\
			&= \lim_{c \to \infty} \ang{cv+v'}{v'-cv} \\
			&= \lim_{c \to \infty} \ang{cv+v'}{v'} + \ang{v'}{v'-cv} \\
			&= \lim_{c \to \infty} \ang{\frac{cv+v'}{\nor{cv+v'}}}{v'} + \ang{v'}{\frac{v'-cv}{\nor{v'-cv}}} \\
			&= \ang{v}{v'} + \ang{v'}{-v}.
		\end{align*}
	
	\item Definimos $\pi := \ang{v}{-v}$ e a unicidade segue da isotropia, pois $\ang{v'}{-v'} = \ang{v}{-v} = \pi$. Agora, da positividade temos $\ang{v'}{-v} \geq 0$, portanto da aditividade limite segue que
		\begin{equation*}
		\ang{v}{v'} \leq \ang{v}{v'} + \ang{v'}{-v} = \ang{v}{-v} = \pi.
		\end{equation*}
	
	\item Da aditividade limite e da isotropia segue que
		\begin{align*}
		\ang{v}{v'} &= \frac{\ang{v}{v'}+\ang{v}{v'}}{2} \\
			&= \frac{\ang{v}{v'}+\ang{v'}{-v}}{2} \\
			&= \frac{\ang{v}{-v}}{2} \\
			&= \frac{\ang{w}{-w}}{2} \\
			&= \ang{w}{w'}.
		\end{align*}
	\end{enumerate}

	\item Segue direto dos itens anteriores.
\end{proof}

Com essa definição, a propriedade do radiano seria equivalente a termos
	\begin{equation*}
	\pi = \frac{\tau}{2},
	\end{equation*}
ou seja, o valor máximo da função $\angf$, denotado $\pi$, é definido como metade do período da função $\cos$, denotado $\tau$. Isso quer dizer que estamos medindo os ângulos em radianos, que são os valores mais naturais para calcular senos e cossenos.

A propriedade lado-ângulo-lado é equivalente a dizer que $\nor{v+v'}$ e $\ang{v}{v+v'}$ são funções de $\nor{v}, \nor{v'}$ e $\ang{v}{v'}$. Essa propriedade é fundamental para relacionar a norma do espaço com a função ângulo e ela que vai, no final, nos permitir mostrar a lei dos cossenos, que essencialmente é equivalente a termos um produto interno. Sem uma propriedade que relacione o ângulo e a norma, não temos porque esperar que tenhamos um produto interno dado pelo produto das normas e do cosseno do ângulo, pois existem normas que não admitem produtos internos, precisamente aquelas que não satisfazem a lei do paralelogramo, o se tivéssemos uma norma dessa, qualquer função ângulo independente de norma, seja ela qual fosse, não nos daria um produto interno.

Após essa discussão, partimos para uma definição da função ângulo que por simplicidade já considera a propriedade de suplementação em vez das propriedades de isotropia e radiano, pois elas são equivalentes.
\clearpage
\begin{definition}
Seja $(\bm V, \nor{\var})$ um espaço normado. Uma \emph{função ângulo} em $(\bm V, \nor{\var})$ é uma função contínua $\fun{\angf}{V \setminus \{0\} \times V \setminus \{0\}}{\intff{0}{\tau \div 2}}$ tal que
	\begin{enumerate}
		\item ($0$-Homogeneidade) Para todos $v,v' \in V$ e $c,c' \in \intaa{0}{\infty}$,
			\begin{equation*}
				\ang{cv}{c'v'} = \ang{v}{v'};
			\end{equation*}
		\item (Separação) Para todos $v,v' \in V \setminus \{0\}$,
			\begin{equation*}
				\ang{v}{v'} = 0
			\end{equation*}
		se, e somente se, existe $c \in \intaa{0}{\infty}$ tal que $v'=cv$;
		\item (Simetria) Para todos $v,v' \in V \setminus \{0\}$,
			\begin{equation*}
				\ang{v}{v'} = \ang{v'}{v};
			\end{equation*}
		\item (Aditividade) Para todos $v,v' \in V \setminus \{0\}$ e $c,c' \in \intfa{0}{\infty}$ tais que $cv+c'v' \neq 0$,
			\begin{equation*}
				\ang{v}{v'} = \ang{v}{cv+c'v'} + \ang{cv+c'v'}{v'};
			\end{equation*}
		\item (Suplementação) Para todos $v,v' \in V \setminus \{0\}$,
			\begin{equation*}
				\ang{v}{v'} + \ang{v'}{-v} = \tau \div 2;
			\end{equation*}
		\item (LAL: lado-ângulo-lado) Para todos $v,v',w,w' \in V \setminus \{0\}$ tais que $v+v' \neq 0$, $w+w' \neq 0$, $\nor{v}=\nor{w}$, $\nor{v'}=\nor{w'}$ e $\ang{v}{v'}=\ang{w}{w'}$, valem
			\begin{enumerate}
			\item $\nor{v+v'}=\nor{w+w'}$;
			\item $\ang{v}{v+v'}=\ang{w}{w+w'}$;
			\end{enumerate}
	\end{enumerate}
Um \emph{espaço angulado} é uma tripla $(\bm V, \nor{\var},\angf)$ em que $(\bm V, \nor{\var})$ é um espaço normado e $\fun{\angf}{V \setminus \{0\} \times V \setminus \{0\}}{\intff{0}{\tau \div 2}}$ uma função ângulo em $(\bm V, \nor{\var})$.
\end{definition}

\begin{exercise}
Seja $(\bm V, \nor{\var},\angf)$ um espaço angulado.
	\begin{enumerate}
	\item (Alternos internos) Para todos $v,v' \in V \setminus \{0\}$,
			\begin{equation*}
			\ang{v}{v'} = \ang{-v}{-v'};
			\end{equation*}
	
	\item (Ângulos externos) Para todos $v,v' \in V \setminus \{0\}$,
			\begin{enumerate}
			\item (Adjacente) $\ang{v'}{v'-v} < \ang{v'}{-v}$;
			\item (Oposto) $\ang{v}{v-v'} < \ang{v'}{-v}$.
			\end{enumerate}
	\end{enumerate}
\end{exercise}

\begin{proposition}
Seja $(\bm V, \nor{\var},\angf)$ um espaço angulado.
	\begin{enumerate}
		\item (LAL negativo) Para todos $v,v',w,w' \in V \setminus \{0\}$ tais que $v-v' \neq 0$, $w-w' \neq 0$, $\nor{v}=\nor{w}$, $\nor{v'}=\nor{w'}$ e $\ang{v}{v'}=\ang{w}{w'}$, valem
			\begin{enumerate}
			\item $\nor{v-v'}=\nor{w-w'}$;
			\item $\ang{v}{v-v'}=\ang{w}{w-w'}$;
			\end{enumerate}
		\item (Triângulo isósceles) Para todos $v,v' \in V \setminus \{0\}$, tais que $\nor{v}=\nor{v'}$,
			\begin{enumerate}
			\item (Positivo) $\ang{v}{v+v'} = \ang{v'}{v+v'}$;
			\item (Negativo) $\ang{v}{v-v'} = \ang{v'}{v'-v}$.
			\end{enumerate}
		
		\item (Oposição lado-ângulo) Para todos $v,v' \in V \setminus \{0\}$ tais que $v'-v \neq 0$,
			\begin{enumerate}
			\item $\nor{v}<\nor{v'}$ se, e somente se, $\ang{v'}{v'-v} < \ang{v}{v-v'}$;
			\item $\nor{v}=\nor{v'}$ se, e somente se, $\ang{v'}{v'-v} = \ang{v}{v-v'}$.
			\end{enumerate}
		
		\item (Monotonicidade lado-ângulo) Para todos $v,v',w,w' \in V \setminus \{0\}$ tais que $\nor{v}=\nor{w}$, $\nor{v'}=\nor{w'}$ e $\ang{v}{v'} < \ang{w}{w'}$,
			\begin{equation*}
			\nor{v'-v} < \nor{w'-w}.
			\end{equation*}
	\end{enumerate}
\end{proposition}
\begin{proof}
	\begin{enumerate}
	\item Como $\nor{v'}=\nor{w'}$, temos $\nor{-v'}=\nor{-w'}$; como $\ang{v}{v'}=\ang{w}{w'}$, segue da suplementação e da simetria que
		\begin{equation*}
		\ang{v}{-v'} = \tau \div 2 - \ang{v}{v'} = \tau \div 2 - \ang{w}{w'} = \ang{w}{-w'}.
		\end{equation*}
	Segue então de LAL que
		\begin{equation*}
		\nor{v-v'} = \nor{v+(-v')} = \nor{w+(-w')} = \nor{w-w'}
		\end{equation*}
	e
		\begin{equation*}
		\ang{v}{v-v'} = \ang{v}{v+(-v')} = \ang{w}{w+(-w')} = \ang{w}{w-w'}.
		\end{equation*}

	\item Como $\nor{v}=\nor{v'}$, $\nor{v'}=\nor{v}$ e, por simetria do ângulo, $\ang{v}{v'} = \ang{v'}{v}$, segue de LAL que
		\begin{equation*}
			\ang{v}{v+v'} = \ang{v'}{v'+v} = \ang{v'}{v+v'}.
		\end{equation*}
	A outra igualdade segue analogamente usando LAL (negativo).
	
	\item Assumamos que $\nor{v} < \nor{v'}$. Seja $c := \nor{v} \div \nor{v'} \in \intaa{0}{1}$. Como $(1-c)>0$ e $v'-v = (1-c)v'+(cv'-v)$, da homogeneidade e da aditividade do ângulo segue que
		\begin{align*}
		\ang{v'}{v'-v}&= \ang{(1-c)v'}{v'-v} \\
			&< \ang{(1-c)v'}{v'-v} + \ang{v'-v}{cv'-v} \\
			&= \ang{(1-c)v'}{cv'-v} \\
			&= \ang{cv'}{cv'-v};
		\end{align*}
	como $\nor{cv'}=c\nor{v'}=\nor{v}$, da proposição do triângulo isósceles (negativo) segue que
		\begin{equation*}
		\ang{cv'}{cv'-v} = \ang{v}{v-cv'};
		\end{equation*}
	como $c>0$, $(1-c) > 0$ e $v-cv' = (1-c)v + c(v-v')$, da aditividade do ângulo segue que
		\begin{equation*}
		\ang{v}{v-cv'} < \ang{v}{v-cv'} + \ang{v-cv'}{v-v'} = \ang{v}{v-v'}.
		\end{equation*}

	Assim, concluímos que
		\begin{align*}
		\ang{v'}{v'-v} &< \ang{cv'}{cv'-v} \\
			&= \ang{v}{v-cv'} \\
			&< \ang{v}{v-v'}.
		\end{align*}

	Assumindo que $\nor{v'} < \nor{v}$, por simetria do enunciado segue que $\ang{v}{v-v'} < \ang{v'}{v-v'}$.

	Assim, se $\ang{v}{v-v'} = \ang{v'}{v-v'}$, então $\nor{v}=\nor{v'}$ e, da proposição do triângulo isósceles segue a recíproca.

	\item Como $\ang{v}{v'} < \ang{w}{w'}$, tomamos\footnote{Isso segue da continuidade da função ângulo $\ang{w}{w+kw'}$ para achar $k$ que satisfaça a condição do ângulo, e então por normalização para igualar a norma de $v'$.} $c,c' \in \intaa{0}{\infty}$ tais que $\nor{v'} = \nor{cw+c'w'}$ e $\ang{v}{v'} = \ang{w}{cw+c'w'}$ e definimos $u := cw+c'w'$. Por LAL (negativo) segue que $\nor{v-v'} = \nor{w-u}$.
	
	Para estimar a norma de $w-u$, consideramos $3$ casos.
		\begin{itemize}
		\item ($c+c' = 1$) Nesse caso temos, $0 < c' = 1-c < 1$, logo
			\begin{align*}
			\nor{v-v'} &= \nor{w-u} \\
				&= \nor{w-(cw+(1-c)w')} \\
				&= \nor{(1-c)(w-w')} \\
				&= \abs{1-c}\nor{w-w'} \\
				&< \nor{w-w'};
			\end{align*}
		
		\item ($c+c'<1$) Como $w'-w = (u-w) + (w'-u)$, segue da positividade e da aditividade que
			\begin{align*}
			\ang{w'-w}{w'-u} &< \ang{u-w}{w'-w} + \ang{w'-w}{w'-u} \\
				&= \ang{u-w}{w'-u};
			\end{align*}
		como $(1-c-c') \div c > 0$, $c' \div c > 0$ e
			\begin{align*}
			u-w &= cw+c'w'-w \\
				&= (c-1)w + c'w'\\
				&= \left(-(1-c-c')-c' \right)w + \left(-\frac{1-c-c'}{c}c' + \frac{c'}{c}(1-c') \right)w' \\
				&= \frac{1-c-c'}{c}(-cw-c'w') + \frac{c'}{c}(w'-cw-c'w') \\
				&= \frac{1-c-c'}{c}(-u) + \frac{c'}{c}(w'-u),
			\end{align*}
		segue da positividade e da aditividade do ângulo que
			\begin{align*}
			\ang{u-w}{w'-u} &< \ang{-u}{u-w} + \ang{u-w}{w'-u}\\
				&= \ang{-u}{w'-u};
			\end{align*}
		como $\nor{-u}=\nor{u}=\nor{w'}$, segue da propriedade do triângulo isósceles (positivo) que
			\begin{equation*}
			\ang{-u}{w'-u} = \ang{w'}{w'-u};
			\end{equation*}
		definimos $k := c' \div (1-c) < 1$ e segue que
			\begin{align*}
			(1-c)(kw'-w) &= (1-c)\frac{c'}{1-c}w'-(1-c)w \\
				&= (cw+c'w')-w \\
				&= u-w;
			\end{align*}
		como $(1-k)w' = (w'-u)-(kw'-u)$, segue da $0$-homogeneidade, da positividade e da aditividade e dos ângulos alternos internos que
			\begin{align*}
			\ang{w'-u}{w'} &= \ang{w'-u}{(1-k)w'} \\
				&= \ang{w'-u}{(w'-u)-(kw'-u)} \\
				&< \ang{w'-u}{-(kw'-u)} \\
				&= \ang{-(u-w')}{-(u-w)} \\
				&= \ang{u-w'}{u-w}.
			\end{align*}

		Assim, concluímos que
			\begin{align*}
			\ang{w'-w}{(w'-w)-(u-w)} &= \ang{w'-w}{w'-u} \\
				&< \ang{u-w}{w'-u} \\
				&< \ang{-u}{w'-u} \\
				&= \ang{w'}{w'-u} \\
				&< \ang{u-w}{u-w'} \\
				&= \ang{u-w}{(u-w)-(w'-w)}
			\end{align*}
		e da proposição anterior de oposição lado-ângulo segue que
			\begin{equation*}
			\nor{v'-v} = \nor{u-w} < \nor{w'-w}.
			\end{equation*}

		\item ($c+c' > 1$) Definimos $k := 1 \div (c+c')$ e segue que
			\begin{align*}
			w'-ku &= w'-k(cw+c'w') \\
				&= (1-kc')w'-kcw \\
				&= \left( \frac{c+c'-c'}{c+c'} \right)w'-kcw \\
				&= \left( \frac{c}{c+c'} \right)w'-kcw \\
				&= kcw'-kcw \\
				&= kc(w'-w).
			\end{align*}
		
		Como $k>0$, $1-k>0$ e $w'-ku = k(w'-u) + (1-k)w'$, segue da $0$-homogeneidade, da positividade e da aditividade que
			\begin{align*}
			\ang{w'-w}{w'-u} &= \ang{w'-u}{w'-w} \\
				&= \ang{w'-u}{w'-ku} \\
				&< \ang{w'-u}{w'-ku} + \ang{w'-ku}{w'} \\
				&= \ang{w'-u}{w'};
			\end{align*}
		como $\nor{w'}=\nor{u}$, segue da proposição do triângulo isósceles (negativo) e da simetria que
			\begin{equation*}
			\ang{w'-u}{w'} = \ang{u-w'}{u};
			\end{equation*}
		como
			\begin{align*}
			(c+c'-1)u = cu+c'w'-cw-c'w' = c'(u-w')+c(u-w),
			\end{align*}
		segue da positividade e da aditividade que
			\begin{align*}
			\ang{u-w'}{u} &< \ang{u-w'}{u} + \ang{u}{u-w} \\
				&= \ang{u-w'}{u-w}.
			\end{align*}
		
		Assim, concluímos que
			\begin{align*}
			\ang{w'-w}{(w'-w)-(u-w)} &= \ang{w'-w}{w'-u} \\
				&< \ang{w'-u}{w'} \\
				&= \ang{u-w'}{u} \\
				&< \ang{u-w'}{u-w} \\
				&= \ang{u-w}{(u-w)-(w'-w)}
			\end{align*}
		e da proposição anterior de oposição lado-ângulo segue que
			\begin{equation*}
			\nor{v'-v} = \nor{u-w} < \nor{w'-w}.
			\end{equation*}
		\end{itemize}

	\end{enumerate}
\end{proof}

\begin{figure}
	\centering

	\begin{tikzpicture}[line cap=round,line join=round,>=triangle 45,x=1.0cm,y=1.0cm]
	\draw(0,0) node {\scriptsize $0$} circle (1cm);
	\draw (0,0) -- (1,0) node[anchor=west] {\scriptsize $v$};
	\draw (0,0) -- (1.63,0.77) node[anchor=west] {\scriptsize $v+v'$};
	\draw (0,0) -- (0.63,0.77) node[anchor=west] {\scriptsize $v'$};
	\draw (0,0) -- (-0.37,0.77) node[anchor=east] {\scriptsize $-v+v'\ \ $};
	\draw (0,0) -- (-1,0) node[anchor=east] {\scriptsize $-v$};
	\draw (0,0) -- (-1.63,-0.77) node[anchor=east] {\scriptsize $-v-v'$};
	\draw (0,0) -- (-0.63,-0.77) node[anchor=east] {\scriptsize $-v'$};
	\draw (0,0) -- (0.37,-0.77) node[anchor=west] {\scriptsize $\ \ v-v'$};
	\end{tikzpicture}

	\caption{Soma e diferença de vetores de mesma norma}
	\label{fig:somavetorescongruentes}
\end{figure}




\begin{proposition}
Seja $(\bm V, \nor{\var},\angf)$ um espaço angulado.
	\begin{enumerate}
	\item (LLL) Para todos $v,v',w,w' \in V \setminus \{0\}$ tais que $\nor{v}=\nor{w}$, $\nor{v'}=\nor{w'}$ e $\nor{v+v'}=\nor{w+w'}$,
		\begin{equation*}
		\ang{v}{v'}=\ang{w}{w'}.
		\end{equation*}
	
	\item (ALA) Para todos $v,v',w,w' \in V \setminus \{0\}$ tais que $\nor{v}=\nor{w}$, $\ang{v}{v'}=\ang{w}{w'}$ e $\ang{v}{v+v'} = \ang{w}{w+w'}$,
		\begin{equation*}
		\nor{v+v'}=\nor{w+w'}
		\end{equation*}
	
	\item (LAA) Para todos $v,v',w,w' \in V \setminus \{0\}$ tais que $\nor{v+v'}=\nor{w+w'}$, $\ang{v}{v'}=\ang{w}{w'}$ e $\ang{v}{v+v'} = \ang{w}{w+w'}$,
		\begin{enumerate}
		\item $\nor{v}=\nor{w}$;
		\item $\ang{v}{v+v'} = \ang{w}{w+w'}$.
		\end{enumerate}
	\end{enumerate}
\end{proposition}









\subparagraph{Perpendicularidade}


\begin{definition}
Seja $(\bm V, \nor{\var},\angf)$ um espaço angulado. Vetores \emph{perpendiculares} são vetores $v,v' \in V \setminus \{0\}$ tais que
	\begin{equation*}
	\ang{v}{v'} = \ang{v'}{-v}.
	\end{equation*}
Denota-se $v \perp v'$.
\end{definition}

\begin{proposition}[Perpendicular]
Seja $(\bm V, \nor{\var},\angf)$ um espaço angulado. Para todos $v,v' \in V \setminus \{0\}$ linearmente independentes, a função
	\begin{align*}
		\func{d}{\R}{\intfa{0}{\infty}}{c}{\nor{v'-cv}}
	\end{align*}
é contínua, $\lim_{c \to \pm\infty} d(c) = \infty$, tem único ponto de mínimo $c_{\perp} \in \R$ e
	\begin{equation*}
		v \perp (v'-c_{\perp}v).
	\end{equation*}
\end{proposition}
\begin{proof}
A função $d$ é contínua pois é composição da norma, subtração e multiplicação por escalar, que são contínuas. Dados $c,k \in \R$, segue da desigualdade triangular que
	\begin{align*}
		d(c) &= \nor{v'-cv} \\
			&\geq \nor{cv-kv} - \nor{v'-kv} \\
			&=\left( \frac{\abs{c-k}\nor{v}}{\nor{v'-kv}} - 1 \right) \nor{v'-kv}.
	\end{align*}
Como
	\begin{equation*}
		\lim_{c \to \pm\infty} \frac{\abs{c-k}\nor{v}}{\nor{v'-kv}} = \infty,
	\end{equation*}
segue que
	\begin{equation*}
		\lim_{c \to \pm\infty} d(c) = \infty.
	\end{equation*}
Da continuidade e dos limites serem infinitos segue que $d$ tem um ponto de mínimo $c_0 \in \R$. Ainda, para todo $s \in \intaa{d(c_0)}{\infty}$, existem $c_-,c_+ \in \R$ tais que $c_- < c_0 < c_+$ e
	\begin{equation*}
		\nor{v'-c_+v} = d(c_+) = s = d(c_-) = \nor{v'-c_-v}.
	\end{equation*}
Definindo $c_{\perp} := \frac{c_-+c_+}{2}$, segue de homogeneidade e da proposição do triângulo isósceles (positivo) que
	\begin{align*}
		\ang{v'-c_+v}{v'-c_{\perp}v} &= \ang{v'-c_+v}{2v'-(c_-+c_+)v} \\
			&= \ang{v'-c_+v}{(v'-c_-v)+(v'-c_+v)} \\
			&= \ang{v'-c_-v}{(v'-c_-v)+(v'-c_+v)} \\
			&= \ang{v'-c_-v}{2v'-(c_-+c_+)v} \\
			&= \ang{v'-c_-v}{v'-c_{\perp}v}.
	\end{align*}
Como $\nor{v'-c_{\perp}v} = \nor{v'-c_{\perp}v}$ e $\nor{v'-c_+v} = \nor{v'-c_-v}$, por LAL (negativo) segue que
	\begin{align*}
		\ang{v'-c_{\perp}v}{\frac{c_--c_+}{2}v} &= \ang{v'-c_{\perp}v}{v'-c_{\perp}v - (v'-c_+v)} \\
			&= \ang{v'-c_{\perp}v}{v'-c_{\perp}v - (v'-c_-v)} \\
			&= \ang{v'-c_{\perp}v}{\frac{c_+-c_-}{2}v}.
	\end{align*}
Por fim, segue da simetria e da homogeneidade que
	\begin{equation*}
		\ang{v}{v'-c_{\perp}v} = \ang{-v}{v'-c_{\perp}v},
	\end{equation*}
logo $v \perp (v'-c_{\perp}v)$.

Notemos que tal $c_{\perp}$ é único: suponhamos que existam $c,c' \in \R$, $c < c'$, tais que $v \perp (v'-cv)$ e $v \perp (v'-c'v)$. Então $c'-c > 0$ e
	\begin{equation*}
		v'-cv = v'-c'v + (c'-c)v,
	\end{equation*}
portanto da aditividade do ângulo segue que
	\begin{equation*}
		\ang{v}{v'-c'v} = \ang{v}{v'-cv} + \ang{v'-cv}{v'-c'v}.
	\end{equation*}
Como $v \perp (v'-cv)$ e $v \perp (v'-c'v)$, segue da unicidade do ângulo reto que
	\begin{equation*}
		\ang{v'-cv}{v'-c'v} = \ang{v}{v'-c'v} - \ang{v}{v'-cv} = 0;
	\end{equation*}
da separação do ângulo, existe $k \in \intaa{0}{\infty}$ $v'-c'v = k(v'-cv)$ e, como $v,v'$ são linearmente independentes, segue que $k=1$ e $c=c'$, o que contradiz $c<c'$.

Por fim, notemos que $d$ é estritamente crescente em $\intaa{c_{\perp}}{\infty}$ e estritamente decrescente em $\intaa{-\infty}{c_{\perp}}$. Mostraremos que é crescente; a demonstração de que é decrescente é análoga. Suponhamos, por absurdo, que existam $k_{-},k_{+} \in \intaa{c_{\perp}}{\infty}$ tais que $k_{-} < k_{+}$ e $d(k_{-}) = d(k_{+})$. Pela mesma construção anterior, $k_{\perp} := \frac{k_{-} + k_{+}}{2}$ satisfaz $v \perp (v'-k_{\perp}v)$. Como $c_{\perp}$ é único com essa propriedade, segue que $k_{\perp} = c_{\perp}$.  Mas $c_{\perp} \leq k_{-} < k_{\perp}$, contradição. Isso mostra que não podem existir tais $k_{-},k_{+}$ distintos, logo da continuidade de $d$ e de $\lim_{c \to \infty} d(c) = \infty$ segue que $d$ estritamente crescente. Analogamente se mostra que $d$ é estritamente decrescente em $\intaa{-\infty}{c_{\perp}}$ e isso implica que $c_{\perp}$ é único ponto de mínimo de $d$, portanto $c_0 = c_{\perp}$ e $v \perp (v'-c_0v)$.
\end{proof}

\begin{corollary}
Para todos $v,v' \in V \setminus \{0\}$ tais que $v \perp v'$,
	\begin{equation*}
	\nor{v+v'} > \nor{v'}
	\end{equation*}
\end{corollary}
\begin{proof}
Como $(v+v')-v = v'$ e $v \perp ((v+v')-v)$, $c=1$ é o ponto de mínimo da função $\nor{(v+v')-cv}$ e então
	\begin{equation*}
	\nor{v+v'} = \nor{(v+v')-0v} > \nor{(v+v')-1v} = \nor{v'}.
	\end{equation*}
\end{proof}

A última proposição nos dá para cada par $v,v' \in V \setminus \{0\}$ de vetores linearmente independentes, existe único $c_{\perp} \in \R$ tal que $v \perp (v'-c_\perp v)$. Isso significa que existe uma função $c_{\perp}(v,v')$. A seguir, mostraremos que essa função é de fato uma função das variáveis $\nor{v}$, $\nor{v'}$ e $\ang{v}{v'}$.

\begin{proposition}
\label{prop:geo.cperpfuncaonormaangulo}
Seja $(\bm V, \nor{\var},\angf)$ um espaço angulado. Para todos $v,v',w,w' \in V \setminus \{0\}$, $v,v'$ e $w,w'$ linearmente independentes, respectivamente, tais que $\nor{v}=\nor{w}$, $\nor{v'}=\nor{w'}$ e $\ang{v}{v'} = \ang{w}{w'}$, vale
	\begin{equation*}
	c_{\perp}(v,v') = c_{\perp}(w,w').
	\end{equation*}
\end{proposition}
\begin{proof}
Por simplicidade, denotemos $c:=c_{\perp}(v,v')$ e $k:=c_{\perp}(w,w')$. Como $v,v'$ e $w,w'$ são linearmente independentes, então $(v'-cv),(w'-kv) \neq 0$. Consideramos o caso em que $c=0$. Nesse caso, $v'-cv = v'$ e portanto
	\begin{equation*}
	\ang{w}{w'} = \ang{v}{v'} = \ang{v}{v'-cv} = \ang{w'}{w'-kw} = \tau \div 4,
	\end{equation*}
logo $w \perp w'$, e segue da unicidade de $k$ que $k=0$. Analogamente, supondo $k=0$ obtemos $c=0$.

Consideramos agora o caso em que $c \neq 0$ e $k \neq 0$. Mostraremos que $c$ e $k$ têm o mesmo sinal. Consideramos, por absurdo, os dois casos em que isso não ocorre.
	\begin{enumerate}
	\item ($k<0<c$) Nesse caso, temos $-k>0$. Como $v' = cv + (v'-cv)$ e $w'-kw = (-k)w + w'$, segue da positividade e da aditividade que
		\begin{align*}
		\ang{v}{v'} &< \ang{v}{v'} + \ang{v'}{v'-cv} \\
			&= \ang{v}{v'-cv} \\
			&= \tau \div 4 \\
			&= \ang{w}{w'-kw} \\
			&= \ang{w}{w'-kw} + \ang{w'-kw}{w'} \\
			&= \ang{w}{w'},
		\end{align*}
	o que é uma contradição.

	\item ($c<0<k$) Obtemos uma contradição de modo análogo ao caso anterior.
	\end{enumerate}

Sendo assim, temos que, se $c>0$ e $k>0$, então
	\begin{equation*}
	\ang{cv}{v'} = \ang{v}{v'} = \ang{w}{w'} = \ang{kw}{w'},
	\end{equation*}
e, se $c<0$ e $k<0$, então
	\begin{equation*}
	\ang{cv}{v'} = \tau \div 2 - \ang{v}{v'} = \tau \div 2 - \ang{w}{w'} = \ang{kw}{w'},
	\end{equation*}
portanto em ambos os casos vale
	\begin{equation*}
	\ang{cv}{v'} = \ang{kw}{w'}.
	\end{equation*}

Assim, como $v' = cv + (v'-cv)$, $w' = kw + (w'-kw)$, e $\nor{v'}=\nor{w'}$, $\ang{cv}{v'} = \ang{kw}{w'}$ e $\ang{v}{v'-cv} = \tau \div 4 = \ang{w}{w'-kw}$, segue do critério LAA que $\nor{cv}=\nor{kw}$. Como $\nor{v}=\nor{w}$, segue que
	\begin{equation*}
	\abs{c} = \frac{\nor{cv}}{\nor{v}} = \frac{\nor{kw}}{\nor{w}} = \abs{k},
	\end{equation*}
portanto $k=c$ ou $k=-c$. Mas sabemos que eles têm o mesmo sinal, portanto $c=k$.
\end{proof}

\begin{proposition}
\label{prop:geo.cperpproporcao}
Seja $(\bm V, \nor{\var},\angf)$ um espaço angulado. Para todos $v,v' \in V \setminus \{0\}$ linearmente independentes e $c,c' \in \intaa{0}{\infty}$
	\begin{equation*}
	c_{\perp}(cv,c'v') = \frac{c'}{c}c_{\perp}(v,v').
	\end{equation*}
\end{proposition}
\begin{proof}
Como
	\begin{equation*}
	v'-c_{\perp}(v,v')v = \frac{1}{c'}\left( c'v'-\frac{c'}{c}c_{\perp}(v,v')cv \right),
	\end{equation*}
segue da $0$-homogeneidade que
	\begin{align*}
		\ang{c'v'}{v'-c_{\perp}(v,v')cv} &= \frac{\tau}{4} \\
		&= \ang{v'}{v'-c_{\perp}(v,v')v} \\
		&= \ang{c'v'}{\frac{1}{c'}\left( c'v'-\frac{c'}{c}c_{\perp}(v,v')cv \right)} \\
		&= \ang{c'v'}{c'v'-\frac{c'}{c}c_{\perp}(v,v')cv}.
	\end{align*}
Da unicidade de $c_{\perp}(cv,c'v')$ segue que
	\begin{equation*}
	c_{\perp}(cv,c'v') = \frac{c'}{c}c_{\perp}(v,v').
	\end{equation*}
\end{proof}

\begin{proposition}
\label{prop:geo.cperpfuncaoangulo}
Seja $(\bm V, \nor{\var},\angf)$ um espaço angulado. Para todos $v,v',w,w' \in V \setminus \{0\}$ tais que $\ang{v}{v'}=\ang{w}{w'}$,
	\begin{equation*}
	\frac{\nor{v}}{\nor{v'}}c_{\perp}(v,v') = \frac{\nor{w}}{\nor{w'}}c_{\perp}(w,w').
	\end{equation*}
\end{proposition}
\begin{proof}
Como
	\begin{equation*}
	\nor{\frac{v}{\nor{v}}} = 1 = \nor{\frac{w}{\nor{w}}}, \qquad \nor{\frac{v'}{\nor{v'}}} = 1 = \nor{\frac{w'}{\nor{w'}}}
	\end{equation*}
e, pela $0$-homogeneidade,
	\begin{equation*}
	\ang{\frac{v}{\nor{v}}}{\nor{\frac{v'}{\nor{v'}}}} = \ang{v}{v'} = \ang{w}{w'} = \ang{\frac{w}{\nor{w}}}{\nor{\frac{w'}{\nor{w'}}}},
	\end{equation*}
segue da proposição~\ref{prop:geo.cperpfuncaonormaangulo} que
	\begin{equation*}
	c_{\perp}\left( \frac{v}{\nor{v}},\frac{v'}{\nor{v'}} \right) = c_{\perp}\left( \frac{w}{\nor{w}},\frac{w'}{\nor{w'}} \right).
	\end{equation*}
Assim, segue da proposição~\ref{prop:geo.cperpproporcao} que
	\begin{align*}
	\frac{\nor{v}}{\nor{v'}}c_{\perp}(v,v') &= c_{\perp}\left( \frac{v}{\nor{v}},\frac{v'}{\nor{v'}} \right) \\
		&= c_{\perp}\left( \frac{w}{\nor{w}},\frac{w'}{\nor{w'}} \right) \\
		&= \frac{\nor{w}}{\nor{w'}}c_{\perp}(w,w').
	\end{align*}
\end{proof}

Essa proposição mostra que $\frac{\nor{v}}{\nor{v'}}c_{\perp}(v,v')$ só depende de $\ang{v}{v'}$, ou seja, que existe uma função
	\begin{align*}
	\func{\ncos}{\intaa{0}{\tau \div 2}}{\R}{a}{\ncos(a)}
	\end{align*}
tal que
	\begin{equation*}
	\ncos(\ang{v}{v'}) = \frac{\nor{v}}{\nor{v'}}c_{\perp}(v,v').
	\end{equation*}

 Mostraremos que essa função é $\ncos = \cos$.

































\cleardoublepage


\begin{proposition}
Seja $(\bm V, \nor{\var},\angf)$ um espaço angulado.
	\begin{enumerate}
		\item (Retificação) Para todos $v,v' \in V \setminus \{0\}$, $\ang{v}{v'}=\ang{-v}{v'}$ se, e somente se,
			\begin{equation*}
				\ang{v}{v'} = \frac{\tau}{4}.
			\end{equation*}
		
		\item (Alternos internos) Para todos $v,v' \in V \setminus \{0\}$,
			\begin{equation*}
				\ang{v}{v'} = \ang{-v}{-v'}.
			\end{equation*}
		
		\item (Monotonicidade complementar) Para todos $v,v' \in V \setminus \{0\}$ linearmente independentes e $c,c' \intaa{0}{\infty}$ tais que $c < c'$,
		\begin{equation*}
			\ang{v}{v+cv'} < \ang{v}{v+c'v'}.
		\end{equation*}
		
		\item (Monotonicidade suplementar) Para todos $v,v' \in V \setminus \{0\}$ linearmente independentes e $c,c' \intaa{0}{\infty}$ tais que $c < c'$,
%			\begin{equation*}
%				\ang{v}{v'} + \ang{v'}{v'+c(-v)} < \ang{v}{v'} + \ang{v'}{v'+c'(-v)}.
%			\end{equation*}
			\begin{equation*}
				\ang{v}{v'+c(-v)} < \ang{v}{v'+c'(-v)}.
			\end{equation*}

		\item Para todos $v,v' \in V \setminus \{0\}$ e $c \intaa{0}{\infty}$,
			\begin{equation*}
				\lim_{c \to \infty} \ang{v}{v+cv'} = \ang{v}{v'};
			\end{equation*}
%			\begin{equation*}
%				\lim_{c \to \infty} \ang{v}{\frac{v+cv'}{\nor{v+cv'}}} = \ang{v}{v'}.
%			\end{equation*}
		\item Para todos $v,v' \in V \setminus \{0\}$ e $c \intaa{0}{\infty}$,
		\begin{equation*}
			\lim_{c \to \infty} \ang{v}{v'+c(-v)} = \ang{v}{-v};
		\end{equation*}

%		\item Para todo $v, v' \in V \setminus \{0\}$ linearmente independentes, a função
%			\begin{align*}
%				\func{\alpha}{\intff{0}{\tau \div 2}}{\intff{0}{\tau \div 2}}{t}{\ang{v}{c(t)v + c'(t)v'}}
%			\end{align*}
%		em que $a := \ang{v}{v'}$,
%			\begin{equation*}
%				c(t) := \cos(t)+\sin(t)\cot(a)
%			\end{equation*}
%		e
%			\begin{equation*}
%				c'(t) := -\sin(t)\csc(a)
%			\end{equation*}
%		é crescente.
	\end{enumerate}
\end{proposition}
\begin{proof}
	\begin{enumerate}
		\item Se $\ang{v}{v'}=\ang{-v}{v'}$, segue da suplementação que
			\begin{equation*}
				\ang{v}{v'} = \frac{2\ang{v}{v'}}{2} = \frac{\ang{v}{v'}+\ang{-v}{v'}}{2} = \frac{\tau}{4}.
			\end{equation*}

			Reciprocamente, se $\ang{v}{v'}=\tau \div 4$, segue da suplementação que
				\begin{equation*}
					\ang{v}{v'} = \frac{\tau}{4} = \frac{\tau}{2} - \ang{v}{v'} = \ang{-v}{v'}.
				\end{equation*}
		
		\item Segue da suplementação e da simetria que
				\begin{equation*}
					\ang{v}{v'} = \frac{\tau}{2} - \ang{-v}{v'} = \frac{\tau}{2} - \left(\frac{\tau}{2} - \ang{-v}{-v'}\right) = \ang{-v}{-v'}.
				\end{equation*}
		
		\item Como $c<c'$, então para todo $k \in \intaa{0}{\infty}$ temos $v+cv' \neq k(v+c'v')$, portanto da separação e da positividade do ângulo segue que $0 < \ang{v+cv'}{v+c'v'}$. Definindo $k := 1 - c \div c'$ e $k' := c \div c'$, temos que $k'>0$, pois $c,c'>0$, e $k>0$, pois $c<c'$, portanto
				\begin{equation*}
					v+cv' = (k+k')v + (k'c')v'= kv+k'(v+c'v').
				\end{equation*}

		Assim da aditividade do ângulo segue que
				\begin{align*}
					\ang{v}{v+cv'} &< \ang{v}{v+cv'} + \ang{v+cv'}{v+c'v'} \\
						&= \ang{v}{v+c'v'}
				\end{align*}
		
		\item Análogo ao item anterior.

		\item Como $\angf$ é homogênea e contínua,
				\begin{equation*}
					\ang{v}{v+cv'} = \ang{v}{\frac{v+cv'}{\nor{v+cv'}}} \conv \ang{v}{v'}.
				\end{equation*}
		
		\item Análogo ao item anterior.
	\end{enumerate}
\end{proof}

\paragraph{Aditividade do produto interno} Sejam $v,v',v'' \in V$. Queremos mostrar que
					\begin{equation*}
						\inte{v+v'}{v''} = \inte{v}{v''} + \inte{v'}{v''}.
					\end{equation*}
				Como supomos que $v''$ é não nulo, isso é equivalente a
					\begin{equation*}
						\nor{v+v'}\cos(\ang{v+v'}{v''}) = \nor{v}\cos(\ang{v}{v''}) + \nor{v'}\cos(\ang{v'}{v''}).
					\end{equation*}
				Separamos em casos.
					\begin{enumerate}
						\item ($v$ e $v'$ são linearmente dependentes) Nesse caso, para algum $c \in \R \setminus \{0\}$, $v'=cv$. O caso em que $c=0$ não ocorre, pois estamos supondo $v' \neq 0$. Da homogeneidade do produto interno, provada no subitem anterior,
							\begin{align*}
								\inte{v+v'}{v''} &= \inte{v+cv}{v''} \\
									&= \inte{(1+c)v}{v''} \\
									&= (1+c)\inte{v}{v''} \\
									&= \inte{v}{v''} + c\inte{v}{v''} \\
									&= \inte{v}{v''} + \inte{cv}{v''} \\
									&= \inte{v}{v''} + \inte{v'}{v''}.
							\end{align*}
						
						\item ($v$ e $v'$ são linearmente independentes e $v''$ é gerado por eles) Nesse caso, existem $c,c' \in \R$ tais que $v'' = cv+c'v' \neq 0$.
						
						TERMINAR
						
						\item ($v$, $v'$ e $v''$ são linearmente independentes)
						
						TERMINAR

					\end{enumerate}



\paragraph{Uma possível generalização multidimensional}


\begin{definition}
Seja $(\bm V, \nor{\var})$ um espaço normado. Uma \emph{função ângulo $3$-dimensional} em $(\bm V, \nor{\var})$ é uma função contínua
	\begin{equation*}
	\fun{\angf}{V \setminus \{0\} \times V \setminus \{0\} \times V \setminus \{0\}}{\intff{0}{\tau}}
	\end{equation*}
tal que
	\begin{enumerate}
		\item (Homogeneidade) Para todos $v,v',v'' \in V$ e $c,c',c'' \in \intaa{0}{\infty}$,
			\begin{equation*}
				\angf(cv,c'v',c''v'') = \angf(v,v',v'');
			\end{equation*}
		\item (Separação) Para todos $v,v',v'' \in V \setminus \{0\}$,
			\begin{equation*}
				\angf(v,v',v'') = 0
			\end{equation*}
		se, e somente se, $\{v,v',v''\}$ é conicamente dependente (existem $c,c',c'' \in \intaa{0}{\infty}$ tais que $cv+c'v'+c''v''=0$);
		\item (Simetria) Para todos $v,v',v'' \in V \setminus \{0\}$ e bijeção $\fun{f}{\{v,v',v''\}}{\{v,v',v''\}}$,
			\begin{equation*}
				\angf(f(v),f(v'),f(v'')) = \angf(v,v',v'');
			\end{equation*}
		\item (Suplementação) Para todos $v,v',v'' \in V \setminus \{0\}$,
			\begin{equation*}
				\angf(v,v',v'') + \angf(-v,v',v'') + \angf(-v,-v',v'') + \angf(v,-v',v'') = \tau;
			\end{equation*}
		\item (Aditividade) Para todos $v,v',v'' \in V \setminus \{0\}$ e $c,c',c'' \in \intaa{0}{\infty}$ tais que $cv+c'v'+c''v'' \neq 0$,
			\begin{align*}
				\angf(v,v',v'') =& \angf(v,v',cv+c'v'+c''v'') \\
					&+ \angf(v,cv+c'v'+c''v'',v'') \\
					&+ \angf(cv+c'v'+c''v'',v',v'');
			\end{align*}
		\item (Ângulo interno) Para todos $v,v',v'',w,w',w'' \in V \setminus \{0\}$ tais que $v+v' \neq 0$, $w+w' \neq 0$, $\nor{v}=\nor{w}$, $\nor{v'}=\nor{w'}$, $\nor{v''}=\nor{w''}$, $\nor{v+v'}=\nor{w+w'}$, $\nor{v'+v''}=\nor{w'+w''}$ e $\angf(v,v',v'')=\angf(w,w',w'')$, então
			\begin{enumerate}
				\item $\nor{v+v''}=\nor{w+w''}$;
				\item $\angf(v,v',v'')=\angf(w,w',w'')$;
			\end{enumerate}
	\end{enumerate}
\end{definition}














\cleardoublepage


\subsection{Espaço projetivo}

O espaço projetivo é o espaço de retas pela origem. Ele pode ser descrito de diferentes formas. Quando $V=\R^d$ (ou um espaço normado), podemos descrevê-lo como o quociente da esfera unitária pela antípoda $-\Id$. Quando $\bm V$ é um espaço com produto interno, podemos induzir em $\pro V$ uma distância, o ângulo entre as retas de $V$. De qualquer forma, esse espaço é um espaço topológico, como será também descrito a seguir.

\subsubsection{Estrutura topológica}

A relação de paralelismo $\parallel$ é uma equivalência não só em $V$, mas em $V \setminus \{0\}$. Além disso, ela não depende de produto interno, está definida para qualquer espaço linear $\bm V$ sobre um corpo $\bm C$ qualquer. Isso permite que se quociente $V \setminus \{0\}$ por $\parallel$, e esse é o \textit{espaço projetivo} de $\bm V$.

\begin{definition}
Seja $\bm V$ um espaço linear sobre um corpo $\bm C$. O \emph{espaço projetivo} de $\bm V$ é o conjunto
	\begin{equation*}
	\pro V := \quo{\left( V \setminus \{0\} \right)}{\parallel}.
	\end{equation*}
Os elementos de $\pro V$ são as \emph{retas} de $\bm V$.
\end{definition}

Esse quociente pode ser entendido também como o quociente pela ação do grupo multiplicativo $C \setminus \{0\}$. Desse modo, temos que
	\begin{equation*}
	\pro V = \quo{\left( V \setminus \{0\} \right)}{(C \setminus \{0\})}.
	\end{equation*}

O caso em que $\bm V$ é um espaço normado nos permite dar mais uma definição equivalente desse espaço. Lembremos que $\S^0 = \{1,-1\}$ é o grupo discreto multiplicativo com $2$ elementos e é subgrupo do grupo multiplicativo $C \setminus \{0\}$. Se $\bm V$ é um espaço normado, a esfera $\S V$ de $\bm V$ está definida. Nesse caso, $\S^0$ age em $\S V$ e o espaço quociente $\quo{\S}{\S^0}$ pode ser identificado com $\pro V$ de um jeito bem natural, de modo que as estruturas definidas em um possam sempre ser passadas para o outro. Temos
	\begin{equation*}
	\pro V \simeq \quo{\S V}{\S^0}.
	\end{equation*}

O isomorfismo é dado por
	\begin{align*}
	\func{h}{\pro V}{\quo{\S V}{\S^0}}{[v]}{\left\{\frac{v}{\nor{v}},-\frac{v}{\nor{v}}\right\}}
	\end{align*}
com inversa
	\begin{align*}
	\func{h\inv}{\quo{\S V}{\S^0}}{\pro V}{\{u,-u\}}{\set{cu}{c \in C \setminus \{0\}}}.
	\end{align*}

\subsubsection{Estrutura métrica}

Quando o espaço linear tem um produto interno, a função ângulo pode ser usada para definir uma função distância no espaço projetivo.

\begin{definition}
Seja $(\bm V,\inte{\var}{\var})$ um espaço com produto interno. O \emph{ângulo} entre retas $r$ e $r'$ de $\bm V$ é o menor ângulo entre vetores de $r$ e $r'$
	\begin{equation*}
	\ang{r}{r'} := \min_{v \in r, v' \in r'} \ang{v}{v'}.
	\end{equation*}
\end{definition}

Mostremos como achar uma expressão para $\ang{r}{r'}$, o que facilitará a demonstração de que essa função é de fato uma distância. Queremos mostrar que
	\begin{equation*}
	\ang{r}{r'} = \cos\inv \left( \frac{\abs{\inte{v}{v'}}}{\nor{v}\nor{v'}} \right),
	\end{equation*}
em que $v \in r, v' \in r'$. Para todos $\bar v \in r$ e $\bar v' \in r'$, existem escalares $c,c' \in C \setminus \{0\}$ tais que $\bar v = cv$ e $\bar v' = c'v'$. Portanto
%	\begin{align*}
%	\cos\inv \left( \frac{\abs{\inte{\bar v}{\bar v'}}}{\nor{\bar v}\nor{\bar v'}} \right) &= \cos\inv \left( \frac{\abs{\inte{cv}{c'v'}}}{\nor{cv}\nor{c'v'}} \right) \\
%		&= \cos\inv \left( \frac{\abs{cc'}}{\abs{c}\abs{c'}}\frac{\abs{\inte{v}{v'}}}{\nor{v}\nor{v'}} \right) \\
%		&= \cos\inv \left( \frac{\abs{\inte{v}{v'}}}{\nor{v}\nor{v'}} \right),
%	\end{align*}
%o que mostra que a expressão é invariante por representante da reta.
%Agora, notemos que, para todos $v \in r, v' \in r'$,
	\begin{align*}
	\ang{r}{r'} &= \min_{\bar v \in r, \bar v' \in r'} \ang{\bar v}{\bar v'} \\
		&= \min_{c,c' \in C \setminus \{0\}} \ang{cv}{cv'} \\
		&= \min \left\{ \ang{v}{v'}, \frac{\tau}{2} - \ang{v}{v'} \right\}.
	\end{align*}
Como $\ang{v}{v'} \in \intff{0}{\frac{\tau}{2}}$, segue que
	\begin{equation*}
	\min \left\{ \ang{v}{v'}, \frac{\tau}{2} - \ang{v}{v'} \right\} \in \intff{0}{\frac{\tau}{4}}.
	\end{equation*}

Assim, escolhendo $v \in r$ e $v' \in r'$ tais que $\inte{v}{v'} \geq 0$, temos $\ang{v}{v'} \in \intff{0}{\frac{\tau}{4}}$, o que implica que
	\begin{equation*}
	\ang{r}{r'} = \min \left\{ \ang{v}{v'}, \frac{\tau}{2} - \ang{v}{v'} \right\} = \cos\inv \left( \frac{\abs{\inte{v}{v'}}}{\nor{v}\nor{v'}} \right).
	\end{equation*}

Podemos restringir ainda mais a escolha dos $v \in r$ e $v' \in r'$ notando que
	\begin{equation*}
	\cos\inv \left( \frac{\abs{\inte{v}{v'}}}{\nor{v}\nor{v'}} \right) = \cos\inv \left( \abs{\inte{\frac{v}{\nor{v}}}{\frac{v'}{\nor{v'}}}} \right)
	\end{equation*}
e que $\frac{v}{\nor{v}},\frac{v'}{\nor{v'}} \in \S$. Como existe $u \in V$ tal que $r \cap \S = \{u,-u\}$, podemos tomar $u \in r \cap \S$ e $u' \in r' \cap \S$ e temos que
	\begin{equation*}
	\ang{r}{r'} = \cos\inv (\abs{\inte{u}{u'}}).
	\end{equation*}

\begin{definition}
Seja $(\bm V,\inte{\var}{\var})$ um espaço com produto interno. A distância em $\pro V$ é a função
	\begin{align*}
	\func{\ang{\var}{\var}}{\pro V \times \pro V}{\intff{0}{\frac{\tau}{4}}}{(r,r')}{\ang{r}{r'}}.
	\end{align*}
\end{definition}

\begin{proposition}
Seja $(\bm V,\inte{\var}{\var})$ um espaço com produto interno. A função $\ang{\var}{\var}$ é uma distância em $\pro V$.
\end{proposition}
\begin{proof}
%%%%%%%%%%%%%%%%%%%%%%%%%%%%%%%%%%%%%%%%%%%%
\begin{comment}

%(Separação) Sejam $r \in \pro V$ e $u \in r \cap \S$. Então
%	\begin{equation*}
%	\ang{r}{r} = \cos\inv (\abs{\inte{u}{u}}) = \cos\inv (1) = 0.
%	\end{equation*}
%Reciprocamente, sejam $r,r' \in \pro V$, $u \in r \cap \S$ e $u' \in r' \cap \S$. Se $\ang{r}{r'} = 0$, então $\abs{\inte{u}{u'}}= 1$, o que implica que $u'=u$ ou $u'=-u$, logo $r=r'$.

%(Simetria) Sejam $r,r' \in \pro V$, $u \in r \cap \S$ e $u' \in r' \cap \S$. Então
%	\begin{equation*}
%	\ang{r}{r'} = \cos\inv (\abs{\inte{u}{u'}}) = \cos\inv (\abs{\inte{u'}{u}}) = \ang{u'}{u}.
%	\end{equation*}

(Desigualdade Triangular) Sejam $r,r',r'' \in \pro V$, $u \in r \cap \S$, $u' \in r' \cap \S$ e $u'' \in r'' \cap \S$. Queremos mostrar que
	\begin{equation*}
	\ang{r}{r''} \leq \ang{r}{r'} + \ang{r'}{r''}.
	\end{equation*}
Isso é o mesmo que	
	\begin{equation*}
	\cos\inv (\abs{\inte{u}{u''}}) \leq \cos\inv (\abs{\inte{u}{u'}}) + \cos\inv (\abs{\inte{u'}{u''}}).
	\end{equation*}
% A função $\cos\colon \intff{0}{\frac{\tau}{4}} \to \intff{0}{1}$ é decrescente, o que implica que

Se mostrarmos que
	\begin{equation*}
	\abs{\inte{u}{u''}} \geq \abs{\inte{u}{u'}} + \abs{\inte{u'}{u''}},
	\end{equation*}
segue da monotonicidade decrescente de $\cos\inv$.
	\begin{align*}
	\abs{\inte{u}{u'}+\inte{u'}{u''}} \leq \abs{\inte{u}{u'}} + \abs{\inte{u'}{u''}}
	\end{align*}
	\begin{align*}
	\inte{u}{u'}+\inte{u''}{u'} = \inte{u+u''}{u'}
	\end{align*}

\end{comment}
%%%%%%%%%%%%%%%%%%%%%%%%%%%%%%%%%%%%%%%%%%%%
(Separação) Sejam $r \in \pro V$ e $v \in r$. Então
	\begin{equation*}
	\ang{r}{r} = \cos\inv \left( \frac{\abs{\inte{v}{v}}}{\nor{v}\nor{v}} \right) = \cos\inv (1) = 0.
	\end{equation*}
Reciprocamente, sejam $r,r' \in \pro V$, $v \in r$ e $v' \in r'$. Se $\ang{r}{r'} = 0$, então $\frac{\abs{\inte{v}{v}}}{\nor{v}\nor{v}}= 1$, o que implica que $v'=v$ ou $v'=-v$, logo $r=r'$.
(Simetria) Sejam $r,r' \in \pro V$, $v \in r$ e $v' \in r'$. Então
	\begin{equation*}
	\ang{r}{r'} = \cos\inv \left( \frac{\abs{\inte{v}{v'}}}{\nor{v}\nor{v'}} \right) = \cos\inv \left( \frac{\abs{\conju{\inte{v'}{v}}}}{\nor{v'}\nor{v}} \right) = \cos\inv \left( \frac{\abs{\inte{v'}{v}}}{\nor{v'}\nor{v}} \right) = \ang{r'}{r}.
	\end{equation*}

(Desigualdade Triangular) Sejam $r,r',r'' \in \pro V$, $v \in r$, $v' \in r'$ e $v'' \in r''$. Queremos mostrar que
	\begin{equation*}
	\ang{r}{r''} \leq \ang{r}{r'} + \ang{r'}{r''}.
	\end{equation*}
%Isso é o mesmo que	
%	\begin{equation*}
%	\cos\inv \left( \frac{\abs{\inte{v}{v''}}}{\nor{v}\nor{v''}} \right) \leq \cos\inv \left( \frac{\abs{\inte{v}{v'}}}{\nor{v}\nor{v'}} \right) + \cos\inv \left( \frac{\abs{\inte{v'}{v''}}}{\nor{v'}\nor{v''}} \right).
%	\end{equation*}
%
% A função $\cos\colon \intff{0}{\frac{\tau}{4}} \to \intff{0}{1}$ é decrescente, o que implica que

Consideremos a projeção $\bar v$ de $v'$ no plano $\ger{\{v,v''\}}$. 	

Se mostrarmos que
	\begin{equation*}
	\abs{\inte{u}{u''}} \geq \abs{\inte{u}{u'}} + \abs{\inte{u'}{u''}},
	\end{equation*}
segue da monotonicidade decrescente de $\cos\inv$.
	\begin{align*}
	\abs{\inte{u}{u'}+\inte{u'}{u''}} \leq \abs{\inte{u}{u'}} + \abs{\inte{u'}{u''}}
	\end{align*}
	\begin{align*}
	\inte{u}{u'}+\inte{u''}{u'} = \inte{u+u''}{u'}
	\end{align*}
\end{proof}

\subsection{Funções ortogonais e conformes}

\begin{definition}
Sejam $(\bm V,\inte{\var}{\var})$ e $(\bm V',\inte{\var}{\var}')$ espaços com produto interno. Uma função \emph{ortogonal} de $\bm V$ para $\bm V'$ é uma função linear $f\colon V \to V'$ tal que, para todos $v,v' \in V$,
	\begin{equation*}
	\inte{f(v)}{f(v')}' = \inte{v}{v'}.
	\end{equation*}
O conjunto dessas funções é $\toplin_{\inte{}{}}(\bm V,\bm V')$.

Uma função \emph{conforme} de $\bm V$ para $\bm V'$ é uma função linear $f\colon V \to V'$ tal que, para todos $v,v' \in V$,
	\begin{equation*}
	\sphericalangle'(f(v),f(v')) = \ang{v}{v'}.
	\end{equation*}
O conjunto dessas funções é $\toplin_{\sphericalangle}(\bm V,\bm V')$.
\end{definition}

Note que, na definição de uma função conforme, está implícito que $\sphericalangle'(f(v),f(v'))$ existe, portanto $f(v) \neq 0$ para todo $v \in V \setminus \{0\}$, o que significa que $f$ é injetiva.

\begin{proposition}
Sejam $(\bm V,\inte{\var}{\var})$ e $(\bm V',\inte{\var}{\var}')$ espaços com produto interno sobre um corpo $\bm C$ de característica diferente de $2$ e $f\colon V \to V'$ uma função linear.
	\begin{enumerate}
	\item $f$ é ortogonal se, e somente se, é uma isometria local:
		\begin{equation*}
		\toplin_{\inte{}{}}(\bm V,\bm V') = \toplin_{\nor{}}(\bm V,\bm V');
		\end{equation*}
%	\item Se $f$ é ortogonal, então é uma isometria local:
%		\begin{equation*}
%		\toplin_{\inte{}{}}(\bm V,\bm V') \subseteq \toplin_{\nor{}}(\bm V,\bm V');
%		\end{equation*}
%	\item Se a característica de $\bm C$ é diferente de $2$ e $f$ é isometria local, então é ortogonal:
%		\begin{equation*}
%		\toplin_{\nor{}}(\bm V,\bm V') \subseteq \toplin_{\inte{}{}}(\bm V,\bm V');
%		\end{equation*}
%	\item Se $f$ é ortogonal, então é conforme:
%		\begin{equation*}
%		\toplin_{\inte{}{}}(\bm V,\bm V') \subseteq \toplin_{\sphericalangle}(\bm V,\bm V');
%		\end{equation*}
%	\item Se $f$ é isometria local e conforme, então é ortogonal
%		\begin{equation*}
%		\toplin_{\sphericalangle}(\bm V,\bm V') \cap \toplin_{\nor{}}(\bm V,\bm V') \subseteq \toplin_{\inte{}{}}(\bm V,\bm V')
%		\end{equation*}
	\item $f$ é conforme se, e somente se, existe $c \in \intaa{0}{\infty}$ tal que, para todos $v,v' \in V$,
		\begin{equation*}
		\inte{f(v)}{f(v')}' = c\inte{v}{v'}.
		\end{equation*}
	\end{enumerate}
\end{proposition}
\begin{proof}
	\begin{enumerate}
	\item Se $f$ é ortogonal, então para todo $v \in V$,
	\begin{equation*}
	\nor{f(v)}' = {\inte{f(v)}{f(v)}'}^\frac{1}{2} = {\inte{v}{v}}^\frac{1}{2} = \nor{v}.
	\end{equation*}
	
Reciprocamente, como a característica de $\bm C$ é diferente de $2$, vale que, para todos $v,v' \in V$,		
		\begin{equation*}
		\inte{v}{v'} = \frac{1}{4} \left( \left(\nor{v+v'}^2 - \nor{v-v'}^2\right) + \ii \left(\nor{v+\ii v'}^2 - \nor{v-\ii v'}^2\right) \right).
		\end{equation*}
Se $f$ é isometria local, então, para todos $v,v' \in V$,
	\begin{align*}
		\inte{f(v)}{f(v')} 	&= \frac{1}{4}\left(\nor{f(v)+f(v'))}^2 - \nor{f(v)-f(v'))}^2\right) \\
		&\qquad\qquad + \frac{\ii}{4} \left(\nor{f(v)+\ii f(v')}^2 - \nor{f(v)-\ii f(v')}^2\right) \\
		&= \frac{1}{4}\left(\nor{f(v+v')}^2 - \nor{f(v-v')}^2\right) \\
		&\qquad\qquad + \frac{\ii}{4}\left(\nor{f(v+\ii v')}^2 - \nor{f(v-\ii v')}^2\right) \ \\
		&= \frac{1}{4} \left( \left(\nor{v+v'}^2 - \nor{v-v'}^2\right) + \ii \left(\nor{v+\ii v'}^2 - \nor{v-\ii v'}^2\right) \right) \\
		&= \inte{v}{v'}.
	\end{align*}
	
%	\item Se o produto interno é real, então, para todos $v,v' \in V$,
%	\begin{align*}
%	\sphericalangle'(f(v),f(v')) = \cos\inv\left(\frac{\inte{f(v)}{f(v')}'}{\nor{f(v)}'\nor{f(v')}'}\right) = \cos\inv\left(\frac{\inte{v}{v'}}{\nor{v}\nor{v'}}\right) = \sphericalangle(v,v').
%	\end{align*}

%	Se o produto interno é complexo,
%	\begin{align*}
%	\sphericalangle'(f(v),f(v')) = \cos\inv\left(\frac{\abs{\inte{f(v)}{f(v')}'}}{\nor{f(v)}'\nor{f(v')}'}\right) = \cos\inv\left(\frac{\abs{\inte{v}{v'}}}{\nor{v}\nor{v'}}\right) = \sphericalangle(v,v').
%	\end{align*}
	
%	\item Como $f$ é conforme, para todos $v,v' \in V$ vale
%		\begin{equation*}
%		\sphericalangle'(f(v),f(v')) = \ang{v}{v'}.
%		\end{equation*}
%Como $\cos\inv$ é bijeção, então
%	\begin{equation*}
%	\frac{\inte{f(v)}{f(v')}'}{\nor{f(v)}'\nor{f(v')}'} = \frac{\inte{v}{v'}}{\nor{v}\nor{v'}}.
%	\end{equation*}
%Como $f$ é ismometria local, $\nor{f(v)}' = \nor{v}$ e $\nor{f(v')}' = \nor{v'}$, logo
%	\begin{equation*}
%	\inte{f(v)}{f(v')}' = \inte{v}{v'}.
%	\end{equation*}
	
	\item Se $f$ é conforme, para todos $v,v' \in V$ vale
		\begin{equation*}
		\sphericalangle'(f(v),f(v')) = \ang{v}{v'}.
		\end{equation*}
Como $\cos\inv$ é bijeção, então
	\begin{equation*}
	\frac{\inte{f(v)}{f(v')}'}{\nor{f(v)}'\nor{f(v')}'} = \frac{\inte{v}{v'}}{\nor{v}\nor{v'}},
	\end{equation*}
o que implica
	\begin{equation*}
	\inte{f(v)}{f(v')}' = \frac{\nor{f(v)}'\nor{f(v')}'}{\nor{v}\nor{v'}}\inte{v}{v'}.
	\end{equation*}
Resta mostrar que $\frac{\nor{f(v)}'\nor{f(v')}'}{\nor{v}\nor{v'}}$ é constante. Note que isso ocorre se, e somente se, $N(v) := \frac{\nor{f(v)}'}{\nor{v}}$ é constante. Para mostrar que essa função é constante, vamos mostrar que sua diferencial é nula. Sejam $v \in V \setminus \{0\}$ e $h \in V$. Como $f$ é linear e $\D \nor{v}(h) = \frac{\inte{v}{h}}{\nor{v}}$ é a diferencial\footnote{Não tenho certeza, mas acredito que isso dependa de a característica de $\bm C$ ser diferente de $2$, pois quando calculamos a diferencial pela regra da cadeia cancelamos fatores de $2$.} de $\nor{\var}$, a diferencial de $N$ é
	\begin{align*}
	\D N|_v (h) &= \frac{\nor{v} \D \nor{f(v)}'(h) - \nor{f(v)}'\D \nor{v}(h)}{\nor{v}^2} \\
		&= \frac{\frac{\nor{v}}{\nor{f(v)}'} \inte{f(v)}{\D f(v)(h)}' - \frac{\nor{f(v)}'}{\nor{v}}\inte{v}{h}}{\nor{v}^2} \\
		&= \frac{\nor{v}^2\inte{f(v)}{f(h)}' - {\nor{f(v)}'}^2\inte{v}{h}}{\nor{f(v)}'\nor{v}^3}.
	\end{align*}
Notemos que, como $f$ preserva ângulo, então se $v \perp v'$ segue que $f(v) \perp f(v')$. Escrevendo $h = \proj_{\parallel v}(h) + \proj_{\perp v}(h) = h^\parallel + h^\perp$, temos que $h^\perp \perp v$, e $h^\parallel \parallel v$ ou $h^\parallel = 0$, logo
	\begin{equation*}
	\D N|_v (h) = \D N|_v (h^\parallel + h^\perp) =  \D N|_v (h^\parallel) + \D N|_v (h^\perp).
	\end{equation*}
Calculemos $\D N|_v (h^\parallel)$.
%Se $h^\parallel = 0$, então $f(h^\parallel) = 0$ e portanto $\inte{v}{h^\parallel} = \inte{f(v)}{f(h^\parallel)} = 0$, logo $\D N|_v (h^\parallel)$. Caso contrário, $h^\parallel \parallel v$, ou seja, 
Como $h^\parallel = cv$, com $c=\frac{\inte{h}{v}}{\nor{v}^2}$, segue que
	\begin{align*}
	\D N|_v (h^\parallel) &= \frac{\nor{v}^2\inte{f(v)}{f(cv)}' - {\nor{f(v)}'}^2\inte{v}{cv}}{\nor{f(v)}'\nor{v}^3} \\
		&= \frac{c\nor{v}^2\inte{f(v)}{f(v)}' - c{\nor{f(v)}'}^2\inte{v}{v}}{\nor{f(v)}'\nor{v}^3} \\
		&= \frac{c\nor{v}^2{\nor{f(v)}'}^2 - c{\nor{f(v)}'}^2\nor{v}^2}{\nor{f(v)}'\nor{v}^3} \\
		&= 0.
	\end{align*}	
Calculemos agora $\D N|_v (h^\perp)$. Como $\inte{h^\perp}{v} = 0$, então $\inte{f(v)}{f(h^\perp)} = 0$, logo
	\begin{equation*}
	\D N|_v (h^\perp) = \frac{\nor{v}^2\inte{f(v)}{f(h^\perp)}' - {\nor{f(v)}'}^2\inte{v}{h^\perp}}{\nor{f(v)}'\nor{v}^3} = 0.
	\end{equation*}
Assim, concluímos que $\D N|_v (h) = \D N|_v (h^\parallel) + \D N|_v (h^\perp) = 0$, ou seja, $\D N|_v = 0$ para todo $v \in V \setminus \{0\}$, o que implica que existe $c' \in \R$ tal que, para todo $v \in V \setminus \{0\}$, $N(v) = \frac{\nor{f(v)}}{\nor{v}} = c'$. Como $f$ preserva ângulos, é injetiva, portanto $f(v)=0$ se, e somente se, $v=0$, o que significa que $c' \neq 0$. Definindo $c := (c')^2$, segue que $c \in \intaa{0}{\infty}$, portanto
	\begin{equation*}
	\inte{f(v)}{f(v')}' = \frac{\nor{f(v)}'\nor{f(v')}'}{\nor{v}\nor{v'}}\inte{v}{v'} = (c')^2\inte{v}{v'} = c\inte{v}{v'}.
	\end{equation*}
	
Reciprocamente, se existe $c \in \intaa{0}{\infty}$ tal que
	\begin{equation*}
	\inte{f(v)}{f(v')}' = c\inte{v}{v'},
	\end{equation*}
então $\nor{f(v)} = \inte{f(v)}{f(v)}^{\frac{1}{2}} = c^{\frac{1}{2}}\inte{v}{v'}^{\frac{1}{2}} = c^{\frac{1}{2}}\nor{v}$ e segue que
	\begin{align*}
	\sphericalangle' \left( f(v),f(v') \right) &= \cos\inv \left( \frac{\inte{f(v)}{f(v')}'}{\nor{f(v)}'\nor{f(v')}'} \right) \\
		&= \cos\inv \left( \frac{c\inte{v}{v'}}{c\nor{v}\nor{v'}} \right) \\
		&= \ang{v}{v'}. \qedhere
	\end{align*}
	\end{enumerate}
\end{proof}



\section{Espaço de funções quadrado somáveis}

Sejam $X$ um conjunto e $\bm C \subseteq \C$. Para todo $p \in \intff{1}{\infty}$, o espaço $\Smvl^p(\bm X,\bm C)$ é um espaço normado completo, mas somente para $p=2$ esse espaço admite um produto interno. As funções absolutamente $2$-somáveis do espaço $Smvl^2(\bm X,\bm C)$ são também chamadas de \emph{funções quadrado somáveis}. Note que, diferente do caso mais geral de funções absolutamente $p$-somáveis, aqui fixamos um subcorpo dos números complexos para usar o conjugação complexa. A generalização do complexo conjugado é mais difícil e detalhada e não será feita aqui.

\begin{definition}
Sejam $X$ um conjunto e $\bm C \subseteq \C$ um corpo. O \emph{produto interno} entre $f,f' \in \Smvl^2(\bm X,\bm C)$ é
	\begin{equation*}
	\inte{f}{f'} := \left( \sum_{x \in X} f(x)\overline{f'(x)} \right)^{2\inv}.
	\end{equation*}
\end{definition}

\begin{proposition}
Sejam $X$ um conjunto e $\bm C \subseteq \C$ um corpo. O espaço $(\Smvl^2(\bm X,\bm C),\inte{\var}{\var})$ é um espaço com produto interno completo.
\end{proposition}