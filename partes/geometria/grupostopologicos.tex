\chapter{Grupos topológicos}

\section{Grupo topológico}

\begin{definition}
Um \emph{grupo topológico} é uma quádrupla $\bm G = ((G,\topo),\times,\div,\id)$ em que $(G,\topo)$ é um espaço topológico, $(G,\times,\div,\id)$ é um grupo e as operações
	\begin{equation*}
	\times\colon G^2 \to G \qquad\text{\ \ e\ \ }\qquad \div\colon G \to G
	\end{equation*}
são funções contínuas.% Um grupo \emph{discreto} é um grupo cuja topologia é discreta.
\end{definition}

Por simplicidade, denotamos $gg' := g \times g'$ e $\div g := g\inv$. Note que não é necessária nenhuma condição sobre a identidade $\id \in G$, pois se vista como uma operação zerária $\id\colon \{0\} \to G$, ela é contínua em qualquer topologia de $G$, pois $\{0\}$ só admite a topologia trivial.

Lembremos que para um grupo $\bm G$ e $g \in G$, a conjugação por $g$ é a função
	\begin{align*}
	\func{\con{g}}{G}{G}{h}{ghg\inv},
	\end{align*}
a translação à direita por $g$ é a função
	\begin{align*}
	\func{\trD{g}}{G}{G}{h}{hg}
	\end{align*}
e a translação à esquerda por $g$ é a função
	\begin{align*}
	\func{\trE{g}}{G}{G}{h}{gh}.
	\end{align*}

\begin{proposition}
Seja $\bm G$ um grupo topológico. As funções $\con{g}$, $\trD{g}$ e $\trE{g}$ são homeomorfismos para todo $h \in G$.
\end{proposition}
\begin{proof}
Para ver que as funções são bijeções, basta notar que $(\con{g})\inv = \con{g \inv}$, $(\trD{g})\inv = \trD{g \inv}$ e $(\trE{g})\inv = \trE{g \inv}$. Para mostrar a continuidade, basta notar que da continuidade de $\times$ e das relações $\trE{g} = \times \circ (\var,g)$ e $\trD{g} = \times \circ (g,\var)$, segue que $\trE{g}$ e $\trD{g}$ são contínuas, e que da continuidade de $\div$ e da relação $\con{g} = \trD{g} \circ \trE{g\inv}$, segue que $\con{g}$ é contínua.
\end{proof}

\begin{example}
Sejam $\bm X$ um espaço topológico e $\bm G$ um grupo topológico. Consideremos o conjunto $\Cont(X,G)$ das funções contínuas de $\bm X$ para $\bm G$ e as operações induzidas pontualmente de $G$ em $\Cont(X,G)$ por $(f \times g)(x) := f(x) \times g(x)$, $(f\inv)(x) := (f(x))\inv$ e $\id(x) := \id$. A quádrupla $(\Cont(X,G),\times,\div,\id)$ é um grupo. Consideramos agora a topologia \emph{compacto-aberto} $\topo$ gerada pelos conjuntos
	\begin{equation*}
	\mathscr A_{K,A} := \set{f \in \Cont(X,G)}{f(K) \subseteq A}
	\end{equation*}
em que $K \subseteq X$ é um compacto e $A \subseteq G$ é um aberto. Então $((\Cont(X,G),\topo),\times,\div,\id)$ é um grupo topológico.
\end{example}

\begin{proposition}
Seja $\bm G$ um grupo topológico.
	\begin{enumerate}
	\item Para todos $g \in G$ e $A \subseteq G$ aberto, $gA$ e $Ag$ são abertos;
	\item Para todos $g \in G$ e $A \subseteq G$ fechado, $gA$ e $Ag$ são fechados;
	\item Para todos $A \subseteq G$ aberto e $C \subseteq G$, $AC$ e $CA$ são abertos;
	\item Para todos $F \subseteq G$ fechado e $K \subseteq G$ compacto, $FK$ e $KF$ são fechados.
	\end{enumerate}
\end{proposition}
\begin{proof}
	\begin{enumerate}
	\item Segue do fato de que $\trE{g}$ e $\trD{g}$ são homeomorfismos e $gA = \trE{g}(A)$, $Ag = \trD{g}(A)$.
	\item Segue do fato de que $\trE{g}$ e $\trD{g}$ são homeomorfismos e $gA = \trE{g}(A)$, $Ag = \trD{g}(A)$.
	\item Segue do fato de que
		\begin{equation*}
		AC = \bigcup_{c \in C} Ac \text{\ \ e\ \ } CA = \bigcup_{c \in C} cA
		\end{equation*}
e de que os conjuntos $cA$ e $Ac$ são abertos para todo $c \in C$.
	\item Seja $x \in \Fec{FK}$. Então existe uma rede $(x_\lambda)_{\lambda \in \Lambda} = (f_\lambda k_\lambda)$ tal que $x = \lim_{\lambda \in \Lambda} x_\lambda$. Como $K$ é compacto, existe uma sub-rede $k_{\lambda_\mu}$ tal que $k := \lim_{\mu \in M} k_{\lambda_\mu} \in K$, o que implica que $k\inv = \lim_{\mu \in M} {k_{\lambda_\mu}}\inv$ pela continuidade da inversa. Pela continuidade do produto, segue que a sub-rede $f_{\lambda_\mu} = (f_{\lambda_\mu}k_{\lambda_\mu}){k_{\lambda_\mu}}\inv$ converge para $f = xk\inv$, e $f \in F$ pois $F$ é fechado. Assim segue que $x=fk$, portanto $FK$ é fechado. A demonstração é análoga para $KF$.
	\end{enumerate}
\end{proof}

Para o exemplo a seguir, seja $m \in \N \setminus \{0\}$ e lembremos que um $m$-multi-índice é um elemento $i \in I$ e sua norma é norma $1$ dada por
	\begin{equation*}
	\abs{i} = \abs{i_0} + \cdots + \abs{i_{m-1}} \in \N.
	\end{equation*}
Além disso, para um grupo $\bm G = (G,+,-,0)$, o suporte de uma função $\fun{s}{I}{G}$, que é um elemento de $s \in G^{I}$, é o conjunto de multi-índices $i \in I$ tais que $s_i \neq 0$, ou seja,
	\begin{equation*}
	\supp(s) = \set{i \in I}{s_i \neq 0}.
	\end{equation*}
A ordem de uma função $s \in I \setminus \{0\}$ é
	\begin{equation*}
	\ords{s} := \min\set{\abs{i}}{i \in \supp(s)},
	\end{equation*}
a menor norma de um índice de uma entrada não $0$. A ordem de $0 \in I$ pode ser definida como $\infty$, mas não usaremos essa definição explicitamente, embora ela seja compatível com a definição de distância a seguir. As mesmas definições valem analogamente para qualquer $I \subseteq I$.

%\begin{proposition}
%Sejam $\bm G = (G,+,-,0)$ um grupo comutativo com a topologia discreta. O grupo produto $\bm G^{\N}$ com a topologia produto é um grupo topológico e sua topologia é dada pela distância
%	\begin{align*}
%	\func{\dist{\var}{\var}}{G^{\N} \times G^{\N}}{\R}{(s,s')}{
%		\dist{x}{x'} :=
%			\begin{cases}
%			0, & s = s'\\
%			2^{-\ords{s-s'}}, & s \neq s',
%			\end{cases}
%	}
%	\end{align*}
%em que $\ords{s} = \min\supp(s) = \min\set{n \in \N}{s_n \neq 0}$.
%\end{proposition}

\begin{proposition}
\label{prop:grupo.comut.discreto.metrica}
Sejam $\bm G = (G,+,-,0)$ um grupo comutativo discreto, $m \in \Z \setminus \{0\}$, $I \subseteq \Z^m$ e $\alpha \in \intaa{0}{1}$. A topologia produto de $\bm G^{I}$ é dada pela distância
	\begin{align*}
	\func{\dist{\var}{\var}_\alpha}{G^{I} \times G^{I}}{\R}{(s,s')}{
		\dist{s}{s'}_\alpha :=
			\begin{cases}
			0, & s = s'\\
			\alpha^{\ords{s'-s}}, & s \neq s'.
			\end{cases}
	}
	\end{align*}
\end{proposition}
\begin{proof}
Mostremos que $\dist{\var}{\var}_\alpha$ é uma distância.
	\begin{itemize}
		\item (Separação) Sejam $s,s' \in G^{I}$. Se $s=s'$, por definição $\dist{s}{s'}_\alpha = 0$. Se $s \neq s'$, então $\dist{s}{s'}_\alpha = \alpha^{\ords{s'-s}}$ e, como $\alpha>0$, $\dist{s}{s'}_\alpha>0$.
		\item (Simetria) Sejam $s,s' \in G^{I}$. Se $s=s'$, então $\dist{s}{s'}_\alpha = 0 = \dist{s'}{s}_\alpha$. Se $s \neq s'$, então $\ords{s'-s} = \ords{-(s'-s)} = \ords{s-s'}$, portanto $\dist{s}{s'}_\alpha = \dist{s'}{s}_\alpha$.
		\item (Desigualdade triangular) Sejam $s,s',s'' \in G^{I}$. Se $s = s'$, então $\dist{s}{s'}_\alpha = 0$, portanto
			\begin{equation*}
			\dist{s}{s''}_\alpha = \dist{s'}{s''}_\alpha \leq 0 + \dist{s'}{s''}_\alpha = \dist{s}{s'}_\alpha + \dist{s'}{s''}_\alpha.
			\end{equation*}
		Se $s' = s''$, então $\dist{s'}{s''}_\alpha = 0$, portanto
			\begin{equation*}
			\dist{s}{s''}_\alpha = \dist{s}{s'}_\alpha \leq \dist{s}{s'}_\alpha + 0 = \dist{s}{s'}_\alpha + \dist{s'}{s''}_\alpha.
			\end{equation*}
		Se $s=s''$, então $\dist{s}{s''}_\alpha = 0$, portanto
			\begin{equation*}
			\dist{s}{s''}_\alpha = 0 \leq \dist{s}{s'}_\alpha + \dist{s'}{s''}_\alpha.
			\end{equation*}
		Se $s \neq s'$, $s' \neq s''$ e $s \neq s''$, então $\dist{s}{s'}_\alpha = \alpha^{\ords{s'-s}}$, $\dist{s'}{s''}_\alpha = \alpha^{\ords{s''-s'}}$ e $\dist{s}{s''}_\alpha = \alpha^{\ords{s''-s}}$. Como $0 \neq s''-s = (s''-s') + (s'-s)$, vale que
			\begin{equation*}
			\ords{s''-s} = \ords{(s''-s') + (s'-s)} \geq \ords{s''-s'} \opmin \ords{s'-s},
			\end{equation*}
		portanto, como $\alpha<1$,
			\begin{align*}
			\dist{s}{s''}_\alpha &= \alpha^{\ords{s''-s}} \\
				&= \alpha^{\ords{(s''-s') + (s'-s)}} \\
				&\leq \alpha^{\ords{s''-s'} \opmin \ords{s'-s}} \\
				&\leq \alpha^{\ords{s''-s'}} + \alpha^{\ords{s'-s}} \\
				&= \dist{s}{s'}_\alpha + \dist{s'}{s''}_\alpha.
			\end{align*}
	\end{itemize}

	Mostraremos que as topologias são iguais.

	\begin{itemize}
	\item (Topologia métrica é mais fina que topologia produto) Não usaremos que a topologia de $G$ é discreta. Seja $U \subseteq G^{I}$ um aberto básico. Por definição da topologia produto, $U$ é da forma
	\begin{equation*}
	U = \prod_{i \in I} U_i,
	\end{equation*}
	e existe um conjunto finito $F \subseteq I$ tal que, para todo $i \notin F$, $U_i = G$. Se $U = \emptyset$, % então, para algum $i \in F$, $U_i = \emptyset$. Nesse caso, 
	definindo $c := 0 \in G^{I}$ e $r := 0 \in \R$, segue que $\bola{c}{r} = \emptyset \subseteq U$. Caso contrário, seja $u \in U$. Se $F = \emptyset$, então $U = G^{I}$; definindo $r:=1 \in \R$, segue que $u \in \bola{u}{r} \subseteq G^{I} = U$. Se $F \neq \emptyset$, defina
		\begin{equation*}
		r := \alpha^{\max\abs{F}} = \alpha^{\max\set{\abs{i}}{i \in F}}.
		\end{equation*}
	Mostremos que $u \in \bola{u}{r} \subseteq U$. Note que $r>0$, pois $\alpha \in \intaa{0}{1}$, logo $u \in \bola{u}{r}$. Seja $x \in \bola{u}{r}$. Se $x=u$, então $x \in U$; caso contrário, $x-u \neq 0$, logo $\dist{u}{x}_\alpha = \alpha^{\ords{x-u}}$. %Como $x \in \bola{u}{r}$, segue que $\alpha^{\ords{x-u}} < r = \alpha^{\max\abs{F}}$, o que é equivalente a $\max\abs{F} < \ords{x-u}$, pois $\alpha \in \intaa{0}{1}$. Por sua vez, $\max\abs{F} < \ords{x-u}$ é equivalente a, para todo $i \in I$ tal que $\abs{i} \leq \max\abs{F}$, $x_i - u_i = 0$, ou seja, $x_i = u_i \in U_i$.
	Como $x \in \bola{u}{r}$, segue que
		\begin{equation*}
		\alpha^{\ords{x-u}} = \dist{u}{x}_\alpha< r = \alpha^{\max\abs{F}},
		\end{equation*}
	o que implica que $\max\abs{F} < \ords{x-u}$, pois $\alpha \in \intaa{0}{1}$. Por sua vez, isso implica que, para todo $i \in I$ tal que $\abs{i} \leq \max\abs{F}$, $x_i - u_i = 0$, ou seja, $x_i = u_i$. Como todo $i \in F$ satisfaz $\abs{i} \leq \max\abs{F}$, segue que $x_i = u_i \in U_i$, o que mostra que $x \in U$. Portanto $\bola{u}{r} \subseteq U$.

	\item (Topologia produto é mais fina que topologia métrica) Usaremos que a topologia de $G$ é discreta. Seja $\bola{c}{r} \subseteq G^{I}$ uma bola aberta. Se $r=0$, $\bola{c}{r} = \emptyset$; defina $U := \emptyset$, e segue que $U \subseteq \bola{c}{r}$. Se $r>0$, $\bola{c}{r} = \emptyset$; defina $k := \teto{\log_{\alpha}(r)}$ e $F := \set{i \in I}{\abs{i} \leq k}$, que é finito. Como a topologia de $G$ é discreta, para todo $i \in F$, $\{c_i\} \subseteq G$ é aberto; defina assim o aberto básico
		\begin{equation*}
		U := \prod_{i \in I} U_i,
		\end{equation*}
	em que $U_i := \{c_i\}$ se $i \in F$ e $U_i = G$ se $i \notin F$. %Mostremos que $u \in U \subseteq \bola{c}{r}$. Seja $x \in U$. Então, para todo $i \in F$, $x_i=c_i$. 
	%Mostremos que $U = \bola{c}{r}$, o que em particular implica $u \in U \subseteq \bola{c}{r}$. Seja $x \in G^{I}$. Então $x \in U$ se, e somente se, para todo $i \in F$, $x_i=c_i$.
	%Mas $x \neq c$ satisfaz $\ords{x-c} > k$ se, e somente se, para todo $i \in F$, $x_i - c_i = 0$, ou seja, $x_i = c_i$. Isso significa que $\dist{c}{x}_\alpha < \alpha^k$ se, e somente se, para todo $i \in F$, $x_i = c_i$. Como $\alpha \in \intaa{0}{1}$, $k = \teto{\log_{\alpha}(r)} \geq \log_{\alpha}(r)$ é equivalente a $\alpha^k \leq \alpha^{\log_{\alpha}(r)} = r$.

	%Mostremos que $U = \bola{c}{r}$.
	%\begin{itemize}
	%	\item ($U \subseteq \bola{c}{r}$) Seja $x \in U$. Então, para todo $i \in F$, $x_i=c_i$, ou seja, $x_i-c_i=0$. Se $x \neq c$, isso implica que $\ords{x-c} > k$ pela definição de $F$; como $\alpha < 1$, segue que $\dist{c}{x}_\alpha = \alpha^{\ords{x-c}} < \alpha^k$. Se $x=c$, então $\dist{c}{x}_\alpha = 0 < \alpha^k$, pois $\alpha > 0$. Em ambos os casos, $\dist{c}{x}_\alpha < \alpha^k$. Como $\alpha < 1$ e $k = \teto{\log_{\alpha}(r)} \geq \log_{\alpha}(r)$, segue que $\alpha^k \leq \alpha^{\log_{\alpha}(r)} = r$, portanto que $\dist{c}{x}_\alpha < \alpha^k \leq r$, logo $x \in \bola{c}{r}$.
	%
	%	\item ($\bola{c}{r} \subseteq U$) Seja $x \in \bola{c}{r}$. Então $\dist{c}{x}_\alpha < r$. Se $x=c$, então, para todo $i \in F$, $x_i=c_i \in \{c_i\}$, logo $x \in U$. Se $x \neq c$, então $\dist{c}{x}_\alpha = \alpha^{\ords{x-c}}$, portanto $\alpha^{\ords{x-c}} < r = \alpha^{\log_{\alpha}(r)}$. Como $\alpha < 1$, segue que $\ords{x-c} > \log_{\alpha}(r)$ e, já que $\ords{x-c} \in \N$, segue que $\ords{x-c} \geq \teto{\log_{\alpha}(r)} = k$. Se $\ords{x-c}>k$, isso implica que, para todo $i \in F$, $x_i-c_i=0$, ou seja, $x_i=c_i$; portanto $x \in U$. Se $\ords{x-c}=k$, então, para todo $i \in I$ tal que $\abs{i}<k$, $x_i-c_i=0$, e para algum $i \in I$ tal que $\abs{i}=k$, $x_i \neq 0$.
	%NOTA: TRAVEI AQUI, RESOLVI ADAPTAR A DEMONSTRAÇÃO.
	%\end{itemize}
	%Assim, para todo $x \in \bola{c}{r}$, vale $x \in U \subseteq \bola{c}{r}$.

	Mostremos que $c \in U \subseteq \bola{c}{r}$. Seja $x \in U$. Então, para todo $i \in F$, $x_i=c_i$, ou seja, $x_i-c_i=0$. Se $x \neq c$, isso implica que $\ords{x-c} > k$ pela definição de $F$; como $\alpha < 1$, segue que $\dist{c}{x}_\alpha = \alpha^{\ords{x-c}} < \alpha^k$. Se $x=c$, então $\dist{c}{x}_\alpha = 0 < \alpha^k$, pois $\alpha > 0$. Em ambos os casos, $\dist{c}{x}_\alpha < \alpha^k$. Como $\alpha < 1$ e $k = \teto{\log_{\alpha}(r)} \geq \log_{\alpha}(r)$, segue que $\alpha^k \leq \alpha^{\log_{\alpha}(r)} = r$, portanto que $\dist{c}{x}_\alpha < \alpha^k \leq r$, logo $x \in \bola{c}{r}$. Isso mostra que $U \subseteq \bola{c}{r}$. Além disso, $c \in U$, pois, para todo $i \in F$, $c_i \in \{c_i\}$, logo $c \in U$.

	Assim, dado $u \in \bola{c}{r}$, tomamos $r'>0$ tal que $\bola{u}{r'} \subseteq \bola{c}{r}$, e pela demonstração anterior existe aberto básico $U' \subseteq G^{I}$ tal que $u \in U' \subseteq \bola{u}{r'} \subseteq \bola{c}{r}$.
	\end{itemize}
\end{proof}


\section{Topologia em grupos}

Uma peculiaridade dos grupos topológico é que, como temos homeomorfismo do grupo para ele mesmo que levam qualquer elemento para a identidade, segue que basta estudar as vizinhanças abertas da identidade para entender a topologia do grupo. Isso fica mais claro com os resultados a seguir.

\begin{definition}
Seja $\bm G$ um grupo. Um \emph{sistema de vizinhanças da identidade} em $\bm G$ é um conjunto $\viz$ de subconjuntos de $G$ tal que
	\begin{enumerate}
	\item Para todo $A \in \viz$, $\id \in A$;
	\item Para todos $A,A' \in \viz$, $A \cap A' \in \viz$;
	\item Para todo $A \in \viz$, existe $B \in \viz$ tal que $B^2 \subseteq A$;
	\item	 Para todo $A \in \viz$, $A\inv \in \viz$;
	\item Para todos $A \in \viz$, $g \in G$, $gAg\inv \in \viz$.	
	\end{enumerate}
\end{definition}

Denotaremos por $\viz := \viz_\id^\circ$ as vizinhanças abertas da identidade $\id$.

\begin{proposition}
Seja $\bm G$ um grupo topológico. O conjunto $\viz$ de vizinhanças abertas de $\id$ é um sistema de vizinhanças da identidade.
\end{proposition}

\begin{definition}
Seja $\bm G$ um grupo. Uma topologia \emph{invariante por translação à esquerda} em $\bm G$ é uma topologia $\topo$ em $G$ tal que, para todos $A \in \topo$ e $g \in G$, $gA \in \topo$. Uma topologia \emph{invariante por translação à direita} em $\bm G$ é uma topologia $\topo$ em $G$ tal que, para todos $A \in \topo$ e $g \in G$, $Ag \in \topo$.
\end{definition}

\begin{proposition}
Sejam $\bm G$ um grupo e $\topo$ uma topologia em $G$.
	\begin{enumerate}
	\item Se $\topo$ é invariante por translação à esquerda em $\bm G$ então a topologia produto $\topo^2$ em $G \times G$ é invariante por translação à esquerda em $\bm G \times \bm G$;
	\item Se $\topo$ é invariante por translação à direita em $\bm G$ então a topologia produto $\topo^2$ em $G \times G$ é invariante por translação à direita em $\bm G \times \bm G$.
	\end{enumerate}
\end{proposition}

\begin{proposition}
Seja $\bm G$ um grupo e $\topo$ uma topologia invariante por translação à esquerda e à direita em $\bm G$. Então $((G,\topo),\times,\div,\id)$ é um grupo topológico se, e somente se,
	\begin{enumerate}
	\item $\times$ é contínua em $(\id,\id)$;
	\item $\div$ é contínua em $\id$.
	\end{enumerate}
\end{proposition}

\begin{proposition}
Sejam $\bm G$ um grupo e $\viz$ um sistema de vizinhanças da identidade em $\bm G$. Existe uma única topologia $\topo$ em $G$ tal que $((G,\topo),\times,\div,\id)$ é um grupo topológico e $\viz$ é um sistema fundamental de vizinhanças de $\id$ em $(G,\topo)$.
\end{proposition}

\begin{proposition}
Seja $\bm G$ um grupo topológico. São equivalentes:
	\begin{enumerate}
	\item $(G,\topo)$ é separado;
	\item $\{\id\}$ é fechado;
	\item $\bigcap_{A \in \viz} A = \{\id\}$.
	\end{enumerate}
\end{proposition}

\section{Distância em grupos}

\begin{definition}
Seja $\bm G$ um grupo. Uma distância \emph{invariante por translação à esquerda} em $\bm G$ é uma distância $\dist{\var}{\var}\colon G \times G \to \intfa{0}{\infty}$ em $G$ tal que, para todo $g \in G$, a translação à esquerda $\trE{g}$ é uma isometria local em $(G,\dist{\var}{\var})$. Uma distância \emph{invariante por translação à direita} em $\bm G$ é uma distância em $G$ tal que, para todo $g \in G$, a translação à direita $\trD{g}$ é uma isometria local. Uma distância \emph{invariante por translação} em $\bm G$ é uma distância invariante à esquerda e à direita em $\bm G$.
\end{definition}

No caso da invariância à esquerda, isso quer dizer que, para todos $g,g',g'' \in G$,
	\begin{equation*}
	\dist{gg'}{gg''} = \dist{g'}{g''}.
	\end{equation*}
O caso à direita é análogo.

\begin{proposition}
Seja $\bm G$ um grupo topológico e $\viz$ um sistema de vizinhanças da identidade enumerável. Existem distâncias $\dist{\var}{\var}_E$ e $\dist{\var}{\var}_D$ em $G$ invariantes por translação à esquerda e à direita em $\bm G$, respectivamente, e compatíveis com a topologia de $\bm G$.
\end{proposition}


\section{Homomorfismos contínuos}

\begin{proposition}
Sejam $\bm G$ e $\bm H$ grupos topológicos e $\phi\colon G \to H$ um homomorfismo de grupos. Então $\phi$ é contínuo se, e somente se, $\phi$ é contínuo na identidade $\id \in G$.
\end{proposition}
\begin{proof}
A ida é evidente; basta mostrar a volta. Como $\phi$ é homomorfismo, $\phi \circ \trE{g} = \trE{\phi(g)} \circ \phi$ para todo $g \in G$. Mas $\trE{\phi(g)} \circ \phi$ é contínua em $\id$, o que implica que $\phi \circ \trE{g}$ é contínuo em $\id$ e, como $\trE{g}$ é um homeomorfismo, segue que $\phi$ é contínuo em $g=\trE{g}(\id)$.
\end{proof}

\begin{proposition}
Sejam $\bm G$ e $\bm G'$ grupos topológicos, $\bm G'$ separado. Uma função $\phi\colon G \to G'$ é homomorfismo contínuo se, e somente se,
$\graf(\phi)$ é subgrupo de $\bm G \times \bm G'$ homeomorfo a $G$ pela projeção $\proj_G\colon G \times G' \to G$.
\end{proposition}

\section{Subgrupos topológicos}

\begin{definition}
Seja $\bm G = ((G,\topo),\times,\div,\id)$ um grupo topológico. Um \emph{subgrupo topológico} de $\bm G$ é uma quádrupla $\bm S = ((S,\topo_S),\times|_S,\div|_S,\id)$ em que $(S,\topo|_S)$ é um subespaço topológico de $(G,\topo)$ e $(S,\times|_S,\div|_S,\id)$ é um subgrupo de $(G,\times,\div,\id)$.
\end{definition}

\begin{proposition}
Seja $\bm G$ um grupo topológico e $\bm S$ um subgrupo de $\bm G$. Então $\bm S$ é um grupo topológico.
\end{proposition}
\begin{proof}
Temos que mostrar que as operações $\times|_S$ e $\div|_S$ são contínuas.
	\begin{enumerate}[wide, labelwidth=!, labelindent=0pt]
	\item[($\times|_S$ é contínua)]
		Seja $C \subseteq G$. Como
		\begin{equation*}
		\times|_S \inv (C \cap S) = \times \inv (C) \cap (S \times S),
		\end{equation*}
		portanto para todo aberto $A \cap S$ de $S$, com $A \subseteq G$ aberto de $G$, segue que $\times|_S \inv (A \cap S) = \times \inv (A) \cap (S \times S)$ é aberto em $S \times S$, logo $\times|_S$ é contínua.
	\item[($\div|_S$ é contínua)]
		Análogo ao caso do produto. \qedhere
	\end{enumerate}
\end{proof}

\begin{proposition}
Seja $\bm G$ um grupo topológico.
	\begin{enumerate}
	\item Se $\bm S$ é subgrupo topológico de $\bm G$, então $\bm{\Fec{S}}$ é um subgrupo topológico de $\bm G$.
	\item Se $\bm N$ é subgrupo topológico normal de $\bm G$, então $\bm{\Fec{N}}$ é um subgrupo topológico normal de $\bm G$.
	\end{enumerate}
\end{proposition}

\begin{proposition}
Sejam $\bm G$ um grupo topológico e $\bm S$ é subgrupo topológico de $\bm G$.
	\begin{enumerate}
	\item Se $\Int{S} \neq \emptyset$, então $S$ é aberto;
	\item Se $S$ é aberto, então $S$ é fechado.
	\end{enumerate}
\end{proposition}
\begin{proof}
	\begin{enumerate}
		\item Seja $g \in S$. Para todo $g' \in S$, o conjunto $g'g\inv S$ é aberto e $g'\in g'g\inv S \subseteq S$, 
		\item 
		\end{enumerate}
\end{proof}

\section{Conexidade em grupos topológicos}

\begin{definition}
Seja $\bm G$ um grupo topológico. A componente conexa de $\id$ é denotada $G_\id$.
\end{definition}

\begin{proposition}
Seja $\bm G$ um grupo topológico. Então $\bm{G_\id}$ é um subgrupo topológico normal e fechado de $\bm G$ e existe bijeção entre as componentes conexas de $\bm G$ e as classes laterais de $G_\id$.
\end{proposition}

\begin{proposition}
Seja $\bm G$ um grupo topológico.
	\begin{enumerate}
	\item Se $\bm G$ é localmente conexo, então $G_\id$ é aberta;
	\item Se $\bm G$ é conexo, para toda vizinhança $A$ de $\id$,
		\begin{equation*}
		G = \bigcup_{n \in \N} A^n.
		\end{equation*}
	\end{enumerate} 
\end{proposition}

\section{Grupo de homeomorfismos}

Seja $\bm X$ um espaço topológico. O espaço $\Iso{\Cont}(X) = \Iso{\Cont}(X,X)$ dos homeomorfismos de $\bm X$ para $\bm X$ é um grupo com produto de composição $\circ$, a inversa de função $\inv$ e a função identidade $\Id$. Uma questão interessante então é definir uma topologia para esse espaço.

Num primeiro momento, podemos notar que podemos munir esse espaço da topologia produto no seguinte sentido. O conjunto $\Iso{\Func}(X)$ de bijeções de $X$ para $X$ é subconjunto do conjunto $\Func(X)$ de funções de $X$ para $X$. Esse conjunto é também denotado $X^X$, o que deixa mais explícito sua estrutura como um espaço produto. Como $\bm X$ é um espaço topológico, podemos explicitar os abertos sub-básicos dessa topologia. Eles consistem em um produto do tipo $\mathcal A_{x_0} := \prod_{x \in X} A_x$, em que $x_0 \in X$, $A_{x_0} \subseteq X$ é aberto e, para todo $x \in X \setminus \{x_0\}$, $A_x = X$. O conjunto de pontos desse aberto de $X^X$ corresponde ao conjunto de funções $f \in \Func(X)$ tais que $f(x_0) \in A_{x_0}$. Os abertos básicos dessa topologia são interseções finitas desses abertos sub-básicos. Pode-se mostrar que essa é a topologia pontual ou finito-aberto, a topologia gerada pelos abertos sub-básicos
	\begin{equation*}
	\mathcal A_{C,A} := \set{f \in \Func(X)}{f(C) \subseteq A},
	\end{equation*}
em que $C \subseteq X$ é um conjunto finito e $A \subseteq X$ é um conjunto aberto. Finalmente, tendo essa topologia produto em $\Func(X)$, podemos munir $\Iso{\Cont}(X)$ com a topologia induzida de subespaço.

No entanto, notemos que a topologia de $X$ só foi utilizado quando $X$ tinha o papel de contradomínio das funções $\Func(X)=\Func(X,X)$; a mesma construção poderia ser feita para o conjunto de funções $\Func(I,X)$, em que $I$ é um conjunto qualquer, e ainda teríamos uma topologia nesse espaço de funções. Se queremos levar em consideração a topologia de $X$ como domínio, devemos construir uma outra topologia. A topologia geralmente adotada nesse caso é a topologia compacto-aberto, a topologia gerada pelos abertos sub-básicos
	\begin{equation*}
	\mathscr A_{K,A} := \set{f \in \Cont(X)}{f(K) \subseteq A},
	\end{equation*}
em que $K \subseteq X$ é um conjunto compacto e $A \subseteq X$ é um conjunto aberto. Note que agora nos restringimos a $\Cont(X)$ em vez de $\Func(X)$. Essa topologia é maior (ou mais fina) que a topologia finito-aberto, pois todo conjunto finito é compacto, e leva em consideração a topologia de $X$ tanto como domínio quanto como contradomínio.

A topologia compacto-aberto não garante, porém, que o grupo de homeomorfismos seja um grupo topológico. Tanto a composição como a inversão de funções pode não ser contínua para alguns espaços topológicos. Existe uma outra topologia que garante isso, a topologia compacto-cocompacto.

%\begin{definition}
%Seja $\bm X$ um espaço topológico. A topologia de $\Iso{\Cont}(X)$ é a topologia compacto-aberto, denotada $\topo_\Cont$.
%\end{definition}

\begin{proposition}
Seja $\bm X$ um espaço topológico. O espaço do homeomorfismos
	\begin{equation*}
	\left( \left( \Iso{\Cont}(X),\topo_\Cont \right),\circ,\inv,\Id \right)
	\end{equation*}
é um grupo topológico.
\end{proposition}
\begin{proof}
Para qualquer conjunto $X$, o espaço de bijeções $( \Iso{\mathcal F}(X),\circ,\inv,\Id )$ é um grupo. No caso de $\bm X$ ser um espaço topológico, temos que $\Iso{\Cont}(\bm X) \subseteq \Iso{\mathcal F}(X)$, pois todo homeomorfismo é uma bijeção. Devemos mostrar que $\Iso{\Cont}(\bm X)$ é um subgrupo. Para isso, notemos que, para todas $f,f' \in \Iso{\Cont}(X)$, $f' \circ f$ é contínua, pois é composta de contínuas, e tem inversa contínua, pois
	\begin{equation*}
	(f' \circ f)\inv = f\inv \circ (f')\inv,
	\end{equation*}
que é contínua porque é composta das contínuas $f\inv$ e $(f')\inv$. Ainda, para toda $f \in \Iso{\Cont}(X)$, por definição temos $f\inv \in \Iso{\Cont}(X)$. Por fim, claramente $\Id$ é contínua, o que mostra que $\Iso{\Cont}(X)$ é um subgrupo de $\Iso{\mathcal F}(X)$.

%A topologia de $\Iso{\Cont}(X)$ é a topologia de subconjunto induzida da topologia produto de $\Iso{\mathcal F}(X)$. Devemos mostrar que as operações $\circ$ e $\inv$ são contínuas.
Devemos agora mostrar que a composição e a inversão de funções são contínuas com respeito à topologia compacto-aberto.
\end{proof}

\section{Ação contínua}

\begin{definition}
Sejam $\bm G$ um grupo topológico e $\bm X$ um espaço topológico. Uma \emph{ação contínua} de $\bm G$ em $\bm X$ é uma ação
	\begin{align*}
	\func{\pt}{G \times X}{X}{(g,x)}{g \pt x}
	\end{align*}
que é uma função contínua\footnote{Com respeito à topologia produto de $G \times X$.}. Denota-se $\pt\colon \bm G \age \bm X$. O grupo $\bm G$ \emph{age continuamente} em $\bm X$.
\end{definition}

Lembremos que uma ação é uma função que satisfaz
	\begin{enumerate}
	\item (Identidade) Para todo $x \in X$,
		\begin{equation*}
		\id\pt x = x;
		\end{equation*}
	\item (Compatibilidade) Para todos $g,g' \in G$ e $x \in X$,
		\begin{equation*}
		(g'g)\pt x = g'\pt(g\pt x).
		\end{equation*}
	\end{enumerate}

Para cada $g \in G$, ação $\pt$ induz uma função contínua
	\begin{align*}
	\func{g\pt}{X}{X}{x}{g\pt x.}
	\end{align*}
Nesse caso, temos um homomorfismo de grupos topológicos (homomorfismo contínuo de grupo)
	\begin{align*}
	\func{\pt}{G}{\Iso{\Cont}(X)}{g}{
		\begin{aligned}[t]
		\func{g\pt}{X}{X}{x}{g\pt x}
		\end{aligned}
	}
	\end{align*}

\begin{comment}

\begin{proof}
\begin{enumerate}
	\item [($\Rightarrow$)] Suponhamos que a ação $\fun{\varphi}{G \times X}{X}$ seja contínua e mostremos que a ação $\fun{\phi}{G}{\Iso{\Cont}(X)}$ é contínua. Para isso,
 
	\item [($\Leftarrow$)]
\end{enumerate}
\end{proof}

\end{comment}