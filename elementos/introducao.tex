\phantomsection
\addcontentsline{toc}{chapter}{Introdução}
\chapter*{Introdução}

Este livro trata de teoria matemática básica e foi escrito para que pudesse ser lidos de dois modos que eu acredito serem complementares. Como é costume em qualquer texto matemático teórico, as definições, proposições e demonstrações são destacadas em ambientes específicos dentro do texto. Entre esses ambientes está a discussão de aspectos relevantes a respeito de cada uma das definições, proposições e demonstrações. No entanto, o leitor interessado somente no encadeamento lógico da teoria pode seguir somente pelos ambientes destacados sem preocupação.

O conteúdo do livro está dividido em três partes. A primeira desenvolve aspectos básicos da teoria de Conjuntos e das teorias de funções e relações. Além disso, são discutidas construções de conjuntos numéricos, que serão importantes como modelos mais gerais estudados à frente. A segunda trata das bases da Álgebra, focando principalmente nas estruturas algébricas mais importante: grupos, anéis, corpos e espaços vetoriais. A terceira parte diz respeito à Topologia e à Geometria, tanto num aspecto amplo da topologia geral como da topologia dos espaços métricos, e preparam para o estudo de espaços reais e da análise em geral, em que são discutidas apresentadas as teorias de diferenciação, integração e medida.