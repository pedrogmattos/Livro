\phantomsection
\addcontentsline{toc}{chapter}{Prefácio}
\chapter*{Prefácio}

Este projeto nasceu de algumas notas de aula pessoais durante um curso de graduação a que assisti no segundo semestre de 2016. O nome do curso era \textit{Anéis e Corpos} e o contato com o formalismo matemático e com os resultados da teoria matemática apresentada, que começou com a definição de um anel e terminou com a demonstração de que não existe solução em radicais de polinômios de grau maior que 4, foram muito estimulantes. Ao mesmo tempo, eu tinha recentemente aprendido a usar o \LaTeX e estava curioso para testar minhas novas técnicas em algum projeto maior. Esses fatores me levaram a começar a digitar.

Eu posso dizer que este projeto é fruto de dois interesses pessoais: um teórico e um estético. O primeiro, certamente, foi minha motivação principal. O interesse por conhecer e descobrir mais matemática é o que me levou a começar este livro. O segundo, no entanto, também teve seu papel. O prazer estético de produzir um livro desde o zero, o que o \LaTeX proporciona a qualquer leigo em diagramação como eu, estimularam-me em vários momentos, inclusive quando o cansaço ou o desinteresse me impediam de progredir com a escrita.

Originalmente, meu objetivo era organizar algumas notas para estudar e compartilhar com colegas de curso. Já então eu pretendia disponibilizar essas notas para os futuros graduandos em matemática, mas os objetivos do projeto ainda não eram tão grandes. Quando comecei a considerar essas ideias \--- escrever um livro e disponibilizá-lo à comunidade \--- fiquei muito animado e, ao longo dos meses seguintes, ampliei rapidamente meus horizontes. Passei da ideia de escrever notas sobre um curso específico ao objetivo (ingenuamente) extenso de escrever sobre toda a matéria da graduação, tirando e pondo uma coisa ou outra. Como era de se esperar, rapidamente percebi que o projeto estava ficando longo demais para se ver um fim próximo.

Muito do livro ainda não está escrito. De fato, quase nada. A falta de tempo me permitiu somente escrever às vezes e, por isso, decidi primeiro digitar um esquema lógico das definições e teoremas para depois completar com comentários, explicações, discussões e dúvidas que tinha na cabeça enquanto o escrevia. Reluto em disponibilizar este material como está principalmente por não querer que ele se trate de uma apostila normativa e seca \--- uma lista de definições, teormeas e demonstrações \--- mas sim querer que se trate de um livro que apresente resultudados teóricos, gere dúvidas relevantes e suscite discussão. Sob certa óptica, esse objetivo pode parecer presunçoso, mas ele é, na verdade, esperançoso.

\vfill

\begin{flushright}
P. G. M.\\
Campinas, 8 de julho de 2017
\end{flushright}
